\environment env_physics
\environment env_MetaPost

% Fonts
%
\setupbodyfont [libertinus]%
% Old style numerals in text
%\definefontfeature[default][default][
%	pnum=yes, % proportial numbers
%	onum=yes, % oldstyle numbers
%	itlc=yes, % italic correction
%]
\setupbodyfont [libertinus,11pt]% Old style numerals in text
\definefontfeature[default][default][
	pnum=yes, % proportial numbers
	onum=yes, % oldstyle numbers
	itlc=yes, % italic correction
]
\definefontfeature[smallcaps][default][ % smallcaps
  %lnum=no,
  smcp=yes, % smallcaps
	script=latn,
]
\definefontfeature[normnum][default][ % normal proportional numbers
  onum=no,pnum=yes,tnum=yes,
]
\definefontfeature[tabnum][default][ % tabular numbers
  onum=no,pnum=no,tnum=yes,
]

%\setupinterlinespace[line=2.6ex]

\setuppapersize[letter]
%\showframe

\setuplayout[
  topspace=.5in,
  backspace=1in,
  height=middle,
  width=middle,
  header=0.25in,
  headerdistance=0.25in,
  footer=0.25in,
  footerdistance=0.25in,
  margin=0.5in,
  margindistance=0.25in,
]
	
\setuppagenumbering
	[location=bottom]

\setuptabulate[header=text, EQ={=}, unit=.33em, blank=3cm]

%\setupheader[text][]
\setupheadertexts[][{Name: \hl[12]\; Period: \hl[2]}]%[Name:\rule{2in}{.4pt} Period:\rule{.25in}{.4pt}]

\define\jwst{{\sc jwst}}

\starttext
%\showlayout[in,cm]

\startalignment[center]
  %\blank[5*big] % Does not do anything
    {\tfc JWST Problem}
  \blank[small]
    {\tf LCHS Physics}
%  \blank[small]
%    {\tf \currentdate[month,day,{, },year]}
  \blank[3*big]
\stopalignment

We would like to find the distance $d$ to the new home of the James Webb Space Telescope and the L2 Lagrange point. This turns out to be easy to set up, and a mess to solve.
The centripetal force required to keep \jwst\ in orbit is
\startformula
  F = p\omega = \mu\omega^2(r+d) = \frac{GM\mu}{r^3}(r+d),
\stopformula
where $\mu$ is the \jwst\ mass and $\omega$ is the angular velocity of both Earth and \jwst\ about the Sun. This centripetal force is provide by the combined gravitational force of Earth and Sun on \jwst. (Only need the magnitude, so no minus signs.)
\startformula
  F = G\frac{M\mu}{(r+d)^2} + G\frac{m\mu}{d^2}
\stopformula
where the Sun's mass is $M$ and Earth's mass is $m$. These forces are the same force.
\startformula\startmathalignment
\NC  \frac{GM\mu}{r^3}(r+d) \NC = G\frac{M\mu}{(r+d)^2} + G\frac{m\mu}{d^2} \NR
\NC  \frac{M}{r^3}(r+d) \NC = \frac{M}{(r+d)^2} + \frac{m}{d^2} \NR
\stopmathalignment\stopformula
Now we just need to solve for $d$. Yikes! Get rid of these crazy denominator by multiplying by $r^3(r+d)^2d^2$. Then expand.
\startformula\startmathalignment
\NC  M(r+d)^3 d^2 \NC = Mr^3d^2 + mr^3(r+d)^2 \NR
\NC  \cancel{Mr^3d^2} + 3Mr^2d^3 +3Mrd^4 +Md^5 \NC = \cancel{Mr^3d^2} + mr^5 + 2mr^4d + mr^3d^2 \NR
\stopmathalignment\stopformula
The first terms on each side canceled each other. Since $r\gg d$, we can keep only the first remaining term on each side – the terms with the highest power of $r$.
\startformula\startmathalignment
\NC  3Mr^2d^3 \NC = mr^5 \NR
\NC  d^3 \NC = \frac{m}{3M}r^3 \NR
\NC  d \NC = r\,\,\sqrt[3]{\frac{m}{3M}} \NR
\stopmathalignment\stopformula
That was not a nice procedure. The Sun has the mass of $332{,}950$ Earth's, which gives
\startformula
  d = 1.00\sci{-2}\,r = 1.5\sci{9}\units{m}.
\stopformula
\stoptext