\startenvironment env_world
% where \product and \component look for TeX input files
\usepath[{volume01,volume02}]

% Document Setup: Font’s, Layout, Special sections

% Fonts
%
\usetypescriptfile[libertinus]
\setupbodyfont [libertinus,11pt]%
% Old style numerals in text
\definefontfeature[default][default][
	pnum=yes, % proportial numbers
	onum=yes, % oldstyle numbers
	itlc=yes, % italic correction
]
\definefontfeature[smallcaps][default][ % smallcaps
  %lnum=no,
  smcp=yes, % smallcaps
	script=latn,
]
\definefontfeature[normnum][default][ % normal proportional numbers
  onum=no,pnum=yes,tnum=yes,
]
\definefontfeature[tabnum][default][ % tabular numbers
  onum=no,pnum=no,tnum=yes,
]


% Layout
%
% Page Layout
\setuppapersize[letter]
%\showframe
\setuplayout[
	topspace=36pt,% Vertical layout
	height=702pt, % 9 3/4 in, text height = 8 in
	header=18pt, % 1/2 in
	headerdistance=27pt, % 3/8 in
	footerdistance=27pt, % 3/8 in
	footer=18pt,% 1/2 in
	backspace=90pt, % Horizontal layout
	leftmargin=36pt, % 1/2 in
	leftmargindistance=9pt, % 3/8 in
	width=306pt, % 4 1/4 in
	rightmargindistance=27pt, % 3/8 in
	rightmargin=144pt,  % 2 in
	]
\definemeasure [fullwidth]% Used for full width figures
	[\textwidth+\rightmarginwidth+\rightmargindistance]

%
% PDF features (HR) GP: This produced bold figure and table refs, so I turned it off.
%
%\startnotmode[print]
%\setupinteraction[state=start,color=,contrastcolor=,]
%\placebookmarks[chapter,section,subsection][chapter]
%\enabledirectives[references.bookmarks.preroll]
%\setupinteractionscreen[
%  option={doublesided,bookmark},
%  width=fit,height=max,
%]
%\stopnotmode
%\startmode[print]
%\setupinteraction[state=stop]
%\setupbackend[
%  format=PDF/X-4,
%  intent={Coated FOGRA39 (ISO 12647-2:2004)}, % European print standard; use SWOP?
%]
%\stopmode


%
% Chapter heads
\setuplabeltext [en] [chapter=Chapter~]
\define[2]\MyChapter
	{\framed[frame=off,width=broad,align=flushleft]{#1 \blank #2}}
\definetext[chapter][footer][pagenumber]
\setuphead [chapter] [
	command=\MyChapter,
	header=high, footer=chapter, page=right,
	indentnext=no,
	numberstyle=scxx, textstyle=tfc,
]
% Versals for chapter openers:
%\usemodule[lettrine] % GP: I switched to initials.
\setupinitial[
	%distance=-1em, % text from initial (initial size)
	n=2,
	hoffset=0.05em, % initial, initial size, positive is left
	font=Serif*default at 27pt,
	before={\blank[none,disable,back]},
]
\define[2]\Initial{\noindentation\placeinitial{#1}\strut\hskip-0.33em\scaps{#2}}
%\define[2]\Initial{\lettrine{#1}{#2}} % GP: Using this until initials work.

% Section heads
\setuphead[section][
	before=\blank,
	after=,
	sectionsegments=none,
	style=\scaps,
]

% Page heads and page numbers
\setuppagenumbering[location=,alternative=doublesided]
\defineconversionset[frontpart:pagenumber][][romannumerals]
\defineconversionset[bodypart:pagenumber][][numbers]
%\defineconversionset[appendix:pagenumber][][Characters]
\setupheadertexts[text][section][][][chapter]
\setupheadertexts[margin][][\hfill\userpagenumber][\userpagenumber \hfill][]

% Indenting
\setupindenting[yes,medium]

% Emphasis
% HR: setup changed, use highlights, not just defines. GP: Great.
\setupbodyfontenvironment[default][em=italicface]
\definehighlight[emph][style={\em}]
\definehighlight[textit][style=italicface]
\definehighlight[scaps][style={\addff{smallcaps}}]
\definehighlight[booktitle][style=italicface]
\define\dash{|\,--\,|}
\define\visviva{\emph{vis viva}}
\define\Visviva{\emph{Vis viva}}

% References within text
% HR: do you need to be LaTeX compatible? GP: No. I prefer to use ConTeXt commands
\define[1]\ref{\in[#1]}
\define[1]\pageref{\at[#1]}

% Quotes
\setupdelimitedtext[blockquote][
	spacebefore=medium, spaceafter=medium, % standard is the size of \blank, medium is a little smaller.
	indenting=no, % blockquotes should not contain multiple paragraphs.
	style=\nowhitespace,
	indentnext=no, % no = never indent after a blockquote, auto = usual paragraph breaks
]

% Citations and Bibliography
\usebtxdefinitions[apa] % Use APS or APA style Bibliography
\usebtxdataset[common/BookBib.bib]
\define[2]\autocite{\endnote{\cite[righttext={{#1}}][#2]}}%\autocite[p.~123]{Leibniz1695}
\define[1]\FNN{{\addff{normnum}#1.}} % footnote number
\setupnote[endnote][
	location=text,
	%textcommand=\FNN, % footnote marker in text
	%symbolcommand={\#1.}, % HR: does this work at all? GP: I don't think so.
] % Endnotes are used for citations
\setupnotation[endnote][numbercommand=\FNN] % footnote number in list
\setupbtx [apa:cite][alternative=authoryears] %

% Formulas
%\setupformulas[spacebefore=small,spaceafter=small]

% Example Problems
\defineenumeration[example]
\setupenumeration[example][
	margin=no,
	headstyle=\ss\sc,
	text=Example, % This text is part of the number, e.g. "Example 2.3"
	style=\ss,
	alternative=serried,
	title=no, % This would be a title after the number
	prefix=yes, % Turns on the prefix
	prefixsegments=chapter, % makes the prefix the chapter number
	way=bychapter, % Numbers within each chapter
	%margin=standard, % Indents block on the left
	width=fit, % Makes the space between the number and the question a good size
	indentnext=no,
]
\definedelimitedtext[solution][
	spacebefore=medium,
	spaceafter=medium,
	style=\rm\it,
	leftmargin=standard, % Indents block on the left
	rightmargin=yes, % Indents block on the right
	indentnext = no,
]

% Exercises
\define\question{\item}
% Answer boxes and highlight boxes are currently turned off.
\define[1]\answer{#1}
\define[1]\highlightbox{#1}

%
% Floats
%

% Epigraphs

% Figures & Tables
%
\setupexternalfigures[
	maxwidth=\measure{fullwidth},
	directory={images},
]
\defineexternalfigure[widefigure][width=\measure{fullwidth}]

\definecounter[table][figure] % table and figure numbers increase together

% Captions - always in the margin
\setupcaptions[figure,table][
	width=\rightmarginwidth,
	location=bottom, % Places caption below the figure
	headstyle=\sc,
	style=\small,
	align=flushleft,
]
% Sizes - Side (default), Text, and Wide
%
% Side floats are the default
\definefloat[sidefloat][sidefloats][figure]
\setupfloat[sidefloat][default=outermargin]
\setupcaption[sidefloat][
	width=\outermarginwidth,
	location=bottom,
]
% Text floats are the width of the text block with the caption in the margin
\definefloat[textfloat][textfloats][figure]
\setupfloat[textfloat][
	location=inner,
	width=\measure{textwidth},
	default={top,bottom},
]
\setupcaption[textfloat][
	width=144pt,
	location={high,outermargin},
]
% Wide Floats are the full width with the caption above or below in the margin
\definefloat[widefloat][widefloats][figure]
\setupfloat[widefloat][
	location=inner,
	width=\measure{fullwidth},
	default={bottom},
]
\setupcaption[widefloat][
%	width=\rightmarginwidth,
	location=outermargin,
%	headstyle=\sc,
%	style=\small,
%	align=flushleft,
%	spacebefore=-3in,%Temporary! Puts caption above for Ch5
	spacebefore=\the\dimexpr\naturalfloatht+\lineheight\relax,% This puts the caption below
]

% Abreviations

\define\ie{{\sl i.e.}}


% Text Fractions
% HR: this might be easier with OpenType features. GP: I'm almost certain it would be.

% numerator
\define\fracone{\getnamedglyphdirect{libertinus}{onesuperior}} % 1
\define\fractwo{\getnamedglyphdirect{libertinus}{twosuperior}} % 2
\define\fracthree{\getnamedglyphdirect{libertinus}{threesuperior}} % 3
\define\fracfour{\getnamedglyphdirect{libertinus}{foursuperior}} % 4
\define\fracfive{\getnamedglyphdirect{libertinus}{fivesuperior}} % 5
\define\fracseven{\getnamedglyphdirect{libertinus}{sevensuperior}} % 7
\define\fraceight{\getnamedglyphdirect{libertinus}{eightsuperior}} % 8
\define\fracnine{\getnamedglyphdirect{libertinus}{ninesuperior}} % 9
\define\fracfifteen{\getnamedglyphdirect{libertinus}{onesuperior}\getnamedglyphdirect{libertinus}{fivesuperior}} % 15
\define\fracsixteen{\getnamedglyphdirect{libertinus}{onesuperior}\getnamedglyphdirect{libertinus}{sixsuperior}} % 16

% /denominator
\define\halves{\textfraction\getnamedglyphdirect{libertinus}{twoinferior}} % /2
\define\thirds{\textfraction\getnamedglyphdirect{libertinus}{threeinferior}} % /3
\define\quarters{\textfraction\getnamedglyphdirect{libertinus}{fourinferior}} % /4
\define\ninths{\textfraction\getnamedglyphdirect{libertinus}{nineinferior}} % /4
\define\twelfths{\textfraction\getnamedglyphdirect{libertinus}{oneinferior}\getnamedglyphdirect{libertinus}{twoinferior}} % /16
\define\sixteenths{\textfraction\getnamedglyphdirect{libertinus}{oneinferior}\getnamedglyphdirect{libertinus}{sixinferior}} % /16


% numerator/denominator
\define\threehalves{\text{\fracthree\halves}} %  3/2
\define\fourthirds{\text{\fracfour\thirds}} %  4/3
\define\fivequarter{\text{\fracfive\quarters}} % 5/4
\define\sixteenquarter{\text{\fracsixteen\quarters}} % 16/4
\define\ninequarter{\text{\fracnine\quarters}} % 9/4
\define\eightninths{\text{\fraceight\ninths}} % 9/4
\define\onetwelfth{\text{\fracone\twelfths}} % 1/12
\define\twotwelfths{\text{\fractwo\twelfths}} % 2/12
\define\fivetwelfths{\text{\fracfive\twelfths}} % 5/12
\define\seventwelfths{\text{\fracseven\twelfths}} % 7/12
\define\fifteensixteenths{\text{\fracfifteen\sixteenths}} % 15/16

% Index
\define[1]\keyterm{{\em #1}\index{#1}} % See "Context Excursion"
%\newcommand{\keyterm}[2][\defaultindexentry]{% Optional argument [#1] goes in index, {#2} is in the text.
%{\newcommand{\defaultindexentry}{#2}\newcommand{\indexentry}{#1}%
%\textit{#2}\index{\indexentry|textbf}}% \textbf{#2}
%}

\stopenvironment
