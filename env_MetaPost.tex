\startenvironment env_MetaPost
\usemodule [graph]

% Specifically for MetaPost figures

\startbuffer[startMP]
\environment env_physics
\environment env_MetaPost
\setupbodyfont [libertinus,11pt]
\setoldstyle % Old style numerals in text
   \startMPpage
\stopbuffer

\startbuffer[stopMP]
   \stopMPpage
\stopbuffer

\define[3]\marginMP{
  \startplacefigure[location=margin, reference=fig:#2, title={#3}]
    #1\typesetbuffer[startMP,#2,stopMP]
  \stopplacefigure
}

% Cancel command for use in equations. Perhaps one command can work in both italic and roman.
\startuniqueMPgraphic{cross out}
  picture cross;
  cross := image(draw (0,0)--(1,1); draw (0,1)--(1,0););
  draw cross xscaled \overlaywidth yscaled \overlayheight withpen pencircle scaled .8pt ;
\stopuniqueMPgraphic

\defineoverlay[canceloverlay][\uniqueMPgraphic{cross out}]

\defineframed[cancelframe][background=canceloverlay, frame=off, offset=overlay]

\define[1]\cancel{%
	\ifmmode
		\mathchoice
		{\cancelframe{$\displaystyle #1$}}
		{\cancelframe{$\textstyle #1$}}
		{\cancelframe{$\scriptstyle #1$}}
		{\cancelframe{$\scriptscriptstyle #1$}}
	\else%
		\cancelframe{#1}%
	\fi%
}

\define[1]\ucan{%
	\ifmmode
		\mathchoice
		{\cancelframe{$\rm\displaystyle #1$}}
		{\cancelframe{$\rm\textstyle #1$}}
		{\cancelframe{$\rm\scriptstyle #1$}}
		{\cancelframe{$\rm\scriptscriptstyle #1$}}
	\else%
		\cancelframe{#1}%
	\fi%
}

\startMPextensions % Extensions apply to all instances that accept them.
% shawdow(picture, visibility) draws a shadow of the picture (but not the picture) visibility should be between 0-1, with 0 being invisible and 1 being a shadow that is reduces the background brightness by 20%. If visibility is greater than 1, it will make the shadow even darker, but 20% seems to be plenty. (change 0.8background to something else to change the maximum darkness.) The picture should be shifted and rotated in the command, not afterwards. E.g. shadow(cow rotate 45 shifted (10cm, 5cm), .3);.
  vardef shadow(expr p, v) = 
    draw redecorated (draw p) withcolor v[background,0.8background];
  enddef; 

picture ball;
	ball := image(
		draw fullcircle scaled 1cm
      			withshademethod "circular"
			withshadecenter (.25,.25)
       		withcolor "white" shadedinto black;
		draw fullcircle scaled 1cm withpen pencircle scaled 0.8pt;);

picture littleball;
	littleball := image(
		draw fullcircle scaled 0.2cm
      			withshademethod "circular"
			withshadecenter (0.2, 0.2)
       		withcolor "white" shadedinto black;
		draw fullcircle scaled 0.2cm withpen pencircle scaled 0.8pt;);

% This is from the feynmp package
%input feynmp

curly_len := 3mm;

vardef pixlen (expr p, n) =
  for k=1 upto length(p): + segment_pixlen (subpath (k-1,k) of p, n) endfor
enddef;

vardef segment_pixlen (expr p, n) =
  for k=1 upto n: + abs (point k/n of p - point (k-1)/n of p) endfor
enddef;

vardef curly expr p =
 save cpp;
 numeric cpp;
 cpp := ceiling (pixlen (p, 10) / curly_len) / length p;
 if cycle p:
   for k=0 upto cpp*length(p) - 1:
     point (k+.33)/cpp of p
           {direction (k+.33)/cpp of p rotated 90} ..
     point (k-.33)/cpp of p
           {direction (k-.33)/cpp of p rotated -90} ..
   endfor
   cycle
 else:
   point 0 of p
         {direction 0 of p rotated -90} ..
   for k=1 upto cpp*length(p) - 1:
     point (k+.33)/cpp of p
           {direction (k+.33)/cpp of p rotated 90} ..
     point (k-.33)/cpp of p
           {direction (k-.33)/cpp of p rotated -90} ..
   endfor
   point infinity of p
         {direction infinity of p rotated 90}
 fi
enddef;
\stopMPextensions

\stopenvironment

% From: meta-ini.mkxl
%
% initializations:
% \startMPinitializations ... \stopMPinitializations
%
% - pass settings from tex to mp (delayed expansion)
% - used by context core (and modules)
% - cumulative definitions
% - flushed each graphic
% - can be disabled per instance in the instance setup
% - managed at the tex end
%
% extensions:
% \startMPextensions {...,...} ... \stopMPextensions
%
% - add mp functionality (immediate expansion)
% - cumulative
% - all instances or subset of instances (optional list in {...,...})
% - can be disabled per instance in the instance setup
% - managed at the lua/mp end
% - could be managed at the tex end but no real reason and also messy
%
% definitions:
% \startMPdefinitions {...} ... \stopMPdefinitions
%
% - add mp functionality (delayed expansion)
% - cumulative
% - per instance given in optional {...}
% - managed at the tex end
%
% inclusions:
% \startMPinclusions [...] {...} ... \stopMPinclusions
%
% - add mp functionality (delayed expansion)
% - cumulative only with optional [+]
% - per instance given in optional {...}
% - managed at the tex end
%
% order of execution:
%
%   definitions
%   extensions
%   inclusions
%   beginfig
%     initializations
%     graphic
%   endfig
%
% See \defineMPinstance and \setupMPinstance on Wiki.
