\startenvironment env_MetaPost
\usemodule [graph]

\startMPinclusions
  pickup pencircle scaled 0.8pt;
  anglelength := 5mm ;
\stopMPinclusions

% Specifically for MetaPost figures

\startbuffer[startMP]
\environment env_physics
\environment env_MetaPost
\setupbodyfont [libertinus,11pt]
\setoldstyle % Old style numerals in text
   \startMPpage
\stopbuffer

\startbuffer[stopMP]
   \stopMPpage
\stopbuffer

\define[3]\marginMP{
  \startplacefigure[location=margin, reference=fig:#2, title={#3}]
    #1\typesetbuffer[startMP,#2,stopMP]
  \stopplacefigure
}

% Cancel command for use in equations. Perhaps one command can work in both italic and roman.
\startuniqueMPgraphic{cross out}
  picture cross;
  cross := image(draw (0,0)--(1,1); draw (0,1)--(1,0););
  draw cross xscaled \overlaywidth yscaled \overlayheight withpen pencircle scaled .8pt ;
\stopuniqueMPgraphic

\defineoverlay[canceloverlay][\uniqueMPgraphic{cross out}]

\defineframed[cancelframe][background=canceloverlay, frame=off, offset=overlay]

\define[1]\cancel{%
	\ifmmode
		\mathchoice
		{\cancelframe{$\displaystyle #1$}}
		{\cancelframe{$\textstyle #1$}}
		{\cancelframe{$\scriptstyle #1$}}
		{\cancelframe{$\scriptscriptstyle #1$}}
	\else%
		\cancelframe{#1}%
	\fi%
}

\define[1]\ucan{%
	\ifmmode
		\mathchoice
		{\cancelframe{$\rm\displaystyle #1$}}
		{\cancelframe{$\rm\textstyle #1$}}
		{\cancelframe{$\rm\scriptstyle #1$}}
		{\cancelframe{$\rm\scriptscriptstyle #1$}}
	\else%
		\cancelframe{#1}%
	\fi%
}

\startMPinclusions 
picture ball;
	ball := image(
		draw fullcircle scaled 1cm
      			withshademethod "circular"
			withshadecenter (.25,.25)
       		withcolor "white" shadedinto black;
		draw fullcircle scaled 1cm withpen pencircle scaled 0.8pt;);

picture littleball;
	littleball := image(
		draw fullcircle scaled 0.2cm
      			withshademethod "circular"
			withshadecenter (0.2, 0.2)
       		withcolor "white" shadedinto black;
		draw fullcircle scaled 0.2cm withpen pencircle scaled 0.8pt;);

% This is from the feynmp package
%input feynmp

curly_len := 3mm;

vardef pixlen (expr p, n) =
  for k=1 upto length(p): + segment_pixlen (subpath (k-1,k) of p, n) endfor
enddef;

vardef segment_pixlen (expr p, n) =
  for k=1 upto n: + abs (point k/n of p - point (k-1)/n of p) endfor
enddef;

vardef curly expr p =
 save cpp;
 numeric cpp;
 cpp := ceiling (pixlen (p, 10) / curly_len) / length p;
 if cycle p:
   for k=0 upto cpp*length(p) - 1:
     point (k+.33)/cpp of p
           {direction (k+.33)/cpp of p rotated 90} ..
     point (k-.33)/cpp of p
           {direction (k-.33)/cpp of p rotated -90} ..
   endfor
   cycle
 else:
   point 0 of p
         {direction 0 of p rotated -90} ..
   for k=1 upto cpp*length(p) - 1:
     point (k+.33)/cpp of p
           {direction (k+.33)/cpp of p rotated 90} ..
     point (k-.33)/cpp of p
           {direction (k-.33)/cpp of p rotated -90} ..
   endfor
   point infinity of p
         {direction infinity of p rotated 90}
 fi
enddef;
\stopMPinclusions

\stopenvironment