% !TEX useAlternatePath
% !TEX useConTeXtSyncParser

\startcomponent EquationsOfMotion.tex
\project project_world
\product prd_volume01extra

\starttext


%%%%%%%%%%%%%%%%%%%%%%%%%%%%%
\startchapter[title=Equations of motion, reference=ch:EoM]
%%%%%%%%%%%%%%%%%%%%%%%%%%%%%

We must take a moment to understand what such an ambitious general solution would mean. It is hard enough to solve one problem at a time. How could Lagrange address every conceivable problem at once?

First, we must understand the goal of a general solution, which is to predict the system's motion. This is accomplished when we know the position coordinates ($x$, $y$, $\theta$, etc.) for every part of the system at every time. For example, we would like to know the position of a cannon ball at every time between when it is fired upward at an angle and when it hits the ground. Galileo gave us the first tool that we need to accomplish this, the position update formulas.

\placefigure[margin][fig:cannonEoMxy] % location
{A cannon ball's curving path is found by repeatedly updating its position coordinates $x$ and $y$.}	% caption text
{\vskip1.4in\hbox{\starttikzpicture
	\draw[white] (0,0)-- ++(5,0); % Sky to make height better
\stoptikzpicture}}

\placewidefloat
  [top,none]
  {This is its caption I need to fix.}
{\hbox{\small\starttikzpicture	% tikz code
%\draw[->,ultra thick] (12,1.5) -- node[above, pos=.6]{$p\si$}(13,1.5);
%\shade[right color=gray,left color=white] (.6,.24) rectangle (0.9,.26);
\draw[shade, ball color = black] (0.78,.61) circle[radius=.05cm]; % Ball initial
\fill[fill=black!70] (0.76,0.47)-- ++(-.12,.16)-- ++(-.51,-.32)-- ++(0.18,-0.24)-- cycle; % Cannon Barrel
\fill[fill = black!70] (0.22,0.19) circle[radius=0.15cm]; % Cannon Back
\draw[fill=gray] (0,0)-- ++(0,0.05)-- ++(.2,.2)-- ++(.2,0)-- ++(.2,-.2)-- ++(0,-0.05)-- cycle; % Cannon Base
\draw[fill = black!70] (0.3,0.25) circle[radius=0.05cm]; % Cannon Pivot
\shade[top color=gray] (0,-.2) rectangle (16.7,0); % Ground
\draw (0,0)--(16.7,0);
\draw[thick,->] ({0.78+(2.25/4)},0.61)node[right]{$dy$} -- ++(0,0.45); % dy
\draw[thick,->] (0.78,0.61)--node[below]{$dx$} ++({2.25/4},0); % dx
\draw[->] (0.78,0.61) parabola [bend pos=0] bend +(6.75,2.53) ({0.78+6.25*2.25},{0.61+6.25*1.6875-6.25*6.25*0.28125}); % Trajectory
\foreach \T in {0.25, 0.5, 0.75, 1, 1.25, 4.75, 5, 5.25, 5.5, 5.75, 6}{% Balls
	\draw[shade, ball color = black][opacity={(5+\T)/20}] ({0.78+\T*2.25},{0.61+\T*1.6875-\T*\T*0.28125}) circle[radius=.05cm]; % Balls
	\draw[thick,->][opacity={(5+\T)/20}] ({0.78+(2.25/4)+\T*2.25},{0.61+\T*1.6875-\T*\T*0.28125})-- ++(0,{0.45-\T*0.15}); % dy
	\draw[thick,->][opacity={(5+\T)/20}] ({0.78+\T*2.25},{0.61+\T*1.6875-\T*\T*0.28125})-- ++({2.25/4},0); % dx
}
\foreach \T in {1.5, 1.75, 2, 2.25, 2.5, 2.75, 3, 3.25, 3.5, 3.75, 4, 4.25, 4.5}{% Balls
	\draw[shade, ball color = black][opacity={(5+\T)/20}] ({0.78+\T*2.25},{0.61+\T*1.6875-\T*\T*0.28125}) circle[radius=.05cm]; % Balls
	\draw[thick][opacity={(5+\T)/20}] ({0.78+(2.25/4)+\T*2.25},{0.61+\T*1.6875-\T*\T*0.28125})-- ++(0,{0.45-\T*0.15}); % dy
	\draw[thick,->][opacity={(5+\T)/20}] ({0.78+\T*2.25},{0.61+\T*1.6875-\T*\T*0.28125})-- ++({2.25/4},0); % dx
}
\draw[shade, ball color = black] ({0.78+6.25*2.25},{0.61+6.25*1.6875-6.25*6.25*0.28125}) circle[radius=.05cm]; % Ball final
%\draw[thick,->] (14.28,0.61)--node[right]{$dy$} ++(0,-1.8); % dy
%\draw[thick,->] (14.28,0.61)--node[below]{$dx$} ++({2.25/4},0); % dx
\stoptikzpicture}}

\startbuffer[cannonEoMp]
\draw[shade, ball color = black!60] (0,0) circle[radius=.25cm]; % Ball
\filldraw (0,0) circle[radius=.02cm]; % Ball cm
\draw[thick,->] (0,0)--node[above left]{$\vec p$} ++(4.8,3.6); % Momentum initial
\draw[thick,->] (4.8,+3.6)--node[right]{$d\vec p$} +(0,-0.3); % dp
\foreach \T in {0.25, 0.5, 0.75, 1, 1.25, 1.5, 1.75, 2, 2.25, 2.5, 2.75, 3, 3.25, 3.5, 3.75, 4, 4.25, 4.5, 4.75, 5, 5.25, 5.5, 5.75, 6}{% Balls
	\draw[thick,->][opacity={(5+\T)/20}] (0,0)-- ++(4.8,{3.6-\T*1.2}); % Momentum
	\draw[thick,->][opacity={(5+\T)/20}] (4.8,{3.6-\T*1.2})-- +(0,-0.3); % dp
}
\draw[thick,->] (0,0)-- ++(4.8,-3.9); % pf
\stopbuffer

\marginTikZ{}{cannonEoMp}{The cannon ball's momentum is updated repeatedly to find the momentum at any later time.} % vskip, name, caption
\startformula
	dx = v_x\,dt
	\qquad
	dy = v_y\,dt
\stopformula
Starting from the cannon ball's initial coordinates, we can use the cannon ball's velocity to find its location a short time later. These position updates are shown in \in{figure}[fig:cannonEoMxy]. The process is repeated over and over to find the position at any time, provided we know the velocity at each moment. However, knowing the velocity is a problem. Galileo studied important examples that have a known velocity – uniform motion, free fall, and simple harmonic motion – but this is far from a general method.

Descartes pointed the way forward with his idea for a conserved quantity of motion. His formula did not work, but two later ideas succeeded. The first was Newton's momentum, updated using his second law. The momentum update formulas for the cannon ball are
\startformula
	dp_x = 0,
	\qquad
	dp_y = -mg\,dt.
\stopformula
The momentum's $x$-component is constant, so its change is always zero. The momentum's $y$-component changes at a constant rate in the downward direction, so its change is always the same size and negative.
These momentum updates are shown in \in{figure}[fig:cannonEoMp].
The updates are all proportional to the small duration $dt$ of our time steps.
%We start with the cannon ball's initial momentum and update it repeatedly to find the momentum at any later time. The momentum gives us the velocity we need to update the position.
Putting this together with Galileo's position update formulas gives the cannon ball's \keyterm{equations of motion}.
\startformula\startalign[m=2,distance=5em]
\NC dx	\NC = \frac{p_x}{m}\,dt	\NC dp_x	\NC = 0	\NR
\NC dy	\NC = \frac{p_y}{m}\,dt	\NC dp_y	\NC = -mg\,dt	\NR
\stopalign\stopformula
Equations of motion must provide a formula to update each coordinate and each momentum. For the two dimensional motion of the cannon ball we must have four update formulas. The right side of the formulas can include known constants (like $m$ and $g$) and the current coordinates and momenta. For the cannon ball's equations of motion, we see that the momentum update formulas contain only constants, and the position update formulas contain constants and momentum components. (I replaced the velocity components with their momentum equivalents.) These are good equations of motion.
%In the words of Lagrange, these are good equations of motion because they are \quotation{all of the necessary equations for the solution of [the] problem."


\startexample[ex:EoMSHO] Write the equation of motion for the system of the cart (mass $m$) and spring (stiffness $k$) shown in \in{figure}[fig:EoMSHO].

\startbuffer[EoMSHO]
	\fill [black!10] (-.23,0) rectangle (4.8,-.15);
	\fill [black!10] (0,0) rectangle (-.23,.6);
	\draw[thin] (0,0) -- (0,.6);
	\startaxis[margin cart track,
			xmin=-24,xmax=24,
			ymax=10,
			]
	\path (0,0) pic {cart}node[above = 5mm]{$m$};
	\draw[decorate,decoration={coil,segment length=3.6pt}] (-24,2.5) --node[above=3pt] {$k$} (-6,2.5);
    \stopaxis
\stopbuffer

\marginTikZ{}{EoMSHO}{A cart connected to a spring for \in{example}[ex:EoMSHO]. This spring's natural length is $18\units{cm}$, so it holds the cart gently at $x=0\units{cm}$.} % vskip, name, caption

\startsolution
Since the cart moves in one dimension, we only need one coordinate $x$ and one momentum $p$. The two update formulas are
\startformula
	dx = v\, dt,
	\qquad
	dp = F\,dt.
\stopformula
The cart's velocity is $v=p/m$ and the force is given by Hooke's law, $F = -kx$. Putting these in the update formulas gives the equations of motion.
\startformula
	dx = \frac{p}{m}\, dt
	\qquad
	dp = -kx\,dt
\stopformula
In this case the momentum changes depend on the position. The change is always toward the central equilibrium at $x=0$. %Solving with by doing repeated updates produces simple ha
\stopsolution
\stopexample

Once we have the equations of motion, we must solve them. For the cannon ball, we start with the cannon ball's initial position and momentum, and then use the update formulas over and over to find the position and momentum at any later time. Sometimes the equations of motion are easy to solve. The equation for $p_x$ is easy to solve. Since the updates are always zero, we know that the momentum's  $x$-component at every time is the same as its initial value (\in{fig.}[fig:cannonEoMp]). Sometimes the equations of motion can be solved with more advanced mathematical tools. In many cases, the equations of motion can only be solved through the tedious process of repeated calculation for many, many small time steps. Today, these repetitive calculation are performed rapidly by computers – once they are given the system's equations of motion.

For the cannon ball, and many other simple systems, Newton's method produces the equations of motion quite easily. However, Newton's approach is not well suited to the two quadrivium problems we hope to solve: the planets' compound motion discovered by Kepler and the musical string's compound vibrations discovered by Mersenne. The quadrivium problems are much easier to approach using the second successful quantity of motion, kinetic energy.

We have already seen how rotational kinetic energy allows us to study spinning or rolling motion, where many connected parts move with different velocities but share a common angular motion. Angular coordinates are also the natural coordinates for planetary motions. We will use angular coordinates and rotational kinetic energy to find and solve the planets' equations of motion in \in{Chapter}[ch:Rotation]. Like a spinning object, a musical string has many parts moving with different velocities. The string's motion is best described in terms of shared motions rather than the individual parts' motions. We will use the shared motion's energy to find and solve the string's equations of motion in \in{Chapter}[ch:Waves]. Before we attack those problems, we must find a connection between the energy and equations of motion.

Energy is not directional, so it works very well for systems with parts moving in many directions, (like spinning and rolling) and in situations where the coordinates are not straight (like angular coordinates). Perhaps kinetic energy could provide the velocities we need for Galileo's position update formulas.

Unfortunately, kinetic energy is not enough. Since kinetic energy is not directional, it cannot give us the velocity components needed for the position update formulas. In the case of the cannon ball, we can easily use conservation of energy to find the cannon ball's speed at any height, but conservation of energy cannot tell us the velocity's direction, which we need to correctly update the position. (In one dimension, where velocity has only one component, kinetic energy is often enough, but we will not use this one-dimensional trick.)



\section{Lagrange's step in the right direction}



\startbuffer[cannondx]
\draw[shade, ball color = black!60] (0,0) circle[radius=.25cm]; % Ball
\filldraw (0,0) circle[radius=.02cm]; % Ball cm
\draw[thick,->] (-2.5,-1)--node[above]{$x$} ++(1,0); % x-axis
\draw[opacity=0] (-2.5,-1)-- ++(5,0); % x-floor to get the ball centered
\draw[thick,->] (-2.5,-1)--node[right]{$y$} +(0,1); % y-axis
\draw[thick,->] (-0.5,0)--(0.5,0)node[right]{$\partial x$} ; % pf
\stopbuffer

\marginTikZ{}{cannondx}{A small horizontal virtual displacement $\partial x$ produces no change in the gravitational potential energy. $\partial U = 0$. } % vskip, name, caption
For the cannon ball, a small virtual displacement $\partial x$ in the $x$-direction is a horizontal displacement (shown in \in{figure}[fig:cannondx]), which does not change the cannon ball's gravitational potential energy at all. For this horizontal displacement, $\partial U = 0$. Lagrange's equation gives the $x$-component of the gravitational force. (I will drop \quotation{Newton} from the suffix for the remainder of this section.)
\startformula
	F_{x} = -\frac{\partial U}{\partial x}
		= -\frac{0}{\partial x}
		= 0
\stopformula
The gravitational force's $x$-components is zero, as we knew. A small virtual displacement $\partial y$ in the $y$-direction is a vertical displacement (shown in \in{figure}[fig:cannondy]), which does change the cannon ball's gravitational potential energy. For this vertical displacement, $\partial U = mg\partial y$, which leads to the $y$ component of the gravitational force.
\startbuffer[cannondy]
\draw[shade, ball color = black!60] (0,0) circle[radius=.25cm]; % Ball
\filldraw (0,0) circle[radius=.02cm]; % Ball cm
\draw[thick,->] (-2.5,-1)--node[above]{$x$} ++(1,0); % x-axis
\draw[opacity=0] (-2.5,-1)-- ++(5,0); % x-floor to get the ball centered
\draw[thick,->] (-2.5,-1)--node[right]{$y$} +(0,1); % y-axis
\draw[thick,->] (0,-0.5)--(0,0.5)node[above]{$\partial y$} ; % pf
\stopbuffer

\marginTikZ{}{cannondy}{A small vertical virtual displacement $\partial x$ increases the gravitational potential energy. $\partial U = mg\partial y$.} % vskip, name, caption
\startformula
	F_{y} = -\frac{\partial U}{\partial y}
		= -\frac{mg\cancel{\partial y}}{\cancel{\partial y}}
		= -mg
\stopformula
\startbuffer[cannonForce]
\draw[shade, ball color = black!60] (0,0) circle[radius=.25cm]; % Ball
\filldraw (0,0) circle[radius=.02cm]; % Ball cm
\draw[thick,->] (-2.5,-2)--node[above]{$x$} ++(1,0); % x-axis
\draw[opacity=0] (-2.5,-2)-- ++(5,0); % x-floor to get the ball centered
\draw[thick,->] (-2.5,-2)--node[right]{$y$} +(0,1); % y-axis
\draw[thick,->] (0,0)--node[right]{$\vec F = \components{0,-mg}$} ++(0,-2); % pf
\stopbuffer

\marginTikZ{\vskip 0.5cm}{cannonForce}{The gravitational force on the cannon ball, with components found using Lagrange's equation.} % vskip, name, caption
\noindent
The gravitational force's $y$-component is $-mg$, again agreeing with what we knew. These two components give the gravitational force vector shown in \in{figure}[fig:cannonForce].
While potential energy itself does not have a direction, Lagrange's procedure produces the force vector, which does have a direction.

Keeping track of the different virtual changes $\partial U$ due to the different virtual displacements $\partial x$ and $\partial y$ can get confusing, but Lagrange has good news. The ratios $\partial U\!/\partial x$ and $\partial U\!/\partial y$ can be easily found using the same derivative rules that we learned for rates of change (like $v = dx/dt$). When computing $\partial U\!/\partial x$ we keep $y$ constant. When computing $\partial U\!/\partial y$ we keep $x$ constant.

\startexample[ex:CannonDirivaties]
Use Lagrange's equation and the derivative rules to find the force components $F_x$ and $F_y$ on the cannon ball.
\startsolution
The $x$ component of the force is given by Lagrange's equation with a virtual displacement in the $x$-direction.
\startformula
	F_x = -\frac{\partial U}{\partial x} = -\ppx U = -\ppx mgy = 0
\stopformula
When calculating $\partial U\!/\partial x$ we keep $y$ constant, which makes $mgy$ constant. The derivative of a constant is zero.

The $y$ component of the force is given by Lagrange's equation with a virtual displacement in the $y$-direction.
\startformula
	F_y = = -\frac{\partial U}{\partial y} = -\ppy U = -\ppy mgy =  -mg\ppy y = -mg
\stopformula
After pulling the coefficient $mg$ out of the derivative, we compute $\ppy y$ using the power rule with $n=1$.
\stopsolution
\stopexample

The ratios $\partial U\!/\partial x$ and $\partial U\!/\partial y$ are called partial derivatives of $U$ because we compute them allowing only one part of our coordinates (either $x$ or $y$) to change while other parts are kept constant. Computing partial derivatives is much easier than dealing with many tiny virtual displacements. As a result, Lagrange replaced many diagrams with partial derivatives.



\stoptext
\stopcomponent