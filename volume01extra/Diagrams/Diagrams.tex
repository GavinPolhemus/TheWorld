% !TEX useOldSyncParser
\startcomponent c_chapter01
\project project_world
\product prd_volume01

\setupsynctex[state=start,method=max] % "method=max" or "min"
\starttext


%%%%%%%%%%%%%%%%%%%%%%%%%%%%%
\startchapter[title=Diagrams, reference=ch:Diagrams]
%%%%%%%%%%%%%%%%%%%%%%%%%%%%%

The diagrams below are produced here for use in other places.

\startbuffer[CupTable]
\environment env_physics
\environment env_MetaPost
\setupbodyfont [libertinus,11pt]
\setoldstyle % Old style numerals in text
%\startTEXpage
\startMPpage%{graph::CupTableGraphs} % I'd like to add minor ticks, 0.667mm long.
path Top;
	Top := unitsquare xyscaled (4cm, -5mm);
%	Leg := unitsquare xyscaled (1mm, -1.8cm) shifted (-0.5mm,0);
%fill Leg shifted (5mm,0) withcolor "lightgray";
%draw Leg shifted (5mm,0) withpen pencircle scaled 0.8pt;
%fill Leg shifted (35mm,0) withcolor "lightgray";
%draw Leg shifted (35mm,0) withpen pencircle scaled 0.8pt;
fill Top withcolor "lightgray";
draw Top withpen pencircle scaled 0.8pt;
picture CCup; path Cup, HandleOut, HandleIn;
	Cup := unitsquare xyscaled (2cm,3cm) shifted (-1cm,0);
	HandleOut := fullcircle xyscaled (1.8cm,2.2cm) shifted (1.2cm,1.5cm);
	HandleIn := fullcircle xyscaled (1cm,1.4cm) shifted (1.2cm,1.5cm);
	CCup := image(
		fill HandleOut withcolor "gray";
		draw HandleOut withpen pencircle scaled 0.8pt;
		unfill HandleIn ;
		draw HandleIn withpen pencircle scaled 0.8pt;
		fill Cup withcolor "gray";
		draw Cup;
		draw Cup withpen pencircle scaled 0.8pt;);
draw CCup shifted (2cm,0.8pt) ;
path dy; dy := (2cm,-3mm) -- (2cm,3mm);
drawarrow dy withpen pencircle scaled 0.8pt;%node[right]{$\partial x$} ; % pf
label.top("$\partial y$", (2cm,3mm));
\stopMPpage
%\stopTEXpage
\stopbuffer

%\startplacefigure[location=margin, reference=fig:CupTable, title={A coffee cup sits on a slightly sloped table (top). The energy graph shows coffee cup's potential energy vs.\ its position on the slightly sloped table (bottom). The cup's total energy $H$ on the graph is the location of the motionless cup, showing the total energy $H$ is equal to the potential energy $U$.}]
%\typesetbuffer[starttikz,CupTable,stoptikz]
\typesetbuffer[CupTable]
%\reuseMPgraphic{graph::CupTableGraphs}
%\stopplacefigure

\startbuffer[TikZ:GalileoPendulumPath2]
\environment env_physics
\environment env_TikZ
\setupbodyfont [libertinus,11pt]
\setoldstyle % Old style numerals in text
\startTEXpage\small
\starttikzpicture% tikz code
%	\draw [help lines, xstep=8, ystep=.34] (-4.3,0) grid (4.3,4.3); % Background grid
%	\draw (-4.3,-0.5) rectangle (4.3,4.5); % Border
	% h axis
	\draw[
		postaction={decorate},
		decoration={
			markings, % Main marks
			mark=between positions 0 and 1 step 1cm with {
				\draw (0,0)
				node[left]{
					\pgfmathparse{
						-10+10*\pgfkeysvalueof{%
							/pgf/decoration/mark info/sequence number%
						}
					}
					\pgfmathprintnumber{\pgfmathresult}
				} -- (0,-4pt);
			},
		}
	] (-4.3,0) -- (-4.3,4);
	\draw[
		postaction={decorate},
		decoration={
			markings, % Main marks
			mark=between positions 0 and 1 step 1mm with {
				\draw (0,0) -- (0,-2pt);
			},
		}
	] (-4.3,0) --node[sloped,above=5mm]{$h$ (cm)} (-4.3,4);
	% U axis
	\draw[
		postaction={decorate},
		decoration={
			markings, % Main marks
			mark=between positions 0 and 1 step 6.8mm with {
				\draw (0,0)
				node[right]{
					\pgfmathparse{
						0.1*(-1+\pgfkeysvalueof{%
							/pgf/decoration/mark info/sequence number%
						})
					}
					\pgfmathprintnumber{\pgfmathresult}
				} -- (0,4pt);
			},
		}
	] (4.3,0) -- (4.3,4.082);
	\draw[
		postaction={decorate},
		decoration={
			markings, % Main marks
			mark=between positions 0 and 1 step 3.4mm with {
				\draw (0,0) -- (0,2pt);
			},
		}
	] (4.3,0) --node[sloped,below=6mm]{$U$ (J)} (4.3,4.082);
	\fill (0,4) circle[radius=.4mm]; % Pivot
%	\node at (0,0) [above left]{B}; % Bottom
%	\draw (0,-0.2) -- (0,4.2); % Central vertical
%	\node at (-3.2,1.6) [above=2mm]{C}; % Left
%	\node at (3.2,1.6) [above=1mm]{D}; % Right
%	\fill (0,2) circle[radius=.4mm]node[left]{E}; % 2nd nail
%	\fill (0,1) circle[radius=.4mm]node[left]{F}; % 3nd nail
%	\node at (1.833,1.6) [above=1mm]{G}; % Right
%	\node at (0.98,1.6) [above=1mm]{I}; % Right
%	\draw (-4.0,1.6) -- (4.0,1.6); % horizontal at max height
	% Pendulum path
%	\draw[] (0,0) arc[start angle=270, end angle=336.4, radius=2cm];
%	\draw[] (0,0) arc[start angle=270, end angle=371.5, radius=1cm];
	% Positive on the right
	\draw[
		postaction={decorate},
		decoration={
			markings, % Main marks
			mark=between positions 0 and 1 step 1cm with {
				\draw (0,0) -- (0,-4pt)
				node[below,transform shape]{
					\pgfmathparse{
						-10+10*\pgfkeysvalueof{%
							/pgf/decoration/mark info/sequence number%
						}
					}
					\pgfmathprintnumber{\pgfmathresult}
				};
			},
		}
	] (0,0) arc[start angle=270, end angle=339, radius=4cm];
	\draw[
		postaction={decorate},
		decoration={
			markings, % Main marks
			mark=between positions 0 and 1 step 0.998mm with {
				\draw (0,0) -- (0,-2pt);
			},
		}
	] (0,0) arc[start angle=270, end angle=339, radius=4cm];
	% Negative on the left
	\draw[
		postaction={decorate},
		decoration={
			markings, % Main marks
			mark=between positions 0 and 1 step 1cm with {
				\draw (0,0) -- (0,4pt)
				node[below, transform shape, rotate=180]{
					\pgfmathparse{
						10-10*\pgfkeysvalueof{%
							/pgf/decoration/mark info/sequence number%
						}
					}
					\pgfmathprintnumber{\pgfmathresult}
				};
			},
		}
	] (0,0) arc[start angle=270, end angle=201, radius=4cm];
	\draw[
		postaction={decorate},
		decoration={
			markings, % Main marks
			mark=between positions 0 and 1 step 0.998mm with {
				\draw (0,0) -- (0,2pt);
			},
		}
	] (0,0) arc[start angle=270, end angle=201, radius=4cm];
	\node at (0,0) [below=5mm]{$s$ (cm)};
	% Pendulum
	\draw[thick] (0,4) --node[sloped,above]{$40\units{cm}$} (-3.2,1.6); % String
	\draw[ball color=white] (-3.2,1.6) circle[radius=2mm]; % Ball , opacity=.5
	\fill (-3.2,1.6) circle[radius=.4mm]; % CoM
	\draw[very thick, ->] (-3.2,1.6) ++(-3mm,4mm) -- ++(6mm,-8mm)node[above right]{$\partial s$}; % ds
\stoptikzpicture
\stopTEXpage
\stopbuffer

%\placetextfloat[top][fig:GalileoPendulumPath2] % location
%{Galileo’s pendulum at the release position $s=\units{-37cm}$.}	 % caption text
%{\noindent
\typesetbuffer[TikZ:GalileoPendulumPath2]
%} % figure contents


\startbuffer[TikZ:GalileoPendulumGraphU]
\environment env_physics
\environment env_TikZ
\setupbodyfont [libertinus,11pt]
\setoldstyle % Old style numerals in text
\startTEXpage\small
\starttikzpicture% tikz code
	\startaxis[
		every tick/.style={darkgray}, % Fixes ticks which are too light in ConTeXt
		major grid style = {darkgray},
			scale only axis,
			x={1mm},y={68mm},
			xmin=-48, xmax=48,
			minor x tick num=9,
			xlabel=$s$ (cm),
			%axis x line=none,
			%axis y line*=right,
			ymin=0, ymax=0.6,
			minor y tick num=9,
			ylabel=Energy (J),
			%grid=both
		]
		\addplot[thick, domain=-50:50] {0.588*(1-cos(deg(x/40)))}node[above left, pos=.9]{$U$};
	\fill(-37,0.235) circle[radius=.4mm];
	\draw[very thick, ->] (-40,0.27) --node[above]{$\partial s$} (-34,0.27); % ds
	\draw[very thick, ->] (-34,0.27) --node[right]{$\partial U$} (-34,0.20); % ds
	\stopaxis
\stoptikzpicture
\stopTEXpage
\stopbuffer

%\placetextfloat[bottom][fig:GalileoPendulumGraphU] % location
%{An energy graph showing the ball’s gravitational potential energy as a function of position $s$ along the curved path.}	% caption text
%{\noindent
\typesetbuffer[TikZ:GalileoPendulumGraphU]
%} % figure contents

\stoptext
\stopcomponent