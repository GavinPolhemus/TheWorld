% !TEX useOldSyncParser
\startcomponent c_chapter01
\project project_world
\product prd_volume01

\setupsynctex[state=start,method=max] % "method=max" or "min"
\starttext


%%%%%%%%%%%%%%%%%%%%%%%%%%%%%
\startchapter[title=Constrained Motion, reference=ch:Constraints]
%%%%%%%%%%%%%%%%%%%%%%%%%%%%%

Problems with constraints require adding unknown reaction forces to the equations of motion, one force per constraint. The constraint equations provide enough information to solve for these reaction forces when finding the motion from Hamilton’s equations.
Hamilton’s equations with these external reaction forces are
\startformula
	d\eta^\alpha = \frac{\partial H}{\partial \pi_\alpha}\,dt
	\qquad
	d\pi_\alpha = \left( - \frac{\partial H}{\partial \eta^\alpha} + F_\alpha\right)\,dt.
\stopformula
There are $m$ coordinates $\eta^\alpha$ and $m$ canonical momenta $\pi_\alpha$ ($\alpha \in \{1\dots m\}$). The constraints are maintained by the reaction force $F$, which has components $F_\alpha$.

Common constraints, like rolling, link the differential changes in position velocities as well as position. Rolling constraints are linear in differential positions, so consider $n$ constraints of the form
\startformula
	e^\mu_\alpha\,\partial\eta^\alpha = 0,
\stopformula
where $\mu \in \{1\dots n\}$. For rolling constraints the $e^\mu_\alpha$ themselves depend only on positions, not momenta or velocities, but this restriction is not necessary for what follows.

The constraint force is unknown, but we know it does no work on the system for any virtual displacement consistent with the constraints.
\startformula
	F_\alpha\,\partial\eta^\alpha = 0
\stopformula
This can be achieved by setting
\startformula
	F_\alpha = \lambda_\mu e^\mu_\alpha,
\stopformula
(so the work becomes $F_\alpha\,\partial\eta^\alpha = \lambda_\mu e^\mu_\alpha\,\partial\eta^\alpha = 0$)
with $n$ unknown multipliers $\lambda_\mu$.
With this external reaction force, Hamilton’s equations become
\startformula
	d\eta^\alpha = \frac{\partial H}{\partial \pi_\alpha}\,dt
	\qquad
	d\pi_\alpha = \left( - \frac{\partial H}{\partial \eta^\alpha} + \lambda_\mu e^\mu_\alpha \right)\,dt
\stopformula
The final result of all of this is $2m$ Hamilton’s Equations and the $n$ constraint equations. These can be used to solve for the $m$ coordinates, $m$ momenta, and $n$ multipliers. For rolling constraints there is no velocity dependence in Hamilton’s equations. The reaction force maintaining all of the constraints has components
\startformula
	F_{\alpha} = \lambda_\mu e^\mu_\alpha.
\stopformula
Frequently, only the positions are constrained explicitly. (These are called holonomic constraints.) The velocities are constrained only as a consequence of the position constraints
\startformula
	f^\mu = \text{const.},
\stopformula
where each $f^\mu$ depends only on position. These constraints can be put in the form above by finding the virtual changes in the $f^\mu$ through the chain rule.
\startformula
	\partial f^\mu = \frac{\partial f^\mu}{\partial\eta^\alpha}\,\partial\eta^\alpha = 0
\stopformula
This is the a constraint linear in differentials with
\startformula
	e^\mu_\alpha = \frac{\partial f^\mu}{\partial\eta^\alpha}
\stopformula
Hamilton’s equations with holonomic constraints become
\startformula
	d\eta^\alpha = \frac{\partial H}{\partial \pi_\alpha}\,dt
	\qquad
	d\pi_\alpha = \left( - \frac{\partial H}{\partial \eta^\alpha} + \lambda^\mu \frac{\partial f_\mu}{\partial\eta^\alpha}\right)\,dt
\stopformula
This is the common result for constrained motion. See, for example, José and Saletan, in \booktitle{Classical Dynamics} (pp. 115-6) which discusses nonholonomic constraints as well. I have only seen these methods in the Lagrangian formulation, but the Hamiltonian formulation is a natural extension. There is no obstacle to mixing holonomic and nonholonomic constraints.

The geometric story appears to be this. The $e^\mu_\alpha$ are differential forms on the configuration space, and the constraint equations require that the velocity vectors give zero when contracted with these forms.  The reaction force is also a differential form, a scalar multiple of the $e^\mu_\alpha$. If the differential form can be written as the differential of a scalar function, $e = df$, then constraints are holonomic, and the forces are monogenic. (See Lanczos pp. 146-7.)

[Examples: Rolling down a slope (find r, theta, p, L, and friction force with four equations of motion and one constraint equation.). Rolling in more dimensions. Piston, rod, and crankshaft. Hydraulics example. Superball with friction.]



\section{Relativistic Hamiltonian method}

Hamiltonian methods are excellent for a full relativistic treatment. The key is to allow the particle to move in four dimensions while geting rid of the Hamiltonian all together. Replace the Hamiltonian with a single constraint that keeps the particle on shell.
\startformula
	{\cal H }= \half g^{\mu\nu}(\pi_mu - eA_mu)(\pi_nu - eA_nu) = \half m^2
\stopformula
This is a single constraint involving momenta and positions (through $A_\mu$). This will lead to one unknown multiplier, but this multiplier can be set to one, fixing the parametrization of the the particle path. (MTW \booktitle{Gravitation,} pp.~488--489, 654, 897--901 and Lanczos \booktitle{The Variational Principles of Mechanics,} pp.~319--320, 327--336)

\section{Diagrams}

\startbuffer[CupTable]
\environment env_physics
\environment env_MetaPost
\setupbodyfont [libertinus,11pt]
\setoldstyle % Old style numerals in text
%\startTEXpage
\startMPpage%{graph::CupTableGraphs} % I'd like to add minor ticks, 0.667mm long.
path Top;
	Top := unitsquare xyscaled (4cm, -5mm);
%	Leg := unitsquare xyscaled (1mm, -1.8cm) shifted (-0.5mm,0);
%fill Leg shifted (5mm,0) withcolor "lightgray";
%draw Leg shifted (5mm,0) withpen pencircle scaled 0.8pt;
%fill Leg shifted (35mm,0) withcolor "lightgray";
%draw Leg shifted (35mm,0) withpen pencircle scaled 0.8pt;
fill Top withcolor "lightgray";
draw Top withpen pencircle scaled 0.8pt;
picture CCup; path Cup, HandleOut, HandleIn;
	Cup := unitsquare xyscaled (2cm,3cm) shifted (-1cm,0);
	HandleOut := fullcircle xyscaled (1.8cm,2.2cm) shifted (1.2cm,1.5cm);
	HandleIn := fullcircle xyscaled (1cm,1.4cm) shifted (1.2cm,1.5cm);
	CCup := image(
		fill HandleOut withcolor "gray";
		draw HandleOut withpen pencircle scaled 0.8pt;
		unfill HandleIn ;
		draw HandleIn withpen pencircle scaled 0.8pt;
		fill Cup withcolor "gray";
		draw Cup;
		draw Cup withpen pencircle scaled 0.8pt;);
draw CCup shifted (2cm,0.8pt) ;
path dy; dy := (2cm,-3mm) -- (2cm,3mm);
drawarrow dy withpen pencircle scaled 0.8pt;%node[right]{$\partial x$} ; % pf
label.top("$\partial y$", (2cm,3mm));
\stopMPpage
%\stopTEXpage
\stopbuffer

%\startplacefigure[location=margin, reference=fig:CupTable, title={A coffee cup sits on a slightly sloped table (top). The energy graph shows coffee cup's potential energy vs.\ its position on the slightly sloped table (bottom). The cup's total energy $H$ on the graph is the location of the motionless cup, showing the total energy $H$ is equal to the potential energy $U$.}]
%\typesetbuffer[starttikz,CupTable,stoptikz]
\typesetbuffer[CupTable]
%\reuseMPgraphic{graph::CupTableGraphs}
%\stopplacefigure

\startbuffer[TikZ:GalileoPendulumPath2]
\environment env_physics
\environment env_TikZ
\setupbodyfont [libertinus,11pt]
\setoldstyle % Old style numerals in text
\startTEXpage\small
\starttikzpicture% tikz code
%	\draw [help lines, xstep=8, ystep=.34] (-4.3,0) grid (4.3,4.3); % Background grid
%	\draw (-4.3,-0.5) rectangle (4.3,4.5); % Border
	% h axis
	\draw[
		postaction={decorate},
		decoration={
			markings, % Main marks
			mark=between positions 0 and 1 step 1cm with {
				\draw (0,0)
				node[left]{
					\pgfmathparse{
						-10+10*\pgfkeysvalueof{%
							/pgf/decoration/mark info/sequence number%
						}
					}
					\pgfmathprintnumber{\pgfmathresult}
				} -- (0,-4pt);
			},
		}
	] (-4.3,0) -- (-4.3,4);
	\draw[
		postaction={decorate},
		decoration={
			markings, % Main marks
			mark=between positions 0 and 1 step 1mm with {
				\draw (0,0) -- (0,-2pt);
			},
		}
	] (-4.3,0) --node[sloped,above=5mm]{$h$ (cm)} (-4.3,4);
	% U axis
	\draw[
		postaction={decorate},
		decoration={
			markings, % Main marks
			mark=between positions 0 and 1 step 6.8mm with {
				\draw (0,0)
				node[right]{
					\pgfmathparse{
						0.1*(-1+\pgfkeysvalueof{%
							/pgf/decoration/mark info/sequence number%
						})
					}
					\pgfmathprintnumber{\pgfmathresult}
				} -- (0,4pt);
			},
		}
	] (4.3,0) -- (4.3,4.082);
	\draw[
		postaction={decorate},
		decoration={
			markings, % Main marks
			mark=between positions 0 and 1 step 3.4mm with {
				\draw (0,0) -- (0,2pt);
			},
		}
	] (4.3,0) --node[sloped,below=6mm]{$U$ (J)} (4.3,4.082);
	\fill (0,4) circle[radius=.4mm]; % Pivot
%	\node at (0,0) [above left]{B}; % Bottom
%	\draw (0,-0.2) -- (0,4.2); % Central vertical
%	\node at (-3.2,1.6) [above=2mm]{C}; % Left
%	\node at (3.2,1.6) [above=1mm]{D}; % Right
%	\fill (0,2) circle[radius=.4mm]node[left]{E}; % 2nd nail
%	\fill (0,1) circle[radius=.4mm]node[left]{F}; % 3nd nail
%	\node at (1.833,1.6) [above=1mm]{G}; % Right
%	\node at (0.98,1.6) [above=1mm]{I}; % Right
%	\draw (-4.0,1.6) -- (4.0,1.6); % horizontal at max height
	% Pendulum path
%	\draw[] (0,0) arc[start angle=270, end angle=336.4, radius=2cm];
%	\draw[] (0,0) arc[start angle=270, end angle=371.5, radius=1cm];
	% Positive on the right
	\draw[
		postaction={decorate},
		decoration={
			markings, % Main marks
			mark=between positions 0 and 1 step 1cm with {
				\draw (0,0) -- (0,-4pt)
				node[below,transform shape]{
					\pgfmathparse{
						-10+10*\pgfkeysvalueof{%
							/pgf/decoration/mark info/sequence number%
						}
					}
					\pgfmathprintnumber{\pgfmathresult}
				};
			},
		}
	] (0,0) arc[start angle=270, end angle=339, radius=4cm];
	\draw[
		postaction={decorate},
		decoration={
			markings, % Main marks
			mark=between positions 0 and 1 step 0.998mm with {
				\draw (0,0) -- (0,-2pt);
			},
		}
	] (0,0) arc[start angle=270, end angle=339, radius=4cm];
	% Negative on the left
	\draw[
		postaction={decorate},
		decoration={
			markings, % Main marks
			mark=between positions 0 and 1 step 1cm with {
				\draw (0,0) -- (0,4pt)
				node[below, transform shape, rotate=180]{
					\pgfmathparse{
						10-10*\pgfkeysvalueof{%
							/pgf/decoration/mark info/sequence number%
						}
					}
					\pgfmathprintnumber{\pgfmathresult}
				};
			},
		}
	] (0,0) arc[start angle=270, end angle=201, radius=4cm];
	\draw[
		postaction={decorate},
		decoration={
			markings, % Main marks
			mark=between positions 0 and 1 step 0.998mm with {
				\draw (0,0) -- (0,2pt);
			},
		}
	] (0,0) arc[start angle=270, end angle=201, radius=4cm];
	\node at (0,0) [below=5mm]{$s$ (cm)};
	% Pendulum
	\draw[thick] (0,4) --node[sloped,above]{$40\units{cm}$} (-3.2,1.6); % String
	\draw[ball color=white] (-3.2,1.6) circle[radius=2mm]; % Ball , opacity=.5
	\fill (-3.2,1.6) circle[radius=.4mm]; % CoM
	\draw[very thick, ->] (-3.2,1.6) ++(-3mm,4mm) -- ++(6mm,-8mm)node[above right]{$\partial s$}; % ds
\stoptikzpicture
\stopTEXpage
\stopbuffer

%\placetextfloat[top][fig:GalileoPendulumPath2] % location
%{Galileo’s pendulum at the release position $s=\units{-37cm}$.}	 % caption text
%{\noindent
\typesetbuffer[TikZ:GalileoPendulumPath2]
%} % figure contents


\startbuffer[TikZ:GalileoPendulumGraphU]
\environment env_physics
\environment env_TikZ
\setupbodyfont [libertinus,11pt]
\setoldstyle % Old style numerals in text
\startTEXpage\small
\starttikzpicture% tikz code
	\startaxis[
		every tick/.style={darkgray}, % Fixes ticks which are too light in ConTeXt
		major grid style = {darkgray},
			scale only axis,
			x={1mm},y={68mm},
			xmin=-48, xmax=48,
			minor x tick num=9,
			xlabel=$s$ (cm),
			%axis x line=none,
			%axis y line*=right,
			ymin=0, ymax=0.6,
			minor y tick num=9,
			ylabel=Energy (J),
			%grid=both
		]
		\addplot[thick, domain=-50:50] {0.588*(1-cos(deg(x/40)))}node[above left, pos=.9]{$U$};
	\fill(-37,0.235) circle[radius=.4mm];
	\draw[very thick, ->] (-40,0.27) --node[above]{$\partial s$} (-34,0.27); % ds
	\draw[very thick, ->] (-34,0.27) --node[right]{$\partial U$} (-34,0.20); % ds
	\stopaxis
\stoptikzpicture
\stopTEXpage
\stopbuffer

%\placetextfloat[bottom][fig:GalileoPendulumGraphU] % location
%{An energy graph showing the ball’s gravitational potential energy as a function of position $s$ along the curved path.}	% caption text
%{\noindent
\typesetbuffer[TikZ:GalileoPendulumGraphU]
%} % figure contents

\stoptext
\stopcomponent