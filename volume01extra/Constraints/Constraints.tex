% !TEX useAlternatePath
% !TEX useConTeXtSyncParser

\startcomponent c_chapter01
\project project_world
\product prd_volume01

\starttext


%%%%%%%%%%%%%%%%%%%%%%%%%%%%%
\startchapter[title=Constrained Motion, reference=ch:Constraints]
%%%%%%%%%%%%%%%%%%%%%%%%%%%%%

Problems with constraints require adding unknown reaction forces to the equations of motion, one force per constraint. The constraint equations provide enough information to solve for these reaction forces when finding the motion from Hamilton’s equations.
Hamilton’s equations with these external reaction forces are
\startformula
	d\eta^\alpha = \frac{\partial H}{\partial \pi_\alpha}\,dt
	\qquad
	d\pi_\alpha = \left( - \frac{\partial H}{\partial \eta^\alpha} + F_\alpha\right)\,dt.
\stopformula
There are $m$ coordinates $\eta^\alpha$ and $m$ canonical momenta $\pi_\alpha$ ($\alpha \in \{1\dots m\}$). The constraints are maintained by the reaction force $F$, which has components $F_\alpha$.

Common constraints, like rolling, link the infinitesimal displacements $d\eta^\alpha$. Rolling constraints are linear in these displacements, so consider $n$ constraints of the form
\startformula
	e^\mu_\alpha\,d\eta^\alpha = 0,
\stopformula
where $\mu \in \{1\dots n\}$. For rolling constraints, the $e^\mu_\alpha$ themselves depend only on positions, not momenta or velocities, but this restriction is not necessary for what follows.

The constraint force is unknown, but we know it does no work on the system.
\startformula
	F_\alpha\,d\eta^\alpha = 0
\stopformula
This is achieved by setting
\startformula
	F_\alpha = \lambda_\mu e^\mu_\alpha,
\stopformula
(so the work becomes $F_\alpha\,d\eta^\alpha = \lambda_\mu e^\mu_\alpha\,d\eta^\alpha = 0$)
with $n$ unknown multipliers $\lambda_\mu$.
With this external reaction force, Hamilton’s equations become
\startformula
	d\eta^\alpha = \frac{\partial H}{\partial \pi_\alpha}\,dt
	\qquad
	d\pi_\alpha = \left( - \frac{\partial H}{\partial \eta^\alpha} + \lambda_\mu e^\mu_\alpha \right)\,dt
\stopformula
The final result is $2m$ Hamilton’s Equations and $n$ constraint equations. These can be used to solve for the $m$ displacements $d\eta^\alpha$, $m$ impulses $d\pi_\alpha$, and $n$ multipliers $\lambda_\mu$. For rolling constraints, there is no velocity dependence in Hamilton’s equations. The reaction force maintaining all of the constraints has components
\startformula
	F_{\alpha} = \lambda_\mu e^\mu_\alpha.
\stopformula
Frequently, only the positions are constrained explicitly.
\startformula
	f^\mu = \text{const.},
\stopformula
where each $f^\mu$ depends only on position.
These are called holonomic constraints.

The displacements are constrained only as a consequence of the position constraints, as can be seen by finding the changes in the $f^\mu$ through the chain rule.
\startformula
	df^\mu = \frac{\partial f^\mu}{\partial\eta^\alpha}\,d\eta^\alpha = 0
\stopformula
This is the a constraint linear in displacements, now with
\startformula
	e^\mu_\alpha = \frac{\partial f^\mu}{\partial\eta^\alpha}.
\stopformula
Hamilton’s equations with holonomic constraints become
\startformula
	d\eta^\alpha = \frac{\partial H}{\partial \pi_\alpha}\,dt
	\qquad
	d\pi_\alpha = \left( - \frac{\partial H}{\partial \eta^\alpha} + \lambda_\mu \frac{\partial f^\mu}{\partial\eta^\alpha}\right)\,dt
\stopformula
This is the common result for constrained motion. See, for example, José and Saletan, in \booktitle{Classical Dynamics} (pp. 115-6) which discusses nonholonomic constraints as well. I have only seen these methods in the Lagrangian formulation, but the Hamiltonian formulation is a natural extension. There is no obstacle to mixing holonomic and nonholonomic constraints.

The geometric story appears to be this. The $e^\mu_\alpha$ are differential forms on the configuration space, and the constraint equations require that the velocity vectors give zero when contracted with these forms.  The reaction force is also a differential form, a scalar multiple of the $e^\mu_\alpha$. If the differential form can be written as the differential of a scalar function, $e = df$, then constraints are holonomic, and the forces are monogenic. (See Lanczos pp. 146-7.)

[Examples: Rolling down a slope (find r, theta, p, L, and friction force with four equations of motion and one constraint equation.). Rolling in more dimensions. Piston, rod, and crankshaft. Hydraulics example. Superball with friction.]



\section{Relativistic Hamiltonian method}

Hamiltonian methods are excellent for a full relativistic treatment. The key is to allow the particle to move in four dimensions while geting rid of the Hamiltonian all together. Replace the Hamiltonian with a single constraint that keeps the particle on shell.
\startformula
	{\cal H }= \half g^{\mu\nu}(\pi_\mu - eA_\mu)(\pi_\nu - eA_\nu) = \half m^2
\stopformula
This is a single constraint involving momenta and positions (through $g^{\mu\nu}$ and $A_\mu$). This will lead to one unknown multiplier, but this multiplier can be set to one, fixing the parametrization of the the particle path. (MTW \booktitle{Gravitation,} pp.~488--489, 654, 897--901 and Lanczos \booktitle{The Variational Principles of Mechanics,} pp.~319--320, 327--336)

\stoptext
\stopcomponent