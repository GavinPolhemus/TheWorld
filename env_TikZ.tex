\startenvironment env_TikZ

\usemodule[tikz]			% TikZ module for diagrams
\usepgflibrary[shapes.misc]	% Node shapes for ConTeXt and pure pgf
\usetikzlibrary[shapes.arrows]
\usemodule[pgfplots]			% PGFPlots module for graphs
\pgfplotsset{compat=1.17}		% Required to get new any features that could cause compatibility issues.
\usetikzlibrary[arrows.meta,backgrounds]
\usetikzlibrary[decorations.text]
\usetikzlibrary[decorations.markings]
\usetikzlibrary[angles] % quotes syntax broken in ConTeXt.
\usetikzlibrary[math]
\usepgfplotslibrary[groupplots] % groupplot has unconventional commands in ConTeXt, fixed below
\unexpanded\def\startgroupplot{\groupplot}
\unexpanded\def\stopgroupplot{\endgroupplot}

\usepgfplotslibrary[fillbetween]
\usetikzlibrary[pgfplots.polar]

\tikzset{>/.tip=Latex[round]} % The tip drawn by -> will be the LaTeX tip, round.
%\tikzset{>/.tip=Stealth[inset=0pt, round]} % The tip drawn by -> will be similar to the MetaPost tip.
\tikzset{tips=proper} % Arrows with zero length will not have heads. 
\tikzset{help lines/.style=middlegray}

\pgfplotsset{
		every tick/.style={middlegray}, % Fixes ticks which are too light in ConTeXt
		major grid style = {middlegray},
}

% Redefined footnote size graphs, replacing LaTeX style commands with ConTeXt commands.
\pgfplotsset{
    footnotesize/.style={
		width=5cm,
		height=,
		%legend style={font=\switchtobodyfont[8pt]},
		%tick label style={font=\switchtobodyfont[8pt]},
		%label style={font=\switchtobodyfont[8pt]},
		%title style={font=\switchtobodyfont[8pt]},
		every axis title shift=0pt,
		max space between ticks=15,
		every mark/.append style={mark size=8},
		major tick length=0.1cm,
		minor tick length=0.066cm,
		every tick/.style={middlegray}, % Fixes ticks which are too light in ConTeXt
		major grid style = {middlegray},
	},
}

% Specifically for TikZ figures

\startbuffer[starttikz]
\environment env_physics
\environment env_TikZ
\setupbodyfont [libertinus,11pt]
\setoldstyle % Old style numerals in text
   \startTEXpage\small
   \starttikzpicture[thick]
\stopbuffer

\startbuffer[stoptikz]
   \stoptikzpicture
   \stopTEXpage
\stopbuffer

% The \marginTikZ command has three arguments.
% #1 Usually a vertical space command, but could be anything
% #2 Buffer's name, also used as reference name
% #3 Caption contents

\define[3]\marginTikZ{
  \startplacefigure[location=margin, reference=fig:#2, title={#3}]
    #1\typesetbuffer[starttikz,#2,stoptikz]
  \stopplacefigure
}

% Cart and a block on a track

% Side view
\pgfplotsset{margin cart track/.style={
	footnotesize,
	%width=\marginparwidth, scale only axis,
	x={1mm},y={1mm},
	xlabel={$x$ (cm)},
	xmin=-0.5,xmax=49.5,
	ymin=0,ymax=6.1,
	hide y axis,
	%xticklabel={\relax},
	minor x tick num=4,
	axis x line=bottom,
	tick align=outside,
	x axis line style={-}
}}
\pgfplotsset{big diagram cart track/.style={
	%\fill [black!10] (0,0) rectangle (13.6,-.15);
	%footnotesize,
	x={1mm},y={1mm},
	xlabel={$x$ (cm)},
	xmin=-1,xmax=151,
	ymin=-10,ymax=35,
	hide y axis,
	minor x tick num=4,
	axis x line=bottom,
	tick align = outside,
	x axis line style={-}
}}
\tikzset{
	cart/.pic={
		\startscope[transparency group]
		\filldraw  [fill=black!20, semithick]
			(-6mm,1mm)
			{[rounded corners=1mm]--(-6mm,5mm)
			--(6mm,5mm)}
			--(6mm,1mm)
			--cycle;
		\filldraw (-4mm,1mm) circle[radius=1mm];
		\filldraw(4mm,1mm) circle[radius=1mm];
		\fill(0mm,2.5mm) circle[radius=.4mm];
		\stopscope
	}
}
\tikzset{
	block/.pic={
		\filldraw  [fill=black!20, rounded corners=1mm, semithick]
			(-3mm,0mm) rectangle (3mm,5mm);
		\fill[fill=black] (0mm,2.5mm) circle[radius=.4mm];
	}
}
\tikzset{
	pics/pendulum/.style={
		code = {
			\coordinate (-center) at (0mm,-245mm);
			\draw[color=black!#1, thick] (-center) -- (0mm,0mm);
			\fill[fill=white] (-center) circle[radius=3mm];
			\path[ball color=white, opacity={#1/100}] (-center) circle[radius=3mm];
			\draw[draw=black!#1] (-center) circle[radius=3mm];
			\fill[fill=black!#1] (-center) circle[radius=.4mm];
			\fill[fill=black] (0mm,0mm) circle[radius=.4mm];
		}
	}
}
\tikzset{
	pics/small pendulum/.style={
		code = {
			\coordinate (-center) at (0mm,-245mm);
			\draw[color=black!#1, thick] (-center) -- (0mm,0mm);
			\fill[fill=white] (-center) circle[radius=2mm];
			\path[ball color=white, opacity={#1/100}] (-center) circle[radius=2mm];
			\draw[draw=black!#1] (-center) circle[radius=2mm];
			\fill[fill=black!#1] (-center) circle[radius=.4mm];
			\fill[fill=black] (0mm,0mm) circle[radius=.4mm];
		}
	}
}
\tikzset{
	dumbbell/.pic={
		\startscope[transparency group]
		\draw[ultra thick,black!67] (1.5cm,0) -- (-1.5cm,0);
		\draw[shade, ball color = black] (-1.5cm,0) circle[radius=.125cm];
		\draw[shade, ball color = gray] (1.5cm,0) circle[radius=.125cm];
		\stopscope
	}
}
\tikzset{
	dumbbell31/.pic={
		\startscope[transparency group]
		\draw[ultra thick,black!67] (1.5cm,0) -- (-1.5cm,0);
		\draw[shade, ball color = black] (-1.6cm,0) circle[radius=.18cm];
		\draw[shade, ball color = gray] (1.6cm,0) circle[radius=.125cm];
		\stopscope
	}
}
\tikzset{
	dumbbell31CoM/.pic={
		\startscope[transparency group]
		\draw[ultra thick,black!67] (2.4cm,0) -- (-0.8cm,0);
		\draw[shade, ball color = black] (-0.8cm,0) circle[radius=.18cm];
		\draw[shade, ball color = gray] (2.4cm,0) circle[radius=.125cm];
		\stopscope
	}
}
\tikzset{
	RightHandOut/.pic={
		\draw[thick] (-.2cm,0)
			-- (-0.2cm,1.2cm) arc [start angle=180, end angle=0, radius=0.2cm]
			-- (0.2,-1.0) -- (1.2,-1.0) arc [start angle=90, end angle=-90, radius=0.2cm]
			-- (-1.0,-1.4) arc [start angle=-90, end angle=-180, radius=0.4cm]
			-- (-1.4,0) ;
		\draw[thick] (0.2,-1.0) arc [start angle=0, end angle=135, x radius=0.6cm, y radius=0.4cm];
		\draw[thick] (-0.4cm,0) circle[radius=0.2cm];
		\draw[thick] (-0.8cm,0) circle[radius=0.2cm];
		\draw[thick] (-1.2cm,0) circle[radius=0.2cm];
		\node at (0.4cm,-1.2cm) {\small$\vec F$};
		\draw[thick,->] (0.7cm,-1.2cm) -- (1.3cm,-1.2cm);
		\node at (0,0.1cm) {\small$q\vec v$};
		\draw[thick,->] (0,0.4cm) -- (0,1.3cm);
		\node at (-0.7cm,-0.4cm) {\small$\vec B$};
		\node at (-0.4cm,0) {\small•};
		\node at (-0.8cm,0) {\small•};
		\node at (-1.2cm,0) {\small•};
	}
}
\tikzset{
	RightHandIn/.pic={
		\draw[thick] (.2cm,0)
			-- (0.2cm,1.2cm) arc [start angle=0, end angle=180, radius=0.2cm]
			-- (-0.2,-1.0) -- (-1.2,-1.0) arc [start angle=90, end angle=270, radius=0.2cm]
			-- (1.0,-1.4) arc [start angle=-90, end angle=0, radius=0.4cm]
			-- (1.4,0) arc [start angle=0, end angle=180, radius=0.2cm]
			arc [start angle=0, end angle=180, radius=0.2cm]
			arc [start angle=0, end angle=180, radius=0.2cm];
		\node at (-0.4cm,-1.2cm) {\small$\vec F$};
		\draw[thick,->] (-0.7cm,-1.2cm) -- (-1.3cm,-1.2cm);
		\node at (0,0.1cm) {\small$q\vec v$};
		\draw[thick,->] (0,0.4cm) -- (0,1.3cm);
		\node at (0.7cm,-0.4cm) {\small$\vec B$};
		\node at (0.4cm,0) {\small\times};
		\node at (0.8cm,0) {\small\times};
		\node at (1.2cm,0) {\small\times};
	}
}
\tikzset{
	argyleruler/.pic={
		\startscope[transparency group]
		\filldraw  [fill=black!30, semithick]
			(0mm,0mm) rectangle (20mm,10mm);
		\foreach \x in {0,10}{
			\fill [black!20](\x mm,5mm) -- ++(5mm,5mm) -- ++(5mm,-5mm) -- ++(-5mm,-5mm) --cycle;
			\draw [white, thick](\x mm,0mm) -- ++(10mm,10mm);
			\draw [white, thick](\x mm,10mm) -- ++(10mm,-10mm);
		}
		\foreach \x in {1,2,...,19}{
			\draw (\x mm,0mm) -- ++(0mm,.5mm);
			\draw (\x mm,10mm) -- ++(0mm,-.5mm);
		}
		\foreach \x in {10,20}{
			\draw (\x mm,0mm) -- ++(0mm,1mm);
			\draw (\x mm,10mm) -- ++(0mm,-1mm);
		}
		\draw [semithick]
			(0mm,0mm) rectangle (20mm,10mm);
		\stopscope
	}
}


\tikzset{
	clock/.pic={
		\draw[fill=white] (0,0) circle[radius=1.6mm];
		\draw (0,0.12) -- ++(0,-.12) -- ++(30:.09);
	}
}

\stopenvironment