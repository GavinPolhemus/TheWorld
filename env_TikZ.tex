% !TEX root = volume01/prd_volume01.tex
\startenvironment env_TikZ

\usemodule[tikz]			% TikZ module for diagrams
\usepgflibrary[shapes.misc]	% Node shapes for ConTeXt and pure pgf
\usemodule[pgfplots]			% PGFPlots module for graphs
\pgfplotsset{compat=1.17}		% Required to get new any features that could cause compatibility issues.
\usetikzlibrary[arrows.meta,backgrounds]
\usetikzlibrary[decorations.text]
\usetikzlibrary[decorations.markings]
\usetikzlibrary[angles,quotes]
\usetikzlibrary[math]
\usepgfplotslibrary[groupplots] % groupplot has unconventional commands in ConTeXt, fixed below
\unexpanded\def\startgroupplot{\groupplot}
\unexpanded\def\stopgroupplot{\endgroupplot}

\usepgfplotslibrary[fillbetween]
%\unprotect
%\def\pgfplotsdefineaxistype{\pgfutil@ifnextchar[{\pgfplotsdefineaxistype@}{\pgfplotsdefineaxistype@[cart]}}%
%\def\pgfplotsdefineaxistype@[#1]#2#3{%
%	\expandafter\def\csname pgfplots@axistype@#2\endcsname{%
%		\pgfkeyssetvalue{/pgfplots/@expected axis cs}{#1}%
%		#3%
%	}%
%	\expandafter\def\csname pgfplots@environment@#2axis\endcsname{%
%		\pgfutil@ifnextchar[%
%			{\csname pgfplots@environment@#2axis@\endcsname}%
%			{\csname pgfplots@environment@#2axis@\endcsname[]}%
%	}%
%	\expandafter\long\expandafter\def\csname pgfplots@environment@#2axis@\endcsname[##1]{%
%		\axis[%
%			##1,%
%			data cs=#1,%  assume that coordinates provided by \addplot are in the correct coordsystem
%			axis type=#2% active the correct axis type
%		]%
%	}%
%	\expandafter\def\csname endpgfplots@environment@#2axis\endcsname{\endaxis}%
%	\edef\pgfplots@glob@TMPc{\expandafter\noexpand\csname #2axis\endcsname\expandafter\noexpand\csname pgfplots@environment@#2axis\endcsname}%
%	\expandafter
%	\tikzaddtikzonlycommandshortcutlet\pgfplots@glob@TMPc
%	%
%	\edef\pgfplots@glob@TMPc{\expandafter\noexpand\csname end#2axis\endcsname\expandafter\noexpand\csname endpgfplots@environment@#2axis\endcsname}%
%	\expandafter
%	\tikzaddtikzonlycommandshortcutlet\pgfplots@glob@TMPc
%	%
%	% ConTeXt compatibility:
%	\t@pgfplots@toka=\expandafter{\csname #2axis\endcsname}%
%	\t@pgfplots@tokb=\expandafter{\csname end#2axis\endcsname}%
%	\expandafter\edef\csname start#2axis\endcsname{\the\t@pgfplots@toka}%
%	\expandafter\edef\csname stop#2axis\endcsname{\the\t@pgfplots@tokb}%
%}%
%\protect
\usetikzlibrary[pgfplots.polar]

\tikzset{>/.tip=Latex} % The tip drawn by -> will be the LaTeX tip.
\tikzset{tips=proper} % Arrows with zero length will not have heads. 

% Redefined footnote size graphs, replacing LaTeX style commands with ConTeXt commands.
\pgfplotsset{
    footnotesize/.style={
		width=5cm,
		height=,
		%legend style={font=\switchtobodyfont[8pt]},
		%tick label style={font=\switchtobodyfont[8pt]},
		%label style={font=\switchtobodyfont[8pt]},
		%title style={font=\switchtobodyfont[8pt]},
		every axis title shift=0pt,
		max space between ticks=15,
		every mark/.append style={mark size=8},
		major tick length=0.1cm,
		minor tick length=0.066cm,
		every tick/.style={middlegray}, % Fixes ticks which are too light in ConTeXt
		major grid style = {middlegray},
	},
}

% Cancel command for use in equations. Perhaps one command can work in both italic and roman.
\tikzset{cancel/.style={
	cross out, draw, semithick, % "cross out" or "strike out"
	inner sep=0pt,outer sep=0pt,
	anchor=text
}}
\define[1]\cancel{
	\ifmmode
		\mathchoice
		{\tikz[baseline] \node[cancel]{\m{\displaystyle #1}};}
		{\tikz[baseline] \node[cancel]{\m{\textstyle #1}};}
		{\tikz[baseline] \node[cancel]{\m{\scriptstyle #1}};}
		{\tikz[baseline] \node[cancel]{\m{\scriptscriptstyle #1}};}
	\else
    	\tikz[baseline] \node[cancel]{#1};
	\fi
}
\define[1]\ucan{
	\ifmmode
		\mathchoice
		{\tikz[baseline] \node[cancel]{\m{\rm \displaystyle #1}};}
		{\tikz[baseline] \node[cancel]{\m{\rm \textstyle #1}};}
		{\tikz[baseline] \node[cancel]{\m{\rm \scriptstyle #1}};}
		{\tikz[baseline] \node[cancel]{\m{\rm \scriptscriptstyle #1}};}
	\else
    	\tikz[baseline] \node[cancel]{#1};
	\fi
}

% Cart and a block on a track

% Side view
\pgfplotsset{margin cart track/.style={
	footnotesize,
	%width=\marginparwidth, scale only axis,
	x={1mm},y={1mm},
	xlabel={\m{x} (cm)},
	xmin=-0.5,xmax=49.5,
	ymin=0,ymax=6.1,
	hide y axis,
	%xticklabel={\relax},
	minor x tick num=4,
	axis x line=bottom,
	tick align=outside,
	x axis line style={-}
}}
\pgfplotsset{big diagram cart track/.style={
	%\fill [black!10] (0,0) rectangle (13.6,-.15);
	%footnotesize,
	x={1mm},y={1mm},
	xlabel={\m{x} (cm)},
	xmin=-1,xmax=151,
	ymin=-10,ymax=35,
	hide y axis,
	minor x tick num=4,
	axis x line=bottom,
	tick align = outside,
	x axis line style={-}
}}
\tikzset{
	cart/.pic={
		\startscope[transparency group]
		\filldraw  [fill=black!20, semithick]
			(-6mm,1mm)
			{[rounded corners=1mm]--(-6mm,5mm)
			--(6mm,5mm)}
			--(6mm,1mm)
			--cycle;
		\filldraw (-4mm,1mm) circle[radius=1mm];
		\filldraw(4mm,1mm) circle[radius=1mm];
		\fill(0mm,2.5mm) circle[radius=.4mm];
		\stopscope
	}
}
\tikzset{
	block/.pic={
		\filldraw  [fill=black!20, rounded corners=1mm, semithick]
			(-3mm,0mm) rectangle (3mm,5mm);
		\fill[fill=black] (0mm,2.5mm) circle[radius=.4mm];
	}
}
\tikzset{
	pics/pendulum/.style={
		code = {
			\coordinate (-center) at (0mm,-245mm);
			\draw[color=black!#1, thick] (-center) -- (0mm,0mm);
			\fill[fill=white] (-center) circle[radius=3mm];
			\path[ball color=white, opacity={#1/100}] (-center) circle[radius=3mm];
			\draw[draw=black!#1] (-center) circle[radius=3mm];
			\fill[fill=black!#1] (-center) circle[radius=.4mm];
			\fill[fill=black] (0mm,0mm) circle[radius=.4mm];
		}
	}
}
\tikzset{
	pics/small pendulum/.style={
		code = {
			\coordinate (-center) at (0mm,-245mm);
			\draw[color=black!#1, thick] (-center) -- (0mm,0mm);
			\fill[fill=white] (-center) circle[radius=2mm];
			\path[ball color=white, opacity={#1/100}] (-center) circle[radius=2mm];
			\draw[draw=black!#1] (-center) circle[radius=2mm];
			\fill[fill=black!#1] (-center) circle[radius=.4mm];
			\fill[fill=black] (0mm,0mm) circle[radius=.4mm];
		}
	}
}
\tikzset{
	dumbbell/.pic={
		\startscope[transparency group]
		\draw[ultra thick,black!67] (1.5cm,0) -- (-1.5cm,0);
		\draw[shade, ball color = black] (-1.5cm,0) circle[radius=.125cm];
		\draw[shade, ball color = gray] (1.5cm,0) circle[radius=.125cm];
		\stopscope
	}
}
\tikzset{
	dumbbell31/.pic={
		\startscope[transparency group]
		\draw[ultra thick,black!67] (1.5cm,0) -- (-1.5cm,0);
		\draw[shade, ball color = black] (-1.6cm,0) circle[radius=.18cm];
		\draw[shade, ball color = gray] (1.6cm,0) circle[radius=.125cm];
		\stopscope
	}
}
\tikzset{
	dumbbell31CoM/.pic={
		\startscope[transparency group]
		\draw[ultra thick,black!67] (2.4cm,0) -- (-0.8cm,0);
		\draw[shade, ball color = black] (-0.8cm,0) circle[radius=.18cm];
		\draw[shade, ball color = gray] (2.4cm,0) circle[radius=.125cm];
		\stopscope
	}
}
\tikzset{
	RightHandOut/.pic={
		\draw[thick] (-.2cm,0)
			-- (-0.2cm,1.2cm) arc [start angle=180, end angle=0, radius=0.2cm]
			-- (0.2,-1.0) -- (1.2,-1.0) arc [start angle=90, end angle=-90, radius=0.2cm]
			-- (-1.0,-1.4) arc [start angle=-90, end angle=-180, radius=0.4cm]
			-- (-1.4,0) ;
		\draw[thick] (0.2,-1.0) arc [start angle=0, end angle=135, x radius=0.6cm, y radius=0.4cm];
		\draw[thick] (-0.4cm,0) circle[radius=0.2cm];
		\draw[thick] (-0.8cm,0) circle[radius=0.2cm];
		\draw[thick] (-1.2cm,0) circle[radius=0.2cm];
		\node at (0.4cm,-1.2cm) {\small\m{\vec F}};
		\draw[thick,->] (0.7cm,-1.2cm) -- (1.3cm,-1.2cm);
		\node at (0,0.1cm) {\small\m{q\vec v}};
		\draw[thick,->] (0,0.4cm) -- (0,1.3cm);
		\node at (-0.7cm,-0.4cm) {\small\m{\vec B}};
		\node at (-0.4cm,0) {\small•};
		\node at (-0.8cm,0) {\small•};
		\node at (-1.2cm,0) {\small•};
	}
}
\tikzset{
	RightHandIn/.pic={
		\draw[thick] (.2cm,0)
			-- (0.2cm,1.2cm) arc [start angle=0, end angle=180, radius=0.2cm]
			-- (-0.2,-1.0) -- (-1.2,-1.0) arc [start angle=90, end angle=270, radius=0.2cm]
			-- (1.0,-1.4) arc [start angle=-90, end angle=0, radius=0.4cm]
			-- (1.4,0) arc [start angle=0, end angle=180, radius=0.2cm]
			arc [start angle=0, end angle=180, radius=0.2cm]
			arc [start angle=0, end angle=180, radius=0.2cm];
		\node at (-0.4cm,-1.2cm) {\small\m{\vec F}};
		\draw[thick,->] (-0.7cm,-1.2cm) -- (-1.3cm,-1.2cm);
		\node at (0,0.1cm) {\small\m{q\vec v}};
		\draw[thick,->] (0,0.4cm) -- (0,1.3cm);
		\node at (0.7cm,-0.4cm) {\small\m{\vec B}};
		\node at (0.4cm,0) {\small\times};
		\node at (0.8cm,0) {\small\times};
		\node at (1.2cm,0) {\small\times};
	}
}

\stopenvironment