% !TEX useOldSyncParser
\startcomponent c_chapter01
\project project_world
\product prd_volume02

\setupsynctex[state=start,method=max] % "method=max" or "min"

%%%%%%%%%%%%%%%%%%%%%%%%%%%%%
\startchapter[title={The Energy Called Heat}, reference=ch:Heat]
%%%%%%%%%%%%%%%%%%%%%%%%%%%%%

%\placefigure[margin,none]{}{\small
%	\startalignment[flushleft]
%By convention sweet and by convention bitter, by convention hot, by convention cold, by convention color; but in reality atoms and void.%\autocite{p.46}{Helmholtz1857}
%	\stopalignment
%	\startalignment[flushright]
%	%{\it On the Physiological Causes\\
%	%	of Harmony in Music}\\
%	{\sc Democritus}\\
%	c.460 -- c.370 \scaps{BCE}
%	\stopalignment
%}

%%%%%%%%%%%%%%%%%%%%%%%%%%%%%

\Initial{W}{hen Thomas Newcomen} and his partner John Calley built the first piston steam engine 1712, there was not a physicist alive who could have explained how it worked. Newcomen and Calley built the engine to pump water out of a mine at the Conygree Coalworks in central England. At the time, there were no powered machines working in the mines – all work was done by laborers and horses. The mine did have a sort of steam pump that used condensing steam to suck ground water out of the mine, preventing the mine from flooding. However, this steam pump was not working and Newcomen was called to fix it. The fix he devised, shown in \in{figure}[fig:NewcomenAtmospheric], was actually an entirely new design. \in{Figure}[fig:NewcomenAtmospheric] shows his new engine in its resting position. On the left is a simple pump driven by a large weight. When the weight is lifted, water flows into the bottom of the pump, deep in the mine (not shown). When the heavy weight is lowered, it forces water up a pipe to the surface. Then the weight must be lifted again. 

\placefigure[margin][fig:NewcomenAtmospheric]{Newcomen engine from \booktitle{Practical physics for secondary schools. Fundamental principles and applications to daily life,} by Newton Henry Black and Harvey Nathaniel Davis, publ. 1913 by Macmillan and Company, p.~219} {\externalfigure[chapter11/NewcomenAtmospheric][width=\rightmarginwidth]}

Lifting the weight is the job of the engine on the right. Water in the boiler \m{A} is heated by a fire below. When the valves \m{V} and \m{V''} are opened (with valve \m{V'} closed), steam rises into the cylinder \m{B} forcing any air out through the valve \m{V''}.

Once the cylinder is filled entirely with steam, the cylinder is sealed by closing valves \m{V} and \m{V''}. Then the valve \m{V'} is opened briefly, allowing cold water from the tank \m{C} to spray into the cylinder, as shown in \in{figure}[fig:NewcomenAtmospheric]. This spray cools the cylinder and the steam, causing the steam to condense into liquid water again. All of the steam condenses into a tiny volume of water, which collects with the sprayed water at the bottom of the cylinder. No air is allowed back into the cylinder, and all of the steam has condensed, so the cylinder is almost empty, with near vacuum above the small pool of  water at the bottom.

Outside the cylinder is the atmosphere, pushing in from all sides. Most importantly, the atmosphere pushes down on the movable piston which forms the top of the cylinder in \in{figure}[fig:NewcomenAtmospheric]. The downward pressure \m{P} produces a huge downward force \m{F} on the piston.
\startformula
	F = PA,
\stopformula
where \m{A} is the piston's area. (In this chapter we will use \m{P} for pressure, not power.) The {\sc si} unit for pressure is a pascal, defined by
\startformula
	1\units{Pa} = 1\units{N/m^2}.
\stopformula
%Atmospheric pressure is about \m{P\sub{atm} = 101\units{kPa} = 1.01\sci{5}\units{Pa}}, but it varies significantly with altitude and weather.

%%%%%%%%%%%%%%%%%%%%%%%%%%%%%%%%%%%%%%%%%%%%%%%%%%%
\startexample[ex:NewcomenForce]
A working replica of Newcomen's engine operates at a museum near the site of Newcomen's original. The piston's radius is \m{26\units{cm}}. What is the force exerted on the piston by the atmosphere?

\startsolution
The piston's area is \m{A= \pi r^2 = \pi(0.26m)^2 = 0.212\units{m^2}}. The force is
\startformula\startmathalignment
\NC F	\NC = PA												\NR
\NC		\NC = (1.01\sci{5}\units{Pa})(0.212\units{m^2})				\NR
\NC		\NC = (1.01\sci{5}\units{N/\ucan{m^2}})(0.212\,\ucan{m^2})	\NR
\NC		\NC = 21\,000\units{N}.
\stopmathalignment\stopformula
This downward force acting on the right side of \in{figure}[fig:NewcomenAtmospheric] is enough to lift a weight of over \m{2000\units{kg}} on the left.
\stopsolution
\stopexample
%%%%%%%%%%%%%%%%%%%%%%%%%%%%%%%%%%%%%%%%%%%%%%%%%%%

The atmosphere's pressure drives the piston downward. This downward motion is the engine's power stroke. It pulls down on the chain at \m{D} in \in{figure}[fig:NewcomenAtmospheric], rocking the large lever \m{FED} around the pivot \m{E}. That lifts the chain at \m{F}, raising the large weight.

Once the weight is lifted, the valve \m{V} is opened, allowing the water to flow down into the boiler. Hot steam from the boiler fills the cylinder \m{B} again, allowing the piston to rise and the weight to fall.
The whole process is then be repeated, spraying cold water into the cylinder to condense the steam for the power stroke. (Since the cylinder is filling with steam directly from the boiler, valve \m{V''} does not need to be opened each cycle.)

Newcomen's design included mechanisms that used the motion of the top beam to open and close the valves automatically at precisely the right times in the cycle. (These mechanisms are not shown in the diagrams.) This allowed the engine to operate continuously, performing each cycle in about five seconds and doing as much work as a team of several horses or dozens of men. Many mines upgraded to Newcomen's reliable and powerful engine. Dozens were installed in Britain and eventually in other parts of Europe.

The engines could be built in a variety of sizes, depending on the mine's needs. Engineers developed the concepts of work and power to quantify their engines' productivity, so they could build the right size engine.


\startbuffer[TikZ:NewcomenPower]
\environment env_physics
\environment env_TikZ
\setupbodyfont [libertinus,11pt]
\setoldstyle \small% Old style numerals in text
\startTEXpage
\starttikzpicture% tikz code
% left cylinder
\draw[thick,fill=black!30] (-2.5,7.5)-- ++(0,-7.5) -- ++(2,0) -- ++(0,7.5) -- ++(-.1,0) -- ++(0,-7.4) -- ++(-1.8,0) -- ++(0,7.4) -- cycle;
\fill[black!30] (-1.7,-0.5) rectangle ++(0.4,0.5);
\draw[thick,] (-1.7,-0.5)-- ++(0,0.5) -- ++(0.4,0) -- ++(0,-0.5);
\node at (-1.5,3.5){\m{V\si}};
% right cylinder
\draw[thick,fill=black!30] (0.5,7.5)-- ++(0,-7.5) -- ++(2,0) -- ++(0,7.5) -- ++(-.1,0) -- ++(0,-7.4) -- ++(-1.8,0) -- ++(0,7.4) -- cycle;
\fill[black!30] (1.3,-0.5) rectangle ++(0.4,0.5);
\draw[thick,] (1.3,-0.5)-- ++(0,0.5) -- ++(0.4,0) -- ++(0,-0.5);
\node at (1.5,0.5){\m{V\sf}};
% left piston
\draw[thick,fill=black!30] (-2.4,7) rectangle ++(1.8,0.2);
\fill[black!30] (-1.6,8.2) rectangle ++(0.2,-1);
\draw[thick,] (-1.6,8.2)-- ++(0,-1) -- ++(0.2,0) -- ++(0,1);
\node at (-1,7.2) [single arrow, fill=black!25, single arrow head extend=.1cm, anchor=tip, shape border rotate=270]{\m{P}};
\node at (-2,7.2) [single arrow, fill=black!25, single arrow head extend=.1cm, anchor=tip, shape border rotate=270]{\m{P}};
% right piston
\draw[thick,fill=black!30] (0.6,1) rectangle ++(1.8,.2);
\fill[black!30] (1.4,8.2) rectangle ++(0.2,-7);
\draw[thick,] (1.4,8.2)-- ++(0,-7) -- ++(0.2,0) -- ++(0,7);
\node at (1,1.2) [single arrow, fill=black!25, single arrow head extend=.1cm, anchor=tip, shape border rotate=270]{\m{P}};
\node at (2,1.2) [single arrow, fill=black!25, single arrow head extend=.1cm, anchor=tip, shape border rotate=270]{\m{P}};
% y coordinate axis
\draw[thick,->] (0,0) -- (0,7.9)node[above]{\m{y}};
\draw[] (.45,1) -- ++(-0.5,0)node[left]{\m{y\sf}};
\draw[] (-.45,7) -- ++(0.5,0)node[right]{\m{y\si}};
\draw[] (.05,0.1) -- ++(-0.1,0)node[left]{\m{0}};
\stoptikzpicture
\stopTEXpage
\stopbuffer

\placefigure[margin][fig:NewcomenPower] % location, label
{The downward power stroke of Newcomen's atmospheric engine is driven by atmospheric pressure pushing down on the piston.} % caption text
{\noindent\typesetbuffer[TikZ:NewcomenPower]} % figure contents


%%%%%%%%%%%%%%%%%%%%%%%%%%%%%%%%%%%%%%%%%%%%%%%%%%%
\startexample[ex:NewcomenPower]
The replica of Newcomen's engine in \in{example}[ex:NewcomenForce] has a piston that travels approximately 1.8\units{m} inside the cylinder, shown in \in{figure}[fig:NewcomenPower]. Like Newcomen's original, it operates at a rate of one cycle every five seconds. Find the work done by the atmosphere on the engine and the average power.

\startsolution
The work can be calculated using the force form \in{example}[ex:NewcomenForce] and a displacement of 1.8\units{m}.
\startformula
W = F\Delta y
	= (21\,000\sci{4}\units{N})(1.8\units{m})	
	= 38\,000\units{J}.
\stopformula
The force and displacement are in the same direction, so the work is positive. In \in{figure}[fig:NewcomenPower] they are both pointed in the negative direction, so we also could have written both as negative. The answer would be the same.

Power is the rate at which the work is done.
\startformula
	\frac{W}{\Delta t} = \frac{38\,000\units{J}.}{5\units{s}} = 7500\units{W}
\stopformula
\stopsolution
\stopexample
%%%%%%%%%%%%%%%%%%%%%%%%%%%%%%%%%%%%%%%%%%%%%%%%%%%

While we used the piston's area and displacement to calculate the work, all that is actually needed is the pressure and the cylinder's change in volume. The volume of the cylinder in \in{figure}[fig:NewcomenPower] is \m{V = Ay}, where \m{A} is the piston's area and \m{y} is its height. Taking care with signs, we should write the downward force as \m{F=-PA}, since pressure and area are positive, but the force is in the negative direction.
\startformula
	W = F \Delta y = -P A(y\sf - y\si) = -P(Ay\sf - Ay\si) = -P(V\sf - V\si) = -P\Delta V
\stopformula
The negative sign deserves some explanation. During the engine's downward power stroke, the volume inside the cylinder is decreasing, so \m{\Delta V} is negative. The pressure is positive, so the work formula \m{W= -P\Delta V} will give positive work for the downward power stroke, as it should.

Work was a new concept to physicists in the eighteenth century – remember, the \visviva\ debate was raging at this time – but the ideas behind the engine's forceful power stroke were well understood. Atmospheric pressure, force, and displacement were all basic physics knowledge at that time.

The upward stroke was where physicists would have started struggling. The piston is lifted by the heavy weight, but this is only possible because steam is allowed into the cylinder, as shown in \in{figure}[fig:NewcomenSteam]. The steam, also at atmospheric pressure, pushes up on the bottom of the piston, balancing the downward force of the outside atmosphere so the piston can rise. The steam is clearly doing work as it lifts the piston against the downward force of the atmosphere, but where does the steam get the energy? Clearly, the energy is from the fire, but it is not obvious how the energy of the fire turns into work done on the piston.

\startbuffer[TikZ:NewcomenSteam]
\environment env_physics
\environment env_TikZ
\setupbodyfont [libertinus,11pt]
\setoldstyle \small% Old style numerals in text
\startTEXpage
\starttikzpicture% tikz code
% left cylinder
\draw[thick,fill=black!30] (-2.5,7.5)-- ++(0,-7.5) -- ++(2,0) -- ++(0,7.5) -- ++(-.1,0) -- ++(0,-7.4) -- ++(-1.8,0) -- ++(0,7.4) -- cycle;
\fill[black!30] (-1.7,-0.5) rectangle ++(0.4,0.5);
\draw[thick,] (-1.7,-0.5)-- ++(0,0.5) -- ++(0.4,0) -- ++(0,-0.5);
\node at (-1.5,0.5){\m{V\si}};
% right cylinder
\draw[thick,fill=black!30] (0.5,7.5)-- ++(0,-7.5) -- ++(2,0) -- ++(0,7.5) -- ++(-.1,0) -- ++(0,-7.4) -- ++(-1.8,0) -- ++(0,7.4) -- cycle;
\fill[black!30] (1.3,-0.5) rectangle ++(0.4,0.5);
\draw[thick,] (1.3,-0.5)-- ++(0,0.5) -- ++(0.4,0) -- ++(0,-0.5);
\node at (1.5,3.5){\m{V\sf}};
% right piston
\draw[thick,fill=black!30] (0.6,7) rectangle ++(1.8,0.2);
\fill[black!30] (1.4,8.2) rectangle ++(0.2,-1);
\draw[thick,] (1.4,8.2)-- ++(0,-1) -- ++(0.2,0) -- ++(0,1);
\node at (1,7.2) [single arrow, fill=black!25, single arrow head extend=.1cm, anchor=tip, shape border rotate=270]{\m{P}};
\node at (2,7.2) [single arrow, fill=black!25, single arrow head extend=.1cm, anchor=tip, shape border rotate=270]{\m{P}};
\node at (1,7) [single arrow, fill=black!25, single arrow head extend=.1cm, anchor=tip, shape border rotate=90]{\m{P}};
\node at (2,7) [single arrow, fill=black!25, single arrow head extend=.1cm, anchor=tip, shape border rotate=90]{\m{P}};
% left piston
\draw[thick,fill=black!30] (-2.4,1) rectangle ++(1.8,.2);
\fill[black!30] (-1.6,8.2) rectangle ++(0.2,-7);
\draw[thick,] (-1.6,8.2)-- ++(0,-7) -- ++(0.2,0) -- ++(0,7);
\node at (-1,1.2) [single arrow, fill=black!25, single arrow head extend=.1cm, anchor=tip, shape border rotate=270]{\m{P}};
\node at (-2,1.2) [single arrow, fill=black!25, single arrow head extend=.1cm, anchor=tip, shape border rotate=270]{\m{P}};
\node at (-1,1) [single arrow, fill=black!25, single arrow head extend=.1cm, anchor=tip, shape border rotate=90]{\m{P}};
\node at (-2,1) [single arrow, fill=black!25, single arrow head extend=.1cm, anchor=tip, shape border rotate=90]{\m{P}};
% y coordinate axis
\draw[thick,->] (0,0) -- (0,7.9)node[above]{\m{y}};
\draw[] (.45,7) -- ++(-0.5,0)node[left]{\m{y\sf}};
\draw[] (-.45,1) -- ++(0.5,0)node[right]{\m{y\si}};
\draw[] (.05,0.1) -- ++(-0.1,0)node[left]{\m{0}};
\stoptikzpicture
\stopTEXpage
\stopbuffer

\placefigure[margin][fig:NewcomenSteam] % location, label
{Steam entering the cylinder provides an upwards force on the piston. It is then lifted by the heavy weight on the left  of \in{figure}[fig:NewcomenAtmospheric].} % caption text
{\noindent\typesetbuffer[TikZ:NewcomenSteam]} % figure contents

\placefigure[margin][fig:DBernoulliGas]{Daniel Bernoulli's description of a gas. The gas molecules are not packed together. Still, their frequent collisions with the piston hold the piston and weight aloft.} {\externalfigure[chapter11/DBernoulliGas][width=\rightmarginwidth]}

Daniel Bernoulli's \booktitle{Hydrodynamica} appeared in 1838. In this great work, he brought together physicists' \visviva, engineers' work, and his own ideas about potential energy. He also offered an insightful and correct model of a gas like steam. He described the gas as being made of many tiny parts, now called molecules, that bounce around furiously – ricocheting off of the containers walls and off of each other, as shown in \in{figure}[fig:DBernoulliGas]. In this model, the tiny molecules are spread throughout the container, but they occupy very little of the space. It looks as if the piston and the weight at the top of \in{figure}[fig:DBernoulliGas] will fall, packing the molecules together at the bottom of the cylinder. Bernoulli realized that if the molecules are traveling fast enough, their repeated impacts on the bottom of the piston will provide enough tiny impulses to keep the piston from falling. This became known as the kinetic theory of gasses because it is the molecule's motion that causes the pressure against the sides of the container. The molecules' sizes and shapes are of little consequence.

Bernoulli took several important steps in developing this model. He recognized that the kinetic energy of the molecules could be affected by the work done by the piston, or by raising the container's temperature. This established a connection between energy and temperature which would eventually be central to understanding engines. Unfortunately,  this part of \booktitle{Hydrodynamica} was largely ignored for a century. Physicists were not interested in engines, and engineers were not reading physics treatises. It was a great century for steam engines, but not a great century for understanding the heat that powered them.

\section{Energy and Temperature in solids}
By the early nineteenth century physicists noticed the industrial revolution happening around them and many decided to catch up with the engineers by investigating the connection between energy and temperature. They found that the connection depended on the material being studied. Heating water over a fire to a certain temperature takes far longer than heating an equal mass of copper, for example. We say that water has a very high \keyterm{heat capacity} because it consumes a great deal of heat for a relatively small increase in temperature and gives off a great deal of heat when its temperature decreases. 

In 1819 at the École Polytechnique in Paris, Pierre Dulong and Alexis Petit were measuring the heat capacities of solid elements when the noticed a surprising chemistry connection. The energy required to raise the temperature of many solid elements was proportional to the amount of substance measured in moles! Moles were a fairly new idea, and the molar masses of many elements were not yet known. (In fact, a great many elements had not yet been identified.) This relationship became known as the Delong-Petit Law. In modern notation it says that the energy in random thermal motions \m{E\sth} is proportional to the temperature \m{T} and the number of moles \m{n}.
\startformula
	E\sub{therm} = 3nRT
\stopformula
The constant \m{R = 8.314\units{J/mol\.K}}. This law works especially well for the many elements that are metals, and it explains why lighter metals have a higher heat capacity than heavier metals. Lighter metals have more moles in the same mass, so they hold more thermal energy.

If you have taken chemistry, you might have noticed the \m{nRT} on the right side of the Delong-Petit Law looks a lot like the right side of the Ideal Gas Law. This is an important connection that should become clear by the end of the chapter. If you did not notice that connection, you are in good company. Delong and Petit did not notice it either because the Ideal Gas Law had not been discovered. Delong and Petit's discovery preceded the Ideal Gas Law by several decades.

The Delong-Petit Law gives us a formula for thermal energy that we can use with conservation of energy. Thermal energy can be combined with any of the other forms of energy – kinetic energy, or various types of potential energy – but for simple heating experiments \m{E\sth} is the only type of energy that matters. Solids do not expand dramatically when heated, so the work due to \m{\Delta V} is tiny compared to the heat, and can be ignored.

%%%%%%%%%%%%%%%%%%%%%%%%%%%%%%%%%%%%%%%%%%%%%%%%%%%
\startexample[ex:DPcopper]
\m{2.0} moles of copper are heated from \m{20\unit{\degree C}} to \m{100\unit{\degree C}}. How much heat \m{Q} was absorbed by the copper?
\startsolution
Start with conservation of energy, then use the Delong-Petit law for the thermal energies.
\startformula\startmathalignment
\NC H\si + \cancel{W} + Q	\NC = H\sf			\NR
\NC E\sub{therm,i} +  Q	\NC = E\sub{therm,f}	\NR
\NC 3nRT\si +  Q	\NC = 3nRT\sf				\NR
\NC Q	\NC = 3nRT\sf - 3nRT\si					\NR
%\NC 	\NC = 3nR(T\sf - T\si)					\NR
\NC 	\NC = 3nR \Delta T					\NR
\stopmathalignment\stopformula
The change in temperature is \m{\Delta T = 80\unit{\degree C}}. Since celsius degrees are the same size as kelvin degrees, the change in kelvin is the same: \m{\Delta T = 80\units{K}}. The heat absorbed by the copper is.
\startformula
 Q = 3(2.0\units{mol})( 8.314\units{J/mol\.K})(80\unit{K})	
 	= 50.\units{J}
\stopformula
\stopsolution
\stopexample
%%%%%%%%%%%%%%%%%%%%%%%%%%%%%%%%%%%%%%%%%%%%%%%%%%%

Dulong and Petit's discovery proved incredibly useful. Chemists had been trying to determine molar masses using chemical reactions. This presented a challenge, because the chemists could not be sure of the chemical formulas, so they often did not know if a mole of a compound in their reactions included one mole, two moles, or more moles of the individual elements. This could lead to huge errors in the molar mass. However, the Dulong-Petit Law, while not incredibly precise, allowed a determination of the number of moles in a sample purely from physical measurements, without any chemical reactions. This method corrected many of the chemists' early mistakes.

%%%%%%%%%%%%%%%%%%%%%%%%%%%%%%%%%%%%%%%%%%%%%%%%%%%
\startexample[ex:DPIron]
A sample of iron is found to lose \m{4.5\units{J}} of heat as it cools from \m{10.0\unit{\degree C}} to \m{0.0\unit{\degree C}}. How many moles of iron are in the sample?
\startsolution
Starting with conservation of energy leads to \m{Q= 3nR \Delta T}, found in \in{example}[ex:DPcopper], this time with \m{\Delta T = -10\units{K}} and \m{Q = -4.5\units{J}} (negative because the iron sample loses energy as heat).
\startformula\startmathalignment
\NC Q	\NC = 3nR \Delta T					\NR
\NC n	\NC = \frac{Q}{3R \Delta T}
	= \frac{-4.5\units{J}}{3( 8.314\units{J/mol\.K})(-10\unit{K})}
	= 0.018\units{mol}					\NR
\stopmathalignment\stopformula
The molar mass is the mass per mole, so dividing this sample's mass by \m{0.018\units{mol}} would give the molar mass.
\stopsolution
\stopexample
%%%%%%%%%%%%%%%%%%%%%%%%%%%%%%%%%%%%%%%%%%%%%%%%%%%
Delong and Petit's original paper looked at only 13 solid elements, but eventually their law was used for dozens of elements. The molar masses checked using the Delong-Petit law were central to the construction of the periodic table in 1871.

For physicists, the Dulong-Petit Law established a clear connection between atoms, temperature, and thermal energy. Somehow, atoms we holding thermal energy, probably in their microscopic random motions, but the details were unresolved. The law could not be extended to liquids and gasses, and since \m{W\approx0} for solids, the law did not provide the connection between heat \m{Q} and work \m{W} needed to understand engines like Newcomen's.

\section{The motion which we call heat}
Rudolf Clausius finally made the connection between energy, heat, and work. His equation relating them is the equation we have called conservation of energy. He was the one who put it all together.
\startformula
	H\si + W + Q = H\sf
\stopformula
This equation is often called the \keyterm{First Law of Thermodynamics}, especially when used in problems involving thermal energy and heat, but it is the same conservation of energy equation that you already know. The full equation is especially important for studying gasses, where heat and work both affect the energy.

Carnot also provided the first full accounting of the kinetic theory of gasses proposed by Daniel Bernoulli over a century earlier. Clausius' paper "On The Nature of the Motion We Call Heat" connects pressure and temperature with the microscopic kinetic energy of gas molecules in a container like the one drawn by Bernoulli.


I have simplified Clausius’ argument in this section by using a single component, \m{v_x}, where he used speed and angle. This simplification avoids the trigonometry and calculus required in his method. He looks at the horizontal motion of a single molecule and analyzes how it will impact one side of the container.
\startblockquote
...during the unit of time, each molecule will strike the side under consideration just as often as during that time it can, by following its peculiar direction, travel from the side in question to the other and back again. Let \m{h} be the distance between the left and right sides; then
\startformula
	\frac{v_x}{2h}
\stopformula
the number of impulses given to the side, \m{v_x} being the velocity of the molecule.
Hence if \m{n} represents the whole number of molecules, the number of shocks imparted by them will be
\startformula
	\frac{nv_x}{2h}
\stopformula
The \m{y} and \m{z} components of the velocity will not be affected by the shock, and will not enter into consideration in determining its intensity; the \m{x} component, however, will be changed by the shock into an equal velocity in the opposite direction. The action of the side upon the molecule, therefore, consists in depriving it in one direction of the velocity \m{v_x}, and of imparting to it a velocity of \m{-v_x} in the opposite direction. Hence the quantity of motion imparted to the molecule will be
\startformula
	2mv_x\,,
\stopformula
where m is the mass of the molecule.
Applying this to all molecules, we obtain during the unit of time, \m{nv_x/2h} times the same action, hence the quantity of motion imparted by the side to all the molecules which strike against it during the unit of time to be
\startformula
	\frac{nmv_x^2}{h}\,.
\stopformula
Let us now conceive the side to be capable of moving freely; then in order that it may not recede before the shocks of the molecules, it must be acted upon on the other side by a counter force, which latter may in fact be regarded as continuous, in consequence of the great number of shocks and the feebleness of each. The intensity of this force must be such as to enable it, during the unit of time, to generate the quantity of motion represented by the above expression. Since all forces, however, are measured by the quantity of motion they can produce in the unit of time, the above expression at once represents this force as well as the pressure exerted by the gas, the latter being equilibrated by the former.

If \m{A} be the superficial area of the side and \m{P} the pressure on the unit of surface, then
\startformula
	P=\frac{nmv_x^2}{Ah}\,.
\stopformula
The product \m{Ah} here involved gives the volume of the vessel or gas; hence representing the same by \m{V}, we have
\startformula
	P=\frac{nmv_x^2}{V}\,.
\stopformula
\stopblockquote
The mean squared velocities in the y-direction and z-direction will be equal, on average, to the mean squared velocity in the \m{x}-direction. Therefore,
\startformula
	v^2 = v_x^2 + v_y^2 + v_z^2 = 3v_x^2\,.
\stopformula
This is the form given by Clausius’ in his paper.

\startblockquote
If we write the last equation in the form
\startformula
	\tfrac{3}{2} PV=\frac{nmv^2}{2}\,,
\stopformula
the right-hand side then denotes the \visviva of the translational motion of the molecules. But, according to Mariotte's and Gay-Lussac's laws,
\startformula
	PV = T \cdot \text{const.} ,
\stopformula
where \m{T} is the absolute temperature; hence
\startformula
	\frac{nmv^2}{2}=T\cdot\text{const.}\,;
\stopformula
and, as before stated, the vis viva of the translational motion is proportional to the absolute temperature.
18. We may now make an interesting application of the above equations by determining the velocity v with which the several molecules of gas move.
The product nm represents the mass of the whole given quantity of gas, whose mass we will call \m{M}. Then g being the force of gravity,
\startformula
	nm = M ;
\stopformula
and we deduce
\startformula
	v^2=\frac{3PV}{M}\,.
\stopformula
To determine \m{v}, we use the density \m{ρ = M/V} of the gas under consideration. Then
\startformula
	v=\sqrt{\frac{3P}{\rho}}\,.
\stopformula

As particular cases, we obtain the following numbers corresponding to the temperature of melting ice and a pressure of 1 atmosphere – 101360 newtons of force per square metre:
	for oxygen: \m{461 m/s},
	for nitrogen: \m{492 m/s},
	for hydrogen: \m{1844 m/s}.
These numbers are the mean velocities which, for the totality of molecules, give the same vis viva as would their actual velocities. At the same time, however, it is possible that the actual velocities of the several molecules differ materially from their mean value.
\stopblockquote
%3In accordance with a practice lately become general, and with what I have myself done in former memoirs, I call the semi-product of the mass into the square of the velocity the vis viva, because it is only with this definition of the notion that we can, without the addition of a coefficient, equate the expressions representing a quantity of work and the increase or decrease of vis viva which corresponds to the same.


\section{Boltzmann's equipartition theorem}

Ludwig Boltzmann: On the Nature of Gas Molecules
It is an interesting fact that thermal energy tends to spread out equally into every accessible place. The average amount of energy stored in each place is proportional to the temperature, and is given by the formula \m{½\,\kB T}. Boltzmann's constant, kB, is the same in all situations. It is defined as exactly.
\startformula
	\kB = 1.380649\sci{-23}\units{ J/K}
\stopformula
You should think of \m{\kB } as a conversion from kelvins to joules. Kelvins are just another unit for measuring average energy. A kelvin is a very tiny amount of energy!

The molecule in the cylinder has three places for energy, \m{K_x = ½ mv_x^2}, \m{K_y = ½ mv_y^2}, and \m{K_z = ½ mv_z^2}. If the  temperature of the gas is \m{T}, then the molecule with have \m{½\,\kB T} of energy in each of these places. Add more molecules and there are more places for the energy, three places per molecule. Even though \m{½\,\kB T} is a very small amount of energy, a large number of molecules can store a significant amount of energy. Be careful with hot things!

7.	What is \m{PV} if there are \m{N} molecules with an average vertical kinetic energy \m{K_y = ½\,\kB T} per molecule? \m{T} is the temperature of the gas.
The pressure will be \m{N} times greater, so the new \m{PV} is
\startformula
PV = 2NKy = NkBT
\stopformula

8.	What is \m{PV} if there are \m{n} moles of gas in the cylinder? Use the definition \m{R=\NA\kB} where \m{\NA} is Avogadro’s number and \m{R} is the gas constant, both of which you remember fondly from chemistry. Also recall \m{N = n\NA}. Your result should look familiar.
\startformula
PV = nNAkBT = nRT
\stopformula
\startformula
PV = nRT
\stopformula

9.	At temperature \m{T}, what is the total energy in the gas. Assume that the molecules' kinetic energy is the only energy in the gas. (For simple, monoatomic gasses this is an excellent assumption.) Call the total energy \m{U}, even though it is actually kinetic energy. Everyone uses \m{U} for the internal energy of the gas.
The total energy in the gas is \m{N} times the Kinetic energy of a single atom
\startformula
U = NK = N(K_x + K_y + K_z) = N(½\,\kB T + ½\,\kB T + ½\,\kB T ) = (3/2)N\kB T = (3/2)RT,
\stopformula

n this activity you will use Boltzmann's insight to find the amount of energy required to heat 63.55 g of copper from 20°C to 100°C. (Why 63.55 g? Because working with four significant figures builds character.)
Boltzmann tells us that the energy in every available spot is ½kBT, where kB = 1.381×10-23 J/K. You need to find out how many spots there are in 63.55 g of copper.	
The number of spots can be found with a simple model of copper. Imagine that the copper atoms are arranged is a simple cubic lattice, and that each atom is connected to its immediate neighbors by springs. Every atom is connected to the atom on its left and its right, in front and in back, above and below. The atoms can bounce around a bit, but they are always pushed back towards their home position by the springs.

1.	Some of the thermal energy is stored in the bouncing kinetic energy of the copper atoms. There are three types of kinetic energy for each atom. List the three symbols for these.




2.	For a lattice containing a large number of copper atoms, how many springs are there per atom? (Hint: There are not six springs per atom!) Explain, with a picture if you like.



3.	How many thermal energy spots are there per atom?



4.	Find the thermal energy per atom for copper at room temperature (20°C = 293 K).



5.	Find the thermal energy per atom for copper at 100°C = 373 K.
6.	How much heat must be added to 63.55 g of copper to heat it from room temperature to 100°C.











7.	What is the specific heat capacity of copper in  J/g °C? (For comparison, the specific heat capacity of water is 4.2 J/g °C.)


\startblockquote
For all simple solids with the exception of carbon, [boron], and silicon the product \m{M\Gamma} is not very different from 6; it is between 5.22 and 6.9. The values obtained for \m{h} are between 1.78 and 2.34. On average, then, the total heat supplied is twice as great as that to increase the mean living force, half of which is used for work, if the forces acting on an atom are proportional to the removal of it from its position of rest. For solids, this is likely to be the case with some approximation. If one wanted to use the latter formula, one could insert the theoretical value \m{Ap /\rho T} for \m{\gamma'-\gamma}. The result is a similar one. 
From the composite solid bodies, which obey Neumann's law, it must be assumed that each of its atoms really has three kinds of mobility.
For simple or compound bodies, however, which deviate significantly from Dulong-Petit or Neumann's laws, it could perhaps be assumed that two or more atoms are so firmly connected that the number of types of mobility of the system they form is smaller than the three-fold number of its atoms. [p.108-9]
\stopblockquote

%\subject{Notes}
%\placefootnotes[criterium=chapter]
\placenotes[endnote][criterium=chapter]

%\subject{Bibliography}
%        \placelistofpublications

\stopchapter
\stopcomponent
%%%%%%%%%%%%%%%%%%%%%%%%%%%%%%%%%%%%%%%%%%%%%%%%%%%
%%%%%%%%%%%%%%%%%%%%%%%%%%%%%%%%%%%%%%%%%%%%%%%%%%%

%$6.241509\sci{18}$ electrons is \emph{negative} one Coulomb. One Coulomb of charge from a one volt battery gives one joule of energy. How many electron volts are in one joule? Converting the other way, one electron volt is how many joules? How many electron volts of energy would be produced by one mole of electrons going through a potential difference on one volt. How many joules? Calories too?


% Templates:

% Margin image
\placefigure[margin][] % Location, Label
{} % Caption
{\externalfigure[chapter03/][width=144pt]} % File

% Margin Figure
\startbuffer[TikZ:NAME]
\environment env_physics
\environment env_TikZ
\setupbodyfont [libertinus,11pt]
\setoldstyle % Old style numerals in text
\startTEXpage\small
\starttikzpicture% tikz code
\stoptikzpicture
\stopTEXpage
\stopbuffer

\placefigure[margin][fig:NAME] % Location, Label
{}	 % caption text
{\noindent\typesetbuffer[TikZ:NAME]}

% Aligned equation
\startformula\startmathalignment
\stopmathalignment\stopformula

% Aligned Equations
\startformula\startmathalignment[m=2,distance=2em]
\stopmathalignment\stopformula
