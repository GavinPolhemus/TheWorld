% !TEX useOldSyncParser
\startcomponent c_chapter01
\project project_world
\product prd_volume02

\setupsynctex[state=start,method=max] % "method=max" or "min"

%%%%%%%%%%%%%%%%%%%%%%%%%%%%%
\startchapter[title={The Motion Called Heat}, reference=ch:Heat]
%%%%%%%%%%%%%%%%%%%%%%%%%%%%%

\placefigure[margin,none]{}{\small
	\startalignment[flushleft]
By convention sweet and by convention bitter, by convention hot, by convention cold, by convention color; but in reality atoms and void.%\autocite{p.46}{Helmholtz1857}
	\stopalignment
	\startalignment[flushright]
	%{\it On the Physiological Causes\\
	%	of Harmony in Music}\\
	{\sc Democritus}\\
	c.460 -- c.370 \scaps{BCE}
	\stopalignment
}

%%%%%%%%%%%%%%%%%%%%%%%%%%%%%

\Initial{D}{emocritus speculated} that everything is made of small, indivisible units he called \quotation{atoms.} 
He was correct, but the project of identifying and classifying these indivisible units got off to a slow start. 


\section{The kind of motion we call heat}


\section{Boltzmann: On the Nature of Gas Molecules}
It is an interesting fact that thermal energy tends to spread out equally into every accessible place. The average amount of energy stored in each place is proportional to the temperature, and is given by the formula \m{½\,\kB T}. Boltzmann's constant, kB, is the same in all situations. It is defined as exactly.
\startformula
	\kB = 1.380649\sci{-23}\units{ J/K}
\stopformula
You should think of \m{\kB } as a conversion from kelvins to joules. Kelvins are just another unit for measuring average energy. A kelvin is a very tiny amount of energy!

The molecule in the cylinder has three places for energy, \m{K_x = ½ mv_x^2}, \m{K_y = ½ mv_y^2}, and \m{K_z = ½ mv_z^2}. If the  temperature of the gas is \m{T}, then the molecule with have \m{½\,\kB T} of energy in each of these places. Add more molecules and there are more places for the energy, three places per molecule. Even though \m{½\,\kB T} is a very small amount of energy, a large number of molecules can store a significant amount of energy. Be careful with hot things!

7.	What is \m{PV} if there are \m{N} molecules with an average vertical kinetic energy \m{K_y = ½\,\kB T} per molecule? \m{T} is the temperature of the gas.
The pressure will be \m{N} times greater, so the new \m{PV} is
\startformula
PV = 2NKy = NkBT
\stopformula

8.	What is \m{PV} if there are \m{n} moles of gas in the cylinder? Use the definition \m{R=\NA\kB} where \m{\NA} is Avogadro’s number and \m{R} is the gas constant, both of which you remember fondly from chemistry. Also recall \m{N = n\NA}. Your result should look familiar.
\startformula
PV = nNAkBT = nRT
\stopformula
\startformula
PV = nRT
\stopformula

9.	At temperature \m{T}, what is the total energy in the gas. Assume that the molecules' kinetic energy is the only energy in the gas. (For simple, monoatomic gasses this is an excellent assumption.) Call the total energy \m{U}, even though it is actually kinetic energy. Everyone uses \m{U} for the internal energy of the gas.
The total energy in the gas is \m{N} times the Kinetic energy of a single atom
\startformula
U = NK = N(K_x + K_y + K_z) = N(½\,\kB T + ½\,\kB T + ½\,\kB T ) = (3/2)N\kB T = (3/2)RT,
\stopformula

n this activity you will use Boltzmann's insight to find the amount of energy required to heat 63.55 g of copper from 20°C to 100°C. (Why 63.55 g? Because working with four significant figures builds character.)
Boltzmann tells us that the energy in every available spot is ½kBT, where kB = 1.381×10-23 J/K. You need to find out how many spots there are in 63.55 g of copper.	
The number of spots can be found with a simple model of copper. Imagine that the copper atoms are arranged is a simple cubic lattice, and that each atom is connected to its immediate neighbors by springs. Every atom is connected to the atom on its left and its right, in front and in back, above and below. The atoms can bounce around a bit, but they are always pushed back towards their home position by the springs.

1.	Some of the thermal energy is stored in the bouncing kinetic energy of the copper atoms. There are three types of kinetic energy for each atom. List the three symbols for these.




2.	For a lattice containing a large number of copper atoms, how many springs are there per atom? (Hint: There are not six springs per atom!) Explain, with a picture if you like.



3.	How many thermal energy spots are there per atom?



4.	Find the thermal energy per atom for copper at room temperature (20°C = 293 K).



5.	Find the thermal energy per atom for copper at 100°C = 373 K.
6.	How much heat must be added to 63.55 g of copper to heat it from room temperature to 100°C.











7.	What is the specific heat capacity of copper in  J/g °C? (For comparison, the specific heat capacity of water is 4.2 J/g °C.)


\startblockquote
For all simple solids with the exception of carbon, [boron], and silicon the product \m{M\Gamma} is not very different from 6; it is between 5.22 and 6.9. The values obtained for \m{h} are between 1.78 and 2.34. On average, then, the total heat supplied is twice as great as that to increase the mean living force, half of which is used for work, if the forces acting on an atom are proportional to the removal of it from its position of rest. For solids, this is likely to be the case with some approximation. If one wanted to use the latter formula, one could insert the theoretical value \m{Ap /\rho T} for \m{\gamma'-\gamma}. The result is a similar one. 
From the composite solid bodies, which obey Neumann's law, it must be assumed that each of its atoms really has three kinds of mobility.
For simple or compound bodies, however, which deviate significantly from Dulong-Petit or Neumann's laws, it could perhaps be assumed that two or more atoms are so firmly connected that the number of types of mobility of the system they form is smaller than the three-fold number of its atoms. [p.108-9]
\stopblockquote

%\subject{Notes}
%\placefootnotes[criterium=chapter]
\placenotes[endnote][criterium=chapter]

%\subject{Bibliography}
%        \placelistofpublications

\stopchapter
\stopcomponent
%%%%%%%%%%%%%%%%%%%%%%%%%%%%%%%%%%%%%%%%%%%%%%%%%%%
%%%%%%%%%%%%%%%%%%%%%%%%%%%%%%%%%%%%%%%%%%%%%%%%%%%

%$6.241509\sci{18}$ electrons is \emph{negative} one Coulomb. One Coulomb of charge from a one volt battery gives one joule of energy. How many electron volts are in one joule? Converting the other way, one electron volt is how many joules? How many electron volts of energy would be produced by one mole of electrons going through a potential difference on one volt. How many joules? Calories too?


% Templates:

% Margin image
\placefigure[margin][] % Location, Label
{} % Caption
{\externalfigure[chapter03/][width=144pt]} % File

% Margin Figure
\startbuffer[TikZ:NAME]
\environment env_physics
\environment env_TikZ
\setupbodyfont [libertinus,11pt]
\setoldstyle % Old style numerals in text
\startTEXpage\small
\starttikzpicture% tikz code
\stoptikzpicture
\stopTEXpage
\stopbuffer

\placefigure[margin][fig:NAME] % Location, Label
{}	 % caption text
{\noindent\typesetbuffer[TikZ:NAME]}

% Aligned equation
\startformula\startmathalignment
\stopmathalignment\stopformula

% Aligned Equations
\startformula\startmathalignment[m=2,distance=2em]
\stopmathalignment\stopformula
