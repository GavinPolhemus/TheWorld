% !TEX useOldSyncParser
\startcomponent c_chapter01
\project project_world
\product prd_volume02

\setupsynctex[state=start,method=max] % "method=max" or "min"

%%%%%%%%%%%%%%%%%%%%%%%%%%%%%
\startchapter[title={Energy Quanta}, reference=ch:Planck]
%%%%%%%%%%%%%%%%%%%%%%%%%%%%%

%\placefigure[margin,none]{}{\small
%	\startalignment[flushleft]
%By convention sweet and by convention bitter, by convention hot, by convention cold, by convention color; but in reality atoms and void.%\autocite{p.46}{Helmholtz1857}
%	\stopalignment
%	\startalignment[flushright]
%	%{\it On the Physiological Causes\\
%	%	of Harmony in Music}\\
%	{\sc Democritus}\\
%	c.460 -- c.370 \scaps{BCE}
%	\stopalignment
%}

%%%%%%%%%%%%%%%%%%%%%%%%%%%%%

\Initial{B}{oltzmann's counting} of degrees of freedom was a great success. It explained the Delong-Petit law discovered fifty years earlier and provided the first explanation of the thermal energies of gases based on their atomic structure. The thermal energies predicted by Boltzmann's counting were not perfect matches with experimental results, but they were close enough to be useful for a large number of substances. 
However, there were some problems.

The most obvious problem with Boltzmann's argument was the inconsistent treatment of atomic bonds. The bonds in a solids are treated as springs that hold some potential energy, but the bond in a diatomic gas molecule is treated as rigid and unable to hold any energy. These two different treatments matched the experimental results for solids and gases, but there was no clear reason why the two should be treated differently.

The greatest experimental challenge for Boltzmann's counting came from the lightest, solid elements – carbon, boron, and silicon – whose thermal energies are far below the value \m{E\sth=3nRT} predicted by Boltzmann and the Dulong-Petit law. Diamond, which is pure carbon, had the lowest value by far, only 0.7nRT, as measured \quotation{not without difficulty or expense} by two Swiss Physicists in 1840. %Pais SitL p. 391
In the 1870s, diamond was found to be even more peculiar because its ability to hold heat depends strongly on temperature. In 1875, just before Boltzmann's important paper, Heinrich Weber showed that all of the light elements obey the Dulong-Petit law at high temperature, even though they do not at room temperature.

Boltzmann offered what seems like a reasonable explanation for some of these results: some bond are too stiff to vibrate, and therefore cannot hold any potential or kinetic energy. In solids, some of these bonds form at low temperature, reducing the number of degrees of freedom. Diatomic gasses, according to this explanation, contain a bond that is also too stiff to vibrate. This explanation fails, however, because Boltzmann had \emph{proven}, based on the rules of mechanics, that stiff bonds hold just as much energy as springy bonds. The vibrations of stiff bonds will have a smaller amplitude, but the vibrations will hold the same amount of energy. 

Maxwell identified yet another problem, this time having to do with the visible light given off by elements when zapped with an electrical spark. Each element and molecule produces light with frequencies specific to the substance. Light is produced by a vibrating charge, so Maxwell concluded that each atom must have electrically charged parts that can vibrate at specific frequencies to produce these characteristic frequencies of light. Each of these vibrating parts would represent another degree of freedom that should be counted in determining the thermal energy. These frequencies are very high, suggesting that they are very stiff, but they clearly are capable of holding energy if they can vibrate enough to produce visible light.

As I stated before, Boltzmann's method of counting produced quite good results for a wide range of substances, so it was reasonable to hope that these various exceptions will eventually be explained away with minor changes or with a better understanding of the atoms' and molecules' inner parts. That is not what happened. These exceptions were actually the earliest signs that the foundations of mechanics – originally laid out by Newton in the \booktitle{Principia} – would have to be ripped out and replaced with something new.

\section{The Ultraviolet Catastrophe}
Boltzmann's equipartition theorem also caused problems in the study of light.
Physicists of the late nineteenth century understood that light is a wave, just as sound is a wave. Monochromatic light contains a single color, just as a pure musical tone has a single pitch. The frequency of monochromatic light is perceived as color (if the frequency is in the visible range) just as the frequency of a pure tone is perceived as pitch (if the frequency is in the audible range). Visible light frequencies are much higher than audible sound frequencies.

Physicists could determine these high frequencies from the wavelength, which they measured using interference effects.
Monochromatic light’s frequency \m{f} is related to its wavelength \m{\lambda} by the usual wave relation.
\startformula
c=\frac{f\lambda}{1\units{cyc}}
\stopformula
where the speed of light \m{c} is much greater than the speed of sound.
\startformula
c = 299 792 458\units{m/s} ≈ 3.00\sci{8}\units{m/s}
\stopformula
These high frequency vibrations carry energy, and they should also obey Boltzmann's equipartition function. In a situation where the light is in thermal equilibrium with it surroundings – like in a glowing hot furnace – every individual frequency of vibration should contain \m{\onehalf \kB T} of thermal energy, just like the vibrations of atoms and molecules. However, unlike atoms and molecules, there are an infinite number of higher and higher frequencies of light, meaning that the electromagnetic field's thermal energy should be infinite!
This ridiculous prediction is called the \keyterm{ultraviolet catastrophe}. (Violet is the highest visible frequency, so ultraviolet is the even higher frequencies.) %While the failures of Boltzmann's counting for a few substances was treated as a minor annoyance by many, the ultraviolet catastrophe 

The experimental study of light's behavior at high temperature revealed a strong relationship between temperature and the frequencies of light which carry energy. When a black object, like a piece of iron, is heated to several hundred degrees, it begins to glow a dull, dark red. Red is the lowest frequency visible light. As the iron's temperature is increased, its glow become brighter, then more orange, and eventually a bright yellow as it gives off higher frequencies of light. The emitted light is not a single frequency, but rather a mixture of all frequencies up to a certain color. When iron first begins to glow, the only visible light that it produces is red. As it gets hotter it gives emits read and orange light, then red, orange, and yellow light, producing a bright, light yellow rather than a monochromatic lemon yellow.

\section{The quantum hypothesis}
Planck explained this temperature dependence by proposing that glowing objects can only release light in small, specific amounts. Einstein took this argument further and said that the energy in light can only exist is these small amounts, which were eventually given the cute name \keyterm{photons}, for particles of light. The amount of energy in one of these photons is directly proportional to the frequency.
\startformula
	E = hf,
\stopformula
where \m{h} is a constant, \m{f} is the frequency, and \m{E} is the energy of the individual photons. Lower frequency red light comes in photons with less energy than higher frequency blue light. This explains why higher frequencies can only be produced at higher temperatures. Boltzmann's equipartition function says that the energy available to each frequency is \m{\onehalf\kB T}. For very high frequencies, where \m{hf\gg\onehalf\kB T}, there is not enough thermal energy to produce even one photon, so even hot iron cannot produce the highest frequency blue photons. Hot iron does produce abundant photons of lower frequencies, with \m{hf\ll\onehalf\kB T}, including yellow, orange, and red photons. As the iron cools, \m{\onehalf\kB T} decreases the frequencies of photons that can be produced also decreases.

The constant \m{h} is \keyterm{Planck's constant}.
\startformula
	h = 6.63\sci{-31}\units{J/Hz} = 4.14\sci{-15}\units{eV/Hz}
\stopformula
The value of \m{h} is quite tiny. Even though visible photons have very high frequencies, they each carry only a tiny amount of energy, comparable to the tiny energy \m{\onehalf\kB T} predicted by Boltzmann's equipartition function.

This model prevents the ultraviolet catastrophe by explaining why no energy goes into the vast range of higher frequencies. It also solves the earlier problems with Boltzmann's predictions for the heat capacities of solids and gasses. Einstein proposed that the all vibrations can hold energy only in discreet amounts equal to \m{hf}. For light one of these discrete pieces is called a photon, but in more general cases is is called a \keyterm{quantum} of every (plura: \keyterm{quanta}). The stiff bonds connecting the atoms in a diatomic gas vibrate at very high frequency. At room temperature, there is not enough energy available to each degree of freedom to produce even one quantum of vibrational energy in these stiff bonds, so the stiff bond holds no energy. In solids made of light elements, the atoms vibrate with a high frequency due to their small mass. Again, at room temperature the the energy available to each degree of freedom is not enough to create a single quantum of energy, so these vibrations do not carry any energy. As the temperature goes up, the energy quanta can be produces and the total thermal energy approaches the value predicted by Boltzmann and the Delong-Petit law, just as Weber observed.

This quantum hypothesis has no foundation in mechanics. Planck's proposal, offered in 1900, represented something totally new for the twentieth century. It introduces a new constant of nature, Planck's constant \m{h}, which relates frequency and energy in a new way. Since Boltzmann had already shown the relationship between temperature and energy, the quantum hypothesis extended this relationship to connect temperature and frequency. Low temperatures can only excite low frequencies, while higher temperatures excite higher frequencies. We see this connection in light, with hotter objects emitting higher frequency visible photons. The sun, which is quite hot, produces photons across the entire visible spectrum.

The quantum hypothesis puzzled physicists for decades, but it also provided a great synthesis. With the connection between frequency an energy, many microscopic properties of atom and molecules began to make sense.

%\placefigure[margin][fig:NewcomenAtmospheric]{Newcomen engine from \booktitle{Practical physics for secondary schools. Fundamental principles and applications to daily life,} by Newton Henry Black and Harvey Nathaniel Davis, publ. 1913 by Macmillan and Company, p.~219} {\externalfigure[chapter11/NewcomenAtmospheric][width=\rightmarginwidth]}
%
%%%%%%%%%%%%%%%%%%%%%%%%%%%%%%%%%%%%%%%%%%%%%%%%%%%%
%\startexample[ex:NewcomenForce]
%A working replica of Newcomen's engine operates at a museum near the site of Newcomen's original. The piston's radius is \m{26\units{cm}}. What is the force exerted on the piston by the atmosphere?
%
%\startsolution
%The piston's area is \m{A= \pi r^2 = \pi(0.26m)^2 = 0.212\units{m^2}}. The force is
%\startformula\startmathalignment
%\NC F	\NC = PA												\NR
%\NC		\NC = (1.01\sci{5}\units{Pa})(0.212\units{m^2})				\NR
%\NC		\NC = (1.01\sci{5}\units{N/\ucan{m^2}})(0.212\,\ucan{m^2})	\NR
%\NC		\NC = 21\,000\units{N}.
%\stopmathalignment\stopformula
%This downward force acting on the right side of \in{figure}[fig:NewcomenAtmospheric] is enough to lift a weight of over \m{2000\units{kg}} on the left.
%\stopsolution
%\stopexample
%%%%%%%%%%%%%%%%%%%%%%%%%%%%%%%%%%%%%%%%%%%%%%%%%%%%
%
%\startbuffer[TikZ:NewcomenPower]
%\environment env_physics
%\environment env_TikZ
%\setupbodyfont [libertinus,11pt]
%\setoldstyle \small% Old style numerals in text
%\startTEXpage
%\starttikzpicture% tikz code
%% left cylinder
%\draw[thick,fill=black!30] (-2.5,7.5)-- ++(0,-7.5) -- ++(2,0) -- ++(0,7.5) -- ++(-.1,0) -- ++(0,-7.4) -- ++(-1.8,0) -- ++(0,7.4) -- cycle;
%\fill[black!30] (-1.7,-0.5) rectangle ++(0.4,0.5);
%\draw[thick,] (-1.7,-0.5)-- ++(0,0.5) -- ++(0.4,0) -- ++(0,-0.5);
%\node at (-1.5,3.5){\m{V\si}};
%% right cylinder
%\draw[thick,fill=black!30] (0.5,7.5)-- ++(0,-7.5) -- ++(2,0) -- ++(0,7.5) -- ++(-.1,0) -- ++(0,-7.4) -- ++(-1.8,0) -- ++(0,7.4) -- cycle;
%\fill[black!30] (1.3,-0.5) rectangle ++(0.4,0.5);
%\draw[thick,] (1.3,-0.5)-- ++(0,0.5) -- ++(0.4,0) -- ++(0,-0.5);
%\node at (1.5,0.5){\m{V\sf}};
%% left piston
%\draw[thick,fill=black!30] (-2.4,7) rectangle ++(1.8,0.2);
%\fill[black!30] (-1.6,8.2) rectangle ++(0.2,-1);
%\draw[thick,] (-1.6,8.2)-- ++(0,-1) -- ++(0.2,0) -- ++(0,1);
%\node at (-1,7.2) [single arrow, fill=black!25, single arrow head extend=.1cm, anchor=tip, shape border rotate=270]{\m{P}};
%\node at (-2,7.2) [single arrow, fill=black!25, single arrow head extend=.1cm, anchor=tip, shape border rotate=270]{\m{P}};
%% right piston
%\draw[thick,fill=black!30] (0.6,1) rectangle ++(1.8,.2);
%\fill[black!30] (1.4,8.2) rectangle ++(0.2,-7);
%\draw[thick,] (1.4,8.2)-- ++(0,-7) -- ++(0.2,0) -- ++(0,7);
%\node at (1,1.2) [single arrow, fill=black!25, single arrow head extend=.1cm, anchor=tip, shape border rotate=270]{\m{P}};
%\node at (2,1.2) [single arrow, fill=black!25, single arrow head extend=.1cm, anchor=tip, shape border rotate=270]{\m{P}};
%% y coordinate axis
%\draw[thick,->] (0,0) -- (0,7.9)node[above]{\m{y}};
%\draw[] (.45,1) -- ++(-0.5,0)node[left]{\m{y\sf}};
%\draw[] (-.45,7) -- ++(0.5,0)node[right]{\m{y\si}};
%\draw[] (.05,0.1) -- ++(-0.1,0)node[left]{\m{0}};
%\stoptikzpicture
%\stopTEXpage
%\stopbuffer
%
%\placefigure[margin][fig:NewcomenPower] % location, label
%{The downward power stroke of Newcomen's atmospheric engine is driven by atmospheric pressure pushing down on the piston.} % caption text
%{\noindent\typesetbuffer[TikZ:NewcomenPower]} % figure contents
%




%\subject{Notes}
%\placefootnotes[criterium=chapter]
\placenotes[endnote][criterium=chapter]

%\subject{Bibliography}
%        \placelistofpublications

\stopchapter
\stopcomponent
%%%%%%%%%%%%%%%%%%%%%%%%%%%%%%%%%%%%%%%%%%%%%%%%%%%
%%%%%%%%%%%%%%%%%%%%%%%%%%%%%%%%%%%%%%%%%%%%%%%%%%%

%$6.241509\sci{18}$ electrons is \emph{negative} one Coulomb. One Coulomb of charge from a one volt battery gives one joule of energy. How many electron volts are in one joule? Converting the other way, one electron volt is how many joules? How many electron volts of energy would be produced by one mole of electrons going through a potential difference on one volt. How many joules? Calories too?


% Templates:

% Margin image
\placefigure[margin][] % Location, Label
{} % Caption
{\externalfigure[chapter03/][width=144pt]} % File

% Margin Figure
\startbuffer[TikZ:NAME]
\environment env_physics
\environment env_TikZ
\setupbodyfont [libertinus,11pt]
\setoldstyle % Old style numerals in text
\startTEXpage\small
\starttikzpicture% tikz code
\stoptikzpicture
\stopTEXpage
\stopbuffer

\placefigure[margin][fig:NAME] % Location, Label
{}	 % caption text
{\noindent\typesetbuffer[TikZ:NAME]}

% Aligned equation
\startformula\startmathalignment
\stopmathalignment\stopformula

% Aligned Equations
\startformula\startmathalignment[m=2,distance=2em]
\stopmathalignment\stopformula
