%D \module
%D   [      file=simpleslides-s-Swoosh,
%D        version=2009.03.30
%D          title=\CONTEXT\ Style File,
%D       subtitle=Presentation Module -- Swoosh Style,
%D         author=Aditya Mahajan and Thomas A. Schmitz,
%D           date=\currentdate,
%D      copyright={Aditya Mahajan and Thomas A. Schmitz}]
%C
%C Copyright 2009 Aditya Mahajan and Thomas A. Schmitz
%C This file may be distributed under the GNU General Public License v. 2.0.

%D This file provides the \quotation{Swoosh} style for the presentation
%D module. It is loaded at runtime.

\writestatus{simpleslides}{loading Swoosh style}

\startmodule[simpleslides-s-Swoosh]

\unprotect

%D The "counter" parameter controls which kind of counter will be used:

\startsetups simpleslides:setup:randomizeyes
\startuseMPgraphic{simpleslides:MP:randomize}
if condition = 1:
  qw := qw shifted (-5pt - c*0.1*uniformdeviate(10), 0) ; 
elseif condition = 2:
  qw := qw shifted (0, c*0.1*uniformdeviate(10)) ; 
fi ;
\stopuseMPgraphic
\stopsetups

\startsetups simpleslides:setups:dot
\setuplayout [simpleslides:layout:horizontal][header=3cm,footer=0cm]
\startuseMPgraphic{simpleslides:MP:counter}
if NOfPages > 1:
  numeric prog ; prog = PageNumber/NOfPages - 0.005 ;
fi ;
if PageNumber > 1:
  if condition = 1:
    z11 = (prog * PaperWidth, PaperHeight) ;
    z12 = (x11, 0) ;
    z13 = (z11 -- z12) intersectionpoint qw ;
    draw z13 withcolor \MPcolor{simpleslides:altcontrastcolor} ;
  elseif condition = 2:
    z14 = (0, (PaperHeight - prog * PaperHeight)) ;
    z15 = (PaperWidth, y14) ;
    z16 = (z14 -- z15) intersectionpoint qw ;
    draw z16 withcolor \MPcolor{simpleslides:altcontrastcolor} ;
  fi ;
fi ;
\stopuseMPgraphic
\stopsetups

\startsetups simpleslides:setups:circle
\setuplayout [simpleslides:layout:horizontal][header=3cm]
\startuseMPgraphic{simpleslides:MP:counter}
save b, s, t, p, circcenter, theta, pic ;
path p[] ;
pair t[] ;
pair s[] ;
pair circcenter ; circcenter = lrcorner Page shifted (-1cm, 1cm) ;
pair zt, zr, zb, zl ;
picture pic ;
b = 1.5cm ;
if PageNumber > 1:
	theta = (PageNumber - 1)/(NOfPages - 1) ;
 	p[4] = fullcircle scaled b rotated 90 ;
	p[4] := p[4] shifted circcenter ;
 	fill p[4] withcolor \MPcolor{simpleslides:altcontrastcolor} ;
 	t[0] = center p[4] ;
 	t[1] = point 1 along p[4] ;
 	t[2] = point -theta along p[4] ;
 	t[3] = point -theta/2 along p[4] ;
 	p[5] = t[0] -- t[1] .. t[3] .. t[2] -- cycle ;
 	fill p[5] withcolor \MPcolor{simpleslides:contrastcolor} ;
 	for i = 1 upto NOfPages :
 		s[i] = point i/(NOfPages -1) along p[4] ;
 		pickup pencircle scaled 1pt ;
 		draw s[i] -- t[0] withcolor \MPcolor{simpleslides:backgroundcolor} ;
 	endfor ;
 	zt = t[0] shifted (0, b * 0.2) ;
	zr = t[0] shifted (b * 0.2, 0) ;
	zb = t[0] shifted (0, -b * 0.2) ;
	zl = t[0] shifted (-b * 0.2, 0) ;
	p[3] = zt .. zr .. zb .. zl .. cycle ;
 	fill p[3] withcolor  \MPcolor{simpleslides:contrastcolor} ;
 	draw p[3] withcolor \MPcolor{simpleslides:backgroundcolor} ;
 	label(textext("\switchtobodyfont[10pt]\startcolor[simpleslides:backgroundcolor]\pagenumber\stopcolor"),center p[3]) ;
fi ;
\stopuseMPgraphic 
\stopsetups

\doifsetupselse{simpleslides:setups:\moduleparameter{simpleslides}{counter}}
    {\setups{simpleslides:setups:\moduleparameter{simpleslides}{counter}}}
    {\setups{simpleslides:setups:circle}}

\doif%
  {\moduleparameter{simpleslides}{randomize}}
  {yes}
  {\setups{simpleslides:setup:randomizeyes}}

%D First we change the page layout, adding more space on the top.

\setuplayout [width=fit,
              margin=0cm,
              height=fit,
              header=3cm, 
              footer=0.8cm, 
              topspace=.6cm, 
              backspace=1cm,
              location=singlesided]

\setuplayout [simpleslides:layout:vertical]  [header=0.8cm]
\setuplayout [simpleslides:layout:title]     [header=0.8cm]

%D We also specify the position of the slidetitle.

\setuplayer[simpleslides:layer:slidetitle]
    [x=10mm,y=2mm]

\setupcombinations[distance=1.85cm]

%D These macros are used for placing figures/pictures:

\define\NormalHeight        {\textheight}
\define\NormalWidth         {.45\textwidth}
\define\PictureFrameHeight  {\textheight}
\define\PictureFrameWidth   {.45\textwidth}

\defineframed[simpleslides:framed]
             [frame=off,offset=0pt,
              top=\vss,bottom=\vss]

%D We define our color scheme:

\definecolor [simpleslides:backgroundcolor]     [r=.88,g=.92,b=.95]
\definecolor [simpleslides:contrastcolor]	[r=.4,g=.6,b=.8]
\definecolor [simpleslides:altcontrastcolor]	[r=.1,g=.1,b=.4]
\definecolor [simpleslides:itemize:color]       [simpleslides:contrastcolor]
\definecolor [simpleslides:textcolor]  		[simpleslides:altcontrastcolor] 

\setupcolors[textcolor=simpleslides:textcolor]

%D We use \METAPOST\ to draw backgrounds.

\startuseMPgraphic{simpleslides:MP:horizontal} 
StartPage ;
fill Page withcolor \MPcolor{simpleslides:backgroundcolor} ;
numeric c ; c = PaperWidth * 0.4 ;
numeric d ; d = 2.2cm ;
numeric e ; e = PaperWidth / 46.9 ;
numeric condition ; condition = 1 ;
save ql, qr, qt, qw, qnw, qc ; path ql, qr, qt, qw, qnw, qc ;
ql = ulcorner Page -- llcorner Page ;
qr = lrcorner Page -- urcorner Page ;
qt = urcorner Page -- ulcorner Page ;
pickup pencircle scaled 4pt ;
z1 = ulcorner Page ;
z2 = (x1, y1 - d - 2*e) ;
z3 = (x1 + 4pt, y2) ;
z4 = (x3 + c, y2 + 2*e) ;
z5 = (x4 + c, y2) ;
z6 = (x5 + c, y2 + 2*e) ;
z7 = (x6 + c, y2) ;
z8 = (x7, y1) ;
qw = z2 .. z3 .. z4 .. z5 .. z6 .. z7 ;
\includeMPgraphic{simpleslides:MP:randomize} ;
z9 = ql intersectionpoint qw ;
z10 = qr intersectionpoint qw ;
qw := qw cutbefore z9 cutafter z10 ;
fill qr & qt & ql -- qw -- cycle withcolor \MPcolor{simpleslides:contrastcolor} ; 
draw qw withcolor white ;
\includeMPgraphic{simpleslides:MP:counter} ;
StopPage ;
\stopuseMPgraphic 

\startuseMPgraphic{simpleslides:MP:vertical} 
StartPage ;
fill Page withcolor \MPcolor{simpleslides:backgroundcolor} ;
numeric c ; c = PaperHeight * 0.4 ;
numeric d ; d = 2.2cm ;
numeric e ; e = PaperHeight / 46.9 ;
if NOfPages > 1:
  numeric prog ; prog = PageNumber/NOfPages - 0.005 ;
fi ;
numeric condition ; condition = 2 ;
save ql, qr, qt, qw, qc, qu, qb ; path ql, qr, qt, qw, qc, qu, qb ;
pickup pencircle scaled 4pt ;
z1 = ulcorner Page ;
z3 = center Page ;
z4 = (x3 - 2*e, y1) ;
z6 = (x3 + 2*e, y4 - c) ;
z7 = (x3 - 2*e, y6 - c) ;
z8 = (x3 + 2*e, y7 - c) ;
z9 = (x3 - 2*e, y8 - c) ;
qw = z4 .. z6 .. z7 .. z8 .. z9 ;
qt = ulcorner Page -- urcorner Page ;
ql = llcorner Page -- ulcorner Page ;
qb = llcorner Page -- lrcorner Page ;
\includeMPgraphic{simpleslides:MP:randomize} ;
z10 = qt intersectionpoint qw ;
qu = ulcorner Page -- z10 ;
z11 = qb intersectionpoint qw ;
qr = z11 -- llcorner Page ;
qw := qw cutbefore z10 cutafter z11 ;
fill qw -- qr & ql & qu -- cycle withcolor \MPcolor{simpleslides:contrastcolor} ;
draw qw withcolor white ;
\includeMPgraphic{simpleslides:MP:counter} ;
StopPage ;
\stopuseMPgraphic 

%D We define these backgrounds as overlays:

\defineoverlay 
  [simpleslides:background:title] 
  [\useMPgraphic{simpleslides:MP:horizontal}] 

\defineoverlay 
  [simpleslides:background:horizontal] 
  [\useMPgraphic{simpleslides:MP:horizontal}] 

\defineoverlay 
  [simpleslides:background:vertical] 
  [\useMPgraphic{simpleslides:MP:vertical}] 

\defineoverlay
  [simpleslides:background:ornament]
  [\useMPgraphic{simpleslides:MP:ornament}]

\setupTitle
  [\c!headcolor={simpleslides:altcontrastcolor}]

%D We want the title to placed in a framed box. We redefine all the keys of
%D \type{\setupTitle}, so that the module is easier to maintain.

\setupSlideTitle
  [\c!alternative=layer,
   \c!color=white,
   \c!align=\v!center,
   \c!width=\textwidth,
   \c!height=2cm,
   \c!after=]


\setupTitle
  [\c!title=,
   \c!author=,
   \c!date=\currentdate,
   \c!headstyle=,
   \c!headcolor={simpleslides:altcontrastcolor},
   \c!align=\v!middle,
   \c!before=\vfill,
   \c!after=\vfill,
   \c!title\c!style={\switchtobodyfont[\TitleSize]},
   \c!title\c!color=simpleslides:altcontrastcolor,
   \c!title\c!align=\v!middle,
   \c!author\c!style=,
   \c!author\c!color=simpleslides:altcontrastcolor,
   \c!author\c!align=\v!middle, 
   \c!date\c!style=,
   \c!date\c!color=simpleslides:altcontrastcolor,
   \c!date\c!align=\v!middle,
   \c!before\c!title=,
   \c!before\c!author=,
   \c!before\c!date=,
   \c!after\c!title={\blank[1*line]},
   \c!after\c!author={\blank[2*line]},
   \c!after\c!date=]

% 
% %D We want the title to be of a specific height
% 
% \setuphead[SlideTitle]
%    [\c!after=,
%     \c!alternative=\v!text,
%     \c!color=white,
%     \c!command=\doSlideTitle]
% 
% \define[2]\doSlideTitle
%   {\setlayer[simpleslides:layer:slidetitle]%
%      {\getvalue{simpleslides:framed}[\c!width=\textwidth,\c!height=1.1cm,
%                 \c!align=\v!right]
%                {#1#2}}} 

%D The symbol for the first level of itemizations. 

\definesymbol[1][\useMPgraphic{simpleslides:itemize:square}]
\setupitemize[1][color=simpleslides:itemize:color]

\protect
\stopmodule

\endinput

