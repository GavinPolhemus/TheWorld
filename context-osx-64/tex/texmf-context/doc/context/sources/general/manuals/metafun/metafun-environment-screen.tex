% language=us runpath=texruns:manuals/metafun
%
% copyright=pragma-ade readme=readme.pdf licence=cc-by-nc-sa

\startenvironment mfun-environment-screen

\enablemode[screen]

\setuppapersize
  [S6][S6]

\setuplayout
  [backspace=60pt,
   topspace=60pt,
   cutspace=0pt,
   header=0pt,
   footer=0pt,
   bottom=20pt,
   bottomdistance=40pt,
   top=20pt,
   topdistance=40pt,
   leftmargin=30pt,
   leftmargindistance=25pt,
   rightmargin=0pt,
   edge=0pt,
   width=middle,
   height=middle]

\setupbodyfont
  [9pt]

\setuppagenumbering
  [alternative=singlesided]

\setupinteraction
  [state=start,
   style=bold,
   color=darkred,
   contrastcolor=darkred,
   symbolset=navigation 3,
   menu=on]

\setupsymbolset
  [navigation 3]

\setupinteractionscreen
  [option=max]

\setupinteractionmenu
  [bottom]
  [unknownreference=yes,
   state=start]

\setuptoptexts
  []
  [{\lightgray \bf Page \pagenumber
    \doifcontent\quad{}{}\hbox{\getmarking[section]}}]

\startinteractionmenu[bottom]
    \txt \bf \lightgray \getmarking[chapter] \\
    \hfill
    \bgroup
    \setupinteraction[color=white,contrastcolor=white]
    \got [CloseDocument] exit \\
    \egroup
    \got [content]  content  \\
    \got [index]    index    \\
%   \got [commands] commands \\
    \got [reference] reference \\
    \setupinteraction[color=white,contrastcolor=white]
    \got [PreviousJump] \symbol[PreviousJump]  \\
    \got [previouspage] \symbol[previouspage]  \\
    \got [nextpage]     \symbol[nextpage]      \\
    \got [NextJump]     \symbol[NextJump]      \\
\stopinteractionmenu

\starttexdefinition unexpanded ChapterCommand #1#2
    \framed [
        background=titled,
        frame=off
    ] {
        #1
        \quad
        #2
    }
\stoptexdefinition

\startuseMPgraphic{PageFrame}
    StartPage ;
        save p, q, ranx, rany, minx, miny, maxx, maxy ;
        pickup pencircle scaled 4pt ;
        pair p[] ; path q[] ; numeric ranx, rany, minx, miny, maxx, maxy ;
        minx := BackSpace/2 ; maxx := PaperWidth -minx ; ranx := minx/2 ;
        miny := TopSpace /2 ; maxy := PaperHeight-miny ; rany := miny/2 ;
        p[0]  := llcorner Page ;
        p[1]  := (minx,0)           randomshifted (ranx,0) ;
        p[2]  := (maxx,0)           randomshifted (ranx,0) ;
        p[3]  := lrcorner Page ;
        p[4]  := (PaperWidth,miny)  randomshifted (0,rany) ;
        p[5]  := (PaperWidth,maxy)  randomshifted (0,rany) ;
        p[6]  := urcorner Page ;
        p[7]  := (maxx,PaperHeight) randomshifted (ranx,0) ;
        p[8]  := (minx,PaperHeight) randomshifted (ranx,0) ;
        p[9]  := ulcorner Page ;
        p[10] := (0,maxy)           randomshifted (0,rany) ;
        p[11] := (0,miny)           randomshifted (0,rany) ;
        def page_color = (.4+uniformdeviate.3)*white enddef ;
        fill Page  withcolor \MPcolor{lightgray} ;
        q[1] := p[9]--p[6]--p[ 5]--p[10]--cycle ;
        q[2] := p[6]--p[3]--p[ 2]--p[ 7]--cycle ;
        q[3] := p[3]--p[0]--p[11]--p[ 4]--cycle ;
        q[4] := p[0]--p[9]--p[ 8]--p[ 1]--cycle ;
        for i=1 upto 4: fill q[i] withcolor page_color ; endfor ;
        q[1] := p[9]--p[8]--((p[8]--p[ 1]) intersectionpoint (p[10]--p[ 5]))--p[10]--cycle ;
        q[2] := p[6]--p[5]--((p[5]--p[10]) intersectionpoint (p[ 2]--p[ 7]))--p[ 7]--cycle ;
        q[3] := p[3]--p[4]--((p[4]--p[11]) intersectionpoint (p[ 7]--p[ 2]))--p[ 2]--cycle ;
        q[4] := p[0]--p[1]--((p[1]--p[ 8]) intersectionpoint (p[ 4]--p[11]))--p[11]--cycle ;
        for i=1 upto 4: fill q[i] withcolor page_color ; endfor ;
        q[1] := p[ 8]--p[1] ;
        q[2] := p[ 7]--p[2] ;
        q[3] := p[10]--p[5] ;
        q[4] := p[11]--p[4] ;
        for i=1 upto 4: draw q[i] withcolor \MPcolor{darkred} ; endfor ;
    StopPage ;
\stopuseMPgraphic

% \setupbackgrounds[page][background=PageFrame]

\setupbackgrounds
  [page]
  [background={PageFrame,backgraphics,foreground,foregraphics}]

\defineoverlay[PageFrame][\useMPgraphic{PageFrame}]

\startMPinclusions
    background := \MPcolor{lightgray} ;
\stopMPinclusions

\stopenvironment
