% language=uk

\environment interaction-style

\startcomponent interaction-annotations

\startchapter[title=Annotations]

Before we discuss interactive features (in following chapter) a few words will be
spent on so called annotations. This term is used in Acrobat and is somewhat
confusing as hyperlinks conceptually are not really annotations while comments
are. The term relates to the way \PDF\ files can have added functionality. It
might help understand the following chapters better when you know what model is
used inside a \PDF.

If you open a \PDF\ file in an editor you will finds lots of objects. Numbers are
an object, as are strings and booleans. Symbols (represented as strings with a
leading slash) are also objects. Objects can be collected in indexed tables
(arrays) and hash tables (dictionaries). Serialized arrays are bounded by square
brackets:

\starttyping
[ (value1) (value2) ]
[ 1 2 3 ]
[ 1 2 (value1) (value3) true /foo ]
\stoptyping

and hashes by double angle brackets:

\starttyping
<<
    /Key1 (value1)
    /Key2 (value2)
    /Key3 123
    /Key4 true
    /Key5 [ 1 2 3 4 ]
>>
\stoptyping

A \PDF\ file is a collection of objects:

\starttyping
1 0 obj
    ...
endobj
\stoptyping

Instead of a number, string, boolean, array or dictionary an object can also be
a stream of bytes:

\starttyping
1 0 obj << /Length 123 >>
stream
... 123 bytes ...
endstream
endobj
\stoptyping

Objects can refer to each other and can be looked up via a so called xref table.
We refer to object with number one and revision zero as follows:

\starttyping
/foo 1 0 R
\stoptyping

When an object is updated it can be added to the end of the file and the version
number can get bumped. A program that does something with the \PDF\ is supposed
to do something clever with these numbers. More often the revision stays zero.

A document is normally a sequence of pages. When a file is opened the cross
reference table is loaded and the so called catalog is looked up. From there
pages can be found. Pages have a content stream and can refer to resources, like
fonts, special color spaces, complex objects (xforms) and among other things
annotations.

The page is rendered from the content stream but that stream has no information
about hyperlinks and such. The \type {/Link} annotation objects that implement
interactivity are independent and kind of layered over the rendered page. They
describe rectangular areas that a viewer can use to intercept mouse clicks. If
you want to see where the actions happens, search for \type {/Dest} and \type
{/Annot} in an (uncompressed) \PDF file. When you process a file with

\starttyping
\nopdfcompression
\stoptyping

you get an uncompressed file and with an appropriate editor you can see where
annotations end up.

The main reason for mentioning these details is that when you are checking a file
for problems you need to look for annotation objects instead of the page stream.
You also need to realize that these annotations in no way interfere with the page
stream. They only exist because a viewer can use that information to add
functionality. Their reference point is the lower left corner of the page. This
is a conceptual difference with \HTML\ where hyperlinks are often in|-|line and
part of the content stream. Both approaches have their advantages and
disadvantages. From the perspective of quality typesetting it makes much sense to
have them overlayed and explicitly defined (in terms of dimensions) but users
will of course in most cases define them inline. This means that in order to make
the \PDF\ some analyzing and juggling has to take place. In \CONTEXT\ we always
have done as much as possible at the \TEX\ (therefore bypassing some limitations
in the engine) end in \MKIV\ we don't use the engine's features at all.

\stopchapter

\stopcomponent
