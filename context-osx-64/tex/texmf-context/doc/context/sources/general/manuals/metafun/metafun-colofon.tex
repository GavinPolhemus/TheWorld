% language=us runpath=texruns:manuals/metafun
%
% copyright=pragma-ade readme=readme.pdf licence=cc-by-nc-sa

\startcomponent metafun-colofon

\environment metafun-environment

\introsubject{For them}

I owe much inspiration to both my parents. My mother Jannie constantly
demonstrated me that computer graphics will never improve nature. She also
converted one of my first \METAPOST\ graphics into a patchwork that will remind
me forever that handcraft is more vivid than computer artwork. My father Hein has
spent a great deal of his life teaching math, and I'm sure he would have loved
\METAPOST. I inherited his love for books. I therefore dedicate this document to
them.

\introsubject{Colofon}

This manual is typeset with \CONTEXT\ \MKIV. No special tricks are used and
everything you see in here, is available for \CONTEXT\ users. The text is typeset
in Palatino and Computer Modern Typewriter. We used \LUATEX\ as \TEX\ processing
engine. Since this document is meant to be printed in color, some examples will
look sub||optimal when printed in black and white.

\introsubject{Graphics}

The artist impression of one of Hasselts canals at \at {page} [canal] is made by
Johan Jonker. The \CDROM\ production process graphic at \at {page} [hacker] is a
scan of a graphic made by Hester de Weert.

\introsubject{Copyright}

\startlines
Hans Hagen, PRAGMA Advanced Document Engineering, Hasselt NL
copyright: 1999-\currentdate[year] / version 4: \currentdate
\stoplines

\introsubject{Publisher}

\startlines
publisher: Boekplan, NL
isbn-ean: 978-94-90688-02-8
website: www.boekplan.nl
\stoplines

\introsubject{Info}

\startlines
internet: www.pragma-ade.com
support: ntg-context@ntg.nl
context: www.contextgarden.net
\stoplines

\stopcomponent
