% language=us runpath=texruns:manuals/fonts

\startcomponent fonts-extensions

\environment fonts-environment

\startchapter[title=Extensions][color=darkorange]

\startsection[title=Introduction]

One of the benefits of using \TEX\ is that you can add your own features and try
to optimize the look and feel. Of course this can also go wrong and output can
look pretty awful when you don't know what you're doing, but on the average it
works out well. In many aspects the move to an \UNICODE\ data path and \OPENTYPE\
fonts is a good one and solves a lot of problems with traditional \TEX\ engines
and helps us to avoid complex and ugly hacks. But, if you look into the source
code of \CONTEXT\ you will notice that there's still quite some complex coding
needed. This is because we want to control mechanisms, even if it's only for
dealing with some border cases. It's also the reason why \LUATEX\ is what it is:
an extensible engine, building on tradition.

As always with \TEX, fonts are an area where many tuning happens and this is also
true in \CONTEXT. In this chapter some of the extensions will be discussed. Some
extensions run on top of the (rather generic) feature mechanism and some are
using dedicated code.

\stopsection

\startsection[title=Italics]

Although \OPENTYPE\ fonts are more rich in features than traditional \TEX\ and
\TYPEONE\ fonts, one important feature is missing: italic correction. This might
sound strange but you need to keep in mind that in practice it's a feature that
needs to be applied manually.

\starttyping
test {\it test\/} test
\stoptyping

It is possible to automate this mechanism and this is what the \type {\em} command
does in \MKII:

\starttyping
test {\em test} test
\stoptyping

This command knows that it switches to italic (or slanted) and when used nested it
knows to switch back. It also knows if a bold italic or slanted font is used. Therefore
it can add italic correction between an italic and upright shape.

An italic correction is bound to a glyph and bound to a font. In \in {figure}
[latinmodern-italic] we see how an italic shape extends out of the bounding box.
This is not the case in Dejavu: watch \in {figure} [dejavu-italic].

\startplacefigure[reference=latinmodern-italic,title={Italic overshoot in Latin Modern.}]
    \startcombination
        \startcontent
            \backgroundline[gray]{\color[maincolor]{\definedfont[lmroman10-regular*default sa 8]test}}
        \stopcontent
        \startcaption
            Latin Modern Roman Regular
        \stopcaption
        \startcontent
            \backgroundline[gray]{\color[maincolor]{\definedfont[lmroman10-italic*default sa 8]test}}
        \stopcontent
        \startcaption
            Latin Modern Roman Italic
        \stopcaption
    \stopcombination
\stopplacefigure

\startplacefigure[reference=dejavu-italic,title={Italic overshoot in Dejavu Serif.}]
    \startcombination
        \startcontent
            \backgroundline[gray]{\color[maincolor]{\definedfont[dejavuserif*default sa 8]test}}
        \stopcontent
        \startcaption
            Dejavu Regular
        \stopcaption
        \startcontent
            \backgroundline[gray]{\color[maincolor]{\definedfont[dejavuserifitalic*default sa 8]test}}
        \stopcontent
        \startcaption
            Dejavu Italic
        \stopcaption
    \stopcombination
\stopplacefigure

This means that the application of italic correction should never been applied without
knowing the font. In  \in {figure} [italic-upright] we see an upright word following
an italic. The space is determined by the upright one.

\startplacefigure[reference=italic-upright,title={Italic followed by upright.}]
    \startcombination
        \startcontent
            \backgroundline
                [gray]
                {\color[maincolor]{\definedfont[lmroman10-italic*default  sa 4]test}
                 \color[maincolor]{\definedfont[lmroman10-regular*default sa 4]\space test}}
        \stopcontent
        \startcaption
            Latin Modern
        \stopcaption
        \startcontent
            \backgroundline
                [gray]
                {\color[maincolor]{\definedfont[dejavuserifitalic*default sa 4]test}%
                 \color[maincolor]{\definedfont[dejavuserif*default       sa 4]\space test}}
        \stopcontent
        \startcaption
            Dejavu
        \stopcaption
    \stopcombination
\stopplacefigure

Because it is to be used with care you need to enable this feature per font, You
also need to explicitly enable the application of this correction. in \in {figure}
[italic-one] we see italic correction in action.

\startbuffer
\definefontfeature
  [italic]
  [default]
  [itlc=yes]
\stopbuffer

\typebuffer

\getbuffer

\startplacefigure[reference=italic-one,title={Italic correction.}]
    \startcombination
        \startcontent
            \backgroundline
                [maincolor]
                {\color[white]{\definedfont[lmroman10-italic*default  sa 4]test}
                 \color[white]{\definedfont[lmroman10-regular*default sa 4]\space test}}
        \stopcontent
        \startcaption
            \backgroundline
                [maincolor]
                {\setupitaliccorrection[text]%
                 \color[white]{\definedfont[lmroman10-italic*italic   sa 4]test}
                 \color[white]{\definedfont[lmroman10-regular*default sa 4]\space test}}
        \stopcaption
        \startcontent
            \backgroundline
                [maincolor]
                {\color[white]{\definedfont[dejavuserifitalic*default sa 4]test}
                 \color[white]{\definedfont[dejavuserif*default       sa 4]\space test}}
        \stopcontent
        \startcaption
            \backgroundline
                [maincolor]
                {\setupitaliccorrection[text]%
                 \color[white]{\definedfont[dejavuserifitalic*italic sa 4]test}
                 \color[white]{\definedfont[dejavuserif*default      sa 4]\space test}}
        \stopcaption
    \stopcombination
\stopplacefigure

This only signals the font constructor that additional italic information has
to be added to the font metrics. As we already mentioned, the application of
correction is driven by the \type {\/} primitive and that one consults the
font metrics. Because the correction is not part of the original font
metrics it is calculated automatically by adding a small value to the
width. This value is calculated as follows:

\starttyping
factor * (parameters.uwidth or 40) / 2
\stoptyping

The \type {uwidth} parameter is sometimes part of the specification but if not, we
take a reasonable default.  The factor is under user control:

\startbuffer
\definefontfeature
  [moreitalic]
  [default]
  [itlc=5]
\stopbuffer

\typebuffer

\getbuffer

This is demonstrated in \in {figure} [italic-two]. You will notice that for Latin
Modern (any) correction makes sense, but for Dejavu it probably makes things look
worse. This is why italic correction is disabled by default. When enabled there
are several variants:

\starttabulate[|Bl|l|]
\NC global \NC always apply correction \NC \NR
\NC text   \NC only apply correction to text \NC \NR
\NC always \NC apply correction between text and boxes \NC \NR
\NC none   \NC forget about correction \NC \NR
\stoptabulate

We keep track of the state using attributes but that comes at a (small) price in terms
of extra memory and runtime. The \type {global} option simply assumes that we always
need to check for correction (of course only for fonts that have this feature enables).
In the given example we used:

\starttyping
\setupitaliccorrection
  [text]
\stoptyping

You can combine keys:

\starttyping
\setupitaliccorrection
  [global,always]
\stoptyping

\startplacefigure[reference=italic-two,title={Italic correction (factor 5).}]
    \startcombination
        \startcontent
            \backgroundline
                [maincolor]
                {\color[white]{\definedfont[lmroman10-italic*default  sa 4]test}
                 \color[white]{\definedfont[lmroman10-regular*default sa 4]\space test}}
        \stopcontent
        \startcaption
            \backgroundline
                [maincolor]
                {\setupitaliccorrection[text]%
                 \color[white]{\definedfont[lmroman10-italic*italic   sa 4]test}
                 \color[white]{\definedfont[lmroman10-regular*default sa 4]\space test}}
        \stopcaption
        \startcontent
            \backgroundline
                [maincolor]
                {\color[white]{\definedfont[dejavuserifitalic*default sa 4]test}
                 \color[white]{\definedfont[dejavuserif*default       sa 4]\space test}}
        \stopcontent
        \startcaption
            \backgroundline
                [maincolor]
                {\setupitaliccorrection[text]%
                 \color[white]{\definedfont[dejavuserifitalic*italic  sa 4]test}
                 \color[white]{\definedfont[dejavuserif*default       sa 4]\space test}}
        \stopcaption
    \stopcombination
\stopplacefigure

The \type {itlc} feature controls if a font gets italic correction applied. In
principle this is all that the user needs to do, given that the mechanism is
enabled. These is an extra feature that controls the implementation:

\starttabulate[|T|T|p|]
\NC itlc        \NC no    \NC don't apply italic correction (default) \NC \NR
\NC             \NC yes   \NC apply italic correction \NC \NR
\NC textitalics \NC no    \NC precalculate italic corrections (permit engine usage) \NC \NR
\NC             \NC yes   \NC precalculate italic corrections (inhibit engine) \NC \NR
\NC             \NC delay \NC delay calculation of corrections \NC \NR
\stoptabulate

When \type {textitalics} is set to \type {yes} or \type {delay} the mechanism
built into the engine is completely disabled. When set to \type {no} the engine
can kick in but normally the alternative method takes precedence so that the
engine sees no reason for further action. You can trace italic corrections with:

\starttyping
\enabletrackers[typesetters.italics]
\stoptyping

\stopsection

\startsection[title=Bounding boxes]

\startbuffer
\definefontfeature
  [withbbox]
  [boundingbox=yes]

\definefont
  [FontWithBB]
  [Normal*withbbox]
\stopbuffer

\start \getbuffer \FontWithBB

There are some features that are rather useless and only make sense when figuring out
issues. An example of such a feature is the following:

\typebuffer

This feature adds a background to each character in a font. In some fonts a glyph
has a tight bounding box, while on other fonts some extra space is put on the left
and right. Keep in mind that this feature blocks colored text.

\par \stop

\stopsection

\startsection[title=Math italics]

In the traditional \TEX\ fonts the width of a glyph was not the real width because
one had to add the italic correction to it. The engine then juggles a bit with
these properties. If you run into fonts that are designed this way, you can do this:

\starttyping
\definefontfeature[mathextra][italicwidths=yes] % fix latin modern
\stoptyping

This might make \type {$\left|V\right| = \left|W\right|$} look better for such
fonts. Of course there can be side effects because these fonts assume a
traditional engine.

\stopsection

\startsection[title=Slanting]

This features (as well as the one described in the next section) are seldom used
but provided because they were introduced in \PDFTEX.

\startbuffer[define]
\definefontfeature
  [abitslanted]
  [default]
  [slant=.1]

\definefontfeature
  [abitmoreslanted]
  [default]
  [slant=.2]
\stopbuffer

\startbuffer[sample]
\definedfont[Normal*abitslanted]This is a bit slanted.
\definedfont[Normal*abitmoreslanted]And this is a bit more slanted.
\stopbuffer

\typebuffer[define,sample]

The result is:

\getbuffer[define]

\startlines
\getbuffer[sample]
\stoplines

\stopsection

\startsection[title=Extending]

The second manipulation is extending the shapes horizontally:

\startbuffer[define]
\definefontfeature
  [abitbolder]
  [default]
  [extend=1.3]

\definefontfeature
  [abitnarrower]
  [default]
  [extend=0.7]
\stopbuffer

\startbuffer[sample]
\definedfont[Normal*abitbolder]This looks a bit bolder.
\definedfont[Normal*abitnarrower]And this is a bit narrower.
\stopbuffer

\typebuffer[define,sample]

The result is:

\getbuffer[define]

\startlines
\getbuffer[sample]
\stoplines

We can also combine slanting and extending:

\startbuffer[define]
\definefontfeature
  [abitofboth]
  [default]
  [extend=1.3,
   slant=.1]
\stopbuffer

\startbuffer[sample]
\definedfont[Normal*abitofboth]This is a bit bolder but also slanted.
\stopbuffer

\typebuffer[define,sample]

If you remember those first needle matrix printers you might recognize the
next rendering:

\getbuffer[define]

\startlines
\getbuffer[sample]
\stoplines

\stopsection

\startsection[title=Fixing] % dimensions

This is a rather special one. First we show a couple of definitions:

\startbuffer
\definefontfeature
  [dimensions-a]
  [default]
  [dimensions={1,1,1}]

\definefontfeature
  [dimensions-b]
  [default]
  [dimensions={1,2,3}]

\definefontfeature
  [dimensions-c]
  [default]
  [dimensions={1,3,2}]

\definefontfeature
  [dimensions-d]
  [default]
  [dimensions={3,3,3}]
\stopbuffer

\typebuffer \getbuffer

When you don't want a dimension to change you leave an entry empty, so
valid entries are for instance: \type {,3,} and \type {1,,}.

As usual you apply such a feature as follows:

\starttyping
\definefont[MyFont][Serif*dimensions-a sa 2]
\stoptyping

Alternatively you can use such a feature on its own:

\starttyping
\definefontfeature
  [dimensions-333]
  [dimensions={3,3,3}]
\definefont[MyFont][Serif*default,dimensions-333 sa 2]
\stoptyping

In \in {figure} [dimensions-side-by-side] you see these four definitions in
action. The leftmost rendering is the default rendering. The three numbers in the
definitions represent the width (in em), height and depth (in ex).

\startplacefigure[reference={dimensions-side-by-side},title={Freezing dimensions of glyphs.}]
    \startcombination[5*1]
        \startcontent \hbox to 7em {\hss\ruledhbox{\definedfont[Serif*default      sa 2]g}\hss}\stopcontent \startcaption default     \stopcaption
        \startcontent \hbox to 7em {\hss\ruledhbox{\definedfont[Serif*dimensions-a sa 2]g}\hss}\stopcontent \startcaption \hbox{1em 1ex 1ex} \stopcaption
        \startcontent \hbox to 7em {\hss\ruledhbox{\definedfont[Serif*dimensions-b sa 2]g}\hss}\stopcontent \startcaption \hbox{1em 2ex 3ex} \stopcaption
        \startcontent \hbox to 7em {\hss\ruledhbox{\definedfont[Serif*dimensions-c sa 2]g}\hss}\stopcontent \startcaption \hbox{1em 3ex 2ex} \stopcaption
        \startcontent \hbox to 7em {\hss\ruledhbox{\definedfont[Serif*dimensions-d sa 2]g}\hss}\stopcontent \startcaption \hbox{3em 3ex 3ex} \stopcaption
    \stopcombination
\stopplacefigure

This feature only makes sense for fonts that need a fixed width, like the
\CJK\ fonts that are used for asian scripts. Normally those fonts already
have fixed dimensions, but this feature can be used to fix problematic
fonts or add some more space. However, for such large fonts this also brings a
larger memory footprint.

A special case is the following:

\startbuffer
\definefontfeature
  [dimensions-e]
  [dimensions=strut]
\stopbuffer

\typebuffer \getbuffer

This will make the height and depth the same as the {\em current} strut height
and depth:

\startbuffer
\ruledhbox{\definedfont[Serif*default,dimensions-e at  8pt]clipped}
\ruledhbox{\definedfont[Serif*default,dimensions-e at 12pt]clipped}
\ruledhbox{\definedfont[Serif*default,dimensions-e at 24pt]clipped}
\stopbuffer

\typebuffer

The dimensions are (in this case) limited:

\startlinecorrection[blank] \dontleavehmode \hpack{\maincolor\inlinebuffer} \stoplinecorrection

Another special case is \type {dimensions=mono} which will make an characters the
fonts em|-|width. This is handy when you define font fallbacks where glyphs come
from a non monospaced font.

\stopsection

\startsection[title=Unicoding]

Nowadays we will mostly use fonts that ship with a \UNICODE\ aware encoding. And
in \CONTEXT, even if we use a \TYPEONE\ font, it gets mapped onto \UNICODE.
However, there are some exceptions, for instance the Zapf Dingbats in \TYPEONE\
format. These have a rather obscure private encoding and the glyph names run from
\type {a1} upto \type {a206} and have no relation to what the glyph represents.

In the case of Dingbats we're somewhat lucky that they ended up in \UNICODE, so
we can relocate the glyphs to match their rightful place. This is done by means
of a goodies file. We already discussed this in \in {section} [goodies] so we
only repeat the usage.

\startbuffer
\definefontfeature
  [dingbats]
  [mode=base,
   goodies=dingbats,
   unicoding=yes]

\definefontsynonym
  [ZapfDingbats]
  [file:uzdr.afm]
  [features=dingbats]
\stopbuffer

\typebuffer \getbuffer

I tend to qualify the Dingbat font in \TEX\ distributions as rather unstable
because of name changes and them either or not being included. Therefore it's best to
use the hard coded name because that triggers the most visible error message when
the font is not found.

A font like this can for instance be used with the glyph placement macros as is
demonstrated below. In the last line we see that a direct \UTF\ input also works
out well.

\starttabulate[|||T|]
\HL
\NC \type{\getglyphdirect     {ZapfDingbats*dingbats}{\number"2701}} \NC \getglyphdirect     {ZapfDingbats*dingbats}{\number"2701} \NC \NC \NR
\NC \type{\getglyphdirect     {ZapfDingbats*dingbats}{\char"2701}}   \NC \getglyphdirect     {ZapfDingbats*dingbats}{\char"2701}   \NC \NC \NR
\NC \type{\getnamedglyphdirect{ZapfDingbats*dingbats}{a1}}           \NC \getnamedglyphdirect{ZapfDingbats*dingbats}{a1}           \NC \NC \NR
\NC \type{\getnamedglyphdirect{ZapfDingbats*dingbats}{a11}}          \NC \getnamedglyphdirect{ZapfDingbats*dingbats}{a11}          \NC \NC \NR
\HL
\NC \type{\getglyphdirect     {ZapfDingbats}{\number"2701}}          \NC \getglyphdirect     {ZapfDingbats}{\number"2701}          \NC unknown \NC \NR
\NC \type{\getglyphdirect     {ZapfDingbats}{\char"2701}}            \NC \getglyphdirect     {ZapfDingbats}{\char"2701}            \NC unknown \NC \NR
\NC \type{\getnamedglyphdirect{ZapfDingbats}{a1}}                    \NC \getnamedglyphdirect{ZapfDingbats}{a1}                    \NC \NC \NR
\NC \type{\getnamedglyphdirect{ZapfDingbats}{a11}}                   \NC \getnamedglyphdirect{ZapfDingbats}{a11}                   \NC \NC \NR
\HL
\NC \type{\definedfont[ZapfDingbats*dingbats]✁}                      \NC \definedfont[ZapfDingbats*dingbats]✁                      \NC \NC \NR
\HL
\stoptabulate

Keep in mind that fonts like Dejavu (that we use here as document font) already
has these characters which is why it shows up in the verbose part of the table.

\stopsection

\startsection[title=Protrusion]

Protrusion is a feature that \LUATEX\ inherits from \PDFTEX. It is sometimes
referred to as hanging punctuation but in our case any character qualifies. Also,
hanging is not frozen but can be tuned in detail. Currently the engine defines
protrusion in terms of the emwidth which is unfortunate and likely to change.
\footnote {In general the low level implementation can be optimized as there are
better mechanisms in \LUATEX.}

It is sometimes believed that protrusion improves for instance narrower columns,
but I'm pretty sure that this is not the case. It is true that it is taken into
account when breaking a paragraph into lines, and that we then have a little bit
more width available, but at the same time it is an extra constraint: if we
protrude we have to do it for each line (and the whole main body of text) so it's
just a different solution space. The main reason for applying this feature is
{\em not} that the lines look better or that we get better looking narrow lines
but that the right and left margins look nicer. Personally I don't like half
protrusion of punctuation and hyphens. Best is to have small values for regular
characters to improve the visual appearance and use full protrusion for hyphens
(and maybe punctuation).

\startsubsubject[title=protrusion classes]

In \CONTEXT\ we've always defined protrusion as a percentage of the width of a
glyph. From \MKII\ we inherit the level of control as well as the ability to
define vectors. The shared properties are collected in so called classes and the
character specific properties in vectors. The following classes are predefined:

\showprotrusionclass

The names are used in the definitions:

\starttyping
\definefontfeature[default][protrusion=quality]
\stoptyping

Currently adding a class only has a \LUA\ interface.

\startbuffer
\startluacode
fonts.protrusions.classes.myown = {
    vector = 'myown',
    factor = 1,
}
\stopluacode
\stopbuffer

\typebuffer \getbuffer

\stopsubsubject

\startsubsubject[title=protrusion vectors]

Vectors are larger but not as large as you might expect. Only a subset of
characters needs to be defined. This is because in practice only latin scripts
are candidates and these scripts have glyphs that look a lot like each other. As
we only operate on the horizontal direction characters like \quote
{aàáâãäå} look the same from the left and right so we only have to define
the protrusion for \quote {a}.

As with classes, you can define your own vectors:

\startbuffer
\startluacode
fonts.protrusions.vectors.myown = table.merged (
    fonts.protrusions.vectors.quality,
    {
        [0x002C] = { 0, 2 }, -- comma
    }
)
\stopluacode
\stopbuffer

\typebuffer \getbuffer

\stopsubsubject

\startsubsubject[title=protrusion vector pure]
    \showprotrusionvector[name=pure]
\stopsubsubject

\startsubsubject[title=protrusion vector punctuation]
    \showprotrusionvector[name=punctuation]
\stopsubsubject

\startsubsubject[title=protrusion vector alpha]
    \showprotrusionvector[name=alpha]
\stopsubsubject

\startsubsubject[title=protrusion vector quality]
    \showprotrusionvector[name=quality]
\stopsubsubject

\startsubsubject[title=examples of protrusion]

Next we show the quality protrusion. For this we use \type {tufte.tex} as
this one for sure will result in punctuation and other candidates for
protrusion.

\startbuffer[define]
\definefontfeature
  [whatever]
  [default]
  [protrusion=quality]

\definefont[MyTestA][Serif*default  at 10pt]
\definefont[MyTestB][Serif*whatever at 10pt]
\stopbuffer

\startbuffer[example]
\startoverlay
    {\ruledvbox \bgroup
        \hsize\textwidth
        \MyTestA
        \setupalign[normal]
        We thrive in information||thick worlds because of our
marvelous and everyday capacity to select, edit,
single out, structure, highlight, group, pair, merge,
harmonize, synthesize, focus, organize, condense,
reduce, boil down, choose, categorize, catalog, classify,
list, abstract, scan, look into, idealize, isolate,
discriminate, distinguish, screen, pigeonhole, pick over,
sort, integrate, blend, inspect, filter, lump, skip,
smooth, chunk, average, approximate, cluster, aggregate,
outline, summarize, itemize, review, dip into,
flip through, browse, glance into, leaf through, skim,
refine, enumerate, glean, synopsize, winnow the wheat
from the chaff and separate the sheep from the goats.

     \egroup}
    {\ruledvbox \bgroup
        \hsize\textwidth
        \MyTestB
        \setupalign[hanging,normal]
        \maincolor
        We thrive in information||thick worlds because of our
marvelous and everyday capacity to select, edit,
single out, structure, highlight, group, pair, merge,
harmonize, synthesize, focus, organize, condense,
reduce, boil down, choose, categorize, catalog, classify,
list, abstract, scan, look into, idealize, isolate,
discriminate, distinguish, screen, pigeonhole, pick over,
sort, integrate, blend, inspect, filter, lump, skip,
smooth, chunk, average, approximate, cluster, aggregate,
outline, summarize, itemize, review, dip into,
flip through, browse, glance into, leaf through, skim,
refine, enumerate, glean, synopsize, winnow the wheat
from the chaff and separate the sheep from the goats.

     \egroup}
\stopoverlay
\stopbuffer

\typebuffer[define]
\getbuffer [define]

We use the following example. The results are shown in \in {figure}
[protrusion:quality]. The colored text is the protruding one.

\typebuffer[example]

\startplacefigure[reference=protrusion:quality,title=The difference between no protrusion and quality protrusion.]
    \getbuffer [example]
\stopplacefigure

The previously defined own class and vector is somewhat more extreme:

\startbuffer[define]
\definefontfeature
  [whatever]
  [default]
  [protrusion=myown]

\definefont[MyTestA][Serif*default  at 10pt]
\definefont[MyTestB][Serif*whatever at 10pt]
\stopbuffer

\typebuffer[define]
\getbuffer [define]

In \in {figure} [protrusion:myown] we see that the somewhat extreem definition of
the comma also pulls the preceding character into the margin.

\startplacefigure[reference=protrusion:myown,title=The influence of extreme protrusion on preceding characters.]
    \getbuffer [example]
\stopplacefigure

\stopsubsubject

\stopsection

\startsection[title=Expansion]

Expansion is also an inheritance of \PDFTEX. \footnote {As with protrusion the
implementation in the engine is somewhat suboptimal and inefficient and will be
upgraded to a more \LUATEX-ish way.} This mechanism selectively expands
characters, normally upto 5\%. One reason for applying it is that we have less
visually incompatible spacing, especially when we have underfull or cramped
lines. For each (broken) line the badness is reconsidered with either shrink or
stretch applied to all characters in that line. So, in the worst case a shrunken
line is followed by a stretched one and that can be visible when the scaling
factors are chosen wrong.

As with protrusion, the solution space is larger but so are the constraints. But
contrary to protrusion here the look and feel of the whole line can be made
better but at the cost of much more runtime and larger (\PDF) files.

\startsubsubject[title=protrusion classes]

The amount of expansion depends in the shape of the character. Vertical strokes
are more sensitive for expansion then horizontal ones. So an \quote {o} can
get a different scaling than an \quote {m}. As with protrusion we have collected
the properties in classes:

\showexpansionclass

The smaller the step, the more instances of a font we get, the better it
looks, and the larger the files become. it's best not to use too many stretch
and shrink steps. A stretch of 2 and shrink of 2 and step of .25 results in
upto 8~instances plus the regular sized one.

\stopsubsubject

\startsubsubject[title=expansion vectors]

We only have one vector: \type {quality}:

\showexpansionvector[name=quality]

\stopsubsubject

\startsubsubject[title=an example of expansion]

We use \type {zapf.tex} as example text, if only because Hermann Zapf introduced
this optimization. Keep in mind that you can combine expansion and protrusion.

\startbuffer[define]
\definefontfeature
  [whatever]
  [default]
  [expansion=quality]

\definefont[MyTestA][Serif*default  at 10pt]
\definefont[MyTestB][Serif*whatever at 10pt]
\stopbuffer

\startbuffer[example]
\startoverlay
    {\ruledvbox \bgroup
        \hsize\textwidth
        \MyTestA
        \setupalign[normal]
        We thrive in information||thick worlds because of our
marvelous and everyday capacity to select, edit,
single out, structure, highlight, group, pair, merge,
harmonize, synthesize, focus, organize, condense,
reduce, boil down, choose, categorize, catalog, classify,
list, abstract, scan, look into, idealize, isolate,
discriminate, distinguish, screen, pigeonhole, pick over,
sort, integrate, blend, inspect, filter, lump, skip,
smooth, chunk, average, approximate, cluster, aggregate,
outline, summarize, itemize, review, dip into,
flip through, browse, glance into, leaf through, skim,
refine, enumerate, glean, synopsize, winnow the wheat
from the chaff and separate the sheep from the goats.

     \egroup}
    {\ruledvbox \bgroup
        \hsize\textwidth
        \MyTestB
        \setupalign[hz,normal]
        \maincolor
        We thrive in information||thick worlds because of our
marvelous and everyday capacity to select, edit,
single out, structure, highlight, group, pair, merge,
harmonize, synthesize, focus, organize, condense,
reduce, boil down, choose, categorize, catalog, classify,
list, abstract, scan, look into, idealize, isolate,
discriminate, distinguish, screen, pigeonhole, pick over,
sort, integrate, blend, inspect, filter, lump, skip,
smooth, chunk, average, approximate, cluster, aggregate,
outline, summarize, itemize, review, dip into,
flip through, browse, glance into, leaf through, skim,
refine, enumerate, glean, synopsize, winnow the wheat
from the chaff and separate the sheep from the goats.

     \egroup}
\stopoverlay
\stopbuffer

\typebuffer[define]
\getbuffer [define]

We use the following example. The results are shown in \in {figure}
[expansion:quality]. The colored text is the protruding one.

\typebuffer[example]

\startplacefigure[reference=expansion:quality,title=The difference between no expansion and quality expansion.]
    \getbuffer [example]
\stopplacefigure

You can see what happens in \in {figure} [expansion:visualized].

\startbuffer[example]
    \setupalign[hz]
    \enabletrackers[*expansion*]
    \definefontfeature[boundingbox][boundingbox={frame,empty}]
    \definedfont[Serif*default,quality,boundingbox @ 12.1pt]
    \samplefile{sapolsky}\par
    \disabletrackers[*expansion*]
\stopbuffer

\typebuffer[example]

\startplacefigure[reference=expansion:visualized,title=The injected expansion kerns.]
    \getbuffer [example]
\stopplacefigure

\stopsubsubject

\startsubsubject[title=Expansion and kerning]

When we expand glyphs we also need to look at the font kerns between them. In the
original implementation taken from \PDFTEX\ expansion was implemented using pseudo
fonts (with expanded glyph widths) and expansion of inter|-|character kerns was
based on font information. In \LUATEX\ we have expansion factors in glyph nodes
instead which is more efficient and gives a cleaner separation between front- and
backend as the backend has no need to consult the font.

For the font kerns we set the kern compensation directly and for that we use the
average expansion factors of the neighbouring fonts so technically we support
kerns between different fonts). This also has the advantage that kerns injected
in node mode are treated well, given that they are tagged as font kern.

So what is the effect (and need) of scaling font kerns? Let's look at an example.
Kerns can be positive but also negative:

\startlinecorrection
\startcombination
    {\vbox {
        \forgetall
        \hpack to 3cm{\hss\ruledhbox{\maincolor V\kern-1ptA}\hss}
        \hpack to 3cm{\hss\ruledhbox{\maincolor V\kern 0ptA}\hss}
    }} {negative}
    {\vbox {
        \forgetall
        \hpack to 3cm{\hss\ruledhbox{\maincolor I\kern.25ptI}\hss}
        \hpack to 3cm{\hss\ruledhbox{\maincolor I\kern  0ptI}\hss}
    }} {positive}
\stopcombination
\stoplinecorrection

If we use a rediculous amount of stretch we get the following. In the top line we
scale the kern, in the bottom line we don't.

\startlinecorrection
\startcombination
    {\vbox {
        \definedfont[file:texgyrepagella-regular.otf at 12pt]%
        \forgetall
        \hpack to 3cm{\maincolor \hss\strut \scale[xscale=5000]{V}\kern-5pt\scale[xscale=5000]{A}\hss}
        \hpack to 3cm{\maincolor \hss\strut \scale[xscale=5000]{V}\kern-1pt\scale[xscale=5000]{A}\hss}
    }} {negative}
    {\vbox {
        \definedfont[file:texgyrepagella-regular.otf at 12pt]%
        \forgetall
        \hpack to 3cm{\maincolor \hss\strut \scale[xscale=5000]{I}\kern1.25pt\scale[xscale=5000]{I}\hss}
        \hpack to 3cm{\maincolor \hss\strut \scale[xscale=5000]{I}\kern0.25pt\scale[xscale=5000]{I}\hss}
    }} {positive}
\stopcombination
\stoplinecorrection

The reason that we mention this is that when we apply \OPENTYPE\ features,
positioning not necessarily result in font kerns. For instance ligatures can be
the result of careful applied kerns and in some scripts kerns are used to connect
glyphs. This means that we best cannot expand kerns by default. How bad is
that? By looking at the examples above one would say \quotation {real bad}.

But say that we have about 1pt of font kerns, then a 5\% expansion (which is
already a lot) amounts to 0.05pt so to \blackrule [width=1pt, height=max,
depth=max] we add \blackrule [width=.05pt, height=max, depth=max] which is so
little that it probably goes unnoticed. Even if we use extreme kerns, as between
VA, in practice the small amount of stretch or shrink added to a font kern goes
unnoticed.

In \in {figure} [hz:natural] we have overlayed the different strategies. The
sample and width is chosen such that we see something. On a display you can
scale up these examples and inspect if there is really something to see,
but on paper zooming in helps, as in \in {figure} [hz:zoomed]. Even then the
effect of expanded kerns is invisible. The used definitions are:

\definecolor[hz:test:tr][r=1,a=1,t=.5]
\definecolor[hz:test:tg][g=1,a=1,t=.5]
\definecolor[hz:test:tb][b=1,a=1,t=.5]

\startbuffer
\setupfontexpansion
    [extremehz]
    [stretch=5,shrink=5,step=.5,vector=default,factor=1]
\setupfontexpansion
    [regularhz]
    [stretch=2,shrink=2,step=.5,vector=default,factor=1]
\setupfontexpansion
    [minimalhz]
    [stretch=2,shrink=2,step=.5,vector=default,factor=.5]

\definefontfeature
    [extremehz] [default]
    [mode=node,expansion=extremehz]
\definefontfeature
    [regularhz] [default]
    [mode=node,expansion=regularhz]
\definefontfeature [minimalhz] [default]
    [mode=node,expansion=minimalhz]

\definefont
    [ExtremeHzFont]
    [file:texgyrepagella-regular.otf*extremehz at 10pt]
\definefont
    [RegularHzFont]
    [file:texgyrepagella-regular.otf*regularhz at 10pt]
\definefont
    [MinimalHzFont]
    [file:texgyrepagella-regular.otf*minimalhz at 10pt]
\stopbuffer

\typebuffer \getbuffer

\edef\HzSampleText{\cldloadfile{ward}}

\def\NoHzSample  {\vbox{\hsize 10cm \color[hz:test:tr]{\setupalign  [nohz]\HzSampleText\par}}}
\def\HzSample    {\vbox{\hsize 10cm \color[hz:test:tg]{\setupalign    [hz]\HzSampleText\par}}}
\def\FullHzSample{\vbox{\hsize 10cm \color[hz:test:tb]{\setupalign[fullhz]\HzSampleText\par}}}

\startplacefigure[reference=hz:natural,title={The two expansion methods compared.}]
    \showfontkerns
    \dontcomplain
    \startcombination[1*3]
        {\ExtremeHzFont\ruledhpack{\startoverlay {\NoHzSample} {\HzSample    } \stopoverlay}} {no hz \& hz}
        {\ExtremeHzFont\ruledhpack{\startoverlay {\NoHzSample} {\FullHzSample} \stopoverlay}} {no hz \& full hz}
        {\ExtremeHzFont\ruledhpack{\startoverlay {\HzSample  } {\FullHzSample} \stopoverlay}} {hz \& full hz}
    \stopcombination
\stopplacefigure

\startplacefigure[reference=hz:zoomed,title={The two expansion methods compared (zoomed in).}]
    \showfontkerns
    \dontcomplain
    \startcombination[3*3]

        {\ExtremeHzFont
         \clip[nx=6,ny=5,x=2,y=2,sx=2]{\startoverlay {\NoHzSample} {\HzSample    } \stopoverlay}} {extreme: no hz \& hz}
        {\ExtremeHzFont
         \clip[nx=6,ny=5,x=2,y=2,sx=2]{\startoverlay {\NoHzSample} {\FullHzSample} \stopoverlay}} {extreme: no hz \& full hz}
        {\ExtremeHzFont
         \clip[nx=6,ny=5,x=2,y=2,sx=2]{\startoverlay {\HzSample  } {\FullHzSample} \stopoverlay}} {extreme: hz \& full hz}

        {\RegularHzFont
         \clip[nx=6,ny=5,x=2,y=2,sx=2]{\startoverlay {\NoHzSample} {\HzSample    } \stopoverlay}} {regular: no hz \& hz}
        {\RegularHzFont
         \clip[nx=6,ny=5,x=2,y=2,sx=2]{\startoverlay {\NoHzSample} {\FullHzSample} \stopoverlay}} {regular: no hz \& full hz}
        {\RegularHzFont
         \clip[nx=6,ny=5,x=2,y=2,sx=2]{\startoverlay {\HzSample  } {\FullHzSample} \stopoverlay}} {regular: hz \& full hz}

        {\MinimalHzFont
         \clip[nx=6,ny=5,x=2,y=2,sx=2]{\startoverlay {\NoHzSample} {\HzSample    } \stopoverlay}} {minimal: no hz \& hz}
        {\MinimalHzFont
         \clip[nx=6,ny=5,x=2,y=2,sx=2]{\startoverlay {\NoHzSample} {\FullHzSample} \stopoverlay}} {minimal: no hz \& full hz}
        {\MinimalHzFont
         \clip[nx=6,ny=5,x=2,y=2,sx=2]{\startoverlay {\HzSample  } {\FullHzSample} \stopoverlay}} {minimal: hz \& full hz}

    \stopcombination
\stopplacefigure

In \CONTEXT\ the \type {hz} alignment option only enables expansion of glyphs,
while \type {fullhz} also applies it to kerns. It will be clear that you can just
stick to using the \type {hz} directive (if you want expansion at all) because
this directive is normally disabled and because most fonts are processed in node
mode.

\stopsubsubject

\stopsection

\startsection[title=Composing]

This feature is seldom needed but can come in handy for old fonts or when
some special language is to be supported. When writing this section I tested
this feature with Dejavu and only two additional characters were added:

\definefontfeature
  [default-plus-compose]
  [compose=yes]

\definefont
  [MyComposedSerif]
  [Serif*default-plus-compose]

% we need to cheat a bit as we don't have the main character in mono

\startlines \MyComposedSerif
\type{fonts > combining > }\hbox to .5em{\hss Ѷ\hss}\type{ (U+00476) = }\hbox to .5em{\hss Ѵ\hss}\type{ (U+00474) + ̏ (U+0030F)}
\type{fonts > combining > }\hbox to .5em{\hss ѷ\hss}\type{ (U+00477) = }\hbox to .5em{\hss ѵ\hss}\type{ (U+00475) + ̏ (U+0030F)}
\stoplines

This trace showed up after giving:

\starttyping
\enabletrackers
  [fonts.composing.define]

\definefontfeature
  [default-plus-compose]
  [compose=yes]

\definefont
  [MyFont]
  [Serif*default-plus-compose]
\stoptyping

Fonts like Latin Modern have lots of glyphs but still lack some. Although the
composer can add some of the missing, some of those new virtual glyphs probably
will never look real good. For instance, putting additional accents on top of
already accented uppercase characters will fail when that character has a rather
tight (or even clipped) boundingbox in order not to spoil the lineheight. You can
get some more insight in the process by turning on tracing:

\starttyping
\enabletrackers[fonts.composing.visualize]
\stoptyping

One reason why composing can be suboptimal is that it uses the boundingbox of the
characters that are combined. If you really depend on a specific font and need
some of the missing characters it makes sense to spend some time on optimizing
the rendering. This can be done via the goodies mechanism. As an example we've
added \type {lm-compose-test.lfg} to the distribution. First we show how it
looks at the \TEX\ end:

\startbuffer
\enabletrackers[fonts.composing.visualize]

\definefontfeature
  [default-plus-compose]
  [compose=yes]

\loadfontgoodies
  [lm-compose-test] % playground

\definefont
  [MyComposedSerif]
  [file:lmroman10regular*default-plus-compose at 48pt]
\stopbuffer

\typebuffer \getbuffer

\blank
\backgroundline
  [halfcolor]
  {\MyComposedSerif B\quad\char"1E02\quad\char"1E04}
\blank

The positions of the dot accents on top and below the capital B is defined
in a goodie file:

\starttyping
return {
  name = "lm-compose-test",
  version = "1.00",
  comment = "Goodies that demonstrate composition.",
  author = "Hans and Mojca",
  copyright = "ConTeXt development team",
  compositions = {
    ["lmroman12-regular"] = compose,
  }
}
\stoptyping

As this is an experimental feature there are several ways to deal with
this. For instance:

\starttyping
local defaultfraction = 10.0

local compose = {
  dy       = defaultfraction,
  [0x1E02] = { -- B dot above
      dy = 150
  },
  [0x1E04] = { -- B dot below
      dy = 150
  },
}
\stoptyping

Here the fraction is relative to the difference between the height of the
accentee and the accent. A better solution is the following:

\starttyping
local compose = {
  [0x1E02] = { -- B dot above
    anchored = "top",
  },
  [0x1E04] = { -- B dot below
    anchored = "bottom",
  },
  [0x0042] = { -- B
    anchors = {
      top = {
        x = 300, y = 700,
      },
      bottom = {
        x = 300, y = -30,
      },
    },
  },
  [0x0307] = {
    anchors = {
      top = {
        x = -250, y =  550,
      },
    },
  },
  [0x0323] = {
    anchors = {
      bottom = {
        x = -250, y =  -80,
      },
    },
  },
}
\stoptyping

This approach is more or less the same as \OPENTYPE\ anchoring. It takes a bit
more effort to define these tables but the result is better.

\stopsection

\startsection[title=Kerning]

Inter|-|character kerning is not supported at the font level and with good
reason. The fact that something is conceptually possible doesn't mean that we
should use or support it. Normally proper kerning (or the lack of it) is part
of a font design and for some scripts different kerning is not even an option.

On the average \TEX\ does a proper job on justification but not all programs
are that capable. As a consequence designers (at least we ran into it) tend to
stick to flush left rendering because they don't trust their system to do a
proper job otherwise. On the other hand they seem to have no problem with
messing up the inter|-|character spacing and even combine that with excessive
inter|-|word spacing {\em if} they want to achieve justification (without
hyphenation). And it can become even worse when extreme glyph expansion (like
hz) is applied.

Anyhow, it will be clear that consider messing with properties like kerning that
are part of the font design is to be done careful.

\definecharacterkerning [extremekerning] [factor=.125]

\start \setcharacterkerning[extremekerning]

For running text additional kerning makes no sense. It not only looks
bad, it also spoils the grayness of a text. When it is applied we need
to deal with special cases. For instance ligatures make no sense so they
should be disabled. Additional kerning should relate to already present
kerning and interword spacing should be adapted accordingly. Embedded
non|-|characters also need to be treated well.

\par \stop

This paragraph was typeset as follows:

\starttyping
\definecharacterkerning [extremekerning] [factor=.125]

\setcharacterkerning[extremekerning] ... text ...
\stoptyping

Where additional kerning can make sense, is in titles. The previous
command can do that job. In addition we have a mechanism that
fills a given space. This mechanism uses the following definition:

\starttyping
\setupcharacterkerning
  [stretched]
  [factor=max,
   width=\availablehsize]
\stoptyping

\startbuffer
\stretched{\bfd to the limit}
\stopbuffer

\typebuffer

\blank \start \color[maincolor]{\getbuffer} \stop \blank

The following does not work:

\startbuffer
\ruledhbox to 5cm{\stretched{\bfd to the limit}}
\stopbuffer

\typebuffer

\blank \start \color[maincolor]{\getbuffer} \stop \blank

But this works ok:

\startbuffer
\setupcharacterkerning
  [stretched]
  [width=]

\stretched{\bfd to the limit}
\stopbuffer

\typebuffer

\blank \start \color[maincolor]{\getbuffer} \stop \blank

You can also say this:

\startbuffer
\stretched[width=]{\bfd to the limit}
\stopbuffer

\typebuffer

\blank \start \color[maincolor]{\getbuffer} \stop \blank

or:

\startbuffer
\ruledhbox{\stretched[width=10cm]{\bfd to the limit}}
\stopbuffer

\typebuffer

\blank \start \color[maincolor]{\getbuffer} \stop \blank

You can get some insight in what kerning does to your font by the following
command:

\startbuffer
\usemodule[typesetting-kerning]

\starttext
  \showcharacterkerningsteps
    [style=Bold,
     sample=how to violate a proper font design,
     text=rubish,
     first=0,
     last=45,
     step=5]
\stoptext
\stopbuffer

\typebuffer

\blank \getbuffer \blank

\stopsection

\startsection[title=Extra font kerns]

Fonts are processed independent of each other. Sometimes that is unfortunate for
kerning, although in practice it won't happen that often. We can enable an
additional kerning mechanism to deal with these cases. The \type
{\setextrafontkerns} command takes one argument between square brackets. The
effect can be seen below:

\startbuffer
    VA {\smallcaps va} V{\smallcaps a}
    VA {\bf VA} V{\bf A} {\bf V}A
    V{\it A}
\stopbuffer

\starttabulate[|Tl|l|p|]
\HL
\BC key \BC result \BC logic \NC \NR
\HL
\NC no kerns \NC \showfontkerns\setextrafontkerns[reset]\subff{f:kern}\inlinebuffer \NC no kerns at all \NC \NR
\NC kerns    \NC \showfontkerns\setextrafontkerns[reset]\inlinebuffer \NC kerns within a font (feature) run \NC \NR
\HL
\NC none     \NC \showfontkerns\setextrafontkerns [none]\inlinebuffer \NC only extra kerns within fonts \NC \NR
\NC min      \NC \showfontkerns\setextrafontkerns  [min]\inlinebuffer \NC minimal kerns within and across fonts \NC \NR
\NC max      \NC \showfontkerns\setextrafontkerns  [max]\inlinebuffer \NC maximum kerns within and across fonts \NC \NR
\NC mixed    \NC \showfontkerns\setextrafontkerns[mixed]\inlinebuffer \NC averaged kerns within and across fonts \NC \NR
\HL
\stoptabulate

The content is defined as:

\typebuffer

This mechanism obeys grouping so you have complete control over where and when
it gets applied. The \type {\showfontkerns} command can be used to trace the
injection of (font) kerns.

\stopsection

\startsection[title=Ligatures]

For some Latin fonts ligature building is quite advanced, take Unifraktur. I have no
problem admitting that I find fraktur hard to read, but this one actually is sort of
an exception. It's also a good candidate for a screen presentation where you mainly
made notes for yourself: no one has to read it, but it looks great, especially if
you consider it to be drawn by a pen.

Anyway, we will use the following code as example (based on some remarks on the
fonts website).

\startbuffer[sample]
sitzen / ſitzen / effe fietsen / ch ck ſt tz ſi fi
\stopbuffer

\typebuffer[sample]

Some ligatures are implemented in the usual way, using the \type {liga} and \type {dlig}
features, others kick in thanks to \type {ccmp}. This fact alone is an illustration that
the low level \OPENTYPE\ ligature feature is not related to ligatures at all but a more
generic mechanism: you can basically combine multiple shapes into one in all features
exposed to the user.

We define a bunch of specific feature sets:

\startbuffer
\definefontfeature
  [unifraktur-a]
  [default]
\definefontfeature
  [unifraktur-b]
  [default]
  [goodies=unifraktur,keepligatures=yes]
\definefontfeature
  [unifraktur-c]
  [default]
  [ccmp=yes]
\definefontfeature
  [unifraktur-d]
  [default]
  [ccmp=yes,goodies=unifraktur,keepligatures=yes]
\definefontfeature
  [unifraktur-e]
  [default]
  [liga=no,rlig=no,clig=no,dlig=no,ccmp=yes,keepligatures=auto]
\stopbuffer

\getbuffer \typebuffer

and also some fonts:

\startbuffer
\definefont[TestA][UnifrakturCook*unifraktur-a sa 0.9]
\definefont[TestB][UnifrakturCook*unifraktur-b sa 0.9]
\definefont[TestC][UnifrakturCook*unifraktur-c sa 0.9]
\definefont[TestD][UnifrakturCook*unifraktur-d sa 0.9]
\definefont[TestE][UnifrakturCook*unifraktur-e sa 0.9]
\stopbuffer

\getbuffer \typebuffer

We show these five alternatives here:

\starttabulate[|T||]
\NC liga                        \NC \TestA\getbuffer[sample] \NC \NR
\NC liga + keepligatures        \NC \TestB\getbuffer[sample] \NC \NR
\NC liga + ccmp                 \NC \TestC\getbuffer[sample] \NC \NR
\NC liga + ccmp + keepligatures \NC \TestD\getbuffer[sample] \NC \NR
\NC ccmp + keepligatures        \NC \TestE\getbuffer[sample] \NC \NR
\stoptabulate

The real fun starts when we want to add extra spacing between characters. Some
ligatures need to get broken and some kept.

\startbuffer
\setupcharacterkerning[kerncharacters][factor=0.5]
\setupcharacterkerning[letterspacing] [factor=0.5]
\stopbuffer

\getbuffer \typebuffer

\enabletrackers[typesetters.kerns.ligatures]

Next we will see how ligatures behave depending on how the mechanisms are set
up. The colors indicate what trickery is used:

\starttabulate[|T||]
\NC \color[darkred]  {red}   \NC kept by dynamic feature \NC \NR
\NC \color[darkgreen]{green} \NC kept by static feature  \NC \NR
\NC \color[darkblue] {blue}  \NC keep by goodie          \NC \NR
\stoptabulate

First we use \type {\kerncharacters}:

\starttabulate[|T||]
\NC liga                        \NC \kerncharacters {\TestA\getbuffer[sample]} \NC \NR
\NC liga + keepligatures        \NC \kerncharacters {\TestB\getbuffer[sample]} \NC \NR
\NC liga + ccmp                 \NC \kerncharacters {\TestC\getbuffer[sample]} \NC \NR
\NC liga + ccmp + keepligatures \NC \kerncharacters {\TestD\getbuffer[sample]} \NC \NR
\NC ccmp + keepligatures        \NC \kerncharacters {\TestE\getbuffer[sample]} \NC \NR
\stoptabulate

In the next example we use \type {\letterspacing}:

\starttabulate[|T||]
\NC liga                        \NC \letterspacing {\TestA\getbuffer[sample]} \NC \NR
\NC liga + keepligatures        \NC \letterspacing {\TestB\getbuffer[sample]} \NC \NR
\NC liga + ccmp                 \NC \letterspacing {\TestC\getbuffer[sample]} \NC \NR
\NC liga + ccmp + keepligatures \NC \letterspacing {\TestD\getbuffer[sample]} \NC \NR
\NC ccmp + keepligatures        \NC \letterspacing {\TestE\getbuffer[sample]} \NC \NR
\stoptabulate

\disabletrackers[typesetters.kerns.ligatures]

The difference is that the letterspacing variant dynamically adds the predefined
featureset \type {letterspacing} which is defined in a similar way as \type
{unifraktur-e}. In the case of this font, this variant is the better one to use.
In fact, this variant probably works okay with most fonts. However, by not hard
coding this behaviour we keep control, as one never knows what the demands are.
When no features are used, information from the (given) goodie file \type
{unifraktur.lfg} is consulted:

\starttyping
letterspacing = {
  -- watch it: zwnj's are used (in the tounicodes too)
  keptligatures = {
    ["c_afii301_k.ccmp"] = true, -- ck
    ["c_afii301_h.ccmp"] = true, -- ch
    ["t_afii301_z.ccmp"] = true, -- tz
    ["uniFB05"]          = true, -- ſt
  },
}
\stoptyping

These kick in when we don't disable ligatures by setting features (case~e).

There are two pseudo features that can help us out when a font doesn't provide
the wanted ligatures but has the right glyphs for building them. The \UNICODE\
database has some information about how characters can be (de)composed and we can
use that information to create virtual glyphs:

\starttyping
\definefontfeature
  [default] [default]
  [char-ligatures=yes,mode=node]
\stoptyping

and:

\starttyping
\definefontfeature
  [default] [default]
  [compat-ligatures=yes,mode=node]
\stoptyping

This feature was added after some discussion on the \CONTEXT\ mailing list about
the following use case.

\startbuffer
\definefontfeature
  [default-l] [default]
  [char-ligatures=yes,
   compat-ligatures=yes,
   mode=node]

\definefont[LigCd][cambria*default]
\definefont[LigPd][texgyrepagellaregular*default]
\definefont[LigCl][cambria*default-l]
\definefont[LigPl][texgyrepagellaregular*default-l]
\stopbuffer

\typebuffer \getbuffer

These definitions result in:

\starttabulate[|l|l|l|l|l|]
\NC                    \NC \type {\LigCd}     \NC \type {\LigPd}     \NC \type {\LigCl}     \NC \type {\LigPl}     \NC \NR
\NC \type{PEL·LÍCULES} \NC \LigCd PEL·LÍCULES \NC \LigPd PEL·LÍCULES \NC \LigCl PEL·LÍCULES \NC \LigPl PEL·LÍCULES \NC \NR
\NC \type{pel·lícules} \NC \LigCd pel·lícules \NC \LigPd pel·lícules \NC \LigCl pel·lícules \NC \LigPl pel·lícules \NC \NR
\NC \type{PEĿLÍCULES}  \NC \LigCd PEĿLÍCULES  \NC \LigPd PEĿLÍCULES  \NC \LigCl PEĿLÍCULES  \NC \LigPl PEĿLÍCULES  \NC \NR
\NC \type{peŀlícules}  \NC \LigCd peŀlícules  \NC \LigPd peŀlícules  \NC \LigCl peŀlícules  \NC \LigPl peŀlícules  \NC \NR
\stoptabulate

Of course one can wonder is this is the right approach and if it's not better to
use a font that provides the needed characters in the first place.

\stopsection

\startsection[title=New features]

\startsubsection[title=Substitution]

It is possible to add new features via \LUA. Here is an example of a single
substitution:

\startbuffer
\startluacode
    fonts.handlers.otf.addfeature {
        name = "stest",
        type = "substitution",
        data = {
            a = "X",
            b = "P",
        }
    }
\stopluacode
\stopbuffer

\typebuffer \getbuffer

We show an overview at the end of this section, but here is a simple example
already. You need to define the feature before defining a font because otherwise
the font will not know about it.

\startbuffer
\definefontfeature[stest][stest=yes]
\definedfont[file:dejavu-serifbold.ttf*default]
abracadabra: \addff{stest}abracadabra
\stopbuffer

\typebuffer \start \blank \maincolor \getbuffer \blank \stop

Instead of (more readable) glyph names you can also give \UNICODE\ numbers:

\starttyping
\startluacode
    fonts.handlers.otf.addfeature {
        name = "stest",
        type = "substitution",
        data = {
            [0x61] = 0x58
            [0x62] = 0x50
        }
    }
\stopluacode
\stoptyping

The definition is quite simple: we just map glyph names (or unicodes) onto
other ones. An alternate is also possible:

\startbuffer
\startluacode
    fonts.handlers.otf.addfeature {
        name = "atest",
        type = "alternate",
        data = {
            a = { "X", "Y" },
            b = { "P", "Q" },
        }
    }
\stopluacode
\stopbuffer

\typebuffer \getbuffer

Less useful is a multiple substitution. Normally this one is part of a chain of
replacements.

\startbuffer
\startluacode
    fonts.handlers.otf.addfeature {
        name = "mtest",
        type = "multiple",
        data = {
            a = { "X", "Y" },
            b = { "P", "Q" },
        }
    }
\stopluacode
\stopbuffer

\typebuffer \getbuffer

A ligature (or multiple to one) is also possible but normally only makes sense when
there is indeed a ligature. We use a similar definition for mapping the \TEX\ input
sequence \type {---} onto an \emdash.

\startbuffer
\startluacode
    fonts.handlers.otf.addfeature {
        name = "ltest",
        type = "ligature",
        data = {
            ['1'] = { "a", "b" },
            ['2'] = { "d", "a" },
        }
    }
\stopluacode
\stopbuffer

\typebuffer \getbuffer

\stopsubsection

\startsubsection[title=Positioning]

You can define a kern feature too but when doing so you need to use measures in
font units.

\startbuffer
\startluacode
    fonts.handlers.otf.addfeature {
        name = "ktest",
        type = "kern",
        data = {
            a = { b = -500 },
        }
    }
\stopluacode
\stopbuffer

\typebuffer \getbuffer

Pairwise positioning is more complex and involves two (optional) arrays
that specify \type {{dx dy wd ht}} for each of the two glyphs. In the next
example we only displace the second glyph.

\startbuffer
\startluacode
    fonts.handlers.otf.addfeature {
        name = "ptest",
        type = "pair",
        data = {
            ["a"] = { ["b"] = { false,  { -1000, 1200, 0, 0 } } },
        }
    }
\stopluacode
\stopbuffer

\typebuffer \getbuffer

Of course you need to know a bit about the metrics of the glyphs involved so in
practice this boils down to trial and error.

A single character (glyph) can also be tweaked, although normally this is done
better in a manipulator when loading the font. Anyway:

\startbuffer
\startluacode
    fonts.handlers.otf.addfeature {
        name = "stest",
        type = "single",
        data = {
            a = { -30, 0, -50, 0 },
        }
    }
\stopluacode
\stopbuffer

\typebuffer \getbuffer

This will reduce the left and right edges and make the glyph a pretty tight one. The
values are for Latin Modern.

\stopsubsection

\startsubsection[title=Examples]

We didn't show usage yet. This is because we need to define a feature before we
define a font. New features will be added to a font when it gets defined.

\startbuffer
\definefontfeature[stest][stest=yes]
\definefontfeature[atest][atest=2]
\definefontfeature[mtest][mtest=yes]
\definefontfeature[ltest][ltest=yes]
\definefontfeature[ktest][ktest=yes]
\definefontfeature[ptest][ptest=yes]
\definefontfeature[ctest][ctest=yes]

\definedfont[file:dejavu-serif.ttf*default]

\starttabulate[|l|l|l|]
\NC operation    \NC feature       \NC              abracadabra \NC \NR
\HL
\NC substitution \NC \type {stest} \NC \addff{stest}abracadabra \NC \NR
\NC alternate    \NC \type {atest} \NC \addff{atest}abracadabra \NC \NR
\NC multiple     \NC \type {mtest} \NC \addff{mtest}abracadabra \NC \NR
\NC ligature     \NC \type {ltest} \NC \addff{ltest}abracadabra \NC \NR
\NC kern         \NC \type {ktest} \NC \addff{ktest}abracadabra \NC \NR
\NC pair         \NC \type {ptest} \NC \addff{ptest}abracadabra \NC \NR
\NC chain sub    \NC \type {ctest} \NC \addff{ctest}abracadabra \NC \NR
\stoptabulate
\stopbuffer

\typebuffer \getbuffer

\stopsubsection

\startsubsection[title=Contexts]

A more complex substitution is the following:

\startbuffer
\startluacode
    fonts.handlers.otf.addfeature {
        name    = "ytest",
        type    = "chainsubstitution",
        lookups = {
            {
                type = "substitution",
                data = {
                    ["b"] = "B",
                    ["c"] = "C",
                },
            },
        },
        data = {
            rules = {
                {
                    before  = { { "a" } },
                    current = { { "b", "c" } },
                    lookups = { 1 },
                },
            },
        },
    }
\stopluacode
\stopbuffer

\typebuffer \getbuffer

Here the dataset is a sequence of rules. There can be a \type {before}, \type
{current} and \type {after} match. The replacements are specified with the \type
{lookups} entry and the numbers are indices in the provided \type {lookups}
table.

Here is another example. This one demonstrates that one can check against spaces
(some fonts kerns against them) and against boundaries as well. The later is
something \CONTEXT\ specific. First we define a feature that create ligatures but
only when we touch a space:

\startbuffer
\startluacode
    fonts.handlers.otf.addfeature {
        name    = "test-a",
        type    = "chainsubstitution",
        lookups = {
            {
                type = "ligature",
                data = {
                    ['1'] = { "a", "b" },
                    ['2'] = { "c", "d" },
                },
            },
        },
        data = {
            rules = {
                {
                    before  = { { " " } },
                    current = { { "a" }, { "b" } },
                    lookups = { 1 },
                },
                {
                    current = { { "c" }, { "d" } },
                    after   = { { " " } },
                    lookups = { 1 },
                },
            },
        },
    }
\stopluacode
\stopbuffer

\typebuffer \getbuffer

The next example also checks against whatever boundary we have.

\startbuffer
\startluacode
    fonts.handlers.otf.addfeature {
        name    = "test-b",
        type    = "chainsubstitution",
        lookups = {
            {
                type = "ligature",
                data = {
                    ['1'] = { "a", "b" },
                    ['2'] = { "c", "d" },
                },
            },
        },
        data = {
            rules = {
                {
                    before  = { { " ", 0xFFFC } },
                    current = { { "a" }, { "b" } },
                    lookups = { 1 },
                },
                {
                    current = { { "c" }, { "d" } },
                    after   = { { 0xFFFC, " " } },
                    lookups = { 1 },
                },
            },
        },
    }
\stopluacode
\stopbuffer

\typebuffer \getbuffer

We can actually simplify this one to:

\startbuffer
\startluacode
    fonts.handlers.otf.addfeature {
        name    = "test-c",
        type    = "chainsubstitution",
        lookups = {
            {
                type = "ligature",
                data = {
                    ['1'] = { "a", "b" },
                    ['2'] = { "c", "d" },
                },
            },
        },
        data = {
            rules = {
                {
                    before  = { { 0xFFFC } },
                    current = { { "a" }, { "b" } },
                    lookups = { 1 },
                },
                {
                    current = { { "c" }, { "d" } },
                    after   = { { 0xFFFC } },
                    lookups = { 1 },
                },
            },
        },
    }
\stopluacode
\stopbuffer

\typebuffer \getbuffer

As a bonus we show how to do more complex things:

\startbuffer
\startluacode
    fonts.handlers.otf.addfeature {
        name    = "test-d",
        type    = "chainsubstitution",
        lookups = {
            {
                type = "substitution",
                data = {
                    ["a"] = "A",
                    ["b"] = "B",
                    ["c"] = "C",
                    ["d"] = "D",
                },
            },
            {
                type = "ligature",
                data = {
                    ['1'] = { "a", "b" },
                    ['2'] = { "c", "d" },
                },
            },
        },
        data = {
            rules = {
                {
                    before  = { { 0xFFFC } },
                    current = { { "a" }, { "b" } },
                    lookups = { 2 },
                },
                {
                    current = { { "c" }, { "d" } },
                    after   = { { 0xFFFC } },
                    lookups = { 2 },
                },
                {
                    current = { { "a" } },
                    after   = { { "b" } },
                    lookups = { 1 },
                },
                {
                    current = { { "c" } },
                    after   = { { "d" } },
                    lookups = { 1 },
                },
            },
        },
    }
\stopluacode
\stopbuffer

\typebuffer \getbuffer

\definefontfeature[test-a][test-a=yes]
\definefontfeature[test-b][test-b=yes]
\definefontfeature[test-c][test-c=yes]
\definefontfeature[test-d][test-d=yes]

\startbuffer
abababcdcd abababcdcd abababcdcd
\stopbuffer

With the test text:

\typebuffer

These four result in:

\blank \start

    \definedfont[file:dejavu-serif.ttf*default]

    \start \addff{test-a} \getbuffer \stop\par
    \start \addff{test-b} \getbuffer \stop\par
    \start \addff{test-c} \getbuffer \stop\par
    \start \addff{test-d} \getbuffer \stop\par

\stop \blank

\stopsubsection

\startsubsection[title={Language dependencies}]

When \OPENTYPE\ was not around we only had to deal with ligatures, smallcaps and
oldstyle and of course kerns. Their number was so small that the term \quote
{features} was not even used. In practice one just loaded a font that had
oldstyle or smallcaps or none of that and was done. There were different fonts and
sold separately.

In \OPENTYPE\ we have more variation and although these fonts can be much more
advanced the lack of standardization (for instance what gets initialized, or what
shapes are in the default slots) can lead to messy setups. Some fonts bind
features to scripts, some don't, which means that:

\starttyping
\definefontfeature[smallcaps][smcp=yes,script=dflt]
\definefontfeature[smallcaps][smcp=yes,script=latn]
\definefontfeature[smallcaps][smcp=yes,script=cyrl]
\stoptyping

are in fact different and you don't know in advance if you need to specify \type
{dflt} or \type {latn}. In practice for a feature like smallcaps there is no
difference between languages, but for ligatures there can be.

When we extend an existing feature we can think of:

\starttyping
\definefontfeature[smallcaps][default][smcp=yes,script=auto]
\definefontfeature[smallcaps][default][smcp=yes,script=*]
\stoptyping

but that can have side effects too (for instance disabling language specific
features). The easiest way to explore this language dependency is to make
a feature of our own.

\startbuffer
\startluacode
fonts.handlers.otf.addfeature {
    name     = "simplify",
    type     = "multiple",
    prepend  = true,
    features = {
        ["*"] = {
            ["deu"] = true
        }
    },
    data     = {
        [utf.byte("ä")] = { "a", "e" },
        [utf.byte("Ä")] = { "A", "E" },
        [utf.byte("ü")] = { "u", "e" },
        [utf.byte("Ü")] = { "U", "E" },
        [utf.byte("ö")] = { "o", "e" },
        [utf.byte("Ö")] = { "O", "E" },
        [utf.byte("ß")] = { "s", "z" },
        [utf.byte("ẞ")] = { "S", "Z" },
    },
}
\stopluacode
\stopbuffer

\typebuffer \getbuffer

Here we implement a language specific feature that we use at the \TEX\ end:

\startbuffer
\definefontfeature
  [simplify-de]
  [simplify=yes,
   language=deu]
\stopbuffer

\typebuffer \getbuffer

that we can use as:

\startbuffer
\definedfont[Serif*default,simplify-de]%
äüöß
{\de äüöß}
{\nl äüöß}
\stopbuffer

\typebuffer

and get: \start \maincolor \inlinebuffer \stop, but as you see, both German and
Dutch get the same treatment, which might not be what you want, because in Dutch
the diearesis has a different meaning.

\startbuffer
\definedfont[Serif*default]%
                       äüöß
{\de\addff{simplify-de}äüöß}
{\nl                   äüöß}
\stopbuffer

\typebuffer

The above is restricts the usage so now we get: \start \maincolor \inlinebuffer
\stop, which is more language bound. You don't need much imagination for
extending this:

\startbuffer
\definefontfeature
  [simplify]
  [simplify=yes,
   language=deu]
\stopbuffer

\typebuffer \getbuffer

\startbuffer
\definedfont[Serif*default]%
                    äüöß
{\de\addff{simplify}äüöß}
{\nl\addff{simplify}äüöß}
\stopbuffer

So what do we expect with the next?

\typebuffer

We get: \start \maincolor \inlinebuffer \stop, and we see that the language
setting is not taken into account! This is because the font already has been set
up with a script and language combination. The solution is to temporary set the
font related language explicitly:

\definefontfeature
  [simplify]
  [simplify=yes]

\startbuffer
\definedfont[Serif*default]%
                                  äüöß
{\de\addfflanguage\addff{simplify}äüöß}
{\nl\addfflanguage\addff{simplify}äüöß}
\stopbuffer

\typebuffer

So we can automatically switch to language specific features if we want to:
\start \maincolor \inlinebuffer \stop.

Let's now move to another level of complexity: support for more than one language
as in fact this example was made for Dutch in the first place, but the German
outcome is a bit more visible.

\startbuffer
\startluacode
fonts.handlers.otf.addfeature {
    name     = "simplify",
    type     = "multiple",
    prepend  = true,
 -- prepend  = "smcp",
    dataset  =
    {
        {
            features = {
                ["*"] = {
                    ["nld"] = true
                }
            },
            data     = {
             -- [utf.byte("ä")] = { "a" },
             -- [utf.byte("Ä")] = { "A" },
             -- [utf.byte("ü")] = { "u" },
             -- [utf.byte("Ü")] = { "U" },
             -- [utf.byte("ö")] = { "o" },
             -- [utf.byte("Ö")] = { "O" },
                [utf.byte("ij")] = { "i", "j" },
                [utf.byte("IJ")] = { "I", "J" },
                [utf.byte("æ")] = { "a", "e" },
                [utf.byte("Æ")] = { "A", "E" },
            },
        },
        {
         -- type     = "multiple", -- local values possible
            features = {
                ["*"] = {
                    ["deu"] = true
                }
            },
            data     = {
                [utf.byte("ä")] = { "a", "e" },
                [utf.byte("Ä")] = { "A", "E" },
                [utf.byte("ü")] = { "u", "e" },
                [utf.byte("Ü")] = { "U", "E" },
                [utf.byte("ö")] = { "o", "e" },
                [utf.byte("Ö")] = { "O", "E" },
                [utf.byte("ß")] = { "s", "z" },
                [utf.byte("ẞ")] = { "S", "Z" },
            },
        }
    }
}
\stopluacode
\stopbuffer

\typebuffer \getbuffer

For this we use the following example:

\startbuffer
\definedfont[Serif*default,simplify]%
                   äüöß ijæ
{\de\addfflanguage äüöß ijæ}
{\nl\addfflanguage äüöß ijæ}
\stopbuffer

\typebuffer

Because the Dutch is hard to check we use an \type {æ} replacement too and
commented the similarities with German: \start \maincolor \inlinebuffer \stop.
But still we're not done, say that we want smallcaps too:

\startbuffer
\definefontfeature[alwayssmcp][smcp=always]%
\definedfont[Serif*default,simplify,alwayssmcp]%
                   äüöß ijæ
{\de\addfflanguage äüöß ijæ}
{\nl\addfflanguage äüöß ijæ}
\stopbuffer

\typebuffer

This comes out as: \start \maincolor \inlinebuffer \stop.

The reason for specifying \type{smcp} as \type {always} is that otherwise we
get language specific smallcaps while often they are not bound to a language
but to the defaults. The good news is that we can do this automatically:

\startbuffer
\setupfonts[language=auto]%
\definefontfeature[alwayssmcp][smcp=always]%
\definedfont[Serif*default,simplify,alwayssmcp]%
     äüöß ijæ
{\de äüöß ijæ}
{\nl äüöß ijæ}
\stopbuffer

\typebuffer

But be aware that this applies to all situations. Here we get: \start \maincolor
\inlinebuffer \stop.

\stopsubsection

\startsubsection[title=Syntax summary]

In the examples we have seen several ways to define features. One of the
differences is that you either set a \type {data} field directly, or that you
specify a dataset. The fields in a dataset entry overload the ones given at the
top level or when not set the top level value will be taken. There is a bit
of (downward compatibility) tolerance built in, but best not depend on that.

\starttyping
fonts.handlers.otf.addfeature {
    name     = "demo",
    features = {
        [<script>] = {
            [<language>] = true
        }
    },
    prepend  = true | featurename | position,
    dataset  = {
        {
            type = "substitution",
            data = {
                [<char|code>] = <char|code>,
            }
        },
        {
            type = "alternate",
            data = {
                [<char|code>] = { <char|code>, <char|code>, ... },
            }
        },
        {
            type = "multiple",
            data = {
                [<char|code>] = { <char|code>, <char|code>, ... },
            }
        },
        {
            type = "ligature",
            data = {
                [<char|code>] = { <char|code>, <char|code>, ... },
            }
        },
        {
            type = "kern",
            data = {
                [<char|code>] = { [<char|code>] = <value> },
            }
        },
        {
            type = "pair",
            data = {
                [<char|code>] = {
                    [<char|code>] = {
                        false | { <value>, <value>, <value>, <value> },
                        false | { <value>, <value>, <value>, <value> }
                    }
                }
            }
        },
        {
            type    = "chainsubstitution",
            lookups = {
                {
                    type = <typename>,
                    data = <mapping>,
                },
            },
            data = {
                rules = {
                    {
                        before  = { { [<char|code>], ... } },
                        current = { { [<char|code>], ... } },
                        after   = { { [<char|code>], ... } },
                        lookups = { <index>, ... },
                    },
                },
            },
        },
    },
}
\stoptyping

\stopsubsection

\startsubsection[title=Extra characters]

\startbuffer[hyphenchars]
\startluacode

    local privateslots = fonts.constructors.privateslots

    local function addspecialhyphen(tfmdata)

        local exheight = tfmdata.parameters.xheight
        local emwidth  = tfmdata.parameters.quad
        local width    = emwidth  /  4
        local height   = exheight / 10
        local depth    = exheight /  2
        local offset   = emwidth  /  6

        tfmdata.characters[privateslots.righthyphenchar] = {
            -- no dimensions
            commands = {

                { "right", offset },

                { "push" },
                { "right", -width },
                { "down", depth },
                { "rule", height, width },
                { "pop" },

                { "right", -width/5 },
                { "down", depth + height },
                { "rule", 3*height, width/5 },

            }
        }

        tfmdata.characters[privateslots.lefthyphenchar] = {
            -- no dimensions
            commands = {

                { "right", -offset },

                { "push" },
                { "down", depth + height },
                { "rule", 3*height, width/5 },
                { "pop" },

                { "down", depth },
                { "rule", height, width },

            }
        }

    end

    fonts.constructors.features.otf.register {
        name        = "specialhyphen",
        description = "special hyphen",
        manipulators = {
            base = addspecialhyphen,
            node = addspecialhyphen,
        }
    }

\stopluacode
\stopbuffer

You can add virtual characters to fonts. Here we give an example that is derived
from an example posted on the mailing list. By default, when we hyphenated a word,
we get this:

\definefont[DemoFont] [Serif*default]

\blank \start \DemoFont \maincolor \hsize 1mm averylongword \par \stop \blank

The default character that is appended at the end and beginning of a line
can be specified as follows:

\startbuffer
\setuplanguage
  [en]
  [righthyphenchar=45,
   lefthyphenchar=45]
\stopbuffer

\typebuffer

So now we get:

\blank \start \getbuffer \DemoFont \maincolor \hsize 1mm averylongword \par \stop \blank

Say that we want a different signal, for instance some rule. Here is how that can
be done:

\typebuffer[hyphenchars]

\getbuffer[hyphenchars]

Watch the way we use private slots. You can best use a unique glyph name as these
numbers are shared between fonts. With:

\startbuffer
\definefontfeature
  [default]
  [default]
  [specialhyphen=yes]
\definefont
  [DemoFont]
  [Serif*default at 24pt]
\setuplanguage
  [en]
  [righthyphenchar=\getprivateglyphslot{righthyphenchar},
   lefthyphenchar=\getprivateglyphslot{lefthyphenchar}]
\stopbuffer

\typebuffer

We get:

\startlinecorrection[blank]
\getbuffer
\framed
  [foregroundstyle=\DemoFont \setupinterlinespace,
   offset=none,
   frame=no,
   width=1mm,
   align={flushleft}]
  {\hsize 1mm \maincolor averylongword\par}
\stoplinecorrection

You need to keep in mind that some of these settings are global but in practice that is
not a real problem. Here is how you reset:

\startbuffer
\definefontfeature
  [default]
  [default]
  [specialhyphen=no]
\setuplanguage
  [en]
  [righthyphenchar=45,
   lefthyphenchar=0]
\stopbuffer

\typebuffer \getbuffer

\stopsubsection

\startsubsection[title=Goodies]

The examples above extend a font in the \TEX\ document (normally a style) but you
can use a goodies file too, for instance \type {cambria.lfg}.

\starttyping
return {
    name = "cambria",
    version = "1.00",
    comment = "Goodies that complement cambria.",
    author = "Hans Hagen",
    copyright = "ConTeXt development team",
    extensions = {
        {
            name = "kern", -- adds to kerns
            type = "pair",
            data = {
                [0x0153] = { -- combining acute
                    [0x0301] = { -- aeligature
                        false,
                        { -500, 0, 0, 0 }
                    }
                },
            }
        }
    }
}
\stoptyping

Here we use the feature name \type {kern} and therefore we don't have to define a
specific (new) feature for it. Such a goodie is then used as follows:

\starttyping
\definefontsynonym
  [Serif]
  [cambria]
  [features=default,
   goodies=cambria]
\stoptyping

You can find such definitions in the \type {type-imp-*.mkiv} files.

\stopsubsection

\stopsection

\startsection[title=Spacing]

% By default the font loader deduces the spacing from the space character or
% other font properties. You can influence this by the \type {space} feature.
%
% \starttyping
% \definefontfeature
%   [korean]
%   [default]
%   [script=hang,
%    language=kor,
%    space=local] % or locl
% \stoptyping
%
% Instead of the usual \type {yes} (which means: use character 32), \type {local}
% or \type {locl} (which means: use a replacement provided by the \type{locl}
% feature), you can also pass a character, so
%
% \starttyping
% \definefontfeature
%   [spacy]
%   [default]
%   [space=A]
% \stoptyping
%
% is valid.

As you probably know, \TEX\ has no space character. When the input is read,
characters tagged as space are intercepted and become glue. Compare this:

\startlinecorrection[blank]
    \startcombination
        {\framed
            [width=3cm,height=15mm,align={middle,lohi},foregroundcolor=maincolor]
            {\dorecurse{5}{test }}}
        {\type{text test...}}
        {\framed
            [width=3cm,height=15mm,align={middle,lohi},foregroundcolor=maincolor]
            {\dorecurse{5}{test\char32\relax}}}
        {\type{text\char32test...}}
    \stopcombination
\stoplinecorrection

Most fonts have a space character and you can actually use it and indeed a space
character will be injected but as it is not glue, the line break algorithm will
not see it as space.

Al the magic done with space characters other than the native space character
(decimal 32) are at some point translated into glue.

\starttabulate[||T|p|]
\NC \bf command \NC \UNICODE \NC width \NC \NR

\NC \type{\nobreakspace}
    \type{\nbsp}                     \NC U+00A0 \NC space           \NC \NR
\NC \type{\ideographicspace}         \NC U+2000 \NC quad/2          \NC \NR
\NC \type{\ideographichalffillspace} \NC U+2001 \NC quad            \NC \NR
\NC \type{\twoperemspace}
    \type{\enspace}                  \NC U+2002 \NC quad/2          \NC \NR
\NC \type{\emspace}
    \type{\quad}                     \NC U+2003 \NC quad            \NC \NR
\NC \type{\threeperemspace}          \NC U+2004 \NC quad/3          \NC \NR
\NC \type{\fourperemspace}           \NC U+2005 \NC quad/4          \NC \NR
\NC \type{\fiveperemspace}           \NC        \NC quad/5          \NC \NR
\NC \type{\sixperemspace}            \NC U+2006 \NC quad/6          \NC \NR
\NC \type{\figurespace}              \NC U+2007 \NC width of zero   \NC \NR
\NC \type{\punctuationspace}         \NC U+2008 \NC width of period \NC \NR
\NC \type{\breakablethinspace}       \NC U+2009 \NC quad/8          \NC \NR
\NC \type{\hairspace}                \NC U+200A \NC quad/8          \NC \NR
\NC \type{\zerowidthspace}           \NC U+200B \NC 0               \NC \NR
\NC \type{\zerowidthnonjoiner}
    \type{\zwnj}                     \NC U+200C \NC 0               \NC \NR
\NC \type{\zerowidthjoiner}
    \type{\zwj}                      \NC U+200D \NC 0               \NC \NR
\NC \type{\narrownobreakspace}       \NC U+202F \NC quad/8          \NC \NR
\NC \type{\zerowidthnobreakspace}    \NC U+FEFF \NC                 \NC \NR
\NC \type{\optionalspace}            \NC        \NC space when not followed by punctuation \NC \NR
\stoptabulate

% "205F  % space/8 (math)

The last one is not un \UNICODE\ and the fifths of an emspace is not in \UNICODE\
either. This emspace (or quad in \TEX\ speak) is a font property. The width of
the space used by \CONTEXT\ is dreived form this value. In case of a monospace
fonts, the following logic is applied:

\startitemize
    \startitem
        When there is a space character, the width of that character is used.
    \stopitem
    \startitem
        Otherwise, when there is an emdash present, the width if that character
        is used.
    \stopitem
    \startitem
        Otherwise, when there is an \type {charwidth} property available (the
        average width), that valua is used.
    \stopitem
\stopitemize

When a proportional font is used, we do as follows:

\startitemize
    \startitem
        When there is a space character, the width of that character is used.
    \stopitem
    \startitem
        Otherwise, when there is an emdash present, the width of that character
        divided by two is used.
    \stopitem
    \startitem
        Otherwise, when there is an \type {charwidth} property available (the
        average width), that value is used.
    \stopitem
\stopitemize

In both cases, when no value is set we use the units of the font (often 1000 or
2048). In \TEX\ a space glue also has stretch and shrink. Here we follow the
traditional \TEX\ logic:

\startitemize
    \startitem
        The stretch is set to half the width of a space but to zero with a mono
        spaced font.
    \stopitem
    \startitem
        The shrink is set to one third of the width of a space but to zero with a
        mono spaced font.
    \stopitem
\stopitemize

The xheight is set to the values specified by the font and when this is unset the
height of the character \type {x} will be used but when this character is not in
the font, we use two fifths of the font's units (normally the same as the
emwidth). The italic angle is also taken from the font (and is of course zero for
a not italic font). Most fonts have these properties set so we seldom have to
fall back to a guess.

\stopsection

\startsection[title=Ligatures]

Not all fonts provide ligature control (normally related to languages), so here is a
trick.

\starttyping
\blockligatures[fi,ff]
\blockligatures[fl]

\definefontfeature
  [default]
  [default]
  [blockligatures=yes]

\setupbodyfont[pagella]

...
\stoptyping

This way it works globally. Of course you can also bind it to a font instance:

\startbuffer
\blockligatures[fi,fl]

\definefontfeature
  [default:blockligs]
  [default]
  [blockligatures=yes]

\definefont[DemoBlockY][Serif*default:blockligs at 20pt]
\definefont[DemoBlockN][Serif*default           at 20pt]

Here we have no ligatures: {\DemoBlockY fi ff fl}, while here we get
them: {\DemoBlockN fi ff fl}. Of course it also depends on the font.
\stopbuffer

\typebuffer \start \showfontkerns \getbuffer \par \stop

There is one limitation: you need to specify the blocked ligatures before a font
gets defined and because we share resources it even has to happen before the
first font gets loaded. So, the \type {\blockligatures} commands go before
setting up the body font. This is no real problem because it's a hack anyway.

The next example combines several tricks:

\startbuffer[definitions]
\startluacode
    fonts.handlers.otf.addfeature {
        name = "kernligatures",
        type = "kern",
        data = {
            f = { i = 50, l = 50 },
        }
    }
\stopluacode

\blockligatures[u:fl:a]

\definefontfeature[default:b][default][blockligatures=yes]
\definefontfeature[default:k][default][blockligatures=yes,kernligatures=yes]

\showfontkerns
\stopbuffer

\startbuffer[demo]
{\definedfont[Brill*default   @ 11pt]auflage}\par
{\definedfont[Brill*default:b @ 11pt]auflage}\par
{\definedfont[Brill*default:k @ 11pt]auflage}\par
\stopbuffer

\typebuffer[definitions,demo] \getbuffer[definitions]

\startlinecorrection
    \externalfigure[demo.buffer][width=4cm]
\stoplinecorrection

Processing fonts is complicated by the fact that a text can be hyphenated. This
complicates for instance ligature building which can cross the pre, post and|/|or
replace bounds. The current implementation does a decent job although there will
always be border cases. And, figuring out what goes wrong is a pain. There are
several ways to trace what happens and here's one. As mentioned, blocking only
works when we haven't not yet defined a font instance, so we use a funny size
here.

\startbuffer
\blockligatures[u:fl:a]

\definefontfeature
  [blockligatures]
  [default]
  [blockligatures=yes]

\startotfcompositionlist{texgyrepagella-regular*blockligatures @ 14.5pt}{l2r}
    \HL
    \showotfcompositionsample{auflage}
    \showotfcompositionsample{a\discretionary{-}{}{}uflage}
    \showotfcompositionsample{au\discretionary{-}{}{}flage}
    \showotfcompositionsample{auf\discretionary{-}{}{}lage}
    \showotfcompositionsample{aufl\discretionary{-}{}{}age}
    \showotfcompositionsample{aufla\discretionary{-}{}{}ge}
    \showotfcompositionsample{auflag\discretionary{-}{}{}e}
    \HL
    \showotfcompositionsample{auflegt}
    \showotfcompositionsample{a\discretionary{-}{}{}uflegt}
    \showotfcompositionsample{au\discretionary{-}{}{}flegt}
    \showotfcompositionsample{auf\discretionary{-}{}{}legt}
    \showotfcompositionsample{aufl\discretionary{-}{}{}egt}
    \showotfcompositionsample{aufle\discretionary{-}{}{}gt}
    \showotfcompositionsample{aufleg\discretionary{-}{}{}t}
    \HL
\stopotfcompositionlist
\stopbuffer

\typebuffer \getbuffer

Here is another example. This one demonstrates that ligatures can force
collapsing of discretionaries.

\startbuffer
\startotfcompositionlist{Serif*default @ 11pt}{l2r}
    \HL
    \showotfcompositionsample{effe}
    \showotfcompositionsample{efficient}
    \HL
\stopotfcompositionlist
\stopbuffer

\typebuffer \getbuffer

\stopsection

\startsection[title=Collections]

    {\em Todo.}

\stopsection

\stopchapter
