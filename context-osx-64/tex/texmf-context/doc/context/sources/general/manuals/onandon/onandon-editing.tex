% language=uk

\startcomponent onandon-editing

\environment onandon-environment

\startchapter[title=Editing]

\startsection[title=Introduction]

% This introduction is similar to the workflows chapter.

Some users like the synctex feature that is built in the \TEX\ engines.
Personally I never use it because it doesn't work well with the kind of documents
I maintain. If you have one document source, and don't shuffle around (reuse)
text too much it probably works out okay but that is not our practice. Here I
will describe how you can enable a more \CONTEXT\ specific synctex support so
that aware \PDF\ viewers can bring you back to the source.

\stopsection

\startsection[title={The premise}]

Most of the time we provide our customers with an authoring workflow consisting
of:

\startitemize[packed]
    \startitem the typesetting engine \CONTEXT \stopitem
    \startitem the styles to generate the desired \PDF\ files \stopitem
    \startitem the text editor \SCITE \stopitem
    \startitem the \SUMATRAPDF\ viewer \stopitem
\stopitemize

For the \MATHML\ we advice the \MATHTYPE\ editor and we provide them with a
customized \MATHML\ translator for the copy & paste actions. When \ASCIIMATH\ is
used to code math no special tools are needed.

What people operate this workflow? Sometimes it's an author, but most of the time
they are editors with a background in copy|-|editing. We call them \XML\ editors,
because they are maintaining the large (sets of) \XML\ documents and edit
directly in the \XML\ sources.

Maybe you'll ask yourself \quotation {Can they do that? Can they edit directly in
the \XML\ resource?} The answer is yes, because after they have hit the
processing key they are rewarded with a publishable \PDF\ document in a demanding
layout.

The \XML\ sources have a dual purpose. They form the basis for:

\startitemize[packed]
    \startitem
        all folio products that are generated in \XML\ to \PDF\ workflow(s)
    \stopitem
    \startitem
        the digital web product(s)
    \stopitem
\stopitemize

The \XML\ editors do their proofing chapter|-|wise. Sometimes a chapter is one
big \XML\ file (10.000 lines is no exception when the chapter contains hundreds
of bloated \MATHML\ snippets). In other projects they have to deal with chapters
that are made up of hundreds (100 upto 500) of smaller \XML\ files.

\stopsection

\startsection[title={The problem}]

Let's keep it simple: there's a typo. Here's what an \XML\ editor will do:

\startitemize[packed]
    \startitem
        start \SCITE
    \stopitem
    \startitem
        open a file
    \stopitem
    \startitem
        correct the typo
    \stopitem
    \startitem
        generate the \PDF
    \stopitem
    \startitem
        proof the \PDF\ and see if his alteration has some undesired side
        effects like text flow of image floating
    \stopitem
\stopitemize

So far so good. When the editor dealing with one big \XML\ file there's no
problem. Hopefully the filename will indicate the specific chapter. He or she
opens the file and searches for the typo. And then correction happens. But what
if there are hundreds of small \XML\ files. How does the editor know in which
file the typo can be found?

First, let's give a few statistics based on two projects that are in a revision
stage.

\starttabulate[|c|c|c|c|]
\HL
\NC
    project \NC
    chapters \NC
    \# of files \NC
    average \# of lines \NC \NR
\HL
\NC
    A \NC
    16 \NC
    16 \NC
    11000 \NC \NR
\NC
    B \NC
    132 \NC
    16000\footnote{132 chapters consisting of $\pm 120$ files.} \NC
    100 \NC \NR
\HL
\stoptabulate

The \XML\ resource passes three stages: a raw, a semi final and a final version.
The raw \XML\ version originates from a web authoring tool that is used by the
author. Then the \PDF\ is proofread and the \XML\ editor goes to work.

\starttabulate[|l|c|c|]
\HL
\NC
    workflow \NC
    \# edit locations and adaptations \NC
    \# runs\footnote {Maybe you can now see why we put quite some effort in
       keeping \CONTEXT\ working at a comfortable speed.} \NC \NR
\HL
\NC
    raw to semifinal \NC
    75 \NC
    105 \NC \NR
\NC
    semifinal to final \NC
    35 \NC
    55 \NC \NR
\HL
\stoptabulate

Keep in mind that altering text may cause text to flow and images to float in a
way that an \XML\ editor will have to finetune and needs multiple runs for one
correction.

Just to give an idea of the work involved. A typical semi final needs some 50
runs where each run takes 20 seconds (assuming 3 runs to get all cross
referencing right). The numbers of explicit pagebreaks is about 5, and (related
to formulas) explicit linebreaks around 8. It takes some 2 hours to get
everything right, which includes checking in detail, fixing some things and if
needed moving content a bit around.

Now we broaden the earlier question into: how can we make the work of an \XML\
editor as easy and efficient as possible?

\stopsection

\startsection[title={Enhancing efficiency}]

Since it is easier to proof content for folio and web via PDF documents we
generate proof \PDF\ files in which the complete content is shown. The proof can
be a massive document. A normal 40 page chapter can explode to 140 pages
visualizing all the content that is coded in the \XML\ file(s).

The content in the proof is shown in an effective way and a functional order.
Let's give a few examples of how we enhance the \XML\ editors effectiveness:

\startitemize[packed]

\startitem
    By default the proof \PDF\ file is interactive which serves testing the tocs
    and the register.
\stopitem
\startitem
    The web hyperlinks are active so their destinatation can be tested.
\stopitem
\startitem
    The questions and their answers are displayed in eachothers proximity. This
    sounds logical but in folio they are two seperate products (theory and
    answer books).
\stopitem
\startitem
    Medium specific content (web or folio) is typographically highligthed. For
    example by colored backgrounds.
\stopitem
\startitem
    When spelling mode is on the \XML\ editor can easily pick out the colored
    misspelled words.
\stopitem
\startitem
    Images can be active areas although this is of no interest to \XML\ editors.
    Clicking the image results in opening the image file in its corresponding
    application for maintenance.
\stopitem
\startitem
    For practical reasons the filenames and paths of the \XML\ files are
    displayed. The filenames are active links and clicking them results in
    opening the destination \XML\ file in \SCITE.
\stopitem
\stopitemize

Okay. The last option is a nice feature. However, the destination file is opened
at the top of the file and you still have to find the typo or whatever
incorrect issue you are looking for.

So a further enhancement in efficiency would be to jump to the typo's
corresponding line in the \XML\ source. This is where \SYNCTEX\ comes into view.
This feature, present in the \TEX\ engines, provides a way to go from \PDF\ to
source by using a secondary file with positions. Unfortunately that mechanism is
hardly useable for \CONTEXT\ because it assumes a page and file handling model
different from what we use. However, as \CONTEXT\ uses \LUATEX, it can also
provide it's own alternative.

\stopsection

% The rest is similar to the workflows chapter.

\startsection[title=What we want]

The \SYNCTEX\ method roughly works as follows. Internally \TEX\ constricts linked
lists of glyphs, kerns, glue, boxes, rules etc. These elements are called nodes.
Some nodes carry information about the file and line where they were created. In
the backend this information gets somehow translated in a (sort of) verbose tree
that describes the makeup in terms of boxes, glue and kerns. From that
information the \SYNCTEX\ parser library, hooked into a \PDF\ viewer, can go back
from a position on the screen to a line in a file. One would expect this to be a
relative simple rectangle based model, but as far as I can see it's way more
complex than that. There are some comments that \CONTEXT\ is not supported well
because it has a layered page model, which indicates that there are some
assumptions about how macro packages are supposed to work. Also the used
heuristics not only involve some specific spot (location) but also involve the
corners and edges. It is therefore not so much a (simple) generic system but a
mechanism geared for a macro package like \LATEX.

Because we have a couple of users who need to edit complex sets of documents,
coded in \TEX\ or \XML, I decided to come up with a variant that doesn't use the
\SYNCTEX\ machinery but manipulates the few \SYNCTEX\ fields directly \footnote {This
is something that in my opinion should have been possible right from the start
but it's too late now to change the system and it would not be used beyond
\CONTEXT\ anyway.} and eventually outputs a straightforward file for the editor.
Of course we need to follow some rules so that the editor can deal with it. It
took a bit of trial and error to get the right information in the support file
needed by the viewer but we got there.

The prerequisites of a decent \CONTEXT\ \quotation {click on preview and goto
editor} are the following:

\startitemize

\startitem
    It only makes sense to click on text in the text flow. Headers and footers
    are often generated from structure, and special typographic elements can
    originate in macros hooked into commands instead of in the source.
\stopitem

\startitem
    Users should not be able to reach environments (styles) and other files
    loaded from the (normally read|-|only) \TEX\ tree, like modules. We don't
    want accidental changes in such files.
\stopitem

\startitem
    We not only have \TEX\ files but also \XML\ files and these can normally
    flush in rather arbitrary ways. Although the concept of lines is sort of
    lost in such a file, there is still a relation between lines and the snippets
    that make out the content of an \XML\ node.
\stopitem

\startitem
    In the case of \XML\ files the overhead related to preserving line
    numbers should be minimal and have no impact on loading and memory when
    these features are not used.
\stopitem

\startitem
    The overhead in terms of an auxiliary file size and complexity as well
    as producing that file should be minimal. It should be easy to turn on and
    off these features. (I'd never turn them on by default.)
\stopitem

\stopitemize

It is unavoidable that we get more run time but I assume that for the average user
that is no big deal. It pays off when you have a workflow when a book (or even a
chapter in a book) is generated from hundreds of small \XML\ files. There is no
overhead when \SYNCTEX\ is not used.

In \CONTEXT\ we don't use the built|-|in \SYNCTEX\ features, that is: we let
filename and line numbers be set but often these are overloaded explicitly. The
output file is not compressed and constructed by \CONTEXT. There is no benefit in
compression and the files are probably smaller than default \SYNCTEX\ anyway.

\stopsection

\startsection[title=Commands]

Although you can enable this mechanism with directives it makes sense to do it
using the following command.

\starttyping
\setupsynctex[state=start]
\stoptyping

The advantage of using an explicit command instead of some command line option is
that in an editor it's easier to disable this trickery. Commenting that line will
speed up processing when needed. This command can also be given in an environment
(style). On the command line you can say

\starttyping
context --synctex somefile.tex
\stoptyping

A third method is to put this at the top of your file:

\starttyping
% synctex=yes
\stoptyping

Often an \XML\ files is very structured and although probably the main body of
text is flushed as a stream, specific elements can be flushed out of order. In
educational documents flushing for instance answers to exercises can happen out of
order. In that case we still need to make sure that we go to the right spot in
the file. It will never be 100\% perfect but it's better than nothing. The
above command will also enable \XML\ support.

If you don't want a file to be accessed, you can block it:

\starttyping
\blocksynctexfile[foo.tex]
\stoptyping

Of course you need to configure the viewer to respond to the request for
editing. In Sumatra combined with SciTE the magic command is:

\starttyping
c:\data\system\scite\wscite\scite.exe "%f" "-goto:%l"
\stoptyping

Such a command is independent of the macro package so you can just consult the
manual or help info that comes with a viewer, given that it supports this linking
back to the source at all.

If you enable tracing (see next section) you can what has become clickable.
Instead of words you can also work with ranges, which not only gives less runtime
but also much smaller \type {.synctex} files. Use

\starttyping
\setupsynctex[state=start,method=min]
\stoptyping

to get words clickable and

\starttyping
\setupsynctex[state=start,method=max]
\stoptyping

if you want somewhat more efficient ranges. The overhead for \type {min} is about
10 percent while \type {max} slows down around 5 percent.

\stopsection

\startsection[title=Tracing]

In case you want to see what gets synced you can enable a tracker:

\starttyping
\enabletrackers[system.synctex.visualize]
\enabletrackers[system.synctex.visualize=real]
\stoptyping

The following tracker outputs some status information about \XML\ flushing. Such
trackers only make sense for developers.

\starttyping
\enabletrackers[system.synctex.xml]
\stoptyping

\stopsection

\startsection[title=Warning]

Don't turn on this feature when you don't need it. This is one of those mechanism
that hits performance badly.

Depending on needs the functionality can be improved and|/|or extended. Of course
you can always use the traditional \SYNCTEX\ method but don't expect it to behave
as described here.

\stopsection

\stopchapter

\stopcomponent
