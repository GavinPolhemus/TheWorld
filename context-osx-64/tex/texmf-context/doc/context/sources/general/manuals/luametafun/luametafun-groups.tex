% language=us runpath=texruns:manuals/luametafun

\environment luametafun-style

\startcomponent luametafun-groups

\startchapter[title={Groups}]

This is just a quick example of an experimental features.

\startbuffer
\startMPcode
    fill fullcircle scaled 2cm shifted ( 5mm,2cm) withcolor "darkblue" ;
    fill fullcircle scaled 2cm shifted (15mm,2cm) withcolor "darkblue" ;

    fill fullcircle scaled 2cm shifted ( 5mm,-2cm) withcolor "darkgreen" ;
    fill fullcircle scaled 2cm shifted (15mm,-2cm) withcolor "darkgreen" ;

    draw image (
        fill fullcircle scaled 4cm                 withcolor "darkred" ;
        fill fullcircle scaled 4cm shifted (2cm,0) withcolor "darkred" ;

        setgroup currentpicture to boundingbox currentpicture
            withtransparency (1,.5) ;
    ) ;

    draw image (
        fill fullcircle scaled 3cm                 withcolor "darkyellow"
            withtransparency (1,.5) ;
        fill fullcircle scaled 3cm shifted (2cm,0) withcolor "darkyellow"
            withtransparency (1,.5) ;
    ) ;

    addbackground withcolor "darkgray" ;
\stopMPcode
\stopbuffer

\typebuffer[option=TEX]

A group create an object that when transparency is applied is treated as
a group.

\startlinecorrection
    \getbuffer
\stoplinecorrection

(Groups might become more powerful in the future, like reusable components but
then some more juggling is needed.)

\stopchapter

\stopcomponent
