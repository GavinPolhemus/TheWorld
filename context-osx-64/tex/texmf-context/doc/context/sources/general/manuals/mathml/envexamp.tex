% language=us runpath=texruns:manuals/mathml

\startenvironment envexamp

% this is an old style. only slightly updated to mkiv

\usemodule[abr-02,mathml]

\setupdocument % defaults
  [color=darkred,
   columns=2,
   title=MathML in \ConTeXt]

\startmode[atpragma]
    \setupbodyfont[lucidanova]
\stopmode

\startnotmode[atpragma]
    \setupbodyfont[palatino]
\stopnotmode

\setupbodyfont
  [10pt]

\definepapersize
  [mml]
  [width=20cm,
   height=20cm]

\setuppapersize
  [mml]
  [mml]

\definecolor[lightgray] [s=.85]
\definecolor[pagegray]  [s=.8]
\definecolor[mediumgray][s=.5]
\definecolor[darkgray]  [s=.4]
\definecolor[darkred]   [r=.65]
\definecolor[darkgreen] [g=.65]
\definecolor[darkblue]  [b=.65]
\definecolor[darkyellow][r=.65,g=.65]

\definepalet
  [XMLcolorpretty]
  [  prettyone=darkred,
     prettytwo=darkgreen,
   prettythree=darkblue,
    prettyfour=darkgray]

\definecolor
  [maincolor]
  [\documentvariable{color}]

\setuptyping
  [option=XML]

\setupwhitespace
  [big]

\setupinteraction
  [state=start,
   menu=on,
   color=maincolor,
   contrastcolor=maincolor]

\setuplayout
  [backspace=1cm,
   topspace=1cm,
   bottomspace=7.5mm,
   header=0pt,
   footer=0pt,
   bottomdistance=.5cm,
   bottom=1cm,
   height=17cm,
   width=middle]

\defineoverlay [mmlpage]  [\uniqueMPgraphic{mmlpage}]
\defineoverlay [mmltitle] [\uniqueMPgraphic{mmltitle}]
\defineoverlay [formula]  [\uniqueMPgraphic{formula}]

\startuseMPgraphic{mml}
    z1 = llcorner Page ;
    z2 = .5[lrcorner Page,urcorner Page] ;
    z3 = ulcorner Page ;
    fill
        Page
        withcolor \MPcolor{lightgray} ;
    fill
        z1 shifted (1cm,0) --
        z2 shifted (0,-.5cm) --
        lrcorner Page -- cycle
        withcolor \MPcolor{pagegray} ;
    fill
        z3 shifted (1cm,0) --
        z2 shifted (0,+.5cm) --
        urcorner Page -- cycle
        withcolor \MPcolor{pagegray} ;
\stopuseMPgraphic

\startuniqueMPgraphic{mmlpage}
    StartPage ;
        \includeMPgraphic{mml}
        fill
            llcorner Page --
            .5[lrcorner Page,urcorner Page] --
            ulcorner Page -- cycle
            withcolor \MPcolor{pagegray} ;
        Page := Page enlarged -.5cm ;
    StopPage ;
\stopuniqueMPgraphic

\startuniqueMPgraphic{mmltitle}
    StartPage ;
        \includeMPgraphic{mml}
        fill z1--z2--z3--cycle withcolor \MPcolor{maincolor} ;
        picture p ;
        p := textext("\documentvariable{title}") ;
        p := p xsized (.75length(z2-z1)) ;
        p := p rotatedaround(center p, angle z2) ;
        p := p shifted -center p shifted .5[z1,z2] ;
        p := p shifted ((unitvector(.5[z1,z2]) rotated 90)*1cm) ;
        draw p withcolor \MPcolor{lightgray} ;
        Page := Page enlarged -.5cm ;
    StopPage ;
\stopuniqueMPgraphic

\startuniqueMPgraphic{formula}
    draw
        OverlayBox
        withpen pensquare scaled 2mm
        withcolor \MPcolor{lightgray} ;
    fill
        OverlayBox
        withcolor \MPcolor{mediumgray} ;
\stopuniqueMPgraphic

\setupbottom
  [style=bold,
   color=darkgray]

\setuplist
  [section]
  [alternative=a,
   interaction=all,
   pagenumber=no,
   width=0pt,
   style=\bfb,
   color=darkgray,
   contrastcolor=darkgray,
   before={\blank[2*big]},
   after={\blank\startcolumns[n=5]\placelist[subsection]\stopcolumns}]

\setuplist
  [subsection]
  [alternative=f,
   interaction=all]

\setuphead
  [section]
  [page=yes,
   style=\bfd,
   color=darkgray,
   number=no,
   after={\blank[2*big]\startcolumns[n=5]\placelist[subsection]\stopcolumns\page}]

\setuphead
  [subsection]
  [after=,
   placehead=empty]

\setuphead
  [subject]
  [style=\bfb,
   color=darkgray,
   after={\blank[2*big]}]

\setupinteractionmenu
  [bottom]
  [state=start,
   frame=off,
   left=\hskip3cm,
   middle=\quad]

\startinteractionmenu[bottom]
    \startbut [content]      content \stopbut
    \startbut [colofon]      colofon \stopbut
    \startbut [index]        index   \stopbut
    \startbut [PreviousJump] go back \stopbut
    \startbut [previouspage] \bfa--  \stopbut
    \startbut [nextpage]     \bfa+   \stopbut
    \hfill
    \starttxt
        \color[darkgray]{\markcontent{\getmarking[section]: }\getmarking[subsection]\removemarkedcontent}
    \stoptxt
\stopinteractionmenu

\starttexdefinition unexpanded ShowFormula #1#2#3
    \showXMLformula{\rawstructurelistuservariable{filename}.xml}
\stoptexdefinition

\starttexdefinition unexpanded showXMLformula #1
    \framed [
        strut=no,
        background=formula,
        foregroundcolor=white,
        frame=off,align=normal,
        width=\hsize
    ] {
        \vbox {
            \processXMLfile{#1}\endgraf
        }
    }
\stoptexdefinition

\starttexdefinition unexpanded showXMLsample #1
    \page
    \bgroup
    \startbaselinecorrection
        \showXMLformula{#1.xml}
    \stopbaselinecorrection
    \startsubsection[reference=#1,title=#1,marking=#1][filename=#1]
        \switchtobodyfont
            [8pt]
        \startcolumns[balance=no,n=\documentvariable{columns}]
            \typefile{#1.xml}
        \stopcolumns
        \vfill
        \page
    \stopsubsection
    \egroup
\stoptexdefinition

\startsetups[document:start]

    \setupbackgrounds
      [page]
      [background=mmltitle]

    \startstandardmakeup
        \setupalign[left]
        \bgroup
            \darkgray \bfd \setupinterlinespace
            Examples  \vfill
            Hans Hagen\par
            PRAGMA ADE\par
            \vskip-\dp\strutbox
            \vskip-1cm
        \egroup
        \vskip\dp\strutbox
        \vskip1pt
    \stopstandardmakeup

    \setupbackgrounds
      [page]
      [background=mmlpage]

    \startsubject[reference=content,title={Content}]
        \placelist[section]
    \stopsubject

    \startsubject[reference=colofon,titlr={Colofon}]
        \getbuffer[colofon]
    \stopsubject

\stopsetups

\startsetups[document:stop]

    \page

    \pagereference[index]

    \setuplist
      [section]
      [alternative=a,
       before={\testpage[5]},
       after={\blank[medium]\placelist[subsection]}]

    \setuplist
      [subsection]
      [alternative=vertical,
       before=\startbaselinecorrection,
       after=\stopbaselinecorrection\blank,
       color=,
       contrastcolor=,
       command=\ShowFormula,
       interaction=all]

    \placelist[section][criterium=text]

\stopsetups

\stopenvironment
