% \enablemode[print]

\usemodule[present-overlap,abr-02]

\startdocument
  [title=e-books,
   subtitle=old wine in new bottles,
   location=\ConTeXt\ Meeting 2011]

\Topic{Some observations}

\StartSteps
\startitemize
\item Most ebooks are just books (or try to be). \FlushStep
\item Only a small portion has (or needs) design. \FlushStep
\item To what extent appreciation matters is hard to measure. \FlushStep
\item Vendor locking is spoiling much. \FlushStep
\item 10 years of low res screens have made readers tolerant. \FlushStep
\item Publishers already lost the edge. \FlushStep
\item Eventually authors will publish themselves. \FlushStep
\stopitemize
\StopSteps

% Does quality matter (just look around you)
% Does livetime matter (much tolerance for potentially crappy old stuff anyway)
% Does timebound look and feel matter (it helps to put into perspective)

\Topic{What is an ebook}

\StartSteps
\startitemize
\item Nicest is it being a \PDF\ (some design). \FlushStep
\item Easiest is it being an \XHTML\ file (with some \CSS). \FlushStep
\item Pointless it is being a frozen app. \FlushStep
\item We can already provide a \PDF\ for paper and screen for quite a while. \FlushStep
\item We can consider providing an \XHTML\ alongside as reflowable variant. \FlushStep
\item Who knows what we can provide in the future. \FlushStep
\stopitemize
\StopSteps

\Topic{The starting point}

\StartSteps
\startitemize
\item No output is better than the input. \FlushStep
\item Fixing bad coding is a waste of energy. \FlushStep
\item Not that many publishers want to invest in coding. \FlushStep
\item Not that many tools enforce structure. \FlushStep
\item The real good devices still have to come but we can be ready for it. \FlushStep
\item The intelligence has to be in the macro package, not in the engine. \FlushStep
\stopitemize
\StopSteps

\Topic{Implementation}

\StartSteps
\startitemize
\item Some users have to produce tagged pdf (to satisfy validators). \FlushStep
\item It helps that many commands in \CONTEXT\ are built upon a relatively small generic set. \FlushStep
\item So, given some basic structure, supporting tags is easy. \FlushStep
\item It integrates in the core. \FlushStep
\item And as a side effect an export was equally easy to support. \FlushStep
\item The overhead is not that large (upto 10\%). \FlushStep
\item Instead of going for freaky solutions (no need for challenges) we assume proper structure. \FlushStep
\item It's not to hard to extend the current features. \FlushStep
\stopitemize
\StopSteps

\Topic{Consequence for users}

\StartSteps
\startitemize
\item Use \type{\startchapter} cum suis, but that was already the \MKIV\ fashion. \FlushStep
\item Preferable use \type{\startitem} instead of \type {\item}. \FlushStep
\item Playing safe means tagging paragraphs with \type {\startparagraph}. \FlushStep
\item Use style environments instead of font switches. \FlushStep
\item Just use whatever structural markup that \CONTEXT\ already provides for ages. \FlushStep
\stopitemize
\StopSteps

\Topic{An example}

\StartSteps
\startitemize
\item A normal input with \TEX\ commands (cld-mkiv.tex) \FlushStep
\item Regular (tagged) output (cld-mkiv.pdf) \FlushStep
\item Structured output (cld-mkiv.xml) \FlushStep
\item Reflowable output (cld-mkiv-export.xhtml) \FlushStep
\item Dynamic styling (cld-mkiv-export.css) \FlushStep
\item Basic template (cld-mkiv-export.template) \FlushStep
\stopitemize
\StopSteps

\stopdocument
