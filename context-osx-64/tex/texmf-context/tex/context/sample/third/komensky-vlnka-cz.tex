Nejkrásnějším dílem Božím je svět, všeliké neviditelné Božské věcí viditelné
ukazující. Nejkrásnější je ale naše duše, jíž je dáno, myšlením zobrazovat ve
svém nitru svět i~věci veškery. Nejkrásnější je řeč, jejíž pomocí vmalováváme
všeliké obrazy své duše v~duši druhého. Nejkrásnější je písmo, jímž zachycujeme
a~natrvalo upevňujeme řeč, samu o~sobě prchavou a~pomíjející, a~jímž jako bychom
ji zadržovali, aby trvala. Nejkrásnější jsou knihy, písmem ladně sestavené, jimiž
zpodobenou moudrost posíláme lidem, místně či časově vzdáleným, ba dokonce
i~pozdnímu potomstvu. Nejkrásnějším darem Božím je vynález tiskových liter, jimž
se knihy nesmírně rychle rozmnožují.

Sazeč, maje před sebou kasu, naplněnou kovovými literkami, uloženými
v~přihrádkách, dívá se do rukopisu, postaveného na vidlici tak, aby pohodlně
viděl, v~levé ruce sázítko, pravou rukou vybírá (z~přihrádek) literky a~skládá je
do sázítka v~slova: když naplní sloupec, přenáší je na desku. Jakmile pak je
naplněn počet sloupců pro jednu stranu archu, rozdělí je příložkami, ováže je
a~stáhne železnými rámy, aby se nerozpadly; a~tím se podle svého mínění své
povinnosti zhostil.

A~tak tito všichni pracujíce vzájemně vykonávající jakoby hračkou podivuhodné
dílo, lidem kdysi nepochopitelné, jen když mají všichni to, čeho je přitom
potřebí, jednak odborné znalosti zbystřené praxí, jednak vytrvalou pozornost
a~píli. Znalosti: poněvadž třebaže je tiskařství řemeslem, je nejsubtilnějším
uměním, skládajícím se z~přečetných maličkostí, na něž nutno dávat dobrý pozor.
Pozornost a~píli: poněvadž zde, kde se stýká tolik maličkostí, velmi snadno se
něco přihodí, co by dílo rušilo, kdyby se nebdělo.

Tolik ve všeobecnosti o~tom, jak se tisknou knihy vůbec. Jestliže by se naskytla
otázka, jak vznikají knihy dobré, bude třeba říci, že je žádoucí za prvé rozumně
sepsaný rukopis, hodný světla a~nikoli tmy; na druhém místě elegantní typy; na
třetím čistý papír; na čtvrtém pak pozorná práce, aby vše až do poslední čárky
bylo zřetelné, rozlišené a~správné.
