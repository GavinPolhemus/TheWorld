\startcomponent luagraph-fit

\environment luagraph-environment

\startchapter [title={Least squares \quote {fit} to data}]

\placeinitial
When drawing discrete data, one often looks to smooth the data or
describe it by some functional form. As used earlier, binning the data
calculating averages not only can reduce the number of points or
sampling resolution of a data set, it also smooths variations or
fluctuations in the sampled data. Another common smoothing function is a
piecewise spline of $n$th order polynomials, the most common being the
cubic spline. Of course, \METAPOST\ can automatically calculate a Bézier
curve passing through a set of points (and control points) using the
\type {..} operator.

In this chapter we describe a least|-|squares \quote {fit} to an $n$th
degree polyomial function and variants thereof that can be achieved
through simple coordinate transformations. The calculation follows
\cite[authoryears] [Bevington1969] and its use will best be seen
through examples. It is not a fit or adjustment with an optimization
search of parameter values with convergence (or non|-|convergence) as in
non|-|linear least squares fitting, rather it is an exact calculation
from the data values for which there is always a solution minimizing the
(weighted) least|-|squares deviation.

The $n$th degree polynomial function is of course:
\placeformula
    \startformula
        y = ∑_{i=0}^n a_ix^i
    \stopformula
A zeroth|-|degree polynomial returns an average; a first|-|degree
polynomial is a straight line or a linear function; a second|-|degree
polynomial is a quadratic function, and so forth. An example:
\startfootnote
The weights are all taken equal in this example.
\stopfootnote

\startbuffer [quarter]
\startMPcode{graph}
color c[] ; c2 := red ; c3 := green ; c4 := blue ;
startgraph(.35TextWidth)
    setscale(6,0,12,1200) ;
    path secondquarter ;
    secondquarter := subpath (.25length r,.5length r) of r ;
    gdraw secondquarter withsymbol 0 ;
    numeric a[], chisq ; % coefficients
    path csq ;
    path qq ;
    for n=2 upto 7 :
        qq := polynomialfit.secondquarter(n,a,chisq,1) ;
        augment.csq(n,chisq) ;
        if known c[n] :
            gdraw qq withpen pencircle scaled 2
            withcolor c[n] withtransparency (1,.5) ;
        fi
    endfor
    autogrid.llft(1,5) ;
    label.lft(textext("Irradiance (W/m²)") rotated 90,frame)
                                    shifted (left*3EmWidth) ;
    label.bot("time (hours)",frame) shifted (down*LineHeight) ;
    drawarrow frame.bot ;
    drawarrow frame.lft ;
stopgraph ;
\stopMPcode
\stopbuffer

\startbuffer [chisq]
\startMPcode{graph}
startgraph(.35TextWidth)
    setscale(1,0,8,3500) ;
    gdraw csq ;
    for n=2 upto 7:
        gdraw point (n-2) of csq withsymbol 10
            if known c[n] : withcolor c[n] fi ;
    endfor
    autogrid.llft() ;
    label.lft("$χ^2$",frame)
        shifted (left*3EmWidth) ;
    label.bot("polynomial terms $n$",frame)
        shifted (down*LineHeight) ;
    drawarrow frame.bot ;
    drawarrow frame.lft ;
stopgraph ;
\stopMPcode
\stopbuffer

\typebuffer [quarter]

\startplacefigure [location=force,
    title={Polynomial fit to the morning hour data:
           \startcaptionitemize [a]
               \startitem $n=2$ or linear (red),
                          $n=3$ or quadratic (green),
                          $n=4$ (blue); \stopitem
               \startitem $χ_n^2$ of each fit. \stopitem
           \stopcaptionitemize}]

    \startcombination [nx=2,ny=1,alternative=label]
        {\getbuffer [quarter]}
        {\getbuffer [chisq]}
    \stopcombination
\stopplacefigure

The second graph uses the command \type {augment.p(x,y)} taken from the
Hobby graph macros that makes point|-|by|-|point additions to a path
variable.

\typebuffer [chisq]

\startsubject [title={Least squares \quote {fit} to other functions}]

A few but very useful and important functions can be reduced to a
polynomial forms through appropriate transformations with certain
limitations. One case is a Gaussian function describing a Normal
distribution:

\placeformula
    \startformula
        y = A \exp\left[-\ln2\left(\frac{x-x_0}{h}\right)^2\right]
    \stopformula

where $x_0$ is the peak center, $A$ is the peak amplitude, and $h$ is a
parameter describing the distribution half|-|width at half|-|maximum.
Such a peak can be either strictly positive ($A>0$) or negative ($A<0$).
\startfootnote
Of course, it could be trivially identically zero $∀x$ when $A=0$.
\stopfootnote
When limited to the positive case (or with a simple change of sign), one
can take the logarithm of the ordinate to obtain:
\startfootnote
The polynomial least|-|squares adjustment is then given a weight for
each point that is equal to the exponential of the ordinate.
\stopfootnote

\placeformula
    \startformula
        \ln y = \ln A - \ln2 \left(\frac{x-x_0}{h}\right)^2
    \stopformula

that one recognizes as a quadratic polynomial with

\placeformula
    \startformula
        a_0 = \ln A - \ln2\left(\frac{x_0}{h}\right)^2
        ~~;~~
        a_1 = \frac{2\ln2 x_0}{h^2}
        ~~;~~
        a_2 = \frac{-\ln2}{h^2}
    \stopformula

from which one finds:
\startfootnote
Note that not all data can be (even approximately) described by a
Gaussian peak, and a polynomial \quote {fit} to the (logarithm of the)
data will certainly not give very good results if the coefficient of the
quadratic ($x^2$) term, $a_2$, that best fits the data is found to be positive.
\stopfootnote

\placeformula
    \startformula
        x_0 = \frac{-a_1}{2a_2}
        ~~;~~
        h = \sqrt{\frac{-\ln2}{a_2}}
        ~~;~~
        A = exp\left(a_0 - \frac{a_1^2}{4a_2}\right)
        ~~.
    \stopformula

Our implementation of the Gaussian function maps these parameters in the
order given above as \type {c0} ($x_0$), \type {c1} ($h$), and \type
{c2} ($A$) where \type {numeric c[] ;} is the numeric pseudo-array for
of the \emphasis {suffix} \type {c}. It takes an optional fourth
parameter (\type {c3}) that is a constant \quote {background}, floor or
bias to be subtracted from the data. Importantly, this parameter is
taken as fixed and is not, cannot be adjusted in the polynomial \quote
{fit}.

Another important case is a Lorentzian peak (which is a Fourier transform
of an exponential decay function):

\placeformula
    \startformula
        y = \frac{A}{1 + \left(\frac{x-x_0}{h}\right)^2}
        ~~.
    \stopformula

This peak is also strictly positive ($A>0$) or negative ($A<0$).
For data values that differ from zero, one can take the inverse:

\placeformula
    \startformula
        \frac1y = \frac1A\left[1 + \left(\frac{x-x_0}{h}\right)^2\right]
    \stopformula

that we also recognize as a quadratic polynomial with

\placeformula
    \startformula
        a_0 = \frac1A\left[1 + \left(\frac{x_0}{h}\right)^2\right]
        ~~;~~
        a_1 = \frac{-2x_0}{A h^2}
        ~~;~~
        a_2 = \frac1{A h^2}
    \stopformula

leading to:

\placeformula
    \startformula
        x_0 = \frac{-a_1}{2a_2}
        ~~;~~
        h = \sqrt{\frac{a_0}{a_2} - \left(\frac{a_1}{2a_2}\right)^2}
        ~~;~~
        A = \frac1{a_0 - \frac{a_1^2}{4a_2}}
        ~~.
    \stopformula

So from a set of data one can use the inverse of the ordinate to
determine coordinates minimizing the least squares deviation to a
Lorentzian peak function, simply discarding points having an ordinate
identical to zero. Such a function may not appropriately describe the
data at all, as will be shown in the example below.

\startbuffer
\startMPcode{graph}
vardef Cosinesquared@#(suffix $)(expr x) =
    if numeric x :
        % y = a_2 cos^2[π/4(x - a_0)/a_1] + a_3
        $2*((cosd(45*(x-$0)/$1))**2) if known $3 : + $3 fi
    elseif path x :
        hide(save p ; path p ; p = x) Cosinesquared.p($,false)
    else :
       if str @# <> "" :
           if path @# : 
               hide(save x_i ; def JOIN=hide(def JOIN=--enddef)enddef)
               for i inpath @# :
                   hide(x_i := xpart pointof i)
                   JOIN (x_i,Cosinesquared($,x_i))
               endfor if cycle @# : JOIN cycle fi
           fi
       fi
    fi
enddef ;

startgraph(.8TextWidth,.6TextWidth)
    setscale(0,0,24,1300) ;
    gdraw r withsymbol 0 ;
    save a ; numeric a[], chisq ; % coefficients
    gdraw   Gaussianfit.r(a,chisq) withpen currentpen scaled 2 ;
    save h ; path h ; h = (a0-a1,Gaussian(a,a0-a1))--
                          (a0+a1,Gaussian(a,a0+a1)) ;
    gdrawdblarrow h ;
    gdotlabel.bot("fwhm\strut", point .5 of h) ;
    a1 := 1.15a1 ; a2 := 1200 ;
    gdraw Cosinesquared.r(a,false)     withcolor blue ;
    gdraw polynomialfit.r(7,a,chisq,1) withcolor cyan ;

    save s ; path s ;
    s = subpath ((8/24)*length r,(17.5/24)*length r) of r ;
    save a ; numeric a[], chisq ; % coefficients
    gdraw Lorentzianfit.s(a,chisq)     withcolor red ;
    gdraw    Lorentzian.r(a,false)     withcolor red
        dashed evenly ;
    autogrid.llft(5) ;
    label.lft(textext("Irradiance (W/m²)") rotated 90,frame)
                                    shifted (left*3EmWidth) ;
    label.bot("time (hours)",frame) shifted (down*LineHeight) ;
    drawarrow frame.bot ;
    drawarrow frame.lft ;
stopgraph ;
\stopMPcode
\stopbuffer

\startplacefigure [location=force,
    title={Gaussian function (in black), Lorentzian (red), $\cos^2$
    (blue) and a sixth|-|order polynomial (cyan).},]
    \getbuffer
\stopplacefigure

The Gaussian (in black) and Lorentzian (in red) peak functions can be
compared to the previously plotted data, neither of which are very good
descriptions. Also plotted is a $\cos^2$ function (in blue), as might
be expected to do a little better;
\startfootnote
Of course, this function is only drawn, not adjusted to the data
using the polynomial \quote {fit}.
\stopfootnote
a sixth|-|order polynomial (shown in cyan), or greater, approximates
this function fairly well.

The Lorentzian fit is applied only to a segment of the data, 8:00–17:30,
with the tails outside of this domain drawn as a dashed line. Including
more data to this functional form would force the hwhm parameter (\type
{a[1]}) to be much smaller and the amplitude (\type {a[2]}) to be much
larger, and this brings up an interesting point: in the inverse space
($1/y$) of the polynomial fit, the peak $1/y_{\mathrm{max}}$ approaches
zero and the polynomial minimum can take on negative values. The
resulting Lorentzian function will not be a good fit due to this sign
inversion.

\typebuffer

Two other simple yet very useful functions are the \type {exponential}
($y = a_1 \exp \left( a_0 x \right)$) and the \type {powerlaw} ($y = a_1
x^{a_0}$), the first being a straight line in a lin|-|log plot and the
second being a straight line in a log|-|log plot.

The following example takes its data from the following \italic {url} of
rooftop photovoltaic solar installations in the City of Fort Collins in
Colorado:\
\goto
{https://opendata.fcgov.com/Environmental-Health/Solar-Installations/3ku5-x4k9}
[url
(https://opendata.fcgov.com/Environmental-Health/Solar-Installations/3ku5-x4k9)].

\startbuffer
\startMPcode {graph}
f1 := "Fort_Collins_Solar_Installations.csv" ;
n1 := lua.mp.CSVDataLoad(f1) ;                                               

startgraph (figure "City-solar-panels-Riverside"
            xsized (TextWidth-4LineHeight))
    label.top("Fort Collins solar PV", frame)
        withcolor darkgreen ;
    setcoordinates(date,log) ;
    dateepoch :=  lua.mp.Time(2000,1,1) ;
    datescale := (lua.mp.Time(2020,1,1) -
                  dateepoch)/(20-.5/365) ;
    zero := dateepoch/datescale ;
    Dateform := "@d/@m/@Y" ;
    path p,q ;
    p := (lua.mp.CSVDataPath(f1,"years",3)
        shifted (-zero,0))
        yscaled (1/1000) ;
    numeric sum ; sum := 0 ;
    pair pt ;
    def dashdash=hide(def dashdash=--enddef) enddef ;
    q := for i=0 upto length p :
          hide(pt := point i of p; sum := sum + ypart pt)
          dashdash (xpart pt,sum)
          endfor ;
    setscale((-1,.001),(20,20)) ;
    autogrid.llft(5,9) withcolor darkgreen ;
    gdraw q withpen currentpen scaled 3 withcolor darkgreen ;
    save a ; numeric a[], chisq ; % coefficients
    gdraw exponentialfit.q(a,chisq) withcolor red ;
    label.lft (textext("installed capacity (MW-DC)\strut")
        rotated 90, frame)
        shifted (left*1.5LineHeight)
        withcolor darkgreen ;
    draw frame ;
stopgraph ;
\stopMPcode
\stopbuffer

The graph is framed over an image (of a 621~kW community solar array
that came online in July 2015).

\startplacefigure [location=top,
    title={Solar PV installations by date. The red line describes an
    exponential growth (a straight line in this semi|-|logarithmic plot).}]
    \getbuffer
    \getbuffer [luasyncdata]
\stopplacefigure

\typebuffer

Other non|-|linear functions that cannot be reduced to a simple
polynomial though some appropriate transformation can be drawn but not
so easily \quote {fit} to a set of data. In such cases, more
sophisticated routines can be successfully employed, examples can be
adaptive gradient searches of a non|-|linear parameter space or maximum
entropy methods that go beyond the scope of this manual.

\stopsubject

\stopchapter

\stopcomponent
