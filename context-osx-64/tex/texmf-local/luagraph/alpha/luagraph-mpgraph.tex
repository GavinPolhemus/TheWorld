\startcomponent luagraph-mpgraph

\environment luagraph-environment

\startchapter [title=John Hobby's graph macros]

\placeinitial
Let's first look at John Hobby's \text {graph} macros, revisited.

\blank [2*line]

\startquotation
We concentrate on the mechanics of producing particular kinds of graphs
because the question of what type of graph is best in a given situation
is covered elsewhere;
e.g., \cite[authoryears] [Cleveland1985,Cleveland1993,Cleveland1993a]
and \cite[authoryears] [Tufte1983].
The goal is to provide at least the power of UNIX\footnote{UNIX is a registered
trademark of UNIX System Laboratories, Inc.} \italic {grap} \cite [Bentley1990],
but within the MetaPost language.
Hence the package is implemented using MetaPost's powerful macro facility.

The \text {graph} macros provide the following functionality:
\startitemize [nowhite]
\item Automatic scaling
\item Automatic generation and labeling of tick marks or grid lines
\item Multiple coordinate systems
\item Linear and logarithmic scales
\item Separate data files
\item Ability to handle numbers outside the usual range
\item Arbitrary plotting symbols
\item Drawing, filling, and labeling commands for graphs
\stopitemize
In addition to these items, the user also has access to all the features
described in the MetaPost user's manual \cite [Hobby1992a].
These include access to almost all the features of PostScript,
ability to use and manipulate typeset text,
ability to solve linear equations,
and data types for points, curves, pictures, and coordinate transformations.
\stopquotation

These objectives remain actual and they have been expanded.
A first step was the simple adaptation of the \text {graph} macros
to double|-|precision \METAPOST, removing all special tricks and manipulations
needed under the standard scaled number system.
A second step, described here, is the liberation from the constraints
imposed by the \METAPOST\ syntax, using \LUA\ advantageously
where appropriate. We have also made the macros more \CONTEXT-friendly.

\startsection [title=A few examples]

\startbuffer
\usemodule [luagraph]
\stopbuffer

\typebuffer
%\getbuffer% done in luagraph-enviornment

in the preamble loads the module and creates the \METAPOST\ instance \type {graph}.

\startbuffer
\startMPcode{graph}
startgraph(.8TextWidth,.6TextWidth)
    autogrid.llft() withcolor .6white ;
    gdraw "agepop91.csv" ;
    draw frame ;
    label.lft(textext("population") rotated 90, frame)
                                    shifted (left*3LineHeight) ;
    label.bot("age (years)", frame) shifted (down*LineHeight) ;
stopgraph ;
\stopMPcode
\stopbuffer

\typebuffer

\startplacefigure
     [title=A graph of the 1991 age distribution in the United States.]
\getbuffer
\stopplacefigure

A few notes about differences between this example and the first example
of the Hobby \type {graph} macro package documentation:

\startitemize

\startitem
We use \type {startgraph} … \type {stopgraph ;} (rather than 
       \type {begingraph} … \type {endgraph ;}).%
\startfootnote
\type {begin} … \type {end} is more appropriate for users of \LATEX!
\stopfootnote
\stopitem

\startitem
The datafile expected is a \type {csv} format, using a comma as a field
separator.
\stopitem

\startitem
We explicitly ask to place a grid, draw a frame and add axis labels.
None of these get included by default.
The \italic {suffix} (\type {.llft}) determines where the grid numbering
gets placed, here at the bottom and left sides of the graph field.
Similarly for the frame: \type {draw frame.llft ;} would only draw the
left and bottom edges, for example.
\stopitem

\stopitemize

Objects get drawn in the order that they appear, so the line
of data overlies the scale grid. (You can see this by zooming in.)
Regular \METAPOST\ commands can be interspersed, and \type {draw} refers
to coordinates in the scale of the page whereas \type {gdraw} refers to
the data coordinate scale. Similarly for \type {label} and \type
{glabel}, etc. Notice that a special \METAPOST\ instance (\type {{graph}})
is used, in which the specific graph macros are defined (in addition to
all \METAFUN\ macros of \type {{doublefun}}).

Let's try some improvement of this graph:

\startbuffer
\startMPcode{graph}
startgraph()
    ticmarklength := 2EmWidth/3 ;
    setscale((0,0),(90,5)) ;
    autogrid.llft(1,2) withcolor .5white ;
    gdraw lua.mp.CSVDataPath("agepop91.csv") yscaled 1e-6 ;
    gplot gpath ;
   %gdraw gpath withsymbol 10 ;
    draw frame ;
    label.lft(textext("population (in millions)") rotated 90, frame)
                                    shifted (left*LineHeight) ;
    label.bot("age (years)", frame) shifted (down*LineHeight) ;
stopgraph ;
\stopMPcode
\stopbuffer

\typebuffer

\startplacefigure
     [title=The 1991 age distribution plotted with filled circles.]
\getbuffer
\stopplacefigure

In this example, the behavior of the autogrid macro is changed from
drawing a full grid to drawing and numbering tic|-|marks, by setting the
internal numeric variable \type {ticmarklength} to a non|-|zero value.
Starting a new graph without specifying the frame dimensions continues
using the previously used graph frame.

The \type {.csv} data file was read once into \LUA\ memory and is
recalled using the \LUA\ command \type{lua.mp.CSVDataPath()} returning a
path that can be scaled and otherwise manipulated. Reading the file once
saves a file \type {.tma} in the form of a \LUA\ table (and this file
can be deleted) as well as a binary file \type {.tmc} that is a
\quotation {compiled} version. Subsequent \CONTEXT\ runs can load this
file very quickly, efficiently, and automatically
as long as the original source file is not changed
(and the \type {.tmc} file is not deleted)!

The data path that was drawn is made available through the symbol
\type {gpath} (of type path, of course). This holds the path
used by the previous \type {gdraw} command.
The command \type {gplot gpath ;} is equivalent to
\type {gdraw gpath withsymbol 0 ;}
Symbol~0 is an open circle
and can also be obtained \type {withsymbol} $○$ ;
Unicode \quote {WHITE CIRCLE} (U+25CB).%
\startfootnote
Symbol \quotation {10} would yield filled circles,
and these can also be obtained \type {withsymbol} $●$ ;
Unicode \quote {BLACK CIRCLE} (U+25CF).
\stopfootnote
Notice that these do not use simply a font character
but rather a \METAPOST\ picture
including an outline \quote {halo},
helping distinguish overlapping data symbols.

The scale of the graph is also explicitly set through the command
\type {setcoordinates()}.
As shown here, two \italic {pairs} are given corresponding to
$(xmin,ymin)$ and $(xmax,ymax)$, and these can alternately
specified as four numbers, not necessarily grouped into pairs.
Any one of these can be the \METAPOST\ symbol \type {whatever},
which will then be auto|-|scaled.

\stopsection

\startsection [title=Coordinate systems]

By default, the mapping between data coordinates and drawing coordinates
is linear, where the relationship between the two is obtained through a
standard \type {transformation} in the \METAPOST\ sense.

The Hobby \type {graph} macros also allows for logarithmic (base~10)
scales, both full and semi|-|log ($x$ or $y$). Using \LUA, we go beyond
that and almost any functional mapping can be used including, notably,
polar coordinates. In fact, the mapping of data need not even be
2D$\rightarrow$2D, for \LUA\ data tables can be $n$|-|dimensional.
For now, however, we have programmed only 2D mappings with include
base~10 and base~2 logarithms, square root, polar ($θ,r$), with more to
come.

\startbuffer
\startMPcode{graph}
startgraph()
    setcoordinates(Log,Log) ;
    setscale(10,.001,500,500) ;
    autogrid.llft(9,9) withcolor .5white ;

    gdraw "matmul.csv" dashed evenly ;
    glabel.ulft("Standard",8) ;
    gdraw lua.mp.CSVDataPath("matmul.csv",1,3) ;
    glabel.lrt ("Strassen",8) ;

    draw frame ;
    label.lft(textext("timing (seconds)") rotated 90, frame)
                                    shifted (left*LineHeight) ;
    label.bot("matrix size", frame) shifted (down*LineHeight) ;
stopgraph ;
\stopMPcode
\stopbuffer

\typebuffer

By default, the first two columns of the data file are extracted;
the second \type {gdraw} uses the \LUA\ function \type
{lua.mp.CSVDataPath()} to extract the first and third columns
in this example.

The logarithm base $10$ scale is illustrated through the placement of
intermediate, \quotation {minor} tic marks, selected for both the $x$-
and $y$-scales (abscissa and ordinate) through the pair \type {(9,9)}
passed to the command \type {autogrid}.

\startplacefigure
     [title=Timings for two matrix multiplication algorithms.]
\getbuffer
\stopplacefigure

\startbuffer
\startMPcode{graph}
startgraph()
    draw frame ;
    setcoordinates(LinLin) ;
    setscale(1980,whatever,1990,whatever) ;
    autogrid.llft() withcolor .5white ;
    gdraw "lead.csv" ;
    glabel.lft("$\longleftarrow~$", 4.2) ;
    label.lft(textext("Emissions (thousands of metric tons)")
                      rotated 90, frame)
                             shifted (left*LineHeight) ;
    label.bot("Year", frame) shifted (down*LineHeight) ;

    setscale(1980,whatever,1990,whatever) ;
    autogrid.rt  () withcolor .5white ;
    gdraw lua.mp.CSVDataPath("lead.csv",1,3) ;
    glabel.rt ("$~\longrightarrow$", 3) ;
    label.rt (textext("Lead concentration (micrograms per cubic meter of air)")
                      rotated 90, frame)
                             shifted (right*1.5LineHeight) ;
stopgraph ;
\stopMPcode
\stopbuffer

\typebuffer

\startplacefigure
     [title=Annual lead emissions and average level at atmospheric
     monitoring stations in the United States.]
\getbuffer
\stopplacefigure


\stopsection

\stopchapter

\stopcomponent
