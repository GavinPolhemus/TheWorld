% !TEX useAlternatePath
% !TEX useConTeXtSyncParser

\startcomponent chapter06
\project project_world
\product prd_volume01

\doifmode{*product}{\setupexternalfigures[directory={chapter06/images}]}

%%%%%%%%%%%%%%%%%%%%%%%%%%%%%
\startchapter[title=Celestial Mechanics,reference=ch:Hamilton]
%%%%%%%%%%%%%%%%%%%%%%%%%%%%%

\placefigure[margin,none]{}{\small
	\startalignment[flushleft]
The theoretical development of the laws of motion of bodies is a problem of such interest and importance, that it has engaged the attention of all the most eminent mathematicians, since the invention of dynamics as a mathematical science by \scaps{Galileo}, and especially since the wonderful extension which was given to that science by \scaps{Newton}. Among the successors of those illustrious men, \scaps{Lagrange} has perhaps done more than any other analyst, to give extent and harmony to such deductive researches, by showing that the most varied consequences respecting the motions of systems of bodies may be derived from one radical formula; the beauty of the method so suiting the dignity of the results, as to make of his great work a kind of scientific poem\dots.
%But the science of force, or of power acting by law in space and time, has undergone already another revolution, and has become already more dynamic, by having almost dismissed the conceptions of solidity and cohesion, and those other material ties, or geometrically imaginably conditions, which \scaps{Lagrange} so happily reasoned on, and by tending more and more to resolve all connexions and actions of bodies into attractions and repulsions of points: and while the science is advancing thus in one direction by the improvement of physical views, it may advance in another direction also by the invention of mathematical methods.
And the method proposed in the present essay, for the deductive study of the motions of attracting or repelling systems, will perhaps be received with indulgence, as an attempt to assist in carrying forward so high an inquiry.%\autocite[p.48]{Galileo1610}
	\stopalignment
	\startalignment[flushright]
	{\it On A General Method In Dynamics}\\
	{\sc William Rowan Hamilton}\\
	1805--1865
	\stopalignment
}

%\section[sec:GravUSolarSys]{Gravitational potential energy in the Solar System}

\Initial{I}{n the year that \booktitle{Hydrodynamica} was published,} 1738, Daniel Bernoulli wrote a paper on the motion of the Moon. Gravitational forces in the solar system had already been studied in some detail using Newton’s methods of momentum and force.
The Moon's motion is difficult to study using Newton's methods,  because it is pulled by both Earth and the Sun. Daniel Bernoulli attacked the problem using potential and kinetic energy, and made significant progress. %In this section, we will solve some easier problems using D.~Bernoulli's method.

\section{Gravitational potential energy in the Solar System}

The gravitational potential energy formula in \in{Chapter}[ch:PotentialEnergy], $U=mgh$, is only useful for relatively small heights near the Earth’s surface. Over the long distances between planets, moons, and the sun, the gravitational potential energy is a bit more complicated. The gravitational potential energy of any two objects is
\startformula
	U = -G\frac{mM}{r},
\stopformula
where $m$ and $M$ are the objects’ masses, $r$ is the distance between the objects’ centers, and $G=6.67\sci{-11}\units{m^3/kg\.s^2}$ is the universal gravitational constant.

This formula is a bit intimidating, but extremely valuable if used correctly. Notice four things about this formula. First, the distance between the objects is in the denominator. If the distance is extremely large, then their gravitational potential energy is extremely small. This is quite convenient because it means that very distant objects can be ignored. When studying the interaction of the Earth and Moon, it is not necessary to include the gravitational potential energy due to distant stars.

Second, the distance is between the objects’ centers. When calculating the gravitational potential energy of an object on the Earth’s surface the distance $r$ is Earth’s radius ($r\sEarth=6.37\sci{6}\units{m}$) not the height above Earth’s surface.

Third, the formula always gives a negative potential energy. This has alarmed many people, including great physicists, but it is not a problem. The gravitational potential energy is zero when the objects are far apart and it goes down as objects get closer together. Down from zero is negative. Kinetic energy is always positive, but there is no reason to be alarmed by negative potential energy.

Finally, the universal gravitational constant $G$ is extremely small, so the gravitational potential energy between every-day objects can be completely ignored. At least one of the objects must have a huge mass in order for the gravitational potential energy to be significant. Earth’s mass of $5.97\sci{24}\units{kg}$ is enough to have a significant effect.

\startexample[ex:BoxEarthGravU] How much work is required to lift a $3.0\units{kg}$ box from Earth’s surface to a location very far from Earth?

\startsolution
	This problem is just like the earlier example where Galileo lifted the pendulum (\in{ex.}[ex:GalileoPendulum1]), but now we must use the gravitational potential energy formula that works for large distances. The gravitational potential energy for the box and Earth are plotted in figure~\ref{fig:BoxEarthGravU}.
	\startformula\startmathalignment
	\NC	H\si + W + \cancel{Q}	\NC = H\sf			\NR
	\NC	K\si + U\si + W			\NC = K\sf + U\sf
	\stopmathalignment\stopformula
	Since the box starts motionless on Earth and ends motionless far away, the initial and final kinetic energies are both zero. The final gravitational potential energy is also zero, because the box is very far from Earth. Only the work and initial potential energy remain. The initial distance between the box’s center and Earth’s center is Earth’s radius.
	\startformula\startmathalignment
	\NC	U\si + W			\NC = 0										\NR
	\NC	-G\frac{mM}{r} + W	\NC = 0										\NR
	\NC	W				\NC = G\frac{mM}{r}								\NR
	\NC					\NC = (6.67\sci{-11}\units{m^3/kg\.s^2})
				\frac{(3.0\units{kg})(5.97\sci{24}\units{kg})}{6.37\sci{6}\units{m}}		\NR
	\NC					\NC = 1.88\sci{8}\units{J}
	\stopmathalignment\stopformula
	Moving the $3\units{kg}$ box from Earth’s surface to a distant location requires a tremendous amount of work!
\stopsolution
\stopexample

\startuseMPgraphic{graph::BoxEarthGravU} % I'd like to add minor ticks, 0.667mm long.
vardef U =
	path p;
		for x = 5 step 0.1 until 40:
			y := -11.90/x; % lua.mp.morse(x);
			augment.p(x,y);
		endfor;
	p enddef;
draw begingraph(4cm,4cm);
	setrange(0,-2, 40, 0);
	for x=auto.x:
		itick.bot(formatted("$@g$", x), x);
		itick.bot(formatted("@s", ""), x) withcolor "middlegray";
		itick.top(formatted("@s", ""), x) withcolor "middlegray";
	endfor
	glabel.lft(textext("Energy ($\sci{8}\units{J}$)") rotated 90,OUT);
	glabel.bot(textext("$r$ ($\sci{6}\units{m}$)"), OUT);
	gdraw(U) withpen pencircle scaled 0.8pt;
	glabel.lrt("$U$",140);
	gfill(unitsquare xyscaled (6.37,-2)) withcolor "lightgray";
	gdraw((6.37,0) -- (6.37,-2)) withpen pencircle scaled 0.8pt;
	for y=0 step -0.5 until -2:%auto.y:
		itick.lft(formatted("$@g$", y), y);
		itick.lft(formatted("@s", ""), y) withcolor "middlegray";
		itick.rt(formatted("@s", ""), y) withcolor "middlegray";
	endfor
endgraph shifted (0,-5cm);
  pickup pencircle scaled 0.8pt ;
  draw externalfigure "EarthEratosthenes.png" scaled 0.127 shifted (-6.37mm,-6.37mm) ;
  draw fullcircle scaled 12.74mm;
  drawarrow (0mm,0) -- (40mm,0);
    label.top  ("$r$", (20mm,0)) ;
\stopuseMPgraphic

\startplacefigure[location=margin, reference=fig:BoxEarthGravU, title={The potential energy of a $3.0\units{kg}$ object in Earth’s vicinity. The potential energy is not shown for locations inside Earth (the gray region).}]
\small\reuseMPgraphic{graph::BoxEarthGravU}
\stopplacefigure

The energy graph in \in{figure}[fig:BoxEarthGravU] shows the box's tremendous increase in gravitational potential energy as it is moved far from Earth. It starts at Earth's surface, with a potential energy $U=-1.88\sci{8}\units{J}$. To move the box far away requires going off the graph, far to the right in \in{figure}[fig:BoxEarthGravU], to the where $U\approx 0\units{J}$. Objects sent to deep space require giant rockets which do the work required to reach the much higher gravitational potential energy.

\startexample[ex:BoxEarthGravU2] The $3.0\units{kg}$ box was placed far from Earth, but even so it eventually is pulled back by Earth’s gravitational attraction. How fast is the box going when it enters the atmosphere, approximately $100\units{km}$ above Earth’s surface?
\startsolution
	Include the gravitational potential energy so that there is no work done on the system.

	\startformula\startmathalignment
	\NC	H\si + \cancel{W} + \cancel{Q}	\NC = H\sf		\NR
	\NC	K\si + U\si					\NC = K\sf + U\sf
	\stopmathalignment\stopformula
		The falling box starts with neither kinetic nor potential energy, so the initial total energy is zero. As the box falls its potential energy becomes more negative and its kinetic energy becomes more positive so that the total is unchanged at zero.
	\startformula\startmathalignment
	\NC	0				\NC = \half mv\sf^2	- G\frac{mM}{r\sf}	\NR
	\NC	\half \cancel{m}v\sf^2	\NC = G\frac{\cancel{m}M}{r\sf}		\NR
	\NC	v\sf				\NC = \sqrt{\frac{2GM}{r\sf}}
	\stopmathalignment\stopformula
The final distance between Earth’s center and the box is Earth’s radius plus $100\units{km}$.
	\startformula\startmathalignment
	\NC	v\sf	\NC = \sqrt{\frac{2(6.67\sci{-11}\units{m^3/kg\.s^2})(5.97\sci{24}\units{kg})}
				{6.37\sci{6}\units{m}+1.00\sci{5}\units{m}}}		\NR
	\NC		\NC = 1.11\sci{4}\units{m/s}
	\stopmathalignment\stopformula
The box enters the atmosphere with a speed of about $11\units{km/s}$, causing it to burn up before hitting the ground.
\stopsolution
\stopexample

Use D.~Bernoulli's new formula for gravitational potential energy whenever the problem involves objects far from Earth's surface. The new formula is more complicated looking that the old $U=mgh$, but use the new formula in exactly the same way when using conservation of energy.

\section{Gravitational potential}
We are going to step out of the historical story for a moment to address one of the most vexing questions about gravity: How does gravity reach across great distances?
Specifically, how does Earth pull objects downward, and how does the Sun hold planets in orbit?

%When you wish to move a box, you make physical contact with it. You might make contact directly by grabbing or kicking the box. You might make contact indirectly by pushing the box with a stick, pulling it with a string, or throwing a foot ball at it. You touch the box, or you touch something that touches the box, or you touch something that touches something else that touches the box. One way or another there must be some physical contact from you to the box. (You could move the box with magnets, but magnets are even more mysterious than gravity.)

Gravity's ability to move things has confounded all of the philosophers and physicists in our story so far. They certainly made plenty of guesses. Aristotle and the Aristotelians thought objects fall to find their correct place in the world. From the time of Aristotle to Copernicus, many thought the planets were held in orbit by crystal spheres. Kepler and Descartes thought the planets were swept along by an invisible fluid filling the solar system and swirling around the Sun.
%(Kepler also gave each planet a magnet and an oar\dash a fun but crazy theory that we will skip.) 
Supernatural explanations were offered as well.
These guesses were all reasonable and informed, based on the information available at the time, but none were verified with experiments or observations. 

A reasonable and informed guess that has not been verified by experiments and observations is a \keyterm{hypothesis}. As just described, there were many hypotheses offered to explain gravity. What was lacking was a \keyterm{theory} of gravity\dash an explanation that had been verified by experiments or observations.

Newton took a a huge step toward understanding gravity in his \booktitle{Pricipia}, where he offered a formula for the strength of the gravitational force between any two objects. This formula (which we will encounter soon) is now called Newton's law of universal gravitation. When a formula's accuracy has been verified by experiments or observations, it is called a \keyterm{law}, like Newton's laws of motion and the conservation laws for momentum and energy.
In the \booktitle{Principia}, Newton demonstrates that his law of universal gravitation explains all observed motions of the planets.

Newton also shows that fluid models like Descartes' do not match the observed motions, but Newton offers no alternative mechanism. Instead, he offers this:
\startblockquote
I have not as yet been able to discover the reason for these properties of gravity,\dots and I do not feign hypotheses\dots.
%For whatever is not deduced from the phenomena must be called a hypothesis; and hypotheses, whether metaphysical or physical, or based on occult qualities, or mechanical, have no place in experimental philosophy. In this philosophy particular propositions are inferred from the phenomena, and afterwards rendered general by induction.
%The impenetrability, mobility and momentum of bodies, and the laws of motion and the law of gravity have been found by this method.\autocite{p.~943. \quotation{Impetus} changed to \quotation{momentum}}{Newton1999}
%And
It is enough that gravity really exists and acts according to the laws that we have set forth and is sufficient to explain all the motions of the heavenly bodies\dots.\autocite{p.~943.}{Newton1999}
\stopblockquote

Newton and his contemporaries certainly wanted to know how gravity reaches across the vast distances of the solar system, but they could predict all of the solar system's motions without knowing. Newton's law of universal gravitation was sufficient.

Today we do know how gravity works, thanks to Einstein. The complete theory of gravity is complicated. Luckily, the most important part of the theory is simple, and was found by Joseph-Louis Lagrange in 1773. Like D.~Bernoulli, Lagrange was attempting to explain details of the Moon's motion. He introduced a function call the \keyterm{gravitational potential}, $\Phi$. The gravitational potential exists everywhere. At every location it has a specific value (possibly zero).

\startuseMPgraphic{graph::EarthGravPotential} % I'd like to add minor ticks, 0.667mm long.
vardef U =
	path p;
		for x = 5 step 0.1 until 40:
			y := -398.7/x; % lua.mp.morse(x);
			augment.p(x,y);
		endfor;
	p enddef;
draw begingraph(4cm,7cm);
	setrange(0,-70, 40, 0);
	for x=auto.x:
		itick.bot(formatted("$@g$", x), x);
		itick.bot(formatted("@s", ""), x) withcolor "middlegray";
		itick.top(formatted("@s", ""), x) withcolor "middlegray";
	endfor
	glabel.lft(textext("Gravitational Potential ($\sci{6}\units{m^2/s^2}$)") rotated 90,OUT);
	glabel.bot(textext("$r$ ($\sci{6}\units{m}$)"), OUT);
	gdraw(U) withpen pencircle scaled 0.8pt;
	glabel.lrt("$\Phi$",140);
	gfill(unitsquare xyscaled (6.37,-70)) withcolor "lightgray";
	gdraw((6.37,0) -- (6.37,-70)) withpen pencircle scaled 0.8pt;
	for y=0 step -10 until -70:%auto.y:
		itick.lft(formatted("$@g$", y), y);
		itick.lft(formatted("@s", ""), y) withcolor "middlegray";
		itick.rt(formatted("@s", ""), y) withcolor "middlegray";
	endfor
endgraph shifted (0,-8cm);
  pickup pencircle scaled 0.8pt ;
  draw externalfigure "EarthEratosthenes.png" scaled 0.127 shifted (-6.37mm,-6.37mm) ;
  draw fullcircle scaled 12.74mm;
  drawarrow (0mm,0) -- (40mm,0);
    label.top  ("$r$", (20mm,0)) ;
\stopuseMPgraphic

\startplacefigure[location=margin, reference=fig:EarthGravPotential, title={Gravitational potential $\Phi$ in the vicinity of Earth.}]
\small\reuseMPgraphic{graph::EarthGravPotential}
\stopplacefigure

If space were completely empty, then the gravitational potential would be zero everywhere ($\Phi=0$).
Any massive object produces a region of negative gravitational potential in its vicinity. The gravitational potential produced by Earth is shown in \in{figure}[fig:EarthGravPotential].
The gravitational potential is most negative close to Earth, and rises toward zero far away. At distance $r$ from the center of a mass $M$, the gravitational potential is 
\startformula
	\Phi = -G\frac{M}{r}.
\stopformula
The distance $r$ can be in any direction. \in{Figure}[fig:Phi3D] shows several directions away from Earth, with the gravitational potential plotted below. Only horizontal directions are shown, so the vertical direction can be used to show the value of $\Phi$. In every direction, the gravitational potential is most negative close to Earth, and rises toward zero far away.

\startbuffer[Phi3D]
\startaxis[
 	  %axis line shift=1cm,
	   %axis lines*=left,
   hide x axis,
    hide y axis,
    hide z axis,
        axis lines=center,
        axis on top,
	 view={0}{10},
        width = 6.5cm,
    %z post scale = {1},
        clip mode = individual,
]
%    \addplot3 [
%        mesh, color = middlegray,
%        z buffer=sort,
%        samples=9,
%        domain=0.1:1,
%        y domain=0:2*pi,
%](
%{x * cos(deg(y))}, {x * sin(deg(y))}, {-10}
%    );
    \addplot3 [
        surf, faceted color = middlegray, color = gray,
        z buffer=sort,
        samples=25,
        domain=6.37:50,
        y domain=0:2*pi,
        samples y=25,
](
{x * cos(deg(y))}, {x * sin(deg(y))}, {-39.87/x}
    );
    \node at (-20,-6,-2.5) {$\Phi$};
%    \draw[->] (0,0,0) --node[below, pos = 0.98]{$r$} (50,0,0);
%    \draw[->] (0,0,0) --node[below, pos = 0.98]{$r$} (-50,0,0);
    \draw[->] (0,0,0) --node[above, pos = 0.98]{$r$} (25,43.3,0);
    \draw[->] (0,0,0) --node[above, pos = 0.98]{$r$} (-43.3,25,0);
    \draw[shade, ball color = white] (0,0,0) circle[radius=3.5mm];%node[] {\Earth};
    \draw[->] (-3.18,-5.5,0) --node[above, pos = 0.98]{$r$} (-25,-43.3,0);
    \draw[->] (5.5,-3.18,0) --node[above, pos = 0.98]{$r$} (43.3,-25,0);
\stopaxis
\stopbuffer

\marginTikZ{}{Phi3D}{Earth's region of negative gravitational potential extends in all directions. } % vskip, name, caption

The shape of $\Phi$ in \in{figure}[fig:Phi3D] looks similar to a trampoline that has been pressed down in the center. When you press down the center of a trampoline, this stretches the trampoline downward to  create a wide dimple which is lowest where you press, and which rises in all directions. The gravitational potential has this trampoline-like property. Massive objects lower the potential, and this stretches the gravitational potential toward negative values, producing the potential shown in \in{figure}[fig:Phi3D]. 

Once you have the gravitational potential around Earth, it is easy to find the potential energy $U$ of another mass $m$ near Earth.
\startformula
  U = m\Phi
\stopformula


The deep negative potential near Earth is often called Earth's \keyterm{potential well}.

%%\Initial{I}{n \booktitle{Hydrodynamica} 
%In Hydrodynamica and his other works, Daniel Bernoulli applied sharp physical insights to practical problems – pipes, fountains, pumps, and other useful machines. Everyone else was going in other directions.
%
%Engineers developed and applied methods specific to specialized applications without looking for broadly applicable physical principles. This practical approach was incredibly successful. Eighteenth century engineers launched the industrial revolution, building powerful working engines decades before physicists could explain them.
%
%Physicists largely ignored the engines, focusing their attention on abstract problems in pure mechanics – planetary motion, spinning tops, and jointed pendulums. These problems had little practical value, but they inspired rapid advances in applied mathematics.
%Leonhard Euler (a close friend of Danial Bernoulli), adapted and expanded Newton's methods to solve many new types of problems. Other physicists introduced new principles – like energy conservation – to solve problems in entirely new ways. These advances produced a rather confusing patchwork of methods and principles.



\section{Rational Mechanics}

Joseph-Louis Lagrange sought to replace the patchwork with a single procedure that could be applied to any problem in mechanics. He delivered his synthesis in \booktitle{Mécanique Analytique}, published in 1788, almost exactly a century after Newton's \booktitle{Principia}. Lagrange describes his intent in the monumental work's preface.

\startblockquote
There already exist several treatises on mechanics, but the purpose of this one is entirely new. I propose to condense the theory of this science and the method of solving the related problems to general formulas whose simple application produces all the necessary equations for the solution of each problem. %I hope that my presentation achieves this purpose and leaves nothing lacking.
%In addition, this work will have another use. The various principles presently available will be assembled and presented from a single point of view in order to facilitate the solution of the problems of mechanics.
\stopblockquote

Lagrange's single procedure could be used to solve any mechanical system, from Galileo's falling rock and interrupted pendulum to Kepler's planetary orbits and Mersenne's mysterious musical string. 

\booktitle{Mécanique Analytique} begins with a through and insightful historical review. Lagrange clearly enjoyed finding gems of truth in many different methods, and he also found inspiration.
He refashioned one ancient idea into a general procedure for relating Newton's momentum methods to D.~Bernoulli's energy methods. While this idea had ancient origins, Lagrange's mastery of calculus allowed him apply this idea in new and powerful ways.

\section{Lagrange's finds force from potential energy}

Lagrange's insight was that Newton's force can be found from the energy graph of D.~Bernoulli's  potential energy.
The force's direction is toward lower potential energy.
The force's magnitude is equal to the potential energy's slope on the energy graph.
%If potential energy's slope is gentle, the force is small. If the potential energy's slope is steep, the force is large.

To understand Lagrange's insight, consider the cart connected to a spring in \in{figure}[fig:HookesLaw].
%%The cart's position is represented by $x$.
The graph in \in{figure}[fig:HookesLaw] shows the spring's potential energy, $U = \onehalf kx^2$, where $x$ is the displacement from equilibrium.
Each diagram in \in{figure}[fig:HookesLaw] shows the cart at a different position. These positions are also marked with dots on the energy graph.
Lagrange's insight allows us to find the spring's force\dash direction and magnitude\dash from the potential energy graph.

\startuseMPgraphic{graph::SHOGraph} % I'd like to add minor ticks, 0.667mm long.
	path U, H, K, TPL, TPR;
	U := (-21,36) ..controls (-7,-12) and (7,-12).. (21,36);
draw begingraph(4.2cm,2.1cm);
	setrange(-21,0, 21, 21);
	itick.lft(formatted("$@g$", 0), 0);
	for x=auto.x:
		itick.top(formatted("$@g$", x), x);
		itick.top(formatted("@s", ""), x) withcolor "middlegray";
		itick.top(formatted("@s", ""), x) withcolor "middlegray";
	endfor
	glabel.lft(textext("Energy") rotated 90,OUT)  shifted (2mm,0);
	glabel.top(textext("$x$ (cm)"), OUT);
	gdraw(U) withpen pencircle scaled 0.8pt;
	glabel.urt("$U$",0.167);
	glabel(mydot,(9/42));
	glabel(mydot,(5/14));
	glabel(mydot,(1/2));
	glabel(mydot,(9/14));
	glabel(mydot,(33/42));
%	gdrawarrow (-3.25,0.44) -- (-2.75,0.44) withpen pencircle scaled 0.8pt;
%	glabel.top("$\partial x$",0.5);
%	gdrawarrow (-2.75,0.44) -- (-2.75,0.315) withpen pencircle scaled 0.8pt;
%	glabel.rt("$\partial U$",0);
endgraph;
\stopuseMPgraphic

\startbuffer[SpringCartHooke]
\startgroupplot[
  group style={
    % group name=my plots,
    group size=1 by 5, % columns by rows
    x descriptions at=edge bottom,
    % y descriptions at=edge right,
    % horizontal sep=0.5cm, 
    vertical sep=0.5cm,
  }, 
  scale only axis,
  x={1mm},
  xmin=-24,xmax=24,
  xtick distance=10,
  minor x tick num=9,
  xlabel=$x$ (cm),
  grid=major,
]
\nextgroupplot[ % 1
  y={1mm},
  ymin=0,ymax=5,
  hide y axis,
  axis x line=bottom,
  tick align = outside,
  x axis line style={-},
  grid = none,
  clip=false,
  axis on top = true,
]
\pic at (-12,0){cart};
\draw[decorate,decoration={coil,segment length=1pt}] (-24,2.5) --node[above=3pt] {$k$} (-18,2.5);
\draw[thick,->] (-12,2.5) -- (-4,2.5)node[above] {$F$};
\fill [black!10,] (-24,0) rectangle (24,-1.5);
\fill [black!10,] (-26.3,-1.5) rectangle (-24,6);
\draw (-24,0) -- (-24,6);
\nextgroupplot[ %2
  y={1mm},
  ymin=0,ymax=5,
  hide y axis,
  axis x line=bottom,
  tick align = outside,
  x axis line style={-},
  grid = none,
  clip=false,
  axis on top = true,
]
\pic at (-6,0){cart};
\draw[decorate,decoration={coil,segment length=2.15pt}] (-24,2.5) -- (-12,2.5);
\draw[thick,->] (-6,2.5) -- (-2,2.5) ;
\fill [black!10,] (-24,0) rectangle (24,-1.5);
\fill [black!10,] (-26.3,-1.5) rectangle (-24,6);
\draw (-24,0) -- (-24,6);
\nextgroupplot[ % 3
  y={1mm},
  ymin=0,ymax=5,
  hide y axis,
  axis x line=bottom,
  tick align = outside,
  x axis line style={-},
  grid = none,
  clip=false,
  axis on top = true,
]
\pic at (0,0){cart};
\draw[decorate,decoration={coil,segment length=3.3pt}] (-24,2.5) -- (-6,2.5);
\fill [black!10,] (-24,0) rectangle (24,-1.5);
\fill [black!10,] (-26.3,-1.5) rectangle (-24,6);
\draw (-24,0) -- (-24,6);
\nextgroupplot[ %4
  y={1mm},
  ymin=0,ymax=5,
  hide y axis,
  axis x line=bottom,
  tick align = outside,
  x axis line style={-},
  grid = none,
  clip=false,
  axis on top = true,
]
\pic at (6,0){cart};
\draw[decorate,decoration={coil,segment length=4.5pt}] (-24,2.5) -- (0,2.5);
\draw[thick,->] (6,2.5) -- (2,2.5) ;
\fill [black!10,] (-24,0) rectangle (24,-1.5);
\fill [black!10,] (-26.3,-1.5) rectangle (-24,6);
\draw (-24,0) -- (-24,6);
\nextgroupplot[ % 5
  y={1mm},
  ymin=0,ymax=5,
  hide y axis,
  axis x line=bottom,
  tick align = outside,
  x axis line style={-},
  grid = none,
  clip=false,
  axis on top = true,
]
\pic at (12,0){cart};
\draw[decorate,decoration={coil,segment length=5.6pt}] (-24,2.5) -- (6,2.5);
\draw[thick,->] (12,2.5) -- (4,2.5) ;
\fill [black!10,] (-24,0) rectangle (24,-1.5);
\fill [black!10,] (-26.3,-1.5) rectangle (-24,6);
\draw (-24,0) -- (-24,6);
\stopgroupplot
\stopbuffer

\startplacefigure[location=margin, reference=fig:HookesLaw, title={The cart at its release position (top), the energy graph for the cart and spring (middle), and the cart's position vs.\ time graph (bottom).}]
\reuseMPgraphic{graph::SHOGraph}
\typesetbuffer[starttikz,SpringCartHooke,stoptikz]
\stopplacefigure

Lagrange tells us that the force's direction is toward lower potential energy.
Using the energy graph in \in{figure}[fig:HookesLaw], we see that lower potential energy is toward the center. When the cart is left of center, lower potential energy is to the right, producing the rightward force shown in the top two diagrams. When the cart is right of center, lower potential energy is to the left, producing the leftward force shown in the bottom two diagrams. When the cart is at the center, neither direction is toward lower potential energy, so no force is shown in the middle diagram.

Lagrange tells us that the force's magnitude is the potential energy graph's slope. Using the energy graph in \in{figure}[fig:HookesLaw], we see that far from the center, the slope is greatest, producing the large forces shown in the top and bottom diagrams. When the cart is closer to the center, the slope is smaller, producing the smaller forces shown in the second and fourth diagrams. When the cart is at the center, the slope is zero, producing no force in the middle diagram.

Finding the exact slope on the energy graph is not easy, but there is a simple formula for the slope called \keyterm{Hooke's law}.
\startformula
  F = - kx
\stopformula
The $k$ in Hooke's law is the same spring stiffness that appears in the spring's potential energy formula above.

We should compare Hooke's law to Lagrange's insight to see that they agree.
First, when the cart is in the center, at $x=0$, Hooke's law gives $F=0$, in agreement with Lagrange. 
Next, check the directions.
The negative sign in Hooke's law guarantees that the force always opposes the displacement. When the cart's displacement $x$ is negative, the force is positive, pushing the cart in the positive direction, back toward the equilibrium point. When the cart's displacement is positive, the force pushes in the negative direction, also back toward the equilibrium point. Hooke's directions agree with Lagrange. Finally, as the the magnitude of $x$ increases, the magnitude of Hooke's force $F$ also increases, again in agreement with Lagrange.
Lagrange's insight explains the spring's force. Hooke's law calculates the spring's force.


%On the right half of the graph, the slope is positive, so any motion in the positive direction leads to higher potential energy.
%The force, which is minus the slope, is therefore negative, pushing toward lower potential energy.
%On the left half of the graph, the slope is negative. Any motion in the positive direction leads toward lower potential energy, so the force, which is minus the slope, pushes in the positive direction, again toward lower potential energy.
%At the center, the slope and the the force are both zero. Neither direction leads toward lower potential energy, and the force does not point in either direction.

Lagrange's insight also allows us to find the long range gravitational force. Long range gravitational potential energy was introduced at the end of the previous chapter (\at{p.}[sec:GravUSolarSys]).
\startformula
	U = -G\frac{mM}{r}
\stopformula
The potential energy of a \unit{3.0kg} box was shown in \in{figure}[fig:BoxEarthGravU], and the energy graph is reproduced here in \in{figure}[fig:BoxEarthGravF].
The diagram at the top of \in{figure}[fig:BoxEarthGravF] shows Earth and the box at three different distances $r$.

\startuseMPgraphic{graph::BoxEarthGravF} % I'd like to add minor ticks, 0.667mm long.
vardef U =
	path p;
		for x = 5 step 0.1 until 40:
			y := -11.90/x; % lua.mp.morse(x);
			augment.p(x,y);
		endfor;
	p enddef;
picture mybox; mybox := image(fill fullsquare scaled 3pt;);
draw begingraph(4cm,4cm);
	setrange(0,-2, 40, 0);
	for x=auto.x:
		itick.bot(formatted("$@g$", x), x);
		itick.bot(formatted("@s", ""), x) withcolor "middlegray";
		itick.top(formatted("@s", ""), x) withcolor "middlegray";
	endfor
	glabel.lft(textext("Energy ($\sci{8}\units{J}$)") rotated 90,OUT);
	glabel.bot(textext("$r$ ($\sci{6}\units{m}$)"), OUT);
	gdraw(U) withpen pencircle scaled 0.8pt;
	glabel.lrt("$U$",140);
	glabel(mydot,(80));
	glabel(mydot,(210));
	glabel(mydot,(340));
	gfill(unitsquare xyscaled (6.37,-2)) withcolor "lightgray";
	gdraw((6.37,0) -- (6.37,-2)) withpen pencircle scaled 0.8pt;
	for y=0 step -0.5 until -2:%auto.y:
		itick.lft(formatted("$@g$", y), y);
		itick.lft(formatted("@s", ""), y) withcolor "middlegray";
		itick.rt(formatted("@s", ""), y) withcolor "middlegray";
	endfor
endgraph shifted (0,-5cm);
  pickup pencircle scaled 0.8pt ;
  draw externalfigure "EarthEratosthenes.png" scaled 0.127 shifted (-6.37mm,-6.37mm) ;
  draw fullcircle scaled 12.74mm;
  drawarrow (13mm,0) -- (-0.5mm,0);
    dotlabel.ulft  ("", (13mm,0)) ;
    draw mybox shifted (13mm,0);
  drawarrow (26mm,0) -- (22.625mm,0);
    dotlabel.top  ("$F$", (26mm,0)) ;
    draw mybox shifted (26mm,0);
  drawarrow (39mm,0) -- (37.5mm,0);
    dotlabel.llft  ("", (39mm,0)) ;
    draw mybox shifted (39mm,0);
\stopuseMPgraphic

\startplacefigure[location=margin, reference=fig:BoxEarthGravF, title={The gravitational force on a $3.0\units{kg}$ object at different distances from Earth (top). The potential energy of a $3.0\units{kg}$ object in Earth’s vicinity (bottom). The potential energy is not shown for locations inside Earth (the gray region).}]
\small\reuseMPgraphic{graph::BoxEarthGravF}
\stopplacefigure

Look at the energy graph to determine the force's direction and relative magnitude.
Lower potential energy is leftward, toward Earth, producing the leftward force shown on each box in the diagram.
%The gravitational potential energy's slope is positive everywhere, so the gravitational force is negative everywhere – pulling the box back in the negative direction, back toward Earth. 
This is similar to the right side of \in{figure}[fig:HookesLaw], where the spring potential energy's positive slope produces a negative spring force – pulling the block back toward the center. The spring's potential energy gets steeper as the spring is stretched, producing a spring force that increases with distance. The gravitational potential energy does the opposite, getting less steep as the box is moved farther from Earth, producing a gravitational force that decreases with distance. The diagram shows the largest force on the box closest to Earth and the smallest force on the box farthest away.

The formula for this gravitational force is \keyterm{Newton's law of universal gravitation}.
\startformula\pagereference[eq:UniversalGrav]
  F = -G\frac{mM}{r^2}
\stopformula
Newton's law of universal gravitation can be used to find the gravitational force between any two objects. It looks almost identical to the formula for gravitational potential energy, but the denominator is $r^2$ instead of $r$. The $r^2$ in the denominator tells us that the force is very small when the distance is large.
%(You already knew that from looking at the energy graph.)
The negative in Newton's law of universal gravitation guarantees that the force is always in the negative $r$ direction, toward Earth.
Lagrange's insight explains the gravitational force. Newton's law of universal gravitation calculates the gravitational force.\pagereference[eq:UniversalGrav]

%The gravitational force is often written $F = -mg$, where $g$ is the strength of the \keyterm{gravitational field}.
%\startformula
%	g = G\frac{M}{r^2},
%\stopformula
%Here on Earth's surface, $g=9.8\units{m/s^2}$. The gravitational field gets weaker with distance because of the $r^2$ in the denominator. The gravitational field formula can be used to find $g$ any distance from any celestial object. Objects with larger mass $M$ have stronger gravitational fields. Less massive objects's, like the Moon have weaker gravitational fields.

Hooke published his law for springs in 1678. Newton presented his law of universal gravitation to the Royal Society in 1686. These two laws were early contributions to the patchwork of methods and principles developed during the seventeenth and eighteenth century. Lagrange's insight was that these laws, and many others, were consequences of the potential energy formulas discovered later by D.~Bernoulli. This insight was just one part of Lagrange's general procedure for mechanics.

Lagrange's general procedure was totally unlike Newton's own methods for working with forces. Newton's \booktitle{Principia} is full of diagrams. Newton's calculations are done with graphical methods, and equations are few. In the preface to his \booktitle{Mécanique Analytique}, Lagrange explains his analytical procedure.
\startblockquote
No figures will be found in this work. The methods I present require neither constructions nor geometrical or mechanical arguments, but solely algebraic operations subject to a regular and uniform procedure. Those who appreciate mathematical analysis will see with pleasure mechanics becoming a new branch of it and hence, will recognize that I have enlarged its domain.
\stopblockquote
In \booktitle{Mécanique Analytique}, Lagrange used his powerful mathematical analysis to analyze any mechanical system, described using any coordinates. Starting with only the formulas for the system's total kinetic energy and total potential energy. Lagrange's \quotation{regular and uniform procedure} produces a complete set of equations of motion for the system.

Lagrange's \booktitle{Mécanique Analytique} was a stunning achievement, unlike anything physics had seen before. Lagrange was viewed as a hero in France under King Louis \convertnumber{KR}{16}. Luckily, Lagrange's fame transcended the turbulent – often violent – politics of France following the French Revolution in 1789. As governments rose and fell around him, Lagrange continued to play a leading role in the advancement of science and mathematics in France.

Lagrange's equations of motion are a bit more tangled than our position and momentum update formulas, so we will not learn the details of his procedure. A general method for producing position and momentum update formulas was developed by William Rowan Hamilton.

\startuseMPgraphic{graph::EarthGravField} % I'd like to add minor ticks, 0.667mm long.
vardef U =
	path p;
		for x = 5 step 0.1 until 40:
			y := -398.7/x; % lua.mp.morse(x);
			augment.p(x,y);
		endfor;
	p enddef;
picture mybox; mybox := image(fill fullsquare scaled 3pt;);
draw begingraph(4cm,7cm);
	setrange(0,-70, 40, 0);
	for x=auto.x:
		itick.bot(formatted("$@g$", x), x);
		itick.bot(formatted("@s", ""), x) withcolor "middlegray";
		itick.top(formatted("@s", ""), x) withcolor "middlegray";
	endfor
	glabel.lft(textext("Gravitational Potential ($\sci{6}\units{m^2/s^2}$)") rotated 90,OUT);
	glabel.bot(textext("$r$ ($\sci{6}\units{m}$)"), OUT);
	gdraw(U) withpen pencircle scaled 0.8pt;
	glabel.lrt("$U$",140);
	glabel(mydot,(80));
	glabel(mydot,(210));
	glabel(mydot,(340));
	gfill(unitsquare xyscaled (6.37,-70)) withcolor "lightgray";
	gdraw((6.37,0) -- (6.37,-70)) withpen pencircle scaled 0.8pt;
	for y=0 step -10 until -70:%auto.y:
		itick.lft(formatted("$@g$", y), y);
		itick.lft(formatted("@s", ""), y) withcolor "middlegray";
		itick.rt(formatted("@s", ""), y) withcolor "middlegray";
	endfor
endgraph shifted (0,-8cm);
  pickup pencircle scaled 0.8pt ;
  draw externalfigure "EarthEratosthenes.png" scaled 0.127 shifted (-6.37mm,-6.37mm) ;
  draw fullcircle scaled 12.74mm;
  drawarrow (13mm,0) -- (-0.5mm,0);
    dotlabel.ulft  ("", (13mm,0)) ;
    %draw mybox shifted (13mm,0);
  drawarrow (26mm,0) -- (22.625mm,0);
    dotlabel.top  ("$g$", (26mm,0)) ;
    %draw mybox shifted (26mm,0);
  drawarrow (39mm,0) -- (37.5mm,0);
    dotlabel.llft  ("", (39mm,0)) ;
    %draw mybox shifted (39mm,0);
\stopuseMPgraphic

\startplacefigure[location=margin, reference=fig:EarthGravField, title={Earth's gravitational field at different distances (top). Earth's gravitational potential (bottom). The potential is not shown for locations inside Earth (the gray region).}]
\small\reuseMPgraphic{graph::EarthGravField}
\stopplacefigure


\section{Hamilton finds momentum from kinetic energy}

While physics advanced on the continent, the situation in England was dire. English physicists were ignoring the industrial revolution transforming their civilization, and ignoring the French mathematical revolution transforming physics.
They remained paralyzed by a fanatic devotion to Newton’s cumbersome, geometric methods and continued to hold a grudge over the Leibniz’s calculus and \visviva. This problem persisted until 1812, when a group of young undergraduate students at Cambridge started the Analytical Society, partly as a joke, to study the methods of French analysis. They learned to solve problems using the continental notation and even translated a few important French books and papers into English. The club met for a few years and essentially disappeared after its members graduated and left Cambridge in 1817.

That probably would have been the end of this little revolt, except that one of the members, George Peacock, was appointed in 1817 to write questions for the rigorous Mathematical Tripos exam taken by all third year mathematics students at Cambridge. He audaciously wrote his questions using the continental notation. This caused a bit of grumbling among the faculty, but they did not interfere. Surprisingly, Peacock was asked to write questions again in 1819. Students recognized this as a complete surrender by the Newtonian faculty. From there, the adoption of continental notation and methods in England was quite swift. Newton remained a revered figure, but his 150 year old dispute with Leibniz was finally laid to rest. %Cannell p. 38

Freed from Newton's methods, the United Kingdom began producing great physicists. Among the first was William Rowan Hamilton. Hamilton was a genius, gifted in languages and mathematics. He learned Hebrew, Latin, and Greek from early childhood. At the age of 15 he read Newton’s \booktitle{Principia}, and a two years later the great French work \booktitle{M\'echanique C\'eleste}, by Pierre-Simon Laplace.%Cannell p. 39

In 1823, at the age of 18, Hamilton entered Trinity College Dublin.
Trinity College had adopted the continental notation in 1812, a bit ahead of Cambridge, and used the French textbooks in mathematics and physics courses. Hamilton embraced the abstract approach. He earned top marks in all of his courses and won many awards, which propelled him to a position as head of the Dunsink Observatory, near Dublin. Although the observatory provided ample opportunity for practical astronomy, Hamilton continued to focus on more theoretical interests, first in optics and then in mechanics.

Hamilton wrote two groundbreaking papers on mechanics, \booktitle{On A General Method In Dynamics} in 1834 and \booktitle{Second Essay On A General Method In Dynamics} in 1835. Hamilton's general method produces the equations of motion for any problem in mechanics, just like Lagrange's procedure, but Hamilton's general method is much easier to understand and execute.

Hamilton's insight was that Newton's momentum can be found from D.~Bernoulli's kinetic energy.
In cartesian coordinates, where the kinetic energy is $K=\onehalf mv^2$, Hamilton's method produces Newton's well know momentum formula, $p=mv$. However, Hamilton's method, like Lagrange's, works in any coordinates. For an angular coordinate $\theta$, where the kinetic energy is $K=\onehalf I\omega^2$, Hamilton's method gives the \keyterm{angular momentum} denoted by $L$.
\startformula
  L = I\omega,
\stopformula
where $I$ is the moment of inertia introduced in \in{chapter}[ch:VisViva] (\at{p.}[eq.Krot]).

\startexample[ex:PottersI]
You are building a potter's wheel, which will have a circular, spinning tabletop for making round bowls, cups and vases, as shown in \in{figure}[fig:PottersWheel]. The tabletop is a heavy disk with moment of inertial $I= \unit{0.10kg m^2}$. You would like the tabletop to spin with angular velocity $\omega = \unit{3.0 rad/s}$. Calculate the tabletop's angular momentum at this speed?
\startuseMPgraphic{PottersWheel}
  path top  ;
  top := fullcircle xyscaled (4cm, 1cm) ;
  %mass := fullsquare xyscaled (1cm,1.5cm) shifted (2cm,-3cm) ;
  pickup pencircle scaled 0.8pt ;
  fill top shifted (0,-0.5cm) withcolor "lightgray" ;
  draw top  shifted (0,-0.5cm);
  fill fullsquare xyscaled (4cm,0.5cm) shifted (0,-0.25cm) withcolor "lightgray";
  draw (2cm,0) -- (2cm,-0.5cm);
  draw (-2cm,0) -- (-2cm,-0.5cm);
  fill top withcolor "lightgray";
  draw top ;
  label  ("$\omega$", (0cm,-0.7cm)) ;
  drawarrow quartercircle rotated -135 xyscaled (4cm, 1cm) shifted (0,-0.35cm);
\stopuseMPgraphic

\startplacefigure[location=margin, reference=fig:PottersWheel, title={A potter's wheel has a round, heavy, rotating top.}]
\small\reuseMPgraphic{PottersWheel}
\stopplacefigure

\startsolution
Use the angular momentum formula, $L = I\omega$, taking care with units. 
\startformula
  L = I\omega = (\unit{0.10 kg m^2})(\unit{3.0 rad/s}) = \unit{0.30 kg m^2 /s}
\stopformula

\stopsolution
\stopexample
Angular quantities do not need angular units. In the last example, radians simply dropped out of the calculation, as radians can. If the angular velocity had been measured in cycles per second or degrees per second, we would have converted those angular units to radians, and then dropped the radians.
All the other units were carried through the calculation to get the final units of angular momentum, \unit{kg m^2/s}. These units are not the same as units for Newton's linear momentum (\unit{kg m/s}). 

%Newton's momentum is one of the most successful in physics. Angular momentum is no less significant. Angular momentum joins momentum and energy as the third conserved quantity in mechanics.

\section{Conservation of angular momentum}
All of the problems we have done with momentum have similar versions with angular momentum. In every case, we start with the conservation of momentum equation.
\startformula
  L\si + \tau\Delta t = L\sf
\stopformula
Since you have so much experience with conservation of momentum, a few examples will suffice to show how the same conservation of momentum law works with angular momentum. Just keep in mind that the quantities are angular quantities whenever you work with an angular coordinate, which may mean different units.

\startexample[ex:PottersThrow]
Your new potter's wheel (with moment of inertial $I= \unit{0.10kg m^2}$), is spinning with angular velocity $\omega = \unit{3.0 rad/s}$. To make your first pot, you grab $\unit{1.7kg}$ of clay and drop it directly in the center of the spinning tabletop, as shown in \in{figure}[fig:PottersThrow]. You do not spin the clay as you drop it, so it has no angular momentum before impact. The clay's moment of inertia is $\unit{0.010kg m^2}$.
There are no outside angular forces on the tabletop and clay. What is the new angular velocity of the tabletop with the clay stuck to it?
\startuseMPgraphic{PottersThrow}
  path top, clay ;
  top := fullcircle xyscaled (4cm, 1cm) ;
  clay := top scaled 0.466 ;
  %mass := fullsquare xyscaled (1cm,1.5cm) shifted (2cm,-3cm) ;
  pickup pencircle scaled 0.8pt ;
  fill top shifted (0,-0.5cm) withcolor "lightgray" ;
  draw top  shifted (0,-0.5cm);
  fill fullsquare xyscaled (4cm,0.5cm) shifted (0,-0.25cm) withcolor "lightgray";
  draw (2cm,0) -- (2cm,-0.5cm);
  draw (-2cm,0) -- (-2cm,-0.5cm);
  fill top withcolor "lightgray";
  draw top ;
  label  ("$\omega$", (0cm,-0.7cm)) ;
  drawarrow quartercircle rotated -135 xyscaled (4cm, 1cm) shifted (0,-0.35cm);
  fill clay withcolor "middlegray" ;
  draw clay ;
  fill tcircle scaled 1.86cm withcolor "middlegray";
  draw halfcircle scaled 1.86cm ;
\stopuseMPgraphic

\startplacefigure[location=margin, reference=fig:PottersThrow, title={A collision between a big piece of clay and the potter's where can be treated using conservation of angular momentum.}]
\small\reuseMPgraphic{PottersThrow}
\stopplacefigure
\startsolution
%This is a collision problem where we are interested in the objects' rotation after the collision.
Use conservation of angular momentum, with no outside angular force. Before the collision, only the tabletop has initial angular momentum.
After the collision, table top and the clay share the angular momentum. The combined moment of inertia for the tabletop and clay is $I\sf = \unit{0.10kg m^2} + \unit{0.010kg m^2} = \unit{0.11kg m^2}$
\startformula\startmathalignment
\NC L\si + \cancel{\tau\Delta t} \NC= L\sf \NR
\NC I\si \omega\si \NC= I\sf \omega\sf \NR
\NC \omega\sf \NC= \frac{I\si}{I\sf} \omega\si
         = \frac{\unit{0.10kg m^2}}{\unit{0.11 kg m^2}}\unit{3.0 rad/s}
         = \unit{2.7rad/s} \NR
\stopmathalignment\stopformula
With the total angular momentum unchanged, the increased moment of inertia results in a decreased angular velocity. (Rotational kinetic energy is not conserved in this inelastic collision.)
\stopsolution
\stopexample

A system's angular momentum is changed by external angular forces. Angular force is called \keyterm{torque}.

\startexample[ex:PottersTorque]
You decide to add an electric motor to spin your potter's wheel. At full speed, the potter's wheel has angular momentum $\unit{0.30 kg m^2 /s}$. 
How much torque must your electric motor provide to spin the potter's wheel from rest to the desired angular velocity in \unit{15s}?
\startsolution
Since we are looking for the time required, conservation of momentum is a good choice. In this case, the motion is angular, so $p$ is angular momentum and $\tau$ is torque.
\startformula\startmathalignment
\NC \cancel{L\si} + \tau\Delta t \NC= L\sf \NR
\NC \tau \NC= \frac{L\sf}{\Delta t}
         = \frac{\unit{0.30 kg m^2 /s}}{\unit{15s}}q
         = \unit{0.020 kg m^2/s^2}\NR
\stopmathalignment\stopformula
\stopsolution
\stopexample

As the last two example show, angular momentum conservation problems are often exactly like linear momentum conservation problems, except for the slightly different units.  There is one situation, however, where angular momentum is quite different. This occurs when a spinning object changes its shape, thereby changing its moment of inertia. Angular momentum conservation still works in this case, but the result is unlike anything we have seen with linear momentum conservation.

\startexample[ex:MerryGoRound1]
Six kids have gotten a playground merry-go-round spinning, have climbed on, and are standing near the outer edge, as shown in \in{figure}[fig:MerryGoRound1]. The merry-go-round's moment of inertial is $\unit{10 kg m^2}$. The six kids standing near the edge have a total moment of inertia of $\unit{240 kg m^2}$. The merry-go-round is spinning with a reasonable speed of $\unit{1.0 rad/s}$. The kids then pull their way to the center, as shown in \in{figure}[fig:MerryGoRound2], reducing their moment of inertia to just $\unit{15kg m^2}$. (The merry-go-round's moment of inertia remains $\unit{10kg m^2}$.) What is the merry-go-round's new angular velocity?
\startuseMPgraphic{MerryGoRound1}
  path top, kidpath ;
  top := fullcircle xyscaled (4cm, 1cm) ;
  kidpath := fullcircle rotated 10 xyscaled (3.2cm, 0.8cm) ;
  picture kid ;
  picture littleball;
	kid := image(
	    path kidtop ; kidtop = fullcircle xyscaled (4mm,1mm) ;
	    fill kidtop withcolor "middlegray" ;
	    draw kidtop ;
	    fill fullsquare xyscaled (4mm,1cm) shifted (0cm,0.5cm) withcolor "middlegray" ;
	    draw (-2mm,0) -- (-2mm,1cm) ;
	    draw (2mm,0) -- (2mm,1cm) ;
	    fill kidtop shifted (0,1cm) withcolor "middlegray" ;
	    draw kidtop shifted (0,1cm) ;
    );
  pickup pencircle scaled 0.8pt ;
  fill top shifted (0,-1mm) withcolor "lightgray" ;
  draw top  shifted (0,-1mm);
  fill fullsquare xyscaled (4cm,1mm) shifted (0,-0.50mm) withcolor "lightgray";
  draw (2cm,0) -- (2cm,-1mm);
  draw (-2cm,0) -- (-2cm,-1mm);
  fill top withcolor "lightgray";
  draw top ;
  label  ("$\omega\si$", (0cm,-0.9cm)) ;
  drawarrow quartercircle rotated -135 xyscaled (4cm, 1cm) shifted (0,-0.2cm);
  for i = 0 step 4/3 until 7:
    draw kid shifted point i of kidpath ;
  endfor
\stopuseMPgraphic

\startplacefigure[location=margin, reference=fig:MerryGoRound1, title={Six kids (shown as cylinders) have climbed onto a playground merry-go-round. They are all standing near the outer edge.}]
\small\reuseMPgraphic{MerryGoRound1}
\stopplacefigure

\startuseMPgraphic{MerryGoRound2}
  path top, kidpath ;
  top := fullcircle xyscaled (4cm, 1cm) ;
  kidpath := fullcircle rotated 70 xyscaled (8mm, 2mm) ;
  picture kid ;
  picture littleball;
	kid := image(
	    path kidtop ; kidtop = fullcircle xyscaled (4mm,1mm) ;
	    fill kidtop withcolor "middlegray" ;
	    draw kidtop ;
	    fill fullsquare xyscaled (4mm,1cm) shifted (0cm,0.5cm) withcolor "middlegray" ;
	    draw (-2mm,0) -- (-2mm,1cm) ;
	    draw (2mm,0) -- (2mm,1cm) ;
	    fill kidtop shifted (0,1cm) withcolor "middlegray" ;
	    draw kidtop shifted (0,1cm) ;
    );
  pickup pencircle scaled 0.8pt ;
  fill top shifted (0,-1mm) withcolor "lightgray" ;
  draw top  shifted (0,-1mm);
  fill fullsquare xyscaled (4cm,1mm) shifted (0,-0.50mm) withcolor "lightgray";
  draw (2cm,0) -- (2cm,-1mm);
  draw (-2cm,0) -- (-2cm,-1mm);
  fill top withcolor "lightgray";
  draw top ;
  label  ("$\omega\sf$", (0cm,-0.9cm)) ;
  drawarrow quartercircle rotated -135 xyscaled (4cm, 1cm) shifted (0,-0.2cm);
  for i = 0 step 4/3 until 3:
    draw kid shifted point i+4/3 of kidpath ;
    draw kid shifted point 8-i of kidpath ;
  endfor
\stopuseMPgraphic

\startplacefigure[location=margin, reference=fig:MerryGoRound2, title={The six kids pull themselves to the center, greatly reducing their moment of inertia.}]
\reuseMPgraphic{MerryGoRound2}
\stopplacefigure
\startsolution
Once again, use conservation of momentum. There are no external torques.
\startformula\startmathalignment
\NC L\si + \cancel{\tau\Delta t} \NC= L\sf \NR
\NC I\si \omega\si \NC= I\sf \omega\sf \NR
\NC \omega\sf \NC= \frac{I\si}{I\sf} \omega\si
         = \frac{\unit{250kg m^2}}{\unit{25 kg m^2}}\unit{1.0 rad/s}
         = \unit{10rad/s} \NR
\stopmathalignment\stopformula
They have gone from just less than one revolution every three seconds to over three revolutions per second!
\stopsolution
\stopexample
Changing a system's moment of inertia can dramatically change its angular velocity. In the example above, the kids were able to reduce their distance from the center to about one-quarter of what it was originally. Because the moment of inertial is proportional to $r^2$, reducing their $r$ to one-quarter will reduce their $I$ to one sixteenth!

Nothing like this happens with linear momentum, because linear momentum is mass times velocity. Objects cannot simply reduce their mass to increase their velocity. Many things do chance their shape and moment of inertia. Ice skaters will pull in their arms to make their spins faster. Gymnasts and divers tuck to flip more quickly.
Slowly rotating air masses become hurricanes or tornados when the air is draw towards the center by an area of extreme low pressure.
Planets, stars, solar systems, and galaxies rotate much more quickly than the large clouds of gas and dust from which they formed.
Conservation of angular momentum is one of the great drivers of motion in the universe, from the largest galaxies down to subatomic particles. 

Kinetic energy can be found from either angular velocity or angular momentum.
\startformula
  K = \half I\omega^2
    = \frac{L^2}{2I}
\stopformula

\section{Circular, centered orbits}

Celestial motion was a primary interest for both Lagrange and Hamilton. They each hoped to solve puzzling celestial motion problems using their new, general, and powerful methods. Their methods start with energy, so we begin by looking at the energy graph for a planet orbiting the Sun, shown in \in{figure}[fig:U3D]. 

The Sun ($\Sun$) is at the center of \in{figure}[fig:U3D]. The planets are shown at increasing distances along the $r$ axis. (Planet symbols are listed in \in{table}[T:Astrosym] on \at{p.}[T:Astrosym].)
\in{Figure}[fig:U3D] is three dimensional to show the potential energy $U$ as planets orbit the Sun. The vertical axis in \in{figure}[fig:U3D] is energy. The planets orbit in the vertical plane at the top of \in{figure}[fig:U3D].
The three dimensional energy graph will allow us to understand the planets' two dimensional orbits.

\in{Figure}[fig:U3D] shows the planets' gravitational potential energy $U$, which is very negative near the Sun and rises toward zero far away.
\startformula
	U = -G\frac{mM}{r}
\stopformula
(We will use $m$ for the mass of a planet and $M$ for the much larger mass of the Sun.) This gravitational potential energy produces a gravitational force toward the Sun.

\startbuffer[U3D]
\startaxis[
 	  %axis line shift=1cm,
	   %axis lines*=left,
   hide x axis,
    hide y axis,
    hide z axis,
        axis lines=center,
        axis on top,
	 view={0}{45},
        width = 7cm,
    %z post scale = {1},
        clip mode = individual,
]
%    \addplot3 [
%        mesh, color = middlegray,
%        z buffer=sort,
%        samples=9,
%        domain=0.1:1,
%        y domain=0:2*pi,
%](
%{x * cos(deg(y))}, {x * sin(deg(y))}, {-10}
%    );
    \addplot3 [
        surf, faceted color = middlegray, color = gray,
        z buffer=sort,
        samples=6,
        domain=2.5:15,
        y domain=0:2*pi,
        samples y=25,
](
{x * cos(deg(y))}, {x * sin(deg(y))}, {-13.27/x}
    );
    \node at (-8,-6,-2.5) {$U$};
    \draw[->] (0,0,0) --node[below, pos = 0.98]{$r$} (16,0,0);
    \draw[shade, ball color = white] (0,0,0) circle[radius=.6mm]node[above=0.8mm] {\Sun};
    \draw[shade, ball color = darkgray] (0.5546,0,0) circle[radius=0.2mm]node[below] {\Mercury};
    \draw[shade, ball color = darkgray] (1.082,0,0) circle[radius=0.3mm]node[above] {\Venus};
    %\draw[] (1.5,0,0) -- (1.5,0,-8.874);
    \draw[shade, ball color = darkgray] (1.496,0,0) circle[radius=0.3mm]node[below=0.4mm] {\Earth};
    \draw[shade, ball color = darkgray] (2.259,0,0) circle[radius=0.25mm]node[above=0.7mm] {\Mars};
    \draw[shade, ball color = darkgray] (7.76,0,0) circle[radius=0.4mm]node[above] {\Jupiter};
    \draw[shade, ball color = darkgray] (14.23,0,0) circle[radius=0.4mm]node[above] {\Saturn};
\stopaxis
\stopbuffer

\marginTikZ{}{U3D}{A planet's gravitational potential energy depends on its distance $r$ from the Sun. A planet released from rest would accelerate toward lower potential energy, crashing into the Sun!} % vskip, name, caption

Following the logic of Lagrange's force formula, we can see from \in{figure}[fig:U3D] that the $r$ component of the gravitational force is negative, pulling planets toward lower potential energy in the direction of the Sun. This gravitational force toward the Sun bends the planets' paths into orbits.

A planet orbiting the Sun at a constant distance $r$ does not see any change in potential energy.
Since the potential energy is constant along a circular orbit, Lagrange's force formula tells us that there is no force in along the circular orbit, meaning no torque $\tau$ along the orbit direction.
Since the gravitational force does not exert any torque around the Sun, the planet's angular momentum $L$ will remain constant.

These two facts\dash a force toward the Sun and constant angular momentum $L$\dash will guide our study of orbits.
A planet's angular momentum $L$ is related to the planet's angular velocity $\omega$.
\startformula
  L = I\omega = mr^2\omega
\stopformula



\startbuffer[KeplerVenus]
\environment env_physics
\environment env_TikZ
\setupbodyfont [libertinus,11pt]
\setoldstyle % Old style numerals in text
\startTEXpage\small
\starttikzpicture% tikz code
\startpolaraxis
 [	xticklabels=\empty,
 	ytick={0,0.5,...,1.5},
 	yticklabels={{},{},$100\units{Gm}$,{}},
 	minor y tick num={1},
	% yminorgrids=true,
	hide x axis,
	ymax = 1.5,
	scale only axis=true, width={5.5cm},
 	tick style={middlegray}, % Fixes ticks which are too light in ConTeXt
	major grid style = {middlegray},
 	% ylabel={Distance from Sun $r$ ($\sci{9}\units{m}$)},
 ]
    \addplot [ % Venus
        thick,
        domain=0:360,
        samples=600,
    ]
        {1.082/(1+0.00676*cos(x-131.77))}
  [yshift=-1.3pt]
    node[pos=0.25] {\Venus}
    ;
	\node [name path=Sun] at (0,0) {\Sun}node[below=1mm]{Sun};
	\draw[->]  (22.5, 1.082) --node[right, pos=.7]{$p$} ({22.5+21}, {13*1.082/12});
	\draw[->]  (22.5, 1.082) --node[above, pos=.7]{$F$} (22.5, {0.7});
	\draw[->]  (67.5, 1.082) -- ({67.5+21}, {13*1.082/12});
	\draw[->]  (67.5, 1.082) -- (67.5, {0.7});
	\draw[->]  (112.5, 1.082) -- ({112.5+21}, {13*1.082/12});
	\draw[->]  (112.5, 1.082) -- (112.5, {0.7});
	\draw[->]  (157.5, 1.082) -- ({157.5+21}, {13*1.082/12});
	\draw[->]  (157.5, 1.082) -- (157.5, {0.7});
	\draw[->]  (202.5, 1.082) -- ({202.5+21}, {13*1.082/12});
	\draw[->]  (202.5, 1.082) -- (202.5, {0.7});
	\draw[->]  (247.5, 1.082) -- ({247.5+21}, {13*1.082/12});
	\draw[->]  (247.5, 1.082) -- (247.5, {0.7});
	\draw[->]  (292.5, 1.082) -- ({292.5+21}, {13*1.082/12});
	\draw[->]  (292.5, 1.082) -- (292.5, {0.7});
	\draw[->]  (337.5, 1.082) -- ({337.5+21}, {13*1.082/12});
	\draw[->]  (337.5, 1.082) -- (337.5, {0.7});
\stoppolaraxis
\stoptikzpicture
\stopTEXpage
\stopbuffer

\marginTikZ{}{KeplerVenus}{Venus' orbit is an almost perfectly centered circle, making uniform circular motion a good approximation of for Venus. (A gigameter is one billion meters: $1\units{Gm} = 10^9\units{m}$).} % vskip, name, caption

\noindent
Since $L$ and $r$ are both constant for a circular orbit, the planet's angular velocity $\omega$ must be constant as well.
A planet following a circular orbit with the Sun at the center moves with uniform circular motion. Venus's orbit is shown as an example in \in{figure}[fig:KeplerVenus].


Although uniform circular motion was considered natural in Kepler's time, Descartes and Newton taught us that circular motion requires a central force to bend the planet's path into an orbit.
The force required depends on the planet's momentum $p$ and angular velocity $\omega$, as given by the centripetal force formula (\at{p.}[eq:centripetalforce])
\startformula
	F = \omega p
\stopformula
Using $p = mv$ and $v = \omega r$ for uniform circular motion, we find.
\startformula
	F = m\omega^2 r
\stopformula

The centripetal force formula gives the force required to produce the circular motion, but does not tell us the source of that force. For orbits in the solar system this force is the gravitational force exerted by the Sun, the force directed towards lower gravitational potential energy in \in{figure}[fig:U3D]. The strength of the gravitational force is given by Newton's law of universal gravitation (\at{p.}[eq:UniversalGrav]).

\startformula
	F %= -\pp{r} U
		%= -\pp{r} \left(-G\frac{mM}{r}\right)
		%= GmM \pp{r} \left(r^{-1}\right)
		%= GmM \left(-1r^{-2}\right)
		= -G\frac{mM}{r^2}
\stopformula
The \emph{magnitude} of this gravitational force is the centripetal force above, allowing us to find the planet's angular velocity.
\startformula\startmathalignment\pagereference[eq:AngularVelocityCircular]
\NC	\cancel{m}\omega^2 r	\NC = G\frac{\cancel{m}M}{r^2}	\NR
\NC	\omega				\NC = \sqrt{\frac{GM}{r^3}}	\NR
\stopmathalignment\stopformula
The planet's distance determines its angular velocity, with much higher angular velocities closer to the Sun. Higher angular velocities give shorter periods, which we find using $T = 2\pi / \omega$.
\startformula
	T = \textfrac{2\pi}{\sqrt{GM}}\,r^{\threehalves}
\stopformula
This is Kepler's celebrated third law, relating each planet's period $T$ to the orbit's radius $r$. A larger radius produces a longer period.
Mercury's orbit takes only 88 days. Venus's orbit is 225 days. Earth's orbit is 365.24 days. Mars's orbit is 687 days. The outer planets' orbits are far longer – nearly twelve years for Jupiter's orbit and almost thirty years for Saturn's.
The planet's mass has no effect on its period. % The Sun's strong gravitational pull on these inner planets bends their paths into small, fast orbits.

%Using the orbit radii from \in{figure}[fig:VisiblePlanets], you can calculate these periods, and you will get good results.

%A centered, circular orbit has only one parameter: its radius $r$. The planet's angular momentum and total energy are directly related to this radius of the other constants ($G$, $M$, and $m$) are known. The angular momentum can be found from the angular velocity $\omega$.

Each planet's angular momentum $L$ is also determined by the orbit's radius.
\startformula
	L = I\omega
	  = mr^2\sqrt{\frac{GM}{r^3}}
	  = m\sqrt{GMr}
\stopformula
Circular orbits with a larger radius have a greater angular angular momentum.

Finally, we find each planet's total energy $H$, which include's the planet's gravitational potential energy $U$ and kinetic energy $K$. The calculation is a bit messy, but the result is surprisingly simple.
\startformula\startmathalignment
\NC H	\NC = U + K			\NR
\NC		\NC =  - G\frac{mM}{r} + \frac{L^2}{2I}	\NR
\NC		\NC =  - G\frac{mM}{r} + \frac{m^2 GMr}{2mr^2}	\NR
\NC		\NC =  - G\frac{mM}{r} + G\frac{mM}{2r}		\NR
\NC		\NC = -G\frac{mM}{2r}						\NR
\stopmathalignment\stopformula
The total energy of a circular orbit is negative, and exactly half of the gravitational potential energy. 
Circular orbits with a larger radius $r$ have a higher (less negative) total energy $H$.

\startuseMPgraphic{graph::CircularH} % I'd like to add minor ticks, 0.667mm long.
vardef U =
	path p;
		for x = 0.5 step 0.1 until 3.1:
			y := -13.3/x; % lua.mp.morse(x);
			augment.p(x,y);
		endfor;
	p enddef;
vardef H =
	path p;
		for x = 0.2 step 0.1 until 3.1:
			y := -13.3/(2x); % lua.mp.morse(x);
			augment.p(x,y);
		endfor;
	p enddef;
draw begingraph(4cm,5cm);
	setrange(0,-20, 3, 0);
	for x=auto.x:
		itick.bot(formatted("$@g$", x), x);
		itick.bot(formatted("@s", ""), x) withcolor "middlegray";
		itick.top(formatted("@s", ""), x) withcolor "middlegray";
	endfor
	glabel.lft(textext("Energy per mass ($\sci{8}\units{J/kg}$)") rotated 90,OUT);
	glabel.bot(textext("Distance from Sun $r$ ($\sci{11}\units{m}$)"), OUT);
	gdraw(U) withpen pencircle scaled 0.8pt;
	glabel.lrt("$U$",10);
	gdraw(H) withpen pencircle scaled 1pt dashed withdots;
	glabel.ulft("$H$",10);
%	glabel(mydot,(80));
%	glabel(mydot,(210));
%	glabel(mydot,(340));
%	gfill(unitsquare xyscaled (6.37,-2)) withcolor "lightgray";
%	gdraw((6.37,0) -- (6.37,-2)) withpen pencircle scaled 0.8pt;
	for y=0 step -2 until -20:%auto.y:
		itick.lft(formatted("$@g$", y), y);
		itick.lft(formatted("@s", ""), y) withcolor "middlegray";
		itick.rt(formatted("@s", ""), y) withcolor "middlegray";
	endfor
endgraph shifted (0,-5cm);
%  pickup pencircle scaled 0.8pt ;
%  draw externalfigure "EarthEratosthenes.png" scaled 0.127 shifted (-6.37mm,-6.37mm) ;
%  draw fullcircle scaled 12.74mm;
%  drawarrow (13mm,0) -- (-0.5mm,0);
%    dotlabel.ulft  ("", (13mm,0)) ;
%  drawarrow (26mm,0) -- (22.625mm,0);
%    dotlabel.top  ("$F\sub{Newton}$", (26mm,0)) ;
%  drawarrow (39mm,0) -- (37.5mm,0);
%    dotlabel.llft  ("", (39mm,0)) ;
\stopuseMPgraphic

\startplacefigure[location=margin, reference=fig:CircularH, title={A planet's total energy $H$ in a circular, centered orbit is exactly half of the planet's gravitational potential energy $U$.}]
\small\reuseMPgraphic{graph::CircularH}
\stopplacefigure

The energy graph in \in{figure}[fig:CircularH] shows the gravitational potential energy and a dotted line at $-GMm/2r$. Each dot represents the total energy of a possible circular orbit. Dots toward the upper right represent larger orbits with greater angular momentum and higher energy. These are the orbits of the outer planets. Dots toward the lower left represent smaller orbits with less angular momentum and lower energy. These are the orbits of the inner planets.

%\section{Separation of compound motion}


% Need references in this chapter!
\subject{Notes}
\placenotes[endnote][criterium=chapter]

\subject{Bibliography}
  \placelistofpublications

\stopchapter
\stopcomponent