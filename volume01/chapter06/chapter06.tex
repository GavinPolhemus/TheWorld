% !TEX useOldSyncParser
\startcomponent c_chapter06
\project project_world
\product prd_volume01

\doifmode{*product}{\setupexternalfigures[directory={chapter06/images}]}

\setupsynctex[state=start,method=max] % "method=max" or "min"

%%%%%%%%%%%%%%%%%%%%%%%%%%%%%
\startchapter[title=Hamilton’s Canonical Equations of Motion,reference=ch:Hamilton]
%%%%%%%%%%%%%%%%%%%%%%%%%%%%%

\placefigure[margin,none]{}{\small
	\startalignment[flushleft]
The theoretical development of the laws of motion of bodies is a problem of such interest and importance, that it has engaged the attention of all the most eminent mathematicians, since the invention of dynamics as a mathematical science by \scaps{Galileo}, and especially since the wonderful extension which was given to that science by \scaps{Newton}. Among the successors of those illustrious men, \scaps{Lagrange} has perhaps done more than any other analyst, to give extent and harmony to such deductive researches, by showing that the most varied consequences respecting the motions of systems of bodies may be derived from one radical formula; the beauty of the method so suiting the dignity of the results, as to make of his great work a kind of scientific poem\dots.
%But the science of force, or of power acting by law in space and time, has undergone already another revolution, and has become already more dynamic, by having almost dismissed the conceptions of solidity and cohesion, and those other material ties, or geometrically imaginably conditions, which \scaps{Lagrange} so happily reasoned on, and by tending more and more to resolve all connexions and actions of bodies into attractions and repulsions of points: and while the science is advancing thus in one direction by the improvement of physical views, it may advance in another direction also by the invention of mathematical methods.
And the method proposed in the present essay, for the deductive study of the motions of attracting or repelling systems, will perhaps be received with indulgence, as an attempt to assist in carrying forward so high an inquiry.%\autocite[p.48]{Galileo1610}
	\stopalignment
	\startalignment[flushright]
	{\it On A General Method In Dynamics}\\
	{\sc William Rowan Hamilton}\\
	1805--1865
	\stopalignment
}

\Initial{I}{n \booktitle{Hydrodynamica} and his other works,} Daniel Bernoulli applied sharp physical insights to practical problems – pipes, fountains, pumps, and other useful machines. Everyone else was going in other directions.

Engineers developed and applied methods specific to specialized applications without looking for broadly applicable physical principles. This practical approach was incredibly successful. Eighteenth century engineers launched the industrial revolution, building powerful working engines decades before physicists could explain them.

Physicists largely ignored the engines, focusing their attention on abstract problems in pure mechanics – planetary motion, spinning tops, and jointed pendulums. These problems had little practical value, but they inspired rapid advances in applied mathematics.
Leonhard Euler (a close friend of Danial Bernoulli), adapted and expanded Newton's methods to solve many new types of problems. Other physicists introduced new principles – like energy conservation – to solve problems in entirely new ways. These advances produced a rather confusing patchwork of methods and principles.

Joseph-Louis Lagrange sought to replace the patchwork with a single method that could be applied to any problem in mechanics. He delivered his synthesis in \booktitle{Mécanique Analytique}, published in 1788, almost exactly a century after Newton's \booktitle{Principia}. Lagrange describes his intent in the monumental work's preface.

\startblockquote
There already exist several treatises on mechanics, but the purpose of this one is entirely new. I propose to condense the theory of this science and the method of solving the related problems to general formulas whose simple application produces all the necessary equations for the solution of each problem. %I hope that my presentation achieves this purpose and leaves nothing lacking.
%In addition, this work will have another use. The various principles presently available will be assembled and presented from a single point of view in order to facilitate the solution of the problems of mechanics.
\stopblockquote

Lagrange's single method could be used to solve any mechanical system, from Galileo's falling rock and interrupted pendulum to Kepler's planetary orbits and Mersenne's mysterious musical string. We must take a moment to understand what such an ambitious general solution would mean. It is hard enough to solve one problem at a time. How could Lagrange address every conceivable problem at once?

\section{Equations of motion}
First, we must understand the goal of a general solution, which is to predict the system's motion. This is accomplished when we know the position coordinates ($x$, $y$, $\theta$, etc.) for every part of the system at every time. For example, we would like to know the position of a cannon ball at every time between when it is fired upward at an angle and when it hits the ground. Galileo gave us the first tool that we need to accomplish this, the position update formulas.

\placefigure[margin][fig:cannonEoMxy] % location
{A cannon ball's curving path is found by repeatedly updating its position coordinates $x$ and $y$.}	% caption text
{\vskip1.4in\hbox{\starttikzpicture
	\draw[white] (0,0)-- ++(5,0); % Sky to make height better
\stoptikzpicture}}

\placewidefloat
  [top,none]
  {This is its caption I need to fix.}
{\hbox{\small\starttikzpicture	% tikz code
%\draw[->,ultra thick] (12,1.5) -- node[above, pos=.6]{$p\si$}(13,1.5);
%\shade[right color=gray,left color=white] (.6,.24) rectangle (0.9,.26);
\draw[shade, ball color = black] (0.78,.61) circle[radius=.05cm]; % Ball initial
\fill[fill=black!70] (0.76,0.47)-- ++(-.12,.16)-- ++(-.51,-.32)-- ++(0.18,-0.24)-- cycle; % Cannon Barrel
\fill[fill = black!70] (0.22,0.19) circle[radius=0.15cm]; % Cannon Back
\draw[fill=gray] (0,0)-- ++(0,0.05)-- ++(.2,.2)-- ++(.2,0)-- ++(.2,-.2)-- ++(0,-0.05)-- cycle; % Cannon Base
\draw[fill = black!70] (0.3,0.25) circle[radius=0.05cm]; % Cannon Pivot
\shade[top color=gray] (0,-.2) rectangle (16.7,0); % Ground
\draw (0,0)--(16.7,0);
\draw[thick,->] ({0.78+(2.25/4)},0.61)node[right]{$dy$} -- ++(0,0.45); % dy
\draw[thick,->] (0.78,0.61)--node[below]{$dx$} ++({2.25/4},0); % dx
\draw[->] (0.78,0.61) parabola [bend pos=0] bend +(6.75,2.53) ({0.78+6.25*2.25},{0.61+6.25*1.6875-6.25*6.25*0.28125}); % Trajectory
\foreach \T in {0.25, 0.5, 0.75, 1, 1.25, 4.75, 5, 5.25, 5.5, 5.75, 6}{% Balls
	\draw[shade, ball color = black][opacity={(5+\T)/20}] ({0.78+\T*2.25},{0.61+\T*1.6875-\T*\T*0.28125}) circle[radius=.05cm]; % Balls
	\draw[thick,->][opacity={(5+\T)/20}] ({0.78+(2.25/4)+\T*2.25},{0.61+\T*1.6875-\T*\T*0.28125})-- ++(0,{0.45-\T*0.15}); % dy
	\draw[thick,->][opacity={(5+\T)/20}] ({0.78+\T*2.25},{0.61+\T*1.6875-\T*\T*0.28125})-- ++({2.25/4},0); % dx
}
\foreach \T in {1.5, 1.75, 2, 2.25, 2.5, 2.75, 3, 3.25, 3.5, 3.75, 4, 4.25, 4.5}{% Balls
	\draw[shade, ball color = black][opacity={(5+\T)/20}] ({0.78+\T*2.25},{0.61+\T*1.6875-\T*\T*0.28125}) circle[radius=.05cm]; % Balls
	\draw[thick][opacity={(5+\T)/20}] ({0.78+(2.25/4)+\T*2.25},{0.61+\T*1.6875-\T*\T*0.28125})-- ++(0,{0.45-\T*0.15}); % dy
	\draw[thick,->][opacity={(5+\T)/20}] ({0.78+\T*2.25},{0.61+\T*1.6875-\T*\T*0.28125})-- ++({2.25/4},0); % dx
}
\draw[shade, ball color = black] ({0.78+6.25*2.25},{0.61+6.25*1.6875-6.25*6.25*0.28125}) circle[radius=.05cm]; % Ball final
%\draw[thick,->] (14.28,0.61)--node[right]{$dy$} ++(0,-1.8); % dy
%\draw[thick,->] (14.28,0.61)--node[below]{$dx$} ++({2.25/4},0); % dx
\stoptikzpicture}}

\startbuffer[cannonEoMp]
\draw[shade, ball color = black!60] (0,0) circle[radius=.25cm]; % Ball
\filldraw (0,0) circle[radius=.02cm]; % Ball cm
\draw[thick,->] (0,0)--node[above left]{$\vec p$} ++(4.8,3.6); % Momentum initial
\draw[thick,->] (4.8,+3.6)--node[right]{$d\vec p$} +(0,-0.3); % dp
\foreach \T in {0.25, 0.5, 0.75, 1, 1.25, 1.5, 1.75, 2, 2.25, 2.5, 2.75, 3, 3.25, 3.5, 3.75, 4, 4.25, 4.5, 4.75, 5, 5.25, 5.5, 5.75, 6}{% Balls
	\draw[thick,->][opacity={(5+\T)/20}] (0,0)-- ++(4.8,{3.6-\T*1.2}); % Momentum
	\draw[thick,->][opacity={(5+\T)/20}] (4.8,{3.6-\T*1.2})-- +(0,-0.3); % dp
}
\draw[thick,->] (0,0)-- ++(4.8,-3.9); % pf
\stopbuffer

\marginTikZ{}{cannonEoMp}{The cannon ball's momentum is updated repeatedly to find the momentum at any later time.} % vskip, name, caption
\startformula
	dx = v_x\,dt
	\qquad
	dy = v_y\,dt
\stopformula
Starting from the cannon ball's initial coordinates, we can use the cannon ball's velocity to find its location a short time later. These position updates are shown in \in{figure}[fig:cannonEoMxy]. The process is repeated over and over to find the position at any time, provided we know the velocity at each moment. However, knowing the velocity is a problem. Galileo studied important examples that have a known velocity – uniform motion, free fall, and simple harmonic motion – but this is far from a general method.

Descartes pointed the way forward with his idea for a conserved quantity of motion. His formula did not work, but two later ideas succeeded. The first was Newton's momentum, updated using his second law. The momentum update formulas for the cannon ball are
\startformula
	dp_x = 0,
	\qquad
	dp_y = -mg\,dt.
\stopformula
The momentum's $x$-component is constant, so its change is always zero. The momentum's $y$-component changes at a constant rate in the downward direction, so its change is always the same size and negative.
These momentum updates are shown in \in{figure}[fig:cannonEoMp].
The updates are all proportional to the small duration $dt$ of our time steps.
%We start with the cannon ball's initial momentum and update it repeatedly to find the momentum at any later time. The momentum gives us the velocity we need to update the position.
Putting this together with Galileo's position update formulas gives the cannon ball's \keyterm{equations of motion}.
\startformula\startalign[m=2,distance=5em]
\NC dx	\NC = \frac{p_x}{m}\,dt	\NC dp_x	\NC = 0	\NR
\NC dy	\NC = \frac{p_y}{m}\,dt	\NC dp_y	\NC = -mg\,dt	\NR
\stopalign\stopformula
Equations of motion must provide a formula to update each coordinate and each momentum. For the two dimensional motion of the cannon ball we must have four update formulas. The right side of the formulas can include known constants (like $m$ and $g$) and the current coordinates and momenta. For the cannon ball's equations of motion, we see that the momentum update formulas contain only constants, and the position update formulas contain constants and momentum components. (I replaced the velocity components with their momentum equivalents.) These are good equations of motion.
%In the the words of Lagrange, these are good equations of motion because they are \quotation{all of the necessary equations for the solution of [the] problem."


\startexample[ex:EoMSHO] Write the equation of motion for the system of the cart (mass $m$) and spring (stiffness $k$) shown in \in{figure}[fig:EoMSHO].

\startbuffer[EoMSHO]
	\fill [black!10] (-.23,0) rectangle (4.8,-.15);
	\fill [black!10] (0,0) rectangle (-.23,.6);
	\draw[thin] (0,0) -- (0,.6);
	\startaxis[margin cart track,
			xmin=-24,xmax=24,
			ymax=10,
			]
	\path (0,0) pic {cart}node[above = 5mm]{$m$};
	\draw[decorate,decoration={coil,segment length=3.6pt}] (-24,2.5) --node[above=3pt] {$k$} (-6,2.5);
    \stopaxis
\stopbuffer

\marginTikZ{}{EoMSHO}{A cart connected to a spring for \in{example}[ex:EoMSHO]. This spring's natural length is $18\units{cm}$, so it holds the cart gently at $x=0\units{cm}$.} % vskip, name, caption

\startsolution
Since the cart moves in one dimension, we only need one coordinate $x$ and one momentum $p$. The two update formulas are
\startformula
	dx = v\, dt,
	\qquad
	dp = F\,dt.
\stopformula
The cart's velocity is $v=p/m$ and the force is given by Hooke's law, $F = -kx$. Putting these in the update formulas gives the equations of motion.
\startformula
	dx = \frac{p}{m}\, dt
	\qquad
	dp = -kx\,dt
\stopformula
In this case the momentum changes depend on the position. The change is always toward the central equilibrium at $x=0$. %Solving with by doing repeated updates produces simple ha
\stopsolution
\stopexample

Once we have the equations of motion, we must solve them. For the cannon ball, we start with the cannon ball's initial position and momentum, and then use the update formulas over and over to find the position and momentum at any later time. Sometimes the equations of motion are easy to solve. The equation for $p_x$ is easy to solve. Since the updates are always zero, we know that the momentum's  $x$-component at every time is the same as its initial value (\in{fig.}[fig:cannonEoMp]). Sometimes the equations of motion can be solved with more advanced mathematical tools. In many cases, the equations of motion can only be solved through the tedious process of repeated calculation for many, many small time steps. Today, these repetitive calculation are performed rapidly by computers – once they are given the system's equations of motion.

For the cannon ball, and many other simple systems, Newton's method produces the equations of motion quite easily. However, Newton's approach is not well suited to the two quadrivium problems we hope to solve: the planets' compound motion discovered by Kepler and the musical string's compound vibrations discovered by Mersenne. The quadrivium problems are much easier to approach using the second successful quantity of motion, kinetic energy.

We have already seen how rotational kinetic energy allows us to study spinning or rolling motion, where many connected parts move with different velocities but share a common angular motion. Angular coordinates are also the natural coordinates for planetary motions. We will use angular coordinates and rotational kinetic energy to find and solve the planets' equations of motion in \in{Chapter}[ch:Rotation]. Like a spinning object, a musical string has many parts moving with different velocities. The string's motion is best described in terms of shared motions rather than the individual parts' motions. We will use the shared motion's energy to find and solve the string's equations of motion in \in{Chapter}[ch:Waves]. Before we attack those problems, we must find a connection between the energy and equations of motion.

Energy is not directional, so it works very well for systems with parts that move in many directions, (like spinning and rolling) and in situations where the coordinates are not straight (like angular coordinates). Perhaps kinetic energy could provide the velocities we need for Galileo's position update formulas.

Unfortunately, kinetic energy is not enough. Since kinetic energy is not directional, it cannot give us the velocity components needed for the position update formulas. In the case of the cannon ball, we can easily use conservation of energy to find the cannon ball's speed at any height, but conservation of energy cannot tell us the velocity's direction, which we need to correctly update the position. (In one dimension, where velocity has only one component, kinetic energy is often enough, but we will not use this one-dimensional trick.)

\section{Lagrange's step in the right direction}

Lagrange found the directional information we need not in the kinetic energy $K$, but hidden in the potential energy $U$. His equation for converting Daniel Bernoulli's potential energy into Newtons force contains a virtual displacement $\partial x$.
\startformula
	F\sub{Newton} = -\frac{\partial U}{\partial x}
\stopformula
The virtual displacement's direction can gives us directional information from $U$. 
Remember that $\partial U$ is the small virtual change in the potential energy due to the small virtual displacement $\partial x$. If there are more coordinates, like $y$ for the cannon ball, then there are more directions for the virtual displacement. Lagrange found that he could get each component of Newton's force by choosing the correct direction for the virtual displacement. He found the $x$-component of Newton's force with a virtual displacement in the $x$-direction, the $y$-component with a virtual displacement in the $y$-direction, and so on for all the coordinates of the system. (These virtual displacements do not actually happen in the physical system, so Lagrange was free to imagine different virtual displacements to get the force's different components.)

\startbuffer[cannondx]
\draw[shade, ball color = black!60] (0,0) circle[radius=.25cm]; % Ball
\filldraw (0,0) circle[radius=.02cm]; % Ball cm
\draw[thick,->] (-2.5,-1)--node[above]{$x$} ++(1,0); % x-axis
\draw[opacity=0] (-2.5,-1)-- ++(5,0); % x-floor to get the ball centered
\draw[thick,->] (-2.5,-1)--node[right]{$y$} +(0,1); % y-axis
\draw[thick,->] (-0.5,0)--(0.5,0)node[right]{$\partial x$} ; % pf
\stopbuffer

\marginTikZ{}{cannondx}{A small horizontal virtual displacement $\partial x$ produces no change in the gravitational potential energy. $\partial U = 0$. } % vskip, name, caption
For the cannon ball, a small virtual displacement $\partial x$ in the $x$-direction is a horizontal displacement (shown in \in{figure}[fig:cannondx]), which does not change the cannon ball's gravitational potential energy at all. For this horizontal displacement, $\partial U = 0$. Lagrange's equation gives the $x$-component of the gravitational force. (I will drop \quotation{Newton} from the suffix for the remainder of this section.)
\startformula
	F_{x} = -\frac{\partial U}{\partial x}
		= -\frac{0}{\partial x}
		= 0
\stopformula
The gravitational force's $x$-components is zero, as we knew. A small virtual displacement $\partial y$ in the $y$-direction is a vertical displacement (shown in \in{figure}[fig:cannondy]), which does change the cannon ball's gravitational potential energy. For this vertical displacement, $\partial U = mg\partial y$, which leads to the $y$ component of the gravitational force.
\startbuffer[cannondy]
\draw[shade, ball color = black!60] (0,0) circle[radius=.25cm]; % Ball
\filldraw (0,0) circle[radius=.02cm]; % Ball cm
\draw[thick,->] (-2.5,-1)--node[above]{$x$} ++(1,0); % x-axis
\draw[opacity=0] (-2.5,-1)-- ++(5,0); % x-floor to get the ball centered
\draw[thick,->] (-2.5,-1)--node[right]{$y$} +(0,1); % y-axis
\draw[thick,->] (0,-0.5)--(0,0.5)node[above]{$\partial y$} ; % pf
\stopbuffer

\marginTikZ{}{cannondy}{A small vertical virtual displacement $\partial x$ increases the gravitational potential energy. $\partial U = mg\partial y$.} % vskip, name, caption
\startformula
	F_{y} = -\frac{\partial U}{\partial y}
		= -\frac{mg\cancel{\partial y}}{\cancel{\partial y}}
		= -mg
\stopformula
\startbuffer[cannonForce]
\draw[shade, ball color = black!60] (0,0) circle[radius=.25cm]; % Ball
\filldraw (0,0) circle[radius=.02cm]; % Ball cm
\draw[thick,->] (-2.5,-2)--node[above]{$x$} ++(1,0); % x-axis
\draw[opacity=0] (-2.5,-2)-- ++(5,0); % x-floor to get the ball centered
\draw[thick,->] (-2.5,-2)--node[right]{$y$} +(0,1); % y-axis
\draw[thick,->] (0,0)--node[right]{$\vec F = \components{0,-mg}$} ++(0,-2); % pf
\stopbuffer

\marginTikZ{\vskip 0.5cm}{cannonForce}{The gravitational force on the cannon ball, with components found using Lagrange's equation.} % vskip, name, caption
\noindent
The gravitational force's $y$-component is $-mg$, again agreeing with what we knew. These two components give the gravitational force vector shown in \in{figure}[fig:cannonForce].
While potential energy itself does not have a direction, Lagrange's procedure produces the force vector, which does have a direction.

Keeping track of the different virtual changes $\partial U$ due to the different virtual displacements $\partial x$ and $\partial y$ can get confusing, but Lagrange has good news. The ratios $\partial U\!/\partial x$ and $\partial U\!/\partial y$ can be easily found using the same derivative rules that we learned for rates of change (like $v = dx/dt$). When computing $\partial U\!/\partial x$ we keep $y$ constant. When computing $\partial U\!/\partial y$ we keep $x$ constant.

\startexample[ex:CannonDirivaties]
Use Lagrange's equation and the derivative rules to find the force components $F_x$ and $F_y$ on the cannon ball.
\startsolution
The $x$ component of the force is given by Lagrange's equation with a virtual displacement in the $x$-direction.
\startformula
	F_x = -\frac{\partial U}{\partial x} = -\ppx U = -\ppx mgy = 0
\stopformula
When calculating $\partial U\!/\partial x$ we keep $y$ constant, which makes $mgy$ constant. The derivative of a constant is zero.

The $y$ component of the force is given by Lagrange's equation with a virtual displacement in the $y$-direction.
\startformula
	F_y = = -\frac{\partial U}{\partial y} = -\ppy U = -\ppy mgy =  -mg\ppy y = -mg
\stopformula
After pulling the coefficient $mg$ out of the derivative, we compute $\ppy y$ using the power rule with $n=1$.
\stopsolution
\stopexample

The ratios $\partial U\!/\partial x$ and $\partial U\!/\partial y$ are called partial derivatives of $U$ because we compute them allowing only one part of our coordinates (either $x$ or $y$) to change while other parts are kept constant. Computing partial derivatives is much easier than dealing with many tiny virtual displacements. As a result, Lagrange replaced many diagrams with partial derivatives.

Lagrange's equations for the components of Newton's forces were totally unlike Newton's own methods for working with forces. Newton's \booktitle{Principia} is full of diagrams. Newton's calculations are done with graphical methods, and equations are few. In the preface to his \booktitle{Mécanique Analytique}, Lagrange explains his analytical method.
\startblockquote
No figures will be found in this work. The methods I present require neither constructions nor geometrical or mechanical arguments, but solely algebraic operations subject to a regular and uniform procedure. Those who appreciate mathematical analysis will see with pleasure mechanics becoming a new branch of it and hence, will recognize that I have enlarged its domain.
\stopblockquote
In \booktitle{Mécanique Analytique}, Lagrange used his powerful mathematical analysis to expand the equations for Newton's forces into a procedure for analyzing any mechanical system, described using any coordinates. Starting with only the formulas for the system's total kinetic energy and total potential energy, Lagrange's \quotation{regular and uniform procedure} produces a complete set of equations of motion for the system. (Lagrange's equations of motion are a bit more tangled than our position and momentum update formulas, so we will not learn the details of his procedure.)

Lagrange's \booktitle{Mécanique Analytique} was a stunning achievement, unlike anything physics had seen before. Lagrange was viewed as a hero in France under King Louis \convertnumber{KR}{16}. Luckily, his fame transcended the turbulent – often violent – politics of France following the French Revolution in 1789. As governments rose and fell around him, Lagrange continued to play a leading role in the advancement of science and mathematics in France.

\section{Hamilton's canonical equations of motion}

While physics advanced on the continent, the situation in England was dire. English physicists were ignoring the industrial revolution transforming their civilization, and ignoring French mathematical revolution transforming physics.
They remained paralyzed by a fanatic devotion to Newton’s cumbersome, geometric methods and continued to hold a grudge over the Leibniz’s calculus and \visviva. This problem persisted until 1812, when a group of young undergraduate students at Cambridge started the Analytical Society, partly as a joke, to study the methods of French analysis. They learned to solve problems using the continental notation and even translated a few important French books and papers into English. The club met for a few years and essentially disappeared after its members graduated and left Cambridge in 1817.

That probably would have been the end of this little revolt, except that one of the members, George Peacock, was appointed in 1817 to write questions for the rigorous Mathematical Tripos exam taken by all third year mathematics students at Cambridge. He audaciously wrote his questions using the continental notation. This caused a bit of grumbling among the faculty, but they did not interfere. Surprisingly, Peacock was asked to write questions again in 1819. Students recognized this as a complete surrender by the Newtonian faculty. From there, the adoption of continental notation and methods in England was quite swift. Newton remained a revered figure, but his 150 year old dispute with Leibniz was finally laid to rest. %Cannell p. 38

The United Kingdom began producing great physicists. Among the first was William Rowan Hamilton. Hamilton was a genius, gifted in languages and mathematics. He learned Hebrew, Latin, and Greek from early childhood. At the age of 15 he read Newton’s \booktitle{Principia}, and a two years later the great French work \booktitle{M\'echanique C\'eleste}, by Pierre-Simon Laplace.%Cannell p. 39

In 1823, at the age of 18, Hamilton entered Trinity College Dublin.
Trinity College had adopted the continental notation in 1812, a bit ahead of Cambridge, and used the French textbooks in mathematics and physics courses. Hamilton embraced the abstract approach. He earned top marks in all of his courses and won many awards, which propelled him to a position as head of the Dunsink Observatory, near Dublin. Although the observatory provided ample opportunity for practical astronomy, Hamilton continued to focus on more theoretical interests, first in optics and then in mechanics.

Hamilton wrote two groundbreaking papers on mechanics, \booktitle{On A General Method In Dynamics} in 1834 and \booktitle{Second Essay On A General Method In Dynamics} in 1835. Hamilton's general method produces the equations of motion for any problem in mechanics, just like Lagrange's procedure, but Hamilton's general method is much easier to understand and execute.
We will study Hamilton's general method in this chapter. First, we will learn to produce equations of motion from a system's total energy formula. Then, we will learn some methods of solving these equations of motion in specific examples. In \in{Chapter}[ch:Rotation], Hamilton's general method will explain the compound motions of planets, and in \in{Chapter}[ch:Waves] it will reveal the compound vibrations of a musical string.

Hamilton's general method brings together the big ideas of position, momentum, and energy in a surprisingly simple way. Hamilton found that he did not need two separate formulas for kinetic energy and potential energy, one formula for the total energy is enough. (He chose the symbol $H$ for total energy, which we have adopted.) Hamilton also recognized momentum's importance, and used momentum in his formula for total energy. For example, to describe the cart-on-a-spring system in \in{figure}[fig:HamiltonSpringCart], Hamilton would write the total energy
\startbuffer[HamiltonSpringCart]
	\fill [black!10] (-.23,0) rectangle (4.8,-.15);
	\fill [black!10] (0,0) rectangle (-.23,.6);
	\draw[thin] (0,0) -- (0,.6);
	\startaxis[margin cart track,
			xmin=-24,xmax=24,
			ymax=10,
			]
	\path (0,0) pic {cart}node[above = 5mm]{$m$};
	\draw[decorate,decoration={coil,segment length=3.6pt}] (-24,2.5) --node[above=3pt] {$k$} (-6,2.5);
    \stopaxis
\stopbuffer

\marginTikZ{}{HamiltonSpringCart}{A cart connected to an ideal spring.} % vskip, name, caption
\startformula
  H = \textfrac{1}{2m}p^2 + \textfrac{k}{2}x^2. %\frac{p^2}{2m} + \half kx^2
\stopformula
The coefficients $1/2m$ and $k/2$ are constants. The dynamical variables $p$ and $x$ describe the system's changing momentum and position.
Euler and Lagrange used velocity rather than momentum, writing kinetic energy as $K=\onehalf mv^2$. Using velocity is reasonable, but it caused the update formulas to get tangled. Hamilton discovered that using momentum allowed him to separate the update formulas, making the equations of motion much easier to understand and solve. When a system's total energy is written in Hamilton's form, with positions and momenta, it is called the system's \keyterm{Hamiltonian}.

%\section{Hamilton's momentum update formula}

%Hamilton's momentum update formula uses and idea similar to Lagrange's equation, 

Hamilton followed Daniel Bernoulli's advice and described all interactions as potential energy, not Newtonian forces. There is a bit of an obstacle to this approach. Lagrange's equation tells us how to convert to Newton's force from potential energy, but it does not tell us how to convert the other way. In fact, some forces (like friction) cannot be described using potential energy and must be left as outside forces acting on the system. We will use $F$ only for these external forces. Any interaction that can be described using potential energy will be included in the energy formula.

The system's total energy $H$ and the the external force $F$ come together in Hamilton's momentum update formula.
\startformula[eq:Hamilton2]
	dp = \left(-\ppx H + F \right) dt
\stopformula
Parts of this new formula should be reassuringly familiar. For a system with only external forces (no energy) this is Newton's momentum update formula, $dp = F\,dt$. The new term on the right looks almost identical to Lagrange's equation, $F\sub{Newton} = -\partial U/\partial x$, but Hamilton replaces the potential energy $U$ with the total energy $H$. This replacement will be essential when using curved coordinates, like the polar coordinate used to describe orbits.%, but it has no effect in most of the systems we have studied.

To see Hamilton's procedure in action, let us find the momentum update formula for the familiar cart-on-a-spring system. First, put the system's Hamiltonian into Hamilton's momentum update formula. The spring is included in the system, so there are no external forces.
\startformula
	dp = \left( -\ppx H + \cancel{F} \right) dt
		%= -\ppx H dt
		= -\ppx \left( \textfrac{1}{2m}p^2 + \textfrac{k}{2}x^2 \right) dt
\stopformula
Use the sum rule and then the coefficient rule to pull the coefficients out of the partial derivatives.
\startformula\startmathalignment
\NC	dp	\NC = -\left( \ppx  \textfrac{1}{2m}p^2 + \ppx\textfrac{k}{2}x^2 \right) dt		\NR
\NC		\NC = -\left( \textfrac{1}{2m} \ppx p^2 + \textfrac{k}{2} \ppx x^2 \right) dt	\NR
\stopmathalignment\stopformula
When calculating the partial derivative $\partial/\partial x$, the variable $p$ is constant. Therefore, the constant rule for derivatives tells us that the first term in parenthesis is zero. The second term is found using the power rule with $n=2$.
\startformula
	dp = -\left( 0 + \textfrac{k}{2} 2x^{(2-1)} \right) dt
		= - kx\,dt
\stopformula
This is the correct position update formula that we found before using Hooke's law. 

Hamilton found that the total energy $H$ also produces the position update formula. 
\startformula
	dx = \pp{p} H\,dt
\stopformula
This formula is not familiar, but with your new derivative skills it will be easy to use.
\startexample[ex:CannonDirivaties]
Use Hamilton's position update formula to find the position update formula for the cart-on-a-spring system in \in{figure}[fig:HamiltonSpringCart].
\startsolution
Put the cart-on-a-spring Hamiltonian into Hamilton's position update formula. Then use derivative rules to pull the coefficients out of the partial derivatives.
\startformula
	dx = \pp{p} H\,dt
		= \pp{p} \left( \textfrac{1}{2m}p^2 + \textfrac{k}{2}x^2 \right) dt
		= \left( \textfrac{1}{2m} \pp{p} p^2 +\textfrac{k}{2}\pp{p}x^2 \right) dt
\stopformula
When calculating the partial derivative $\partial/\partial p$, the variable $x$ is constant. Therefore, the constant rule for derivatives tells us that the second term in parenthesis is zero. The first term is found using the power rule with $n=2$.
\startformula
	dx = \left( \textfrac{1}{2m} 2 p^{(2-1)} +0 \right) dt
		= \textfrac{1}{m} p\,dt
\stopformula
This is the familiar position update formula $dx = v\,dt$ with the velocity $v = p/m$.
\stopsolution
\stopexample

Hamilton's general method produces the complete equations of motion for any mechanical system from the system's Hamiltonian – the system's total energy written in terms of the positions and momenta. Hamilton's method is a favorite with physicists because of its power, beauty, and practicality. The method it is also excellent for students. Writing the Hamiltonian is often fairly easy, even for complicated system; the method often requires only basic skills with derivatives; and the equations of motion are separated into their most useful form. Hamilton's method is not necessary for a simple system like the cart on a spring, but it is ideal for our quadrivium problems. 

\stopchapter
\stopcomponent

Before we apply Hamilton's method to the quadrivium problems, we need a couple specific skills. In the next section, we will find angular equations of motion for the familiar pendulum. In the following section, we will look at a system with several degrees of freedom. Hamilton's method produces equations of motion that we can completely solve to find the system's compound motion. 

\section{Angular coordinates, angular momentum, and torque}

\startbuffer[TikZ:GalileoPendulumAngle]
\environment env_physics
\environment env_TikZ
\setupbodyfont [libertinus,11pt]
\setoldstyle % Old style numerals in text
\startTEXpage\small
\starttikzpicture% tikz code
%	\draw [help lines, xstep=8, ystep=.34] (-4.3,0) grid (4.3,4.3); % Background grid
%	\draw (-4.3,-0.5) rectangle (4.3,4.5); % Border
	% h axis
	\draw[
		postaction={decorate},
		decoration={
			markings, % Main marks
			mark=between positions 0 and 1 step 1cm with {
				\draw (0,0)
				node[left]{
					\pgfmathparse{
						10-10*\pgfkeysvalueof{%
							/pgf/decoration/mark info/sequence number%
						}
					}
					\pgfmathprintnumber{\pgfmathresult}
				} -- (0,4pt);
			},
		}
	] (-4.3,4) -- (-4.3,0);
	\draw[
		postaction={decorate},
		decoration={
			markings, % Main marks
			mark=between positions 0 and 1 step 1mm with {
				\draw (0,0) -- (0, -2pt);
			},
		}
	] (-4.3,0) --node[sloped,above=5mm]{$h$ (cm)} (-4.3,4);
	% U axis
	\draw[
		postaction={decorate},
		decoration={
			markings, % Main marks
			mark=between positions 0 and 1 step 6.8mm with {
				\draw (0,0)
				node[right]{
					\pgfmathparse{
						0.1*(1-\pgfkeysvalueof{%
							/pgf/decoration/mark info/sequence number%
						})
					}
					\pgfmathprintnumber{\pgfmathresult}
				} -- (0,-4pt);
			},
		}
	] (4.3,4) -- (4.3,-0.082);
	\draw[
		postaction={decorate},
		decoration={
			markings, % Main marks
			mark=between positions 0 and 1 step 3.4mm with {
				\draw (0,0) -- (0,2pt);
			},
		}
	] (4.3,-0.082) --node[sloped, below=6mm]{$U$ (J)} (4.3,4);
	\fill (0,4) circle[radius=.4mm]; % Pivot
%	\node at (0,0) [above left]{B}; % Bottom
	\draw[middlegray] (0,4) -- (0,3); % Central vertical
	\draw[middlegray] (0,2) -- (0,0); % Central vertical
%	\node at (-3.2,1.6) [above=2mm]{C}; % Left
%	\node at (3.2,1.6) [above=1mm]{D}; % Right
%	\fill (0,2) circle[radius=.4mm]node[left]{E}; % 2nd nail
%	\fill (0,1) circle[radius=.4mm]node[left]{F}; % 3nd nail
%	\node at (1.833,1.6) [above=1mm]{G}; % Right
%	\node at (0.98,1.6) [above=1mm]{I}; % Right
	\draw[middlegray] (-4.0,4) -- (4.0,4); % x axis
	\draw[middlegray] (-4.0,0.8) -- (4.0,0.8); % horizontal at max height
	% Pendulum path
	\draw[middlegray] (0,0) arc[start angle=270, end angle=360, radius=4cm];
	\draw[middlegray] (0,0) arc[start angle=270, end angle=180, radius=4cm];
	\draw[<->] (-0.6,4) --node[sloped, below, pos = 0.6]{$R\cos(\theta)$} (-0.6,0.8); % height
	% Theta marks
	% Positive on the right
	\draw[
		postaction={decorate},
		decoration={
			markings, % Main marks
			mark=between positions 0 and 1 step 1cm with {
				\draw (0,0) -- (0,-4pt)
				node[below,transform shape]{
					\pgfmathparse{
						-1+\pgfkeysvalueof{%
							/pgf/decoration/mark info/sequence number%
						}
					}
					\pgfmathprintnumber{\pgfmathresult}
				};
			},
		}
	] (0,3) arc[start angle=270, end angle=360, radius=1cm];
	\draw[
		postaction={decorate},
		decoration={
			markings, % Main marks
			mark=between positions 0 and 1 step 0.998mm with {
				\draw (0,0) -- (0,-2pt);
			},
		}
	] (0,3) arc[start angle=270, end angle=360, radius=1cm];
	% Negative on the left
	\draw[
		postaction={decorate},
		decoration={
			markings, % Main marks
			mark=between positions 0 and 1 step 1cm with {
				\draw (0,0) -- (0,4pt)
				node[below, transform shape, rotate=180]{
					\pgfmathparse{
						1-\pgfkeysvalueof{%
							/pgf/decoration/mark info/sequence number%
						}
					}
					\pgfmathprintnumber{\pgfmathresult}
				};
			},
		}
	] (0,3) arc[start angle=270, end angle=180, radius=1cm];
	\draw[
		postaction={decorate},
		decoration={
			markings, % Main marks
			mark=between positions 0 and 1 step 0.998mm with {
				\draw (0,0) -- (0,2pt);
			},
		}
	] (0,3) arc[start angle=270, end angle=180, radius=1cm];
	\node at (0,3) [below=5mm]{$\theta$ (rad)};
	% Pendulum
	\draw[thick] (0,4) --node[sloped,above]{$R$} (2.4,0.8); % String
	\draw[ball color=white] (2.4,0.8) circle[radius=2mm]; % Ball , opacity=.5
	\fill (2.4,0.8) circle[radius=.4mm]; % CoM
	%\draw[very thick, ->] (-3.2,1.6) ++(-3mm,4mm) -- ++(6mm,-8mm)node[above right]{$\partial s$}; % ds
\stoptikzpicture
\stopTEXpage
\stopbuffer

\placetextfloat[bottom][fig:GalileoPendulumAngle] % location
{The pendulum's height is measured from the pivot, so height and potential energy are negative as long as the pendulum is below the pivot.}	 % caption text
{\noindent\typesetbuffer[TikZ:GalileoPendulumAngle]} % figure contents

A pendulum's position is naturally described by the angle it makes with the vertical. \in{Figure}[fig:GalileoPendulumAngle] shows a pendulum with its angular position in radians. We will find the pendulum's equations of motion using the angular coordinate $\theta$, and the \keyterm{angular momentum} $p_\theta$. First, we will find the Hamiltonian. Then, we will use Hamilton's equations to find the update formulas for $\theta$ and $p_\theta$.

The Hamiltonian is the pendulum's total energy. When working with the the angle it is convenient to set the pivot at  height $h=0$, making the pendulum's height negative whenever it is below the pivot, as shown in \in{figure}[fig:GalileoPendulumAngle]. The pendulum's potential energy is then $U = -mgR\cos(\theta)$. The pendulum's rotational kinetic energy (\at{p.}[eq.Krot]) is $K = \onehalf I\omega^2$, where the moment of inertial $I$ plays the role of rotational mass. Unfortunately, this is written in terms of angular velocity $\omega$. The Hamiltonian must be written in terms of angular momentum $p_\theta$. The obvious guess is $K = p_\theta^2/2I$, where $I$ again plays the role of rotational mass. This guess is correct. The Hamiltonian is
\startformula
  H = \textfrac{1}{2I}p_\theta^2 - mgR\cos(\theta).
\stopformula
The first term – the rotational kinetic energy – will be necessary whenever we use an angular coordinate. It would be good to remember. The second term is specific to the potential energy of a pendulum.

Now that we have the Hamiltonian, we can use Hamilton's equations to find the equations of motion. %These will be the update formulas for $\theta$ and $p_\theta$.
(No external forces)
\startformula
	d\theta = \pp{p_\theta} H\,dt
	\qquad
	dp_\theta = \left(-\ppth H + \cancel{F_\theta} \right) dt
\stopformula
We will find the angle update formula first.
\startformula
	d\theta %= \pp{p_\theta} H\,dt
		= \pp{p_\theta} \left(\textfrac{1}{2I}p_\theta^2 - mgR\cos(\theta) \right) dt
		= \left( \textfrac{1}{2I} \pp{p_\theta} p_\theta^2 - 0 \right) dt
		= \textfrac{1}{I} p_\theta\,dt
\stopformula
This is the position update formula for angular motion, $d\theta = \omega\,dt$, with angular velocity $\omega = p_\theta / I$.
We see that angular momentum $p_\theta$ is related to the angular velocity in the same way that Newton's momentum is related to velocity.
\startformula
	p_\theta = I\omega
\stopformula
All of the steps above will be relevant to any problem with an angular coordinate. The momentum update formula, which we turn to next, is a bit trickier and specific to the pendulum. Follow the process, but you will not need to reproduce the details in your own calculations.

We begin, of course, with Hamilton's momentum update formula and apply the derivative rules.
\startformula
	dp_\theta %= \left( -\ppth H + \cancel{F} \right) dt
		%= -\ppth H dt
		= -\ppth \left( \textfrac{1}{2I}p_\theta^2 - mgR\cos(\theta) \right) dt
		= mgR \ppth \cos(\theta)\,dt
\stopformula
Completing this calculation requires one calculus fact:
\startformula
	\ppth \cos(\theta) = -\sin(\theta)
\stopformula
(In a calculus course you would learn the derivatives of many special functions. You do not need to learn them here.) With this fact we can finish the calculation.
\startformula
dp_\theta %= mgR \ppth \cos(\theta)\,dt
	= -mgR \sin(\theta)\,dt
\stopformula
%\startformula\startmathalignment
%\NC	dp_\theta	\NC = -\left( \ppth  \textfrac{1}{2I}p_\theta^2 - \ppth mgR\cos(\theta) \right) dt		\NR
%\NC		\NC = -\left( \textfrac{1}{2I} \ppx p_\theta^2 - mgR \ppth \cos(\theta) \right) dt	\NR
%\NC		\NC = mgR \ppth \cos(\theta)\,dt	\NR
%\NC		\NC = -mgR \sin(\theta)\,dt	\NR
%\stopmathalignment\stopformula
This is the usual momentum update formula $p_\theta = F_\theta\,dt$ with an external angular force $F_\theta = -mgR\sin(\theta)$. Angular force is called \keyterm{torque}. %, and is often represented by $\tau$ (the greek letter tau).
The torque is negative when the angle $\theta$ is positive, and the torque is positive when $\theta$ is negative. The resulting change $dp_\theta$ in the angular momentum is always toward the equilibrium at $\theta = 0$, much like the cart on a spring. Both systems oscillate in a similar manner.

The $\sin(\theta)$ in the momentum update formula makes the pendulum's equations of motion difficult to solve by hand. A computer using the equations of motion to do rapid, repeated updates would be the best way to model the pendulums behavior precisely.

\section{Throwing Newton's globes}
Our final example combines two earlier examples. We return to the cannon ball moving in the $x$ and $y$-directions, but we replace the cannon ball with Newton's connected globes from \in{Chapter}[ch:VisViva] (\at{pp.}[GlobesStart]--\at[GlobesStop]). These globes rotate about their center of mass. Their orientation is described by the angle $\theta$ that the rod makes with the horizon.
\in{Figure}[fig:ThrowNewtonGlobes] shows a bit of the globes' complicated motion. We could begin drawing angles and coordinates, but this could become a mess quite quickly. Instead, we will embrace Hamilton's analytical method. 
\startbuffer[TikZ:ThrowNewtonGlobes]
\environment env_physics
\environment env_TikZ
\setupbodyfont [libertinus,11pt]
\setoldstyle % Old style numerals in text
\startTEXpage
\def\angles{522,504,...,-90}
\starttikzpicture% tikz code
	\clip (-14,-4) rectangle (2.5,1.7);% Clipping Rectangle
	\foreach \T in \angles {% Dumbbells
		\draw[-{Straight Barb[scale length=.5]}] ({-3.14*(\T+18)/120},{-((\T-200)/200)^2}) -- ({-3.14*\T/120},{-((\T-218)/200)^2})pic[{}
		-{},rotate={\T},opacity={.5-(\T/1200)}]{dumbbell};
	}
\stoptikzpicture
\stopTEXpage
\stopbuffer

%\placefigure[margin][fig:spinningmoving] % location
%{As the joined globes revolve, their center moves uniformly in a straight line.}	% caption text
%{\noindent\typesetbuffer[TikZ:spinningmoving]} % figure contents

\placefigure[margin][fig:ThrowNewtonGlobes] % location
{Newton's joined globes move in three coordinates, $x$, $y$, and $\theta$. They revolve with constant angular velocity $\omega$ as their center follows a parabolic path like the cannon ball.}	% caption text
{\vskip5.1in\hbox{\starttikzpicture
	\draw[white] (0,0)-- ++(5,0); % Sky to make height better
\stoptikzpicture}}

\placewidefloat[bottom,none]
{This is its caption I need to fix.}
{\hbox{\noindent\typesetbuffer[TikZ:ThrowNewtonGlobes]}} % figure contents
First, we must write the Hamiltonian. This includes three contributions to the kinetic energy, one from each of the three momenta $p_x$, $p_y$, and $p_\theta$. The only potential energy contribution comes from the height $y$ of center of mass. 

\startformula
  H = \textfrac{1}{2m}p_x^2 + \textfrac{1}{2m}p_y^2 + \textfrac{1}{2I}p_\theta^2 + mgy.
\stopformula
Take a moment to be sure that you know what each term in the Hamiltonian represents. Notice that the $p_\theta$ term has the moment of inertia $I$ in place of the mass.

From the Hamiltonian we will find six equations of motion – three update formulas for the three coordinates $x$, $y$, and $\theta$, and three update formulas for the there momenta $p_x$, $p_y$, and $p_\theta$. We will find the momentum update formulas first, beginning with $p_x$.
\startformula
	dp_x = \left( -\ppx H + \cancel{F} \right) dt
		%= -\ppx H dt
		= -\ppx \left( \textfrac{1}{2m}p_x^2 + \textfrac{1}{2m}p_y^2 + \textfrac{1}{2I}p_\theta^2 + mgy \right) dt
		= 0
\stopformula
None of the terms in the Hamiltonian depend on the horizontal position $x$ so the change $dp_x$ in the horizontal momentum is always zero. This means that $p_x$ is a constant of the motion, just as it was for the cannon ball.

A similar simplification happens with the update formula for the angular momentum $p_\theta$.
\startformula
	dp_\theta = \left( -\ppth H + \cancel{F} \right) dt
		%= -\ppx H dt
		= -\ppth \left( \textfrac{1}{2m}p_x^2 + \textfrac{1}{2m}p_y^2 + \textfrac{1}{2I}p_\theta^2 + mgy \right) dt
		= 0
\stopformula
None of the terms in the Hamiltonian depend on the angle $\theta$ so the change $dp_\theta$ in the angular momentum is always zero. This means that $p_\theta$ is a constant of the motion. 
%Solve: $p_\theta = p_{0\theta} = I\omega_{0}$.

%Solve: $p_x = p_{0x} = m v_{0x}$.

\startformula
	dx = \pp{p_x} H\,dt
		= \pp{p_x} \left( \textfrac{1}{2m}p_x^2 + \textfrac{1}{2m}p_y^2 + \textfrac{1}{2I}p_\theta^2 + mgy \right) dt
		= \textfrac{1}{m} p_x\,dt
\stopformula
The momentum $p_x$ is a constant of the motion, but we do not treat it as a constant when calculating the partial derivative $\pp{p_x}$. Partial derivatives compare virtual changes which do not necessarily happen in the real system. In this case, we are considering the virtual momentum change $\partial p_x$ even though the actual momentum change $d p_x$ is zero.

Solve: $x = x_0 + v_{0x} t$.


\startformula
	d\theta = \pp{p_\theta} H\,dt
		= \pp{p} \left( \textfrac{1}{2m}p_x^2 + \textfrac{1}{2m}p_y^2 + \textfrac{1}{2I}p_\theta^2 + mgy \right) dt
		= \textfrac{1}{m} p\,dt
\stopformula
Solve: $\theta = \theta_0 + \omega_0 t$.

\startformula
	dp_y = \left( -\ppx H + \cancel{F} \right) dt
		%= -\ppx H dt
		= -\ppy \left( \textfrac{1}{2m}p_x^2 + \textfrac{1}{2m}p_y^2 + \textfrac{1}{2I}p_\theta^2 + mgy \right) dt
		= -mg\,dt
\stopformula

Solve: $p_y = p_{0y} - mgt = m(v_{0y} - gt)$.

\startformula
	dy = \pp{p_y} H\,dt
		= \pp{p_y} \left( \textfrac{1}{2m}p_x^2 + \textfrac{1}{2m}p_y^2 + \textfrac{1}{2I}p_\theta^2 + mgy \right) dt
		= \textfrac{1}{m} p_y\,dt
\stopformula
Solve: $y = y_0 + v_{0y}t - \half mgt^2$.

\stopchapter
\stopcomponent


\section{Playing with the equations' formatting}

\startformula
	dx = \pp{p} H\,dt
	\qquad
	dp = \left(-\ppx H + F \right) dt
\stopformula
\startformula
	dx_i = \pp{p_i} H\,dt
	\qquad
	dp_i = \left(-\pp{x_i} H + F_i \right) dt
\stopformula

\section{Hamilton's total triumph}

\startformula\startmathalignment
\NC	dH	\NC = \frac{\partial H}{\partial x}\,dx + \frac{\partial H}{\partial p}\,dp	\NR
\NC		\NC = \frac{\partial H}{\partial x}\frac{\partial H}{\partial p}\,dt
				+ \frac{\partial H}{\partial p}\left(-\frac{\partial H}{\partial x} + F \right) dt	\NR
\NC		\NC = \frac{\partial H}{\partial x}\frac{\partial H}{\partial p}\,dt
				- \frac{\partial H}{\partial p}\frac{\partial H}{\partial x}\,dt
					+ F\,\frac{\partial H}{\partial p}\,dt	\NR
\NC		\NC = F\,dx	\NR
\stopmathalignment\stopformula

\startformula\startmathalignment
\NC	dH	\NC = \sum_i^n \left[ \frac{\partial H}{\partial x_i}\,dx_i + \frac{\partial H}{\partial p_i}\,dp_i \right]	\NR
\NC		\NC = \sum_i^n \left[ \frac{\partial H}{\partial x_i}\frac{\partial H}{\partial p_i}\,dt
				+ \frac{\partial H}{\partial p_i}\left(-\frac{\partial H}{\partial x_i} + F_i \right) dt \right]	\NR
\NC		\NC = \sum_i^n \left[ \frac{\partial H}{\partial x_i}\frac{\partial H}{\partial p_i}\,dt
				- \frac{\partial H}{\partial p_i}\frac{\partial H}{\partial x_i}\,dt
					+ F_i\,\frac{\partial H}{\partial p_i}\,dt \right]	\NR
\NC		\NC = \sum_i^n F_i\,dx_i	\NR
\stopmathalignment\stopformula

\subject{Notes}
%\placefootnotes[criterium=chapter]
\placenotes[endnote][criterium=chapter]

%\subject{Bibliography}
%        \placelistofpublications


\stopchapter
\stopcomponent


%
%
%\startbuffer[TikZ:RigidPendulumPath1]
%\environment env_physics
%\environment env_TikZ
%\setupbodyfont [libertinus,11pt]
%\setoldstyle % Old style numerals in text
%\startTEXpage\small
%\starttikzpicture% tikz code
%	\draw [help lines, xstep=8, ystep=.34] (-4,0) grid (4.6,8.2); % Background grid
%%	\draw (-4.3,-0.5) rectangle (4.3,4.5); % Border
%	% h axis
%	\draw[
%		postaction={decorate},
%		decoration={
%			markings, % Main marks
%			mark=between positions 0 and 1 step 1cm with {
%				\draw (0,0)
%				node[left]{
%					\pgfmathparse{
%						-10+10*\pgfkeysvalueof{%
%							/pgf/decoration/mark info/sequence number%
%						}
%					}
%					\pgfmathprintnumber{\pgfmathresult}
%				} -- (0,-4pt);
%			},
%		}
%	] (-4,0) -- (-4,8);
%	\draw[
%		postaction={decorate},
%		decoration={
%			markings, % Main marks
%			mark=between positions 0 and 1 step 1mm with {
%				\draw (0,0) -- (0,-2pt);
%			},
%		}
%	] (-4,0) --node[sloped,above=5mm]{$h$ (cm)} (-4,8);
%	% U axis
%	\draw[
%		postaction={decorate},
%		decoration={
%			markings, % Main marks
%			mark=between positions 0 and 1 step 6.8mm with {
%				\draw (0,0)
%				node[right]{
%					\pgfmathparse{
%						0.1*(-1+\pgfkeysvalueof{%
%							/pgf/decoration/mark info/sequence number%
%						})
%					}
%					\pgfmathprintnumber{\pgfmathresult}
%				} -- (0,4pt);
%			},
%		}
%	] (4.6,0) -- (4.6, 8.164);
%	\draw[
%		postaction={decorate},
%		decoration={
%			markings, % Main marks
%			mark=between positions 0 and 1 step 3.4mm with {
%				\draw (0,0) -- (0,2pt);
%			},
%		}
%	] (4.6,0) --node[sloped,below=6mm]{$U$ (J)} (4.6, 8.164);
%	\fill (0,4) circle[radius=.4mm]; % Pivot
%%	\node at (0,0) [above left]{B}; % Bottom
%%	\draw (0,-0.2) -- (0,4.2); % Central vertical
%%	\node at (-3.2,1.6) [above=2mm]{C}; % Left
%%	\node at (3.2,1.6) [above=1mm]{D}; % Right
%%	\fill (0,2) circle[radius=.4mm]node[left]{E}; % 2nd nail
%%	\fill (0,1) circle[radius=.4mm]node[left]{F}; % 3nd nail
%%	\node at (1.833,1.6) [above=1mm]{G}; % Right
%%	\node at (0.98,1.6) [above=1mm]{I}; % Right
%%	\draw (-4.0,1.6) -- (4.0,1.6); % horizontal at max height
%	% Pendulum path
%%	\draw[] (0,0) arc[start angle=270, end angle=336.4, radius=2cm];
%%	\draw[] (0,0) arc[start angle=270, end angle=371.5, radius=1cm];
%	% Positive on the right
%	\draw[
%		postaction={decorate},
%		decoration={
%			markings, % Main marks
%			mark=between positions 0 and 1 step 1cm with {
%				\draw (0,0) -- (0,-4pt)
%				node[below,transform shape]{
%					\pgfmathparse{
%						-10+10*\pgfkeysvalueof{%
%							/pgf/decoration/mark info/sequence number%
%						}
%					}
%					\pgfmathprintnumber{\pgfmathresult}
%				};
%			},
%		}
%	] (0,0) arc[start angle=-90, end angle=150, radius=4cm];
%	\draw[
%		postaction={decorate},
%		decoration={
%			markings, % Main marks
%			mark=between positions 0 and 1 step 1mm with {
%				\draw (0,0) -- (0,-2pt);
%			},
%		}
%	] (0,0) arc[start angle=-90, end angle=150, radius=4cm];
%	% Negative on the left
%	\draw[
%		postaction={decorate},
%		decoration={
%			markings, % Main marks
%			mark=between positions 0 and 1 step 1cm with {
%				\draw (0,0) -- (0,4pt)
%				node[below, transform shape, rotate=180]{
%					\pgfmathparse{
%						10-10*\pgfkeysvalueof{%
%							/pgf/decoration/mark info/sequence number%
%						}
%					}
%					\pgfmathprintnumber{\pgfmathresult}
%				};
%			},
%		}
%	] (0,0) arc[start angle=270, end angle=201, radius=4cm];
%	\draw[
%		postaction={decorate},
%		decoration={
%			markings, % Main marks
%			mark=between positions 0 and 1 step 1mm with {
%				\draw (0,0) -- (0,2pt);
%			},
%		}
%	] (0,0) arc[start angle=270, end angle=201, radius=4cm];
%	\node at (0,0) [below=5mm]{$s$ (cm)};
%	% Pendulum
%	\draw[thick] (0,4) --node[sloped,above]{$40\units{cm}$} (0,0); % String
%	\draw[ball color=white] (0,0) circle[radius=2mm]; % Ball , opacity=.5
%	\fill (0,0) circle[radius=.4mm]; % CoM
%\stoptikzpicture
%\stopTEXpage
%\stopbuffer
%
%\placetextfloat[top][fig:RigidPendulumPath1] % location
%{Galileo’s pendulum with the position $s$ shown along the ball’s curved path.}	 % caption text
%{\noindent\typesetbuffer[TikZ:RigidPendulumPath1]} % figure contents
%
%\startbuffer[TikZ:RigidPendulumGraphU]
%\environment env_physics
%\environment env_TikZ
%\setupbodyfont [libertinus,11pt]
%\setoldstyle % Old style numerals in text
%\startTEXpage\small
%\starttikzpicture% tikz code
%	\startaxis[
%			scale only axis,
%			x={1mm},y={68mm},
%			xmin=-24, xmax=134,
%			minor x tick num=1,
%			xlabel=$s$ (cm),
%			%axis x line=none,
%			%axis y line*=right,
%			ymin=-0.05, ymax=1.2,
%			minor y tick num=3,
%			ylabel=Energy (J),
%			grid=both
%		]
%		\addplot[thick, domain=-24:134] {0.588*(1-cos(deg(x/40)))}node[above left, pos=.4]{$U$};
%		\addplot[thick, domain=-24:134] {0.588*2}node[below, pos=.6]{Tangent at maximum $U$};
%		\addplot[thick, domain=-24:134] {0}node[above, pos=.5]{Tangent at minimum $U$};
%		%\addplot[thick, domain=0:75] {0.2205*(1-cos(deg(x/15))};
%%		\draw[thin](-37,0) --node[pos=.7, below, sloped]{Release Position} (-37,.7);
%%		\addplot[thick, red, domain=-37:37] {0.235}node[above, pos=.3]{$H$};
%%		\addplot[thick, red, domain=-37:37] {0.235-0.588*(1-cos(deg(x/40))}node[below right, pos=.6]{$K$};
%%		\addplot[thick, red, domain=0:23] {0.22-0.2205*(1-cos(deg(x/15))};
%%	\draw[red, thin](37,0) --node[pos=.7, above, sloped]{Turning Point} (37,.7);
%	\stopaxis
%\stoptikzpicture
%\stopTEXpage
%\stopbuffer
%
%\placefigure[margin][fig:RigidPendulumGraphU] % location
%{An energy graph showing the ball’s gravitational potential energy as a function of position $s$ along the curved path.}	% caption text
%{\vskip27pt\hbox{\starttikzpicture
%	\draw[white] (0,0)-- ++(5,0); % Sky to make height better
%\stoptikzpicture}}
%
%\placewidefloat[bottom,none]
%{This is its caption I need to fix.}
%{\hbox{\noindent\typesetbuffer[TikZ:RigidPendulumGraphU]}} % figure contents
%



\startbuffer[TikZ:CartSlopeSpring]
\environment env_physics
\environment env_TikZ
\setupbodyfont [libertinus,11pt]
\setoldstyle % Old style numerals in text
\startTEXpage\small
\starttikzpicture% tikz code
\startaxis[
	big diagram cart track,
	xmin=-54,xmax=54,
	ymin=0,
	ymax=50,
	%axis x line=center,
	style={rotate=5.7},
	clip=false
]
\pic[rotate=5.7] at (6,0){cart};
%\pic[rotate=5.7] at (15,0){block};
%\fill [black!10, on layer={axis background}] (-1,0) rectangle (151,-1.5);
%\pic at (49,0){block};
\draw[decorate,decoration={coil,segment length=5pt}] (-50,2.5) --node[above=3pt] {$k$} (0,2.5);
%\draw[decorate,decoration={coil,segment length=6pt}] (151,2.5) -- (52,2.5);
\draw (-50,0) -- (-50,6);
\stopaxis
\stoptikzpicture
\stopTEXpage
\stopbuffer

\placetextfloat[bottom][fig:CartSlopeSpring] % location
{Galileo’s pendulum with the position $s$ shown along the ball’s curved path.}	 % caption text
{\noindent\typesetbuffer[TikZ:CartSlopeSpring]} % figure contents




Lagrange’s equation connects forces to potential energy. If an apple’s gravitational potential energy decreases when the apple goes down, then there is a gravitational force pushing the apple down. If a spring’s potential energy increases when it is stretched, then the spring exerts a force pulling back towards its unstretched length. If the gravitational potential energy of a planet is lower when it is closer to the sun, then there is a gravitational force pulling the planet towards the sun. In fact, the equation gives the magnitude as well as the direction of the force. Using Leibniz’s notation, Lagrange’s equation is
\highlightbox{
\startformula[eq:Hamilton2]
	F = -\frac{\partial U}{\partial x}
\stopformula
}
The force can easily be found on an energy vs.~position graph.
\startformula[eq:Laplace]
	F = -\frac{\text{tangent’s $\Delta U$}}{\text{tangent’s $\Delta x$}}
\stopformula

\section{Gravitational Force}

\placefigure[margin][fig:BoxEarthGravU] % location
{The gravitational potential energy of a $3.0\units{kg}$ box as a function of its height $y$. The slope of the graph is $mg=29.4\units{J/m}$, so the $y$-components of the gravitational force is $F_y=-29.4\units{N}$.}	% caption text
{\starttikzpicture
\startaxis
 [footnotesize, width=2.20in, height=2in,
   xlabel={$y$ ($\units{m}$)},
   xmin=0, xmax=6,
   ylabel={$U$ ($\unit{J}$)},
   ymin=0, ymax=200,
   %ytick={-10,-8,...,0},
 ]
 \addplot[
   thick,
   domain=0:10,
   samples=2
  ]
  {29.4*x}
  ;
\stopaxis
\stoptikzpicture}

Force is directional, so forces will be represented by vectors. Gravitational force is downwards. We will use $y$ as our vertical component in most cases, so force is in the negative $y$ direction.
\startformula
	\vec F\sub{g} = \components{0, -mg , 0}
\stopformula

\placefigure[margin][fig:BoxEarthGravU] % location
{The gravitational potential energy of a $3.0\units{kg}$ box as a function of its horizontal position $x$, shown for three different heights. The slope of the graph is zero in every case, so the $x$-component of the gravitational force  is $F_x = 0\units{N}$.
}	% caption text
{\starttikzpicture
\startaxis
 [footnotesize, width=2.20in, height=2in,
   xlabel={$x$ ($\units{m}$)},
   xmin=0, xmax=6,
   ylabel={$U$ ($\unit{J}$)},
   ymin=0, ymax=200,
   %ytick={-10,-8,...,0},
 ]
 \addplot[
   thick,
   domain=0:6,
   samples=2
  ]
  {29.4}
  node[above,pos=0.5]{$y=1\units{m}$}
  ;
 \addplot[
   thick,
   domain=0:6,
   samples=2
  ]
  {29.4*3}
  node[above,pos=0.5]{$y=3\units{m}$}
  ;
 \addplot[
   thick,
   domain=0:6,
   samples=2
  ]
  {29.4*5}
  node[above,pos=0.5]{$y=5\units{m}$}
  ;
% \addplot[
%   thick,
%   domain=0:6,
%   samples=2
%  ]
%  {29.4*6}
%  node[below,pos=0.5]{$y=6\units{m}$}
%  ;
\stopaxis
\stoptikzpicture}

where $g = 9.8\units{N/kg}$ and the $y$ direction is upwards. From the equation we can see that the gravitational force does not have any $x$ or $z$ component and the $y$ component is negative. The gravitational force vector points straight down.


\section{Projectile motion}

Knowing the height of the drop, $h$, and the speed of the cart, $v\si$, there are several questions we might hope to answer.
\startitemize[1,packed,broad]
\startitem What is the cart’s speed, $\vabs{\vec{v}_{\text{f}}}$, when it hits the ground. \stopitem
\startitem What is the cart’s velocity, $\vec{v}_{\text{f}}$, when it hits the ground. \stopitem
\startitem How long will it take for the cart to reach the ground ($\Delta t$). \stopitem
\startitem How far away from the table with the cart hit the ground ($\Delta x$). \stopitem
\stopitemize
We have three basic equations that will be useful: the conservation equations for energy and momentum and the position update formula. Let’s see what each of these can tell us about the motion of the cart.

\placefigure[margin][] % location
{The path of a cart going off of a cliff.}	% caption text
{\starttikzpicture
	\startaxis[%axis equal,
		footnotesize,
		width=2.25in,%2.20in,
		y={1cm},x={1cm},
		xlabel={$x$ (m)},
		xmin=0, xmax=4,
		xtick={0,1,...,4},
		%minor x tick num=9,
		ylabel={$y$ (m)},
		ymin=0, ymax=6,
		ytick={0,1,...,6},
		%minor y tick num=4,
		]
  \addplot[samples=100, variable=\t, domain=0:1]
    ({4*t}, {5-5*t^2});
  \addplot[samples=6, domain=0:1,
    % the default choice ’variable=\x’ leads to
    % unexpected results here!
  	mark = *, mark size={.4pt},
    variable=\t,
    quiver={
        u={3},
        v={-7.5*t},
        scale arrows=0.2}, thick,
        ->]
    ({4*t}, {5-5*t^2});
	\stopaxis
\stoptikzpicture}

% Aligned Equations
\startformula\startmathalignment[m=2,distance=2em]
\stopmathalignment\stopformula

Starting with conservation of energy, we will consider the cart’s kinetic energy, $K$, and gravitational potential energy, $U$.
\startformula\startmathalignment[m=2,distance=2em]
\NC	E\sf \NC = E\si + \cancel{W} + \cancel{Q}	\NC \NC \text{no outside work or heat}\NR
\NC	K\sf + \cancel{U\sf} \NC = K\si + U\si		\NC \NC \text{set $U=0$ at the floor}\NR
\NC	\half m\abs{\vec{v}\sf}^2 \NC = \half mv\si^2 + mgh
			\NC \NC \text{formulae for $K$ and $U$}	\NR
\NC	\abs{\vec{v}\sf} \NC = \sqrt{v\si^2 + 2gh}
			\NC \NC \text{solve for speed}
\stopmathalignment\stopformula
Compare this to the formula derived in Section \ref{sec:freefall}. Aside from the new vector notation, the formulae are almost identical. This is not a coincidence. Energy is not directional, so kinetic energy only depends on speed. This makes energy easy to work with, especially if all you want at the end is speed.

To find the velocity rather than just the speed, we will use conservation of momentum. Rather than working with the full vector notation, we will find it helpful to break the equation into three component equations.
\startformula\startmathalignment[m=3,distance=2em]
\NC	\NC \NC 	\vec{p}\sf \NC = \vec{p}\si + \vec{F}\Delta t	\NR[1.5ex]
\NC	p\sub{f,$x$} \NC = p\sub{i,$x$} + \cancel{F_x\Delta t}	\NC
	p\sub{f,$y$} \NC = \cancel{p\sub{i,$y$}} + F_y\Delta t	\NC
	p\sub{f,$z$} \NC = \cancel{p\sub{i,$z$}} + \cancel{F_z\Delta t}	\NR
\NC	p\sub{f,$x$} \NC = p\sub{i,$x$}	\NC
	p\sub{f,$y$} \NC = -mg\Delta t	\NC
	p\sub{f,$z$} \NC = 0
\stopmathalignment\stopformula
Since the force is entirely in the $y$ direction, the $x$ and $z$ components of the force are zero. Force is required to change momentum, so the $x$ and $z$ components of the momentum do not change, meaning $x$ and $z$ components of the velocity are also constant.
\startformula\startmathalignment[m=2,distance=2em]
%	\NC \NC 	\vec{p}\sf \NC = \vec{p}\si + \vec{F}\Delta t	\NR
\NC	p\sub{f,$x$} \NC = p\sub{i,$x$} 	\NC
%	p\sub{f,$y$} \NC = p\sub{i,$y$} + F_y\Delta t	\NC
	p\sub{f,$z$} \NC = 0	\NR
\NC	mv\sub{f,$x$} \NC = mv\sub{i,$x$} 	\NC
%	mv\sub{f,$y$} \NC = mv\sub{i,$y$} + F_y\Delta t	\NC
	mv\sub{f,$z$} \NC = 0	\NR
\NC	v\sub{f,$x$} \NC = v\si 	\NC
%	v\sub{f,$y$} \NC = v\sub{i,$y$} + F_y\Delta t	\NC
	v\sub{f,$z$} \NC = 0
\stopmathalignment\stopformula
That is two components of the final velocity. What about the $y$ component of the final velocity? Since the $y$ component of the force is not zero, we have to actually calculate the final $y$ component of momentum using the force of gravity and the time of the fall. Unfortunately, we don’t know the time of the fall; we only know the height.

This is our first example of a very common problem when working in more than one dimension. Often the coordinates can be chosen so that motion in some of the directions is completely simple (like the constant momentum in the $x$ and $z$ directions), but something interesting is happening in the remaining direction. The trick is to return to conservation of energy, and use the simple motion to better understand the more complicated motion.

\startformula\startmathalignment[m=2,distance=2em]
\NC	E\sf \NC = E\si + \cancel{W} + \cancel{Q}	\NC \NC \text{no outside work or heat}\NR
\NC	K\sf + \cancel{U\sf} \NC = K\si + U\si		\NC \NC \text{set $U=0$ at the floor}\NR
\NC	K\sub{f,x} + K\sub{f,y} + K\sub{f,z}
		\NC = K\sub{i,x} + K\sub{i,y} + K\sub{i,z} + U\si		\NC \NC \text{break $K$ into pieces}	\NR
\NC	K\sub{f,x} + K\sub{f,y} + \cancel{K\sub{f,z}}
		\NC = K\sub{i,x} + \cancel{K\sub{i,y}} + \cancel{K\sub{i,z}} + U\si		\NC \NC \text{some pieces are zero}	\NR
\NC	K\sub{f,x} + K\sub{f,y}
		\NC = K\sub{i,x} + U\si		%\NC \NC \text{some pieces are zero}
\stopmathalignment\stopformula
This looks like a bit of a mess, but it simplifies rapidly. The $x$ component of the momentum is constant, so the $x$ piece of the kinetic energy is also constant.
\startformula
	K\sub{f,x} = \frac{p\sub{f,x}^2}{2m} = \frac{p\sub{i,x}^2}{2m} = K\sub{i,x}
\stopformula
This part of the kinetic energy just cancels between the two sides.
\startformula\startmathalignment[m=2,distance=2em]
\NC	\cancel{K\sub{f,x}} + K\sub{f,y}
		\NC = \cancel{K\sub{i,x}} + U\si	\NC \NC \text{subtract $K_x$ from both sides}\NR
\NC	\half mv\sub{f,y}^2 \NC = mgh		\NC \NC \text{formulae for $K$ and $U$}\NR
\NC	v\sub{f,y} \NC = \pm\sqrt{2gh}			\NC \NC \text{solve for $v\sub{f,y}$}\NR
\NC	v\sub{f,y} \NC = -\sqrt{2gh}			\NC \NC \text{negative as it hits the ground}
\stopmathalignment\stopformula
Now we have all three components of the final velocity.

Next we can find the time of the fall by returning to the $y$ component of the conservation of momentum formula. That component could not help us find $v\sub{f,y}$ because we did not have $\Delta t$, but now we can use it to find $\Delta t$ since we have $v\sub{f,y}$.
\startformula\startmathalignment[m=2,distance=2em]
\NC		p\sub{f,$y$} \NC = -mg\Delta t	\NC \NC \text{conservation of momentum}\NR
\NC		mv\sub{f,$y$} \NC = - mg\Delta t	\NC \NC \text{formulae for $\vec{p}$}\NR
\NC		\Delta t \NC = - \frac{v\sub{f,$y$}}{g} \NC \NC \text{solve for $\Delta t$}	\NR
\NC		\Delta t \NC = - \frac{-\sqrt{2gh}}{g} \NC \NC \text{$v\sub{f,$y$}$ from above}	\NR
\NC		\Delta t \NC = \sqrt{\frac{2h}{g}} \NC \NC \text{simplify}
\stopmathalignment\stopformula

Finally, we find the displacement in the $x$ direction using the position update formula. This works because the velocity in the $x$ direction is constant.
\startformula\startmathalignment%[m=2,distance=2em]
\NC	\Delta x \NC = v_x\Delta t	\NR
\NC	\Delta x \NC = v\si\sqrt{\frac{2h}{g}}
\stopmathalignment\stopformula

\startexample[ex:BalloonLaunch] In their quest for knowledge, physics students launch a water balloon from the top of the school onto the soccer field below. The water balloon has a mass of $0.50\units{kg}$ and is launched with with a speed of $15\units{m/s}$ and an angle $53\degree$ above the horizontal, as shown in figure~\ref{fig:BalloonLaunch}. What is the water ballon’s speed when it hits the ground?

\startbuffer[TikZ:BalloonLaunch]
\environment env_physics
\environment env_TikZ
\setupbodyfont [libertinus,11pt]
\setoldstyle % Old style numerals in text
\startTEXpage\small
\starttikzpicture% tikz code
	\startaxis[%axis equal,
		footnotesize,
		width=2.25in,%\marginparwidth,
		y={0.1333cm},x={0.1333cm},
		xlabel={$x$ (m)},
		xmin=0, xmax=30,
		%xtick={0,1,...,4},
		%minor x tick num=9,
		ylabel={$y$ (m)},
		ymin=0, ymax=20,
		%ytick={0,1,...,6},
		%minor y tick num=4,
		clip=false,
		]
  \addplot[samples=100, variable=\t, domain=0:3.21]
    ({9*t}, {12+12*t-4.9*t^2});
  \addplot[samples=10, domain=0:3,
    % the default choice ’variable=\x’ leads to
    % unexpected results here!
  	mark = *, mark size={.4pt},
    variable=\t,
    quiver={
        u={9},
        v={12-9.8*t},
        scale arrows=0.333}, thick,
        ->]
    ({9*t}, {12+12*t-4.9*t^2});
  	\draw[fill=black!20] (0,0) rectangle (2,12);
	\stopaxis
\stoptikzpicture
\stopTEXpage
\stopbuffer

\placefigure[margin][fig:BalloonLaunch] % location
{The path of the projectile in example~\ref{ex:BalloonLaunch}}	% caption text
{\noindent\typesetbuffer[TikZ:BalloonLaunch]} % figure contents


\startsolution
	The water ballon starts with both potential and kinetic energy, but only has kinetic energy at the end.
	\startformula\startmathalignment
	\NC	H\si + \cancel{W} + \cancel{Q}	\NC = H\sf						\NR
	\NC	K\si + U\si					\NC = K\sf + \cancel{U\sf}			\NR
	\NC	\half mv\si^2 + mgh			\NC = \half mv\sf^2				\NR
	\NC	v\sf						\NC = \sqrt{v\si^2 + 2gh}			\NR
	\NC						\NC = \sqrt{(15\units{m/s})^2+2(9.8\units{m/s^2})(12.0\units{m})}\NR
	\NC					\NC = 21\units{m/s}
	\stopmathalignment\stopformula
	The balloon is traveling quite a bit faster when it gets down to the soccer field.
\stopsolution
\stopexample
	Since energy is not directional there is no need to break anything into components. The balloon speed at impact is not affected by the launch angle. The launch angle will affect the distance and the time aloft, but the impact speed depends only on the initial speed and height.


\section{Spring Forces}


\section{Universal Gravitation}

\placefigure[margin][] % location
{The path of a cart going off of a cliff.}	% caption text
{\starttikzpicture
	\startaxis[%axis equal,
		footnotesize,
		width=2.25in,%2.20in,
		y={1cm},x={1cm},
		xlabel={$x$ (m)},
		xmin=0, xmax=4,
		xtick={0,1,...,4},
		%minor x tick num=9,
		ylabel={$y$ (m)},
		ymin=0, ymax=6,
		ytick={0,1,...,6},
		%minor y tick num=4,
		]
  \addplot[samples=100, variable=\t, domain=0:1]
    ({4*t}, {5-5*t^2});
  \addplot[samples=6, domain=0:1,
    % the default choice ’variable=\x’ leads to
    % unexpected results here!
  	mark = *, mark size={.4pt},
    variable=\t,
    quiver={
        u={3},
        v={-7.5*t},
        scale arrows=0.2}, thick,
        ->]
    ({4*t}, {5-5*t^2});
	\stopaxis
\stoptikzpicture}


\section{Hamilton’s First Equation for Photons}
When introducing the formulas for momentum ($p=mv$) and kinetic energy ($K=p^2/2m$), I mentioned that these formulas will need to be replaced when we discuss particles traveling close to the speed of light.
Hamilton’s first equation does not tell us what the new formulas will be, but it does tell us how to get the new momentum formula once we have the new kinetic energy formula. For photons, particles of light, this is method is quite simple. A photon’s kinetic energy is proportional its momentum by
\startformula
	K=\abs{p}c
\stopformula
where $c=3.00\sci{8}\units{m/s}$ is the speed of light. The kinetic energy of photons is quite noticeable. In direct sunlight photons warm your face by delivering about six joules of kinetic energy every second. The total momentum of all of those photons is far too small to notice
\startformula
	\abs{p} = \frac{K}{c}
		= \frac{6\units{J}}{3\sci{8}\units{m/s}}
		= \frac{6\units{kg\.m^2/s^2}}{3\sci{8}\units{m/s}}
		= 2\sci{-8}\units{kg\.m/s}
\stopformula
You feel the warming energy of the photons, but you are not pushed by their momentum.

Hamilton’s first equation says that the photon’s velocity is determined from the total energy
\startformula[eq:Hamilton1]
	v = \frac{\partial H}{\partial p}
\stopformula
This is the slope of the energy vs.~momentum graph. The right side of the graph has a positive slope $c$. Photons with positive momentum have a positive velocity which is the speed of light. Good thing that particles of light have a velocity equal to the speed of light! On the left side of the graph the slope is negative $c$. Photons with momentum in the negative direction have a velocity which is $-c$, the speed of light, but in the negative direction.



\section{Contact Forces}
Objects in our everyday lives exert forces on each other when they come in contact.

\subsection{Spring Force}%: $F\sub{s} = kx$
Now that we are pros with vectors, we can write this as a vector equation.
$\vec{F}\sub{s} = - k\vec{x}^2$

\subsection{Friction Force}%: $F\sub{f} = \mu N$

This is interesting because the force of friction depends on another force, $N$.

\subsection{Other forces}

\section{Work}%: $\m{$W\sub{AB}=\vec F\sub{AB}\dotp \Delta \vec r$
In nearly all of the problems we will do, the energy is transferred through work.  When one object, call it A, pushes or pulls another object, B, then A transfers some energy to B.  The work done by A on B is given by
\startformula[work]
	W\sub{AB}=\vec F\sub{AB}\dotp \Delta \vec r = F\sub{AB} d \cos \theta
\stopformula
The force that A exerts on B is $\vec F\sub{AB}$, and the distance that B moves is $d$.
The angle $\theta$ is measured between the force vector, $\vec F\sub{AB}$, and the direction vector, $\Delta \vec r$.

Naturally, any energy that A gives to B can also be thought of as energy that B takes from A.  When B is pushed or pulled by A, we say that B has done negative work on A.  Mathematically, this means
\startformula
	W\sub{BA}=-W\sub{AB}
\stopformula
This consistent with formula for $W\sub{AB}$ because we also know that $\vec F\sub{BA}=-\vec F\sub{AB}$.
One exception is when A and B move different distances, perhaps because they are sliding against each other.  In this case some of the energy is going into something else, usually heat.

%\chapter{Work: Mechanical Energy Transfer}
%
%We can use the formulas for kinetic energy and momentum to get $P = \vec F \dotp \vec v$ (which seems to get time involved in a totally unnecessary way), but that requires taking a derivative.  I could start with $dW=\vec F\dotp d\vec x$.  As with all of my stuff with $d$, the math to back up this approach is differential forms, with $d$ being the exterior derivative.  However, for the junior high version I don’t have any other integrals, so I probably just need to stick with either $W=Fd$ or $W=\vec F \dotp \Delta \vec r$.  I have no idea how to get that from my energy and momentum formulas.
%
%Since I am doing vectors in the momentum section, it might be nice to round out vector algebra with the dot product.
%
%\subsection{The Vector Dot Product}
%I may have to teach the dot product before this.
%
%\subsection{Work}
%$W = \vec F \dotp \Delta \vec r$ % (Integrals of Vectors)
%
%Work out the gravity example, explain the spring example.
%
%\subsection{Using Force for Power}
%$P = \vec F \dotp \vec v$% or



\section{Power}%: $\Delta E = Pt$
%Now we enter the exciting wold of derivatives.  We didn’t mention that velocity is a derivative so this is our first exposure.  This is a good place to do it though.  We can introduce $\Delta E$, and then go to  the limit of really short times.  This should be done with enough care to teach derivatives to students who have not been exposed before.  It will avoid the issues surrounding the fact that some functions do not have derivatives at some points.  I'm not sure why even I care about that.
The rate at which energy flows into or out of a system is called the power.  A 60W lightbulb converts 60J of electricity to 60J of light and heat every second.

We learned in the introduction that rates are found by dividing the change in a quantity by the time it took.  Let’s divide the conservation of energy equation by the time between the start and end, $\Delta t = t\sub{start}-t\sub{end}$.
\startformula
	H\si + W + Q = H\sf
\stopformula

  The total power going into a system minus the total power coming out gives the rate of change in the energy of the system.
\startformula[power]
	P\sub{in} - P\sub{out} = \frac{\Delta H}{\Delta t}
\stopformula
Where $\Delta t$ is the amount of time over which the power is flowing.  This is just another statement of conservation of energy, but written in terms of rates.
%Here we get some great problems that are often beyond a typical introductory physics course.  For example, we can find the power that can be generated by a water wheel under a waterfall of a given height and flow.  These will be hard and students must be prepared with care so that they aren’t doing the "per unit time" thing if it can be avoided.

%\section{Rates}
%I could introduce rates in the introduction to conservation laws (cookies per day).  But I think it is best to wait until here, when they will have don’t some problems and gotten the conservation of mass concept down pretty well.  I could also wait on rates until after we have done energy and momentum, but that seems pretty late and it breaks up the organization by topic into organization by math.  At some point I need to give a good description of speed.  Clearly that is needed for kinetic energy, but perhaps it should be described along side volume, area and density in the introduction.

%Sometimes we don’t want to just compare the starting and ending states, we want to talk about how fast something is happening, or the rate at which it happens.

%\startformula
%	R\sub{change} = \frac{M\sub{end}-M\sub{start}}{t}
%\stopformula

%Example: a steam engine needs to produce a certain volume of steam at a some pressure to run.  How fast will it consume water (I'll just give the steam density).

%\section{Review}

%Last week we learned the integral form of the conservation of energy equation.
%\startformula
%	E(t_1)+ \int_{t_1}^{t_2} \! P\sub{in} \, dt -  \int_{t_1}^{t_2} \! P\sub{out} \, dt = E(t_2)
%\stopformula
%Another useful form can be found by taking the derivative with respect to $t_2$
%\begin{align}
%	\tfrac {d}{dt_2}\left(E(t_1)
%		+ \int_{t_1}^{t_2} \! P\sub{in} \, dt
%		-  \int_{t_1}^{t_2} \! P\sub{out} \, dt \right)
%	\NC = \frac {d}{dt_2} E(t_2) \NR
%	P\sub{in} (t_2)-  P\sub{out}(t_2)
%	\NC = \frac {d}{dt_2} E(t_2)
%\end{align}
%Since there is only one time in the final equation, we can write
%\startformula
%	\frac {d}{dt} E(t) = P\sub{in} (t)-  P\sub{out}(t)
%\stopformula
%This is the instantaneous form of the conservation of energy equation, which says that the rate at which the system’s energy is changing is equal to the rate at which energy is being added minus the rate at which energy is being taken out.  We can get back to the integral form by integrating the instantaneous form.

%You may remember that there were actually two integrals that we could use for the energy added and taken out, the above integral of Power with respect to time, and an integral of force over distance.  We can equate these two forms and take a derivative to get another useful relation.  Here I will use the total power, $P = P\sub{in} - P\sub{out}$.

%\begin{align}
%	\int_{t_1}^{t_2} \! P \, dt
%		\NC =  \int_{\vec{x}_1}^{\vec{x}_2} \! \vec{F} \dotp d\vec{x} \NR
%	\tfrac {d}{dt_2} \int_{t_1}^{t_2} \! P \, dt
%		\NC = \tfrac {d}{dt_2} \int_{\vec{x}_1}^{\vec{x}_2} \! \vec{F} \dotp d\vec{s} \NR
%	P(t_2) \NC = \tfrac {d\vec{x}_2}{dt_2} \dotp \tfrac {d}{d\vec{x}_2}
%		\int_{\vec{x}_1}^{\vec{x}_2} \! \vec{F} \dotp d\vec{s} \NR
%	\NC = \vec{v} \dotp \vec{F}
%\end{align}


\section{Constant Force}
%\setchapterfolder{Ch15}


%Potentials are more fundamental than the fields or the forces.
%
%This chapter should mirror the previous chapter, but with the addition of the potentials. Start by giving $E$ as a function of $m$, $\vec{p}$ and the fields. Then give $v$ as a function of those and the fields.
%
%Some discussion of what is in and what is out in an energy calculation. Before we had just the energy (rest plus kinetic). Now we potential. Many problems can be done with either work or potential, so that should be shown.
%



%\section{The rate of momentum change is the net force}
%\highlightbox{
%%\startformula
%%	\vec F\sub{net} = \frac{\Delta\vec p}{\Delta t}
%%\stopformula
%\begin{gather}
%	\vec F\sub{net} = \frac{\Delta\vec p}{\Delta t} \NR
%	\vec p_1 = \vec p_0 + \vec F\sub{net}\Delta t
%%	E\sub{f} \NC = \vec E\sub{i} + \vec{F}\sub{net}\dotp\Delta\vec{r}
%\end{gather}
%}%\end{shaded}
%

Let us return again to the dawn of the scientific revolution to confront the  most direct objection to Galileo’s heliocentric cosmology: Earth does not \emph{seem} to be moving. In his \booktitle{Dialogue Concerning Two World Systems}, Galileo quotes the most compelling form of this argument, made by Aristotle.


\startblockquote
	As the strongest reason, everyone produces the one from heavy bodies, which when falling down from on high move in a straight line perpendicular to Earth’s surface. This is regarded as an unanswerable argument that Earth is motionless. For, if it were in a state of diurnal rotation, and a rock were dropped from the top of a tower, then during the time taken by the rock in its fall, the tower (being carried by Earth’s turning), would advance many hundreds of cubits toward the east and the rock should hit the ground that distance away from the tower’s base. \autocite{p.~215}{Galileo1632}
\stopblockquote

Recall that Earth’s size was known (roughly), so the tower’s proposed speed was  about $460\units{m/s}$ toward the east, an amazing speed! %(See Example \ref{}.) refer to example problem in circular motion section
A rock dropped from a $50\units{m}$ tower (roughly the height of the Leaning Tower of Pisa), would take about $3.2\units{s}$ to reach the ground.%(Example \ref{})
In that time the tower would move $1500\units{m}$ to the east. If Aristotle’s reasoning is correct, the rock should hit the ground $1.5\units{km}$ to the west of the tower’s base.

Rather than following Galileo’s brilliant and lengthy refutation, we will confront Aristotle’s argument using conservation of momentum and the mathematical language of vectors.%, which will greatly reduce our effort.
The momentum of the rock is changed by the force of gravity.
\startformula
	\vec{p}\sf = \vec{p}\si + \vec{F}\Delta t
\stopformula
The force in this case is the force of gravity, which is directed downwards and has magnitude $mg$.
\startformula
	\vec{F} = \coordinates{0,-mg,0}
\stopformula

The constant force of gravity changes the rock’s momentum. However, we have learned that the $y$-component of the force will only affect the $y$-component of the momentum. Since the force has no $x$-component, the momentum’s $x$-component is unchanged by the force.

If Earth’s surface were stationary, as Aristotle believed, the a rock at held at the top of the tower would have no horizontal momentum, $p\sub{i$x$}=0$. When released, it continues to have no horizontal momentum, falling directly downward and landing at the base of the tower. Since this is what does happen, we seem to have strong evidence for Aristotle’s view.

However, if the tower is moving, due to Earth’s rotation, then the rock held at the top of the tower would have significant horizontal momentum in order to move along with the tower, $p\sub{i$x$}=mv\sub{tower}$. When released that stone would keep this horizontal momentum, falling toward the ground but simultaneously moving horizontally along with the tower. As rock falls, the tower moves $1.5\units{km}$ to the east, and the rock also travels exactly the same distance, striking the ground right at the base of the tower, exactly as we expect.

Galileo argued that the horizontal and vertical motion were separate motions, having no effect on each other. In modern language we refer to them as different components of the momentum. Galileo return’s to this idea in \booktitle{Two New Sciences}, where he provides a mathematical description of projectile motion, which composed of both horizontal and vertical motion under the constant force of gravity.



\startformula
	p\sub{f$x$} = p\sub{i$x$} + F_x\Delta t
\stopformula
Since the $x$-component of the gravitational force is zero, we find that the $x$-component of the momentum is constant: $p\sub{f$x$} = p\sub{i$x$}$.


In a collision, the change in momentum is usually abrupt. The forces between the colliding objects can be enormous, but it acts for a very short time. In this chapter we will consider the opposite extreme: constant forces which, while often smaller in magnitude, are relentless in their action.

Gravity is the classic example. Pulling on objects constantly, gravity is almost impossible to escape. \quotation{What goes up, must come down,} due to the persistence of the gravitational force. Of course, since entering the space age, we have generated a long list of objects that have gone up without coming down. Some are in orbit around Earth, others in orbit around the sun or other planets. A few are leaving the solar system entirely. But these are the exceptions that prove the rule. These runaways are only able to escape because Earth’s gravity becomes weaker far away.

\subject{Notes}
%\placefootnotes[criterium=chapter]
\placenotes[endnote][criterium=chapter]

%\subject{Bibliography}
%        \placelistofpublications


\stopchapter
\stopcomponent

\section{Problems}

Here are some problems inspired by the tzero, a high performance electric car made by AC Propulsion.
\startquestions

\problem The tzero has a mass of 1000kg and can accelerate from zero to 30m/sec in 4sec.  If it accelerates at constant power, what is the power?

\problem At maximum power, how long will it take to accelerate to 30m/sec while towing a 500kg trailer?

\problem Once going at 30 m/sec, the breaks are used to stop.  They provide a stopping force that is 80\% of the weight of the car.  What is the stopping distance?  (No trailer, anymore.)

\problem If the tzero is stopped instead by a big spring with spring constant 2N/m, what is the stopping distance?

\problem  If the tzero is stopped by costing up a $30^\circ$ slope, what is the stopping distance?

\problem  What is the stopping distance if it uses both the spring and the breaks?

\problem The tzero accelerates at full power for 10sec.  Then, seeing the grand canyon only 50m ahead, the driver slams on the breaks.  Does he stop in time?  If not, how fast is he going when he hits the canyon floor 75m below?

\problem The tzero is, for unknown reasons, towing the 500kg trailer with a 50m tow rope.  Although moving slowly, it again inexplicably falls off a high cliff.  Ignoring friction, how fast is the trailer going when it is pulled over the edge by the falling tzero.

\problem
Lifting this book from the floor to the table takes about 10 J of energy.  How much energy would it take to lift the book to a high shelf?

\problem
Lifting this book from the floor to the table takes about 10 J of energy.  How much energy would it take to lift two books onto the table?

\problem
Lifting this book from the floor to the table takes about 10 J of energy.  How much energy would it take to lift the book onto the table if we are on the moon where gravity is one sixth as strong?

\problem
What are the units of energy (J)?

\problem
What are the units of $mc^2$ if $c$ is the speed of light?

\problem
How much energy do I give my 20 kg son, Eric, when I put him on my shoulders (about 1.5 m up)?

\problem
How high would I have to lift a 1000 kg car for it to have a gravitational potential energy of $1.8 \sci{6} \units{J}$?

\problem
A 40 kg person is jogging at $3 \units{m/sec}$.  What is his kinetic energy?

\problem
A car has a kinetic energy of $1.8 \sci{6} \units{J}$ at a speed of 60 m/sec.  What is it’s mass?

\problem
A baseball and a tennis ball are traveling with the same kinetic energy.
If the baseball weighs twice as much, then what is the speed of the tennis ball?

\problem
Two objects of different masses are dropped from the Tower of Pisa.  Which is going faster when it gets to the ground?

\problem
At the park two kids climb to the top of a frictionless slide.  One child slides down, but the other child falls off when he gets to the top.  Which child gets to the ground first?

\problem
A giant sling-shot has a spring constant of $4 \units{N/m}$.  How far does it have to be stretched to store 200 J of energy?

\problem
A physicist used a large spring in his garage to stop his car.  The spring he bought will stop his car in 2 m.  He buys an SUV which is twice as massive as his car, though he drives it into the garage at the same speed.  How many springs will he need to stop the SUV in the same distance?

\problem
The physicist of the last problem gives his car to his daughter, who enters the garage twice as fast.  How many springs will she need to stop in the same distance?

% Here there should be a problem like the next, but with the velocity right after launch.

\problem
A spring is used to launch a ball to a height h.  If the spring were compressed twice as much how high would the ball go?

% The sling shot problems go here

\problem
A kitchen scale has a spring constant of $10^4 \units{N/m}$.  A 4 kg turkey is set on it, without compressing the spring, and then released.  How far does the scale go down before it stops?

\problem
How fast is the turkey going when it passes the halfway point?  This time use $m$, $k$ and $g$.

\problem
A kitchen scale has a spring constant of $10^4 \units{N/m}$.  A 4 kg turkey is dropped on it from a height of 6 mm.  How far does the scale go down before it stops?

\problem
My 20 kg son can climb 2 m in 4 sec.  What is the power he puts into climbing?

\problem
The elevator in my building rises at 3 m/sec when filled to capacity (1000 kg).  What is the power of the motor?  The counterweight is 500 kg.

\problem
A car accelerates with constant power.  After 9 sec it has reached 30 km/sec.  Much longer will it take to reach 60 km/sec?

\problem
A miller finds a waterfall with a flow of 10 kg/sec and a drop of 2 m.  He builds a water wheel.  How much power does he get?
\stopquestions


%\stopchapter
%\stopcomponent


% Templates:

% Epigraph
\placefigure[margin,none]{}{\small
	\startalignment[flushleft]
	\stopalignment
	\startalignment[flushright]
	{\it }\\
	{\sc }\\
	--
	\stopalignment
}

% Margin image
\placefigure[margin][] % Location, Label
{} % Caption
{\externalfigure[chapter03/][width=144pt]} % File

% Margin Figure
\placefigure[margin][] % location
{}	% caption text
{\starttikzpicture	% tikz code
\stoptikzpicture}

% Aligned equation
\startformula\startmathalignment
\stopmathalignment\stopformula

% Aligned Equations
\startformula\startmathalignment[m=2,distance=2em]
\stopmathalignment\stopformula
