% !TEX useAlternatePath
% !TEX useConTeXtSyncParser
\startcomponent c_chapter01
\project project_world
\product prd_volume01

\doifmode{*product}{\setupexternalfigures[directory={chapter01/images}]}

\setupsynctex[state=start,method=max] % "method=max" or "min"

\starttext

%%%%%%%%%%%%%%%%%%%%%%%%%%%%%
\startchapter[title=The Quadrivium, reference=ch:Music]
%%%%%%%%%%%%%%%%%%%%%%%%%%%%%
\placefigure[margin,none]{}{\small
	\startalignment[flushleft]
	Pythagoras conceived that the first attention that should be given to everyone should be addressed to the senses, as when one perceives beautiful figures and forms, or hears beautiful rhythms and melodies.  Consequently he laid down that the first erudition was that which subsists through music’s melodies and rhythms, and from these he obtained remedies of human manners and passions, and restored the pristine harmony of the faculties of the soul.
	\stopalignment
	\startalignment[flushright]
	{\it Life of Pythagoras}\\
	{\sc Iamblichus of Chalcis}\\
	c.245–c.325
	\stopalignment
}

\Initial{O}{ur first exposure} to the beauty and precision of mathematics is through music. As infants, our sophisticated, language processing brains discover relationships of rhythm and pitch in both language and song. As our musical appreciation grows, we hear rhythmic patterns of repetition and variation, and harmonic patterns of consonance and dissonance that stir feelings of joy, foreboding, courage, and transcendence. When you sing and dance, you are probably not thinking about mathematical formulas and theorems (I admit to being odd in this way), but music’s artistic expressiveness and mathematical precision are inseparable.

Music’s precise, expressive nature caught the attention of many early civilizations.
Ancient Egyptians, Babylonians, Chinese, and others found ways to blend different pitches to produces full, rich sounds that give music its expressive power.
As they experimented with different musical instruments, they discovered a mathematical pattern for pipes and strings that produce these beautiful blending pitches.

In the west, this musical discovery was attributed to Pythagoras,\index{Pythagoras} a Greek philosopher in the sixth century \scaps{bc}.  None of Pythagoras’s writings survive, so it is impossible to know if this attribution is correct.
But his followers, the Pythagoreans, did possess a simple mathematical model for predicting which lengths of pipes and strings would produce pitches that blend together comfortably.
The Pythagorean model says {\em simple ratios produce consonant pitches.}

\placefigure[margin][fig:Pythagoras]{Pythagoras (right) and follower Philolaus play pipes demonstrating consonant sounds. The lengths of the pipes are labeled in this woodcut from Franchino Gafurio’s \booktitle{Theoria musice} (1492). Philolaus lived about a century after Pythagoras, so they did not actually perform together.} {\externalfigure[gaffurio_pythagoras_pipes][width=\rightmarginwidth]}

Pythagoras and his follower Philolaus demonstrate their model in \in{figure}[fig:Pythagoras], using six pipes of different lengths.
The longest pipes produce the lowest pitches; the shortest pipes produce the highest.
The pipes' lengths are shown, so we can calculate the ratios.
Philolaus and Pythagoras are playing pipes whose lengths are in a simple two-to-one ratio. The pipes' two pitches are \keyterm{consonant}, which means they blend together comfortably. (The Latin root {\it con-} means \quotation{with} and {\it sonare} means \quotation{to sound.} Consonant pitches \quotation{sound together.}) 
Pipes that are not related by simple ratios produce clashing, or \keyterm{dissonant} sounds. ({\it dis-} means \quotation{apart.} Dissonant pitches \quotation{sound apart.})

A \emph{simple ratio,} according to the Pythagoreans, includes only the numbers one through four. For example, four-to-three is a simple ratio, but five-to-four is not.
The number four represented completeness, according to the Pythagoreans, so the numbers one through four completed their musical system. 
%In their mathematical model, any ration of the numbers one through four was simple enough to be a consonance.

Even if you know nothing about music, you can use the Pythagoreans' model to predict which pairs of pipes will produce consonant pitches, and which pairs will produce dissonant pitches. I invite you to demonstrate your new musical knowledge by solving two example problems.

	%%%%%%%%%%%%%%%%%%%%%%%%%%%%%%%%%%%%%%%%%%%%%
\startexample[ex:PyConsonance]
Continuing their demonstration, Pythagoras plays the pipe with length 8 and Philolaus plays the pipe with length 12. According the Pythagorean model, do they produce consonant pitches?
\startsolution
The ratio of the lengths is twelve-to-eight, which includes numbers bigger than four. However, this ratio simplifies.
\startformula
	\frac{12}{8}
		= \frac{3}{2}
\stopformula
The simplified three-to-two ratio includes only numbers four or less, so these pipes produce consonant pitches. \stopsolution
\stopexample
	%%%%%%%%%%%%%%%%%%%%%%%%%%%%%%%%%%%%%%%%%%%%%

	%%%%%%%%%%%%%%%%%%%%%%%%%%%%%%%%%%%%%%%%%%%%%
\startexample[ex:PyConsonance]
Pythagoras and Philolaus invite you select a pipe and join their demonstration. They will play pipes with lengths 8 and 12. You want to select a pipe that is consonant with both of their pitches. According to the Pythagorean model, which pipe should you avoid?
\startsolution
Avoid the pipe with length 9, because the pitches produced by the 9 and 8 pipes will be dissonant. Any other pipe will produce a pitch consonant with both the 8 and 12 pipes.
\stopsolution
\stopexample
	%%%%%%%%%%%%%%%%%%%%%%%%%%%%%%%%%%%%%%%%%%%%%

The Pythagorean model of consonance is a \keyterm{mathematical model}\dash it connects the purely mathematical idea of simple ratios to the physical phenomena of consonant sounds.
The model's predictions can be tested by building and playing pipes of different lengths.
Indeed, pipes in a three-to-two ratio produce consonant pitches, while pipes in a sixteen-to-nine ratio produce dissonant pitches.

The Pythagorean model of consonance faced some experimental challenges, leading to revisions of the model.
Before we look at these challenges and revisions, I will introduce another Pythagorean model\dash their mathematical model of the world.
%These two Pythagorean models, after some revisions, were the foundation for the models of motion that emerged in the seventeenth century.

The Pythagoreans built their mathematical model of the world to explain the cycles of nature – day and night, the seasons of the year, the Moon’s phases, and the motions of the visible planets. Simple ratios did not help: A year is about 365\onequarter\ days. A lunar month is about 29\onehalf\ days. There are about 12\onethird\ lunar months in a year. The motions of the five visible planets – Mercury, Venus, Mars, Jupiter, and Saturn – are even more complicated.

\placefigure[margin][fig:LunarEclipse]{During a lunar eclipse, Earth casts a shadow on the Moon. The shadow’s round shape reveals Earth’s round shape.} {\externalfigure[LunarEclipseMay2022][width=\rightmarginwidth]}

The Pythagoreans did recognize one interesting pattern: the objects that rule the rhythms of nature are all round. The Moon is obviously round. The Sun is round, as can be seen when its brightness is diminished by a thick veil of clouds. The Pythagoreans even knew Earth to be round because Earth casts its round shadow on the Moon during a lunar eclipse, shown in figure \in[fig:LunarEclipse].
%Earth’s spherical shape was common knowledge among the educated even in ancient times. 

Extrapolating boldly, the Pythagoreans concluded that everything in astronomy is circular, including the paths followed by celestial bodies. Therefore, the Sun, Moon, planets, and stars must follow circular paths around our spherical Earth. This was the Pythagorean’s second mathematical model: {\em celestial objects circle Earth.} The model connects a mathematical abstraction, the perfect circle, to the physical motions of astronomy.

For the Pythagoreans, the abstract, mathematical disciplines of arithmetic and geometry were closely linked to the observable, physical disciplines of music and astronomy.
These four disciplines\dash music, arithmetic, astronomy, and geometry\dash were the \quotation{four paths} to knowledge, known together as the {\em quadrivium.}
The quadrivium was a beautiful system which brought order to Europeans’ understanding of the world until the end of the sixteenth century.

In the first half of this, chapter we will look at the challenges and revisions of the Pythagorean musical model, leading to the modern equal-tempered scale.
In the second half of the chapter, we will look at the challenges and revisions of the Pythagorean astronomical model, leading to acceptance of the Sun-centered solar system. 
Each model's evolution is unique, but there are important similarities. Following these two stories will help us understand how and why mathematical models evolve.
We will need that insight as we develop our new mathematical model of motion in the following chapters.

%As we investigate the Pythagorean’s mathematical model of musical consonance, there is one thing that I want you to remember: this model is fantastic! Simple ratios really do produce consonant pitches. We will look at some different ideas about how to apply this model in musical practice, as well some conflicting ideas about why the model works, but the model itself is a solid foundation for all music theory.
%
%We return to Pythagoras and his follower Philolaus, demonstrating their mathematic model of consonance in \in{figure}[fig:Pythagoras].
%They hold additional pipes that can produce every simple ratio in the Pythagorean musical system.

%\placefigure[margin][fig:PythagoreanPipesCons]{Pythagoras and Philolaus’s pipes, represented by vertical lines, can be paired in many ways to produce the five simple ratios of the Pythagorean musical system. These pairs will produce consonant pitches. Arrows that are the same length show the same ratios.} {\externalfigure[PythagoreanPipesCons]}%[width=\rightmarginwidth]}
%
%\placefigure[margin][fig:PythagoreanPipesDis]{Some pairs do not form simple ratios. These pairs will produce dissonant pitches.} {\externalfigure[PythagoreanPipesDis]}%[width=\rightmarginwidth]}
%\noindent
%Figure \in[fig:PythagoreanPipesCons] shows Pythagoras and Philolaus’s six pipes and the all of the simple ratios that can be made by pairing them.
%Each of these pairings will produces consonant pitches.
%
%Not every pair in Pythagoras and Philolaus's pipe collection produces consonant pitches. The three pairs in figure \in[fig:PythagoreanPipesDis] are not simple ratios of small numbers. Each of these pairs will produce dissonant pitches.

%\placetextfloat[top][fig:MonochordTyndall]{Monochord shown in John Tyndall’s 1875 textbook, \booktitle{Sound.}} {\externalfigure[MonochordTyndallSound][width=\textwidth]}
%
%\startbuffer[monochord]
%	% First monochord
%	\draw[fill=black!10] (0,4.75) rectangle (1.5,-4.75); % Box
%	\draw (0.5,4.5)--(1,4.5); % Top bridge
%	\draw (0.5,-4.5)--(1,-4.5); % Bottom bridge
%	\draw[opacity=.5] (.75,4.5)--(.75,-4.5);
%	\draw[opacity=.5,domain=-90:90] plot ({cos(\x)/10 + 0.75},{\x/20});
%	\draw[opacity=.5,domain=-90:90] plot ({-cos(\x)/10 + 0.75},{\x/20});
%	\draw[thin] (.25,4.5) -- (.45,4.5);
%	\draw[thin] (.25,-4.5) -- (.45,-4.5);
%	\draw [thin,<->](.32, 4.5) -- node[fill=black!10,inner sep=2pt]{$L$}  (.32, -4.5);
%	% Second monochord
%	\draw[fill=black!10] (1.75,4.75) rectangle (3.25,-4.75); % Box
%	\draw (2.25,4.5)--(2.75,4.5); % Top bridge
%	\draw (2.25,-4.5)--(2.75,-4.5); % Bottom bridge
%	\draw (2.5,4.5)--(2.5,1.5);
%	\draw[opacity=.5] (2.5,1.5)--(2.5,-4.5);
%	\draw[thin] (2,1.5) -- (2.35,1.5);
%	\draw[thin] (2,-4.5) -- (2.2,-4.5);
%	\draw[opacity=.5,domain=-90:90] plot ({cos(\x)/10 + 2.5},{\x/30-1.5});
%	\draw[opacity=.5,domain=-90:90] plot ({-cos(\x)/10 + 2.5},{\x/30-1.5});
%	\draw [thin,<->](2.07, 1.5) -- node[fill=black!10,inner sep=2pt]{$\tfrac{2}{3}L$}  (2.07, -4.5);
%	\draw[fill=white] (2.5,1.5) circle[radius=.1]; % Finger placement
%	% Third monochord
%	\draw[fill=black!10] (3.5,4.75) rectangle (5,-4.75); % Box
%	\draw (4,4.5)--(4.5,4.5); % Top bridge
%	\draw (4,-4.5)--(4.5,-4.5); % Bottom bridge
%	\draw (4.25,4.5)--(4.25,0); % Straight string
%	\draw[opacity=.5] (4.25,0)--(4.25,-4.5);
%	\draw[opacity=.5,domain=-90:90] plot ({cos(\x)/10 + 4.25},{\x/40-2.25});
%	\draw[opacity=.5,domain=-90:90] plot ({-cos(\x)/10 + 4.25},{\x/40-2.25});
%	% Dimension of vibrating string
%	\draw[thin] (3.75,0) -- (4.1,0);
%	\draw[thin] (3.75,-4.5) -- (3.95,-4.5);
%	\draw [thin,<->](3.82, 0) -- node[fill=black!10,inner sep=2pt]{$\tfrac{1}{2}L$}  (3.82, -4.5);
%	\draw[fill=white] (4.25,0) circle[radius=.1]; % Finger placement
%\stopbuffer
%
%\marginTikZ{\vskip 1in}{monochord}{A monochord produces the lowest pitch when the full length of string is played.  Placing a finger on the string reduces the string’s length and raises its pitch. The three lengths shown can be paired to form the simple ratios 2:1, 3:2, and 4:3.}

%%%%%%%%%%%%%%%%%%%%%%%%%%%%%%%%%%%%%%%%%%%%%%
\section{Ptolemy's rational musical scale}
%%%%%%%%%%%%%%%%%%%%%%%%%%%%%%%%%%%%%%%%%%%%%%

Pythagoras and Philolaus's pipes (\in{fig.}[fig:Pythagoras]) are excellent for demonstrating consonance and dissonance, but not for performing music.
Performance instruments play many specific pitches determined by the lengths of its pipes or strings, or by the placement of holes or frets.
Ideally, many of these pitches can be combined in consonance, which means many of the pipe's or string's lengths are related by simple ratios.
Finding such a set of lengths is a challenging mathematical problem.
Many cultures adopted different solutions, giving their music distinctive sounds.

The ancient Greeks produced several solutions.
The Pythagoreans took a strictly mathematical approach, using only ratios of one, two, three, and four.
Their method was extremely influential, providing the basis for music through the middle ages.
Aristoxenus, in the fourth century \scaps{bc}, took a more experimental approach, determining consonant pitches by listening to whether they sound consonant.
Mathematics was a useful guide, but Aristoxenus believed music is good if it sounds good.
He shared his ideas in \booktitle{Harmonic Elements}. His book was ridiculed and then largely ignored by music theorists, but his ideas continued to be influential, as we will see.

Ptolemy, writing near the end of the classical period in the second century \scaps{ce}, took a more balanced view, seeking to satisfy both sense and reason. 
He performed experiments using an octochord, shown in \in{figure}[fig:TinyPiano].
This instrument has eight identical strings, all tuned to the same note.
To produce other pitches, Ptolemy placed bridges under the strings.
Each bridge reduces a string's length, raising its pitch.
By adjusting each bridge's location, Ptolemy could easily experiment with many different combinations of lengths to determine which combinations sound best.
(This is much easier than cutting new pipes!)
With all eight strings side-by-side, Ptolemy could easily play multiple strings at once, or even play simple tunes to determine which combinations worked best in actual musical performance.
His octochord provided a great little laboratory for musical experiments.
\in{Figure}[fig:TinyPiano] shows an especially successful combination of lengths, which eventually evolved into the modern C major scale.

Before we investigate the ratios that make this combination so successful, we need to master a bit of musical terminology. The scale will help. 
Below the octochord \in{figure}[fig:TinyPiano] are several representations of the C major scale\dash a keyboard, notes on a staff, and two common naming conventions.
I will identify specific pitches by their letter names.
(The other representations are there to aid anyone already familiar with them.)
%You do not need to learn them to understand this chapter or to work the problems.

The separation between to pitches is called the \keyterm{interval} between the pitches. Pitches that are close together on the scale are separated by a small interval, while pitches that are farther apart are separated by a larger interval.
%Specific intervals are named based on the separation between the two pitches' on the keyboard in \in{figure}[fig:TinyPiano].
Specific intervals are named according to the number of pitches they encompass. %An octave encompasses all eight pitches shown in figure \in[fig:TinyPiano8th5th]. 
%All of the pitches needed for musical counting are shown on the keyboard below the pipes in \in{figure}[fig:PythagoreanPipesArrows]. (I will use the modern letter names shown on the keyboard to refer to the pitches of the ancient scales, even though the frequencies are not quite the same.) 
The most important interval is an \keyterm{octave}, which encompasses eight pitches.
For example, the eight pitches C–D–E–F–G–A–B–C make an octave from C to C. A \keyterm{fifth} encompasses five pitches, like C–D–E–F–G, the interval from C to G. A \keyterm{fourth} encompasses four pitches, like the interval from G up to C. A \keyterm{third} encompasses three. A \keyterm{second,} which encompasses only two adjacent pitches, is often called a \keyterm{step.}

\placefigure[margin][fig:GrandPiano]{A grand piano has long, low strings on the left and shorter, higher strings on the right. \tfx\\Photo: © Copyright Steinway \& Sons}
{\externalfigure[3-1422x800][width=\rightmarginwidth]}
% https://steinway.co.uk/wp-content/uploads/2019/09/3-1422x800.jpg
% https://www.steinway.com/pianos/steinway/grand/model-d
% contact: https://www.steinway.com/about/contact
To see why Ptolemy's scale became so popular, let's look at a few of the intervals, starting with the largest and most consonant, the octave.
The interval from the low C to the high C in \in{figure}[fig:TinyPiano] is one octave. The strings' lengths are in a two-to-one ratio, which produces very consonant pitches. 
Two pitches separated by an octave are so consonant that they are given the same name.

When building performance instruments, octaves are used to extend the scale to higher and lower notes, beyond the eight on the octochord.
Lengths in each lower octave are double the lengths in the higher octave. 
Lengths in each higher octave are half the lengths in the lower octave.
In this way, every pitch will be consonant with the pitch an octave above or below.
A modern grand piano spans over seven octaves, from a very low A to a very high C.
%The dramatic range of string lengths can be seen in \in{figure}[fig:GrandPiano], making this piano  
The exponential increase in string length going down the keyboard is quite obvious in the concert grand piano in \in{figure}[fig:GrandPiano], which is nine feet long!
%Large organs have pipes that are even longer to produce their lowest pitches.
%If the strings were identical thickness and tension, the lengths would double seven times going from the highest C to the lowest C – making the longest string 128 times longer than the shortest! For the very lowest pitches, the strings are thicker and looser, which also makes them lower, so they don’t have to be quite so long. 

Ptolemy's scale contains many strings in ratios of three-to-two or four-to-three. These consonant pairs are separated by fifths or fourths.

%%%%%%%%%%%%%%%%%%%%%%%%%%%%%%%%%%%%%%%%%%%%%
\startexample[ex:TinyPianoPtEx]
You wish to produce a calm, consonant three-to-two ratio, with the low note being Ptolemy's low C. What high note should you use? What is the interval?
%\placefigure[margin][fig:TinyPianoPtEx]{The Ptolomaic scale with the interval sought in example \in[ex:TinyPianoPtEx].} {\externalfigure[TinyPianoPtEx][width=\rightmarginwidth]}
\startsolution
To find the length of the higher pitched string, we set up a ratio equation using the length of the low C, which is $60$, and the unknown length of the higher string, which we call $l$.
\startformula
	\frac{3}{2}
		= \frac{60}{l}
\stopformula
We find $l=40$, which is the length of G. The interval from low C to G is a fifth.
\stopsolution
\stopexample
%%%%%%%%%%%%%%%%%%%%%%%%%%%%%%%%%%%%%%%%%%%%%

Likewise, to make a consonant four-to-three ratio starting on G, you would find $l=30$, which is the length of Ptolemy's high C. The interval from G to high C is a fourth.

Most of Ptolemy's fifths and fourths are consonant, but not all of them.
(Perhaps you can find a dissonant fifth or fourth in Ptolemy's scale.)
A fifth that is a perfect three-to-two ratio is called a \keyterm{perfect fifth}.
A fourth that is a perfect four-to-three ratio is called a \keyterm{perfect fourth}.
Musicians love the many perfect fifths and fourths in Ptolemy's scale.
All of the consonant ratios from the Pythagorean model can be made from Ptolemy's octaves, perfect fifths, and perfect fourths, and with so many of these, musicians could blend consonant pitches in starting from almost any note on the scale.

Ptolemy's scale also offers something new\dash consonant thirds! To see why these are so exciting, let's check the ratios for some of Ptolemy's thirds. The third from low C to E has the ratio
\startformula
	\frac{60}{48}
		= \frac{5}{4}.
\stopformula
According the the Pythagorean model, this third should not produce consonant pitches. However, in Ptolemy's experiments, the pitches were consonant. This consonant interval is called a \keyterm{major third.}
The third from E to G offers another experiment.
\startformula
	\frac{48}{40}
		= \frac{6}{5}
\stopformula
Ptolemy found that this ratio also produces consonant pitches.
This consonant third is a \keyterm{minor third.} (The thirds' names come from the fact that the minor third's ratio, $6/5$, is smaller than the major third's ratio, $5/4$.)
Almost every third in the Ptolemaic scale is a consonant major or minor third.
%(You should find the one exception.)

Ptolemy's experiments with consonant thirds forced a small but significant revision of the Pythagorean model of consonance. Ptolemy still believed that simple ratios produce consonant sounds, but simple ratios could contain numbers up to six.
Ptolemy's scale, with these additional consonances, was a tremendous gift for musicians. It gave them tremendous opportunities to blend consonant pitches in their music, allowing them to create a huge variety of sounds.

The many consonant intervals allowed by Ptolemy's revision are listed in \in{table}[T:PtIntervals].
Ptolemy wanted a scale that would allow musicians as many options for consonance as possible.  
Musician's love Ptolemy's consonant thirds.

\placetable[margin][T:PtIntervals] % Label
    {Consonant intervals of Ptolemy's scale. Zarlino's equal-tempered scale contains the same consonant intervals.} % Caption
    {\vskip9pt\small\starttabulate[|l|c|c|]
\FL[2]%\toprule
\NC Interval		\NC Steps				\NC Ratio		\NR
\HL
\NC Octave		\NC $5$ whole + $2$ half		\NC $2:1$		\NR
\NC Perfect 5th	\NC $3$ whole + $1$ half		\NC $3:2$		\NR
\NC Perfect 4th	\NC $2$ whole + $1$ half		\NC $4:3$		\NR
\NC Major 3rd	\NC $2$ whole					\NC $5:4$		\NR
\NC Minor 3rd	\NC $1$ whole + $1$ half		\NC $6:5$		\NR
\LL[2]%\bottomrule
\stoptabulate}

Ptolemy shared this treasure chest of consonant intervals in his great musical treatise, \booktitle{Harmonics}. Unfortunately, Europe lost \booktitle{Harmonics} before the middle ages. Europeans spent the next twelve centuries subsisting on octaves, fifths, and fourths – without consonant thirds.
(Fortunately, a most of \booktitle{Harmonics} survived, perhaps in the Islamic world.)

%Ptolemy's scale has five larger steps and two smaller steps. The larger steps, with ratios of $9/8$ and $10/9$, are \keyterm{whole steps.}  The whole steps are obvious on the keyboard because there is always a black key in the middle of a whole step. (The black keys are sharps and flats. There were many arguments about the exact ratios for the sharps and flats, which we will avoid.)
%The two smaller steps, with ratios of $16/15$, are \keyterm{half steps.} Half steps are not subdivided, so there is no black key in the half steps.

%These three lengths are enough to produce a three pitch chord, C–G–C, which contains all three of the simplest ratios. The chord is consonant with a wide-open, clear sound.
%%Aaron Copland’s \booktitle{Fanfare for the Common Man} opens with a soaring sequence.
%The fifth and fourth can trade places, still making an octave as in figure \in[fig:TinyPiano4thWhole]. This gives us the F length, which is ¾. The chord C–F–C also has an open sound, but it is darker than the C–G–C chord. Both chords contain the same three simple ratios, but the different arrangement alters the sound.

%\placefigure[margin][fig:TinyPiano4thWhole]{Flipping the order of the perfect fourth and perfect fifth gives a different division of the octave. A whole step is the difference between the perfect fourth and perfect fifth.} {\externalfigure[TinyPiano4thWhole][width=\rightmarginwidth]}

%Not every fifth is a perfect fifth. Perfect fifths have three whole steps and one half step, as in \in{example}[ex:TinyPianoPtEx] above. The fifth spanning B–C–D–E–F has two whole steps and two half steps, making it significantly smaller than the perfect fifths. This small B–F fifth is called a diminished fifth, and it is very dissonant. Similarly, the fourth from F to B is not perfect. Perfect fourths have two whole steps and one half step, but the F–B fourth has three whole steps. This is an augmented fourth, and it is also very dissonant. 

%%%%%%%%%%%%%%%%%%%%%%%%%%%%%%%%%%%%%%%%%%%%%%
%\startexample[ex:TinyPianoPtEx5th]
%What interval is the sum of a major third and a minor third?
%\startsolution
%Two thirds make a fifth. To determine if it is a perfect fifth, we must add the thirds by multiplying their ratios. \startformula
%	\text{major 3rd} + \text{minor 3rd}
%		= \frac{5}{4} \times \frac{6}{5}
%		= \frac{3}{2}
%		= \text{perfect fifth}
%\stopformula
%A major third and a minor third add to a make a perfect fifth.
%\stopsolution
%\stopexample
%	%%%%%%%%%%%%%%%%%%%%%%%%%%%%%%%%%%%%%%%%%%%%%

%%%%%%%%%%%%%%%%%%%%%%%%%%%%%%%%%%%%%%%%%%%%%%
\section{Zarlino's rational musical scale}
%%%%%%%%%%%%%%%%%%%%%%%%%%%%%%%%%%%%%%%%%%%%%%

As Europe emerged into the Renaissance, musicians became more adventurous and began using thirds in their compositions. Music theorists reinvented the Ptolemaic scale, which became known as \keyterm{just intonation.} Then, part of the lost \booktitle{Harmonics} was found, adding Ptolemy’s authority to the system of just intonation. These developments were integrated into a sophisticated musical theory by the great Italian master Gioseffo Zarlino in \booktitle{Le Istitutioni harmoniche,} published in 1558.
Zarlino fully embraced the Pythagorean idea that simple ratios produce consonant sounds, with Ptolemy's revision that simple ratios include numbers up to six.

One of Zarlino’s most accomplished students was the musician Vincenzo Galilei.
Galilei challenged the role of simple ratios by producing experimental evidence against it. In one experiment, he altered the tension in the strings by hanging different weights from them, as in figure \in[fig:GaffurioPythagorasStrings]. Increasing the weight increases the tension and raises the pitch. Using identical strings, Galilei found weights in the simple ratios of two-to-one and three-to-two produced dissonant pitches, but weights in the ratio of nine-to-four produced consonant pitches.
In another experiment, Galilei altered the weights of the strings themselves, comparing thick, heavy strings with thin light strings, all stretched to the same length and tension. Lighter strings produce a higher pitch than heavy strings. Again he found exactly the same results as he had for the hanging weights. Simple ratios do not always produce consonant pitches!

\placefigure[margin][fig:GaffurioPythagorasStrings]{Pythagoras demonstrating consonant sounds in strings with different weights attached. This experiment does not produce the results that the Pythagoreans predicted. (Also from Franchino Gafurio’s \booktitle{Theoria musice.}} {\externalfigure[gaffurio_pythagoras_strings][width=\rightmarginwidth]}

For advocates of the Pythagorean model, this challenge was embarrassing but helpful.
It was embarrassing because many books on music theory, from ancient times right up until Galilei’s day, suggested that these experiments had been done, and the results supported the model.
The woodcut in figure \in[fig:GaffurioPythagorasStrings], made decades before Galilei was born, depicts Pythagoras performing this experiment as a demonstration of the Pythagorean model!
It is not entirely clear why this problem went unnoticed for over two thousand years.
Some people probably noticed, and concluded that they should only use the model for lengths, not weights.
Others simply assumed the experiment would produce the desired result, and never bothered to check.

Galilei’s challenge was quickly met by the realization that the ratios must be applied to the instrument’s rate of vibration.
This rate is called the \keyterm{frequency}.
The mathematical model was refined to say \emph{simple frequency ratios produce consonant pitches}.
Galilei’s experimental challenge brought about an extremely helpful clarification that turned attention from how the instrument is built to how it moves to produce sound.

To use this new mathematical model of musical consonance, we must quantify frequency, which measures how quickly the instrument's vibrating motion is repeated.  Each repetition is called a \keyterm{cycle}, so frequency is measured in cycles per second.  The long, heavy strings in a grand piano vibrate with a frequency of a few dozen cycles per second, producing low pitches.  The shortest strings’ frequencies are thousands of cycles per second, producing high pitches.
The unit of measure for frequency is \keyterm{hertz}, abbreviated as Hz.

\startformula
1\units{Hz} = 1\units{cyc/s}
\stopformula
Humans can hear a huge range of frequencies, from about from $5\units{Hz}$ up to $20{,}000\units{Hz}$.  The highest frequencies are only audible to young people.

\placetable[margin][T:Frequencies] % Label
    {Frequencies of Zarlino's Just Intonation and Galelei's equal-tempered scales.} % Caption
    {\vskip9pt\small\starttabulate[|l|c|c|]
\FL[2]%\toprule
\NC Just		\NC Equal-tempered				\NC Ratio		\NR
\HL
\NC 528	\NC 	\NC 523.25	\NR
\NC 495	\NC 	\NC 493.88	\NR
\NC 440	\NC		\NC 440.00	\NR
\NC 396	\NC 	\NC 392.00	\NR
\NC\dorecurse{3}{352}	\NC 	\NC 261.63	\NR
\LL[2]%\bottomrule
\stoptabulate}

Many details of an instrument’s construction affect its sound by influencing the way it vibrates. Galilei’s experiments taught us that properties of the instruments, like length and tension, may affect the frequency of vibration in surprising ways.
The frequency produced by a string or pipe is inversely proportional to the length, so a simple three-to-two ratio in lengths will produce a consonant two-to-three ratio of frequencies, as the Pythagoreans knew. However, the frequency produced by a string is proportional to the square root of the string’s tension. A simple two-to-three ratio of tensions produces a dissonant $\sqrt 2$-to-$\sqrt 3$ ratio of frequencies. A four-to-nine ratio of tensions produces a consonant two-to-three ratio of frequencies.
We will eventually understand these surprising relationships through our new physics of motion.

%Nearly all instruments use lengths to produce pitches. Wind instruments use pipes cut to different lengths like an organ, stretched like a trombone, or made different lengths by covering and uncovering holes like a flute.
%String instruments have many strings like a piano (figure \in[fig:GrandPiano]) or a few strings that can be shortened with fingers like a violin. In all of these instruments, the frequencies are inversely proportional to the lengths.
%Performance instruments are designed to produce a full, rich sound, while use different lengths to produce different frequencies, just like the different lengths of Pythagoras and Philolaus’ pipes.

%%%%%%%%%%%%%%%%%%%%%%%%%%%%%%%%%%%%%%%%%%%%%
\section{Galilei's irrational musical scale}
%%%%%%%%%%%%%%%%%%%%%%%%%%%%%%%%%%%%%%%%%%%%%

Just intonation does contain a couple of unwelcome surprises. The interval from D to A looks like a perfect fifth on the keyboard – three whole steps and one half step. However, the ratio of frequencies is not a consonant \threehalves\ ratio, instead it is a dissonant ratio:
\startformula
	\frac{440\units{Hz}}{297\units{Hz}}
		= \frac{40}{27}.
\stopformula
This fifth is called the wolf fifth. If a composer puts this fifth in a piece, at the performance it will howl like a wolf! I will let you find the wolf forth hiding in Zarlino's scale.
(Ptolemy's scale had the same wolf ratios, hiding in slightly different places.)
The wolf intervals can be avoided, but this limits composers' options considerably.
%and they are a small price to pay for the exciting selection of consonant thirds in the Ptolemaic scale.

\placefigure[margin][fig:TinyPianoCompleteEq]{The equal-tempered tuning is built from twelfth roots of two.} {\externalfigure[TinyPianoCompleteEq][width=\rightmarginwidth]}

To eliminate the wolves, Galilei proposed a radical alternative to Zarlino's scale.
Following a suggestion of Aristoxenus, Galilei divided the octave into twelve identical \keyterm{half steps.} Galilei calculated that each half step would change the frequency by a factor of $\sqrt[12]{2}$. This method, called \keyterm{equal temperament,} is shown in figure \in[fig:TinyPianoCompleteEq].
The frequencies of the keyboard's black keys (the sharps and flats) fit naturally into equal temperament.

%%%%%%%%%%%%%%%%%%%%%%%%%%%%%%%%%%%%%%%%%%%%%%
%\startexample[ex:TinyPianoPyEx]
%Find the ratio for the whole step in equal temperament.
%\startsolution
%Two half steps make a whole step. We add the half steps by multiplying their ratios.
%\startformula
%	\text{half step} + \text{half step}
%		= \sqrt[12]{2} \times \sqrt[12]{2}
%		= 2^{\onetwelfth} \times 2^{\onetwelfth}
%		%= 2^{\twotwelfths}
%		= 2^{\onesixth}
%		= \sqrt[6]{2}
%\stopformula
%The equal-tempered whole step is $\sqrt[6]{2}$.
%\stopsolution
%\stopexample
%%%%%%%%%%%%%%%%%%%%%%%%%%%%%%%%%%%%%%%%%%%%%%
%%%%%%%%%%%%%%%%%%%%%%%%%%%%%%%%%%%%%%%%%%%%%
\startexample[ex:TinyPianoPyEx]
Find the ratio for the equal-tempered perfect fifth.
\startsolution
The perfect fifth is three whole steps and one half step, which is a total of seven half steps.
\startformula
	7\;\text{half steps}
		= \left(\!\sqrt[12]{2}\right)^7
		= 2^{\seventwelfths}
		\approx 1.498
\stopformula
The equal-tempered perfect fifth is $2^{\seventwelfths}$. This is not a simple fraction, but it is very close to $3/2$, the ratio of a just perfect fifth. The equal-tempered perfect fifth sounds perfect to most listeners.
\stopsolution
\stopexample
%%%%%%%%%%%%%%%%%%%%%%%%%%%%%%%%%%%%%%%%%%%%%

\placetable[margin][T:Intervals] % Label
    {Consonant intervals of the Ptolemy's scale and equal temperament.} % Caption
    {\vskip9pt\small\starttabulate[|l|c|c|]
\FL[2]%\toprule
\NC Interval		\NC Just						\NC Equal								\NR
\HL
\NC Octave			\NC $2$							\NC $2$									\NR
\NC Perfect 5th		\NC $3/2 = 1.5\phantom{3\dots}$	\NC $2^{\seventwelfths} \approx 1.498$	\NR
\NC Perfect 4th		\NC $4/3 = 1.33\dots$			\NC $2^{\fivetwelfths} \approx 1.335$	\NR
\NC Major 3rd		\NC $5/4 = 1.25\phantom{\dots}$	\NC $2^{\onethird}\, \approx 1.260$		\NR
\NC Minor 3rd		\NC $6/5 = 1.2\phantom{3\dots}$	\NC $2^{\onequarter}\, \approx 1.189$	\NR
\LL[2]%\bottomrule
\stoptabulate}

\noindent
In equal temperament, the octave is the only simple ratio. In fact, no other interval is a ratio of whole numbers – the ratios are all irrational roots of two. Yet equal-tempered fifths, fourths, and thirds are nearly as consonant as Ptolemy's intervals. Table \in[T:Intervals] shows the ratios as decimal numbers for just intonation and equal temperament. The worst match is the minor third, which is off by less than one percent. The perfect fifth is off by approximately $0.1\%$. These tiny differences are hardly noticeable.

Better yet, all of the intervals can be started on any pitch. No wolves lurk in the equal-tempered scale! This makes equal temperament extremely practical for composers and musicians. Galilei and his friends found and composed works demonstrating the practical benefits of equal temperament.

Many music purists were not impressed with equal temperament. Ptolemy's consonances do sound better, but the difference is small. The obvious way to please everyone is to adjust the step size of equal temperament so the intervals perfectly match the simple ratios. Unfortunately, this is mathematically impossible. The mathematics of rational numbers proves that certain consonances will not quite line up. The mismatch can be concentrated in a few wolf intervals, which must be avoided in compositions, or the mismatch can be spread out among many intervals in the hope that they will be too small to notice. Ptolemy took the extreme of having the fewest, most howling wolves. Equal temperament takes the other extreme, spreading the mismatch equally throughout the scale so that it is everywhere a barely audible purr.

Theorists and musicians experimented with several other tunings which sought to preserve as many perfect consonances as possible while keeping the wolves from becoming too disruptive. In the end, equal temperament became the standard. Nearly all western instruments are tuned to the equal-tempered scale.

The success of equal temperament eliminated the Pythagorean simple ratios from music, yet it also represents a great triumph of for the Pythagorean’s mathematical model. The equal-tempered tuning succeeds because it produces frequency ratios that are very close to Pythagoras and Ptolemy’s simple ratios.
Based on the experimental success of the equal-tempered scale, the mathematical model of musical consonance was revised again. \emph{Frequencies which approximate simple ratios are consonant.} Experiments show that ratios within about one percent of a simple ratio is good enough to sound consonant, although trained ears will notice the slight mismatch.

%%%%%%%%%%%%%%%%%%%%%%%%%%%%%%%%%%%%%%%%%%%%%
\section{Frequency relations}
%%%%%%%%%%%%%%%%%%%%%%%%%%%%%%%%%%%%%%%%%%%%%
Consonance, dissonance, scales, and tuning are all about ratios.
The individual frequencies hardly matter.
If all the frequencies in \in{figure}[T:Frequencies] were ten percent higher, nothing about the ratios would change.
All of the consonant intervals would be just as consonant.
%The same is true for any other instruments – as long as they are tuned in agreement when they play together.
However, ensuring that instruments can play together requires choosing specific frequencies and building instruments to produce those specific frequencies.

Audible frequencies are difficult to measure because the vibrations that produce audible sound are extremely fast.
Even the lowest audible frequencies are many cycles per second.
Marin Mersenne, in the generation after Galilei, was one of the first to carefully study the motions responsible for music.
He determined the frequencies of audible pitches indirectly.
First, he used a very long string vibrating at an inaudible $4\units{Hz}$, which was slow enough for him to measure.
Then he reduced the length by halves, which increased the pitch by octaves, until the string produced an audible  pitch.
Knowing that the frequency doubled with each octave allowed him to determine the audible frequency.

Whether the vibration is fast or slow, the most accurate way to find a vibration’s frequency is by timing a large number of cycles.
The number of cycles is represented by $N$
The total time, or \keyterm{duration}, is represented by $\Delta t$.
Frequency is the number of cycles divided by the duration of those cycles.

\startformula
f = \frac{N}{\Delta t}
\stopformula

This will give the frequency in cycles per second, or hertz, the usual unit of frequency.

The greek letter $\Delta$ (delta) represents difference or change.  In this case, $\Delta t$ is the difference in time between when you start counting the cycles and when you stop.  For example, if you start timing when your timer says $15\units{s}$ and stop when the timer says $45\units{s}$, then the duration is $\Delta t = 30\units{s}$.  The formula for duration is

\startformula
\Delta t = t\sf - t\si,
\stopformula

where $t\si$ is the initial time and $t\sf$ is the final time.  The subscripts \quote{i} and \quote{f} will be used frequently to represent initial and final values.  Note that final time comes first in the formula, and the initial time comes last.  This order, final minus initial, always gives a positive duration.

	%%%%%%%%%%%%%%%% EXAMPLE 1.3 %%%%%%%%%%%%%%%%
\startexample[ex:frequency]
	To determine the frequency of a violin string, you record the plucking of the string with a high-speed video camera.  Watching it in slow motion, you see the string plucked at $3.25\units{s}$ into the video and count $55$ cycles before ending at $3.50\units{s}$.  What is the frequency of the string’s vibration?

\startsolution
	The duration is $\Delta t = t\sf - t\si = 3.50\units{s}-3.25\units{s}= 0.25\units{s}$.  The frequency is:

	\startformula
		f = \frac{N}{\Delta t}
			= \frac{55\units{cyc}}{0.25\units{s}}
			= 220\units{cyc/s}
			= \answer{220\units{Hz}}.
	\stopformula
	Include cycles for the units of $N$ to get the correct units in your answer!
\stopsolution
\stopexample
%%%%%%%%%%%%%%%%%%%%%%%%%%%%%%%%%%%%%%%%%%%%%


The time required to complete one cycle is called the \keyterm{period}, represented by $T$.
The period can be related to the frequency using the frequency formula with the duration of one cycle being the period $T$.  This gives the extremely useful \keyterm{frequency relations}:

\startformula
	f = \frac{N}{\Delta t} = \frac{1\units{cyc}}{T}
\stopformula

Short periods correspond to high frequencies. Longer periods correspond to lower frequencies.

	%%%%%%%%%%%%%%%% EXAMPLE 1.6 %%%%%%%%%%%%%%%%
\startexample[ex:period]
	A piano string vibrates with a frequency of $220.0\units{Hz}$.  What is this vibration’s period?
\startsolution
	Use the frequency relations, crossing out the part we don’t need.

	\startformula
		f = \cancel{\frac{N}{\Delta t}} = \frac{1\units{cyc}}{T}
	\stopformula

	To solve for the period, multiply both sides by $T\!/\!f$.

	\startformula \startmathalignment
	\NC \frac{T}{\cancel{f}}\cancel{f} \NC= \frac{1\units{cyc}}{\cancel{T}}\frac{\cancel{T}}{f}	\NR
	\NC T \NC= \frac{1\units{cyc}}{f} \NR
	\stopmathalignment \stopformula

	Plug in values, and calculate.

	\startformula
		T = \frac{1\units{cyc}}{f}
			= \frac{1\units{cyc}}{220.0\units{Hz}}
			= \frac{1\units{\ucan{cyc}}}{220.0\units{\ucan{cyc}/s}}
			= \answer{4.545\units{ms}}
	\stopformula

	The string has a period of $4.545\sci{-3}\units{s}$.%, as shown in figure~\in[fig:vibrationgraph].
\stopsolution
\stopexample
	%%%%%%%%%%%%%%%%%%%%%%%%%%%%%%%%%%%%%%%%%%%%%

The standard pitches of the modern equal-tempered scale are based on a middle A with a frequency of $440\units{Hz}$.
Higher As’ frequencies are found by doubling; lower As’ frequencies are found by halving. 
Frequencies of other notes can be found by using the ratios of equal temperament.


	%%%%%%%%%%%%%%%%%%%%%%%%%%%%%%%%%%%%%%%%%%%%%
\section{Ancient Greeks and music in the modern age}
	%%%%%%%%%%%%%%%%%%%%%%%%%%%%%%%%%%%%%%%%%%%%%

The Pythagoreans produced one of the most successful mathematical models of all time. It was expanded by Ptolemy and clarified in response to Galilei, but the basic connection between the mathematical abstraction of simple ratios and the perceived, physical reality of consonant pitches remains unchanged. However, none of this tells us \emph{why} the model is successful. What makes simple frequency ratios sound so good? We do not have a \emph{theory} of consonance which would explain the importance of simple ratios. This is the deepest and most beautiful mystery we will solve in this book. Galilei points us in the right direction: the solution is in the physics of motion.

Galilei offered one more challenge to his teacher, Zarlino, and the musical tradition built on simple frequency ratios. The abundance of consonances in just intonation allowed composers to write elaborate music with many voices singing many different pitches simultaneously. A skilled composer, writing for a professional choir or orchestra, could have these voices changing pitches at different times, some going up while others go down, creating tension through controlled dissonance and then resolution through converging consonance. Zarlino’s music theories focused these grand polyphonic works.

While Galilei appreciated the intellectual beauty of polyphony, his study of ancient Greek music, which was predominately monophonic, left him wondering if something emotional was lost in the multitude of simultaneous sounds.
Galileo and his friends proposed a return to the tradition of Greek theatrical music, characterized by expressive melodies with intelligible words, embellished to engage and move the audience. They called their new form {\it opera}. It was a huge success.

%%%%%%%%%%%%%%%%%%%%%%%%%%%%%%%%%
\section[AncientWorldModels]{Celestial objects do not circle Earth}
%%%%%%%%%%%%%%%%%%%%%%%%%%%%%%%%%

Before we investigate the Pythagorean’s mathematical model of the cosmos, there is one thing that I want you to remember: this model is terrible! Celestial objects do not circle Earth. Only the Moon goes around Earth, and none of the objects circle, they travel along ellipses. Some of these ellipses are close to circular, so a model made from circles could be a rough starting point for understanding the Solar System. A model in which all of the objects go around Earth is irredeemable. This became clear almost immediately to anyone making regular observations of the planets.

% textwidth figure
\placetextfloat[bottom][fig:Ptolemaicsystem]{The geocentric model from Peter Apian’s \booktitle{Cosmographia,} 1524. Earth sits motionless in the center while the heavens spin around it. The outermost sphere, labeled \emph{Primu Mobile} or \quotation{primary mover,} spins most quickly, dragging the inner spheres along at slightly slower rates. This diagram, like others in the coming pages, uses common symbols for the visible planets. Table \in[T:Astrosym] shows these symbols, as well as the symbols of two more recently discovered planets.} {\externalfigure[ptolemaicsystem][width=\textwidth]}

In the Pythagorean model (figure \in[fig:Ptolemaicsystem]) the Sun, Moon, planets, and stars all orbit the central, motionless Earth. This is a \keyterm{geocentric model}. (\quotation{Geo-} means Earth, so geocentric is \quotation{Earth centered.})
The Sun completes one revolution around Earth every day. The Moon, which is closer to Earth takes slightly more than a day to complete a revolution, about twenty-four hours and fifty minutes. The stars, which are farther away, complete a revolution in slightly less time than the Sun, about twenty-three hours and fifty-six minutes.
%This pattern of longer revolutions for closer objects and shorter revolutions for father objects holds for the planets Mars, Jupiter, and Saturn.
%However, it only applies to their average revolution times.
%All of these objects have complicated rhythms of speeding up and slowing down.


\define[2]\Astro{\NC #2 \NC #1 } % starfont

\placetable
    [margin,here]
    [T:Astrosym]
    {Astronomical Symbols for some important objects in our Solar System. Most of these symbols can be found in figure \in[fig:Ptolemaicsystem] above. The two outer planets, Uranus and Neptune, were discovered in 1781 and 1846, respectively.}
{\noindent% \vskip330pt
\starttabulate[|l|c|c|l|c|]
\FL[2]
\Astro{\Sun}{Sun}\NC
\Astro{\Mars}{Mars}\NR
\Astro{\Mercury}{Mercury}\NC
\Astro{\Jupiter}{Jupiter}\NR
\Astro{\Venus}{Venus} \NC
\Astro{\Saturn}{Saturn}\NR
\Astro{\Earth}{Earth} \NC
\Astro{\Uranus}{Uranus}\NR
\Astro{\Moon}{Moon} \NC
\Astro{\Neptune}{Neptune}\NR
\LL[2]
\stoptabulate}
In the geocentric model, the Sun, Moon, and stars all behave reasonably, traveling at nearly constant rates rates from east to west while moving somewhat north and south with the seasons. The planets, in contrast, have the most erratic movements, speeding up and slowing down, sometimes falling behind but at times moving faster than the stars. Venus and Mercury seem to be tethered to the Sun, sometimes falling behind but then aways overtaking the Sun, leading for a while, and falling behind again.
In addition, the planets get significantly brighter and then dimmer, as if they are approaching Earth and then receding away.

% textwidth figure
\placetextfloat[bottom][fig:CopernicanSystem]{The heliocentric model in \booktitle{On the Revolutions of the Heavenly Spheres,} 1543. In this model the distant stars are immobile. Planets orbit the Sun, with inner planets orbiting more quickly than outer planets. The Moon is the one exception, orbiting Earth (\emph{Terra} in the diagram) rather than the Sun (\emph{Sol.}).} {\externalfigure[Copernican_heliocentrism_theory_diagram][width=\textwidth]}

These motions inspired ancient Greeks to try several completely different models of the world (referring to Earth and the visible heavens). The most obvious modification is to put Mercury and Venus in orbit around the Sun, explaining why they never stray far from the Sun, and also why they move closer and then farther from Earth. Another model of the world puts Mars, Jupiter and Saturn in large orbits around the Sun, explaining their erratic behavior.

The most daring modification was proposed by Aristarchus in the third century \scaps{bc}. He allowed Earth to join the other planets orbiting the Sun, as shown in figure \in[fig:CopernicanSystem]. This is the \keyterm{heliocentric model}. (\quotation{Helio-} means Sun.) In this model Earth moves in two ways. It rotates around its own axis, completing one rotation every day, and it orbits the Sun, completing one revolution every year. All of the other known planets – Mercury, Venus, Mars, Jupiter, and Saturn – also orbit the Sun. (Since all of these planets appeared as tiny dots in the sky, no one knew if they were rotating. We now know that they all rotate.)
In the heliocentric model, the Moon does not orbit the Sun like the other planets do; it orbits Earth, completing one revolution each lunar month (about $29\onehalf$ days).
The heliocentric model became widely known among astronomers, but it was not widely accepted. One reason was the Moon’s exceptional behavior. Why would everything revolve around the central Sun, except for the Moon?

About this time, Eratosthenes measured Earth’s size –  an incredible achievement.
%Educated people of the time already knew that Earth was round
Eratosthenes had moved to Egypt to be librarian at the renowned Library of Alexandria, where the Nile flows into the Mediterranean Sea.
This put Eratosthenes in the perfect location to make this measurement.
Egypt is close enough to the equator for the Sun to pass almost directly overhead in the summer.
Eratosthenes learned there was a well in southern Egypt, in the town of Syene, where the noon Sun was \emph{exactly} overhead on the summer solstice, so that someone looking into the well would see the Sun’s reflection in the water far below.
Farther north, in Alexandria, the Sun was not quite directly overhead due to the Earth’s curvature. At noon on the summer solstice, Eratosthenes measured the shadow cast by a vertical post in Alexandria and found the Sun was one fiftieth of a circle (about seven degrees) away from vertical. This meant  the distance from the Syene to Alexandria was one fiftieth of Earth’s circumference, as shown in figure \in[fig:EarthRadius].

\startbuffer[EarthRadius]
\clip (-1.25,-.5) rectangle (3.75,17);
\foreach \x in {-1,-.5,...,3.5}
	\draw[black!50,postaction={decorate}, decoration={markings,% switch on markings
mark=between positions 1cm and 3cm step 1.55cm with {\arrow{stealth}}}] (\x,17)--(\x,13.96);
\path (1.25,16.75) node[below,fill=white,inner sep=2pt]{Parallel rays from the Sun}; % Rays label
\draw[thick, decoration={random steps, segment length=1mm, amplitude=.2mm},fill=black!10] (.1,13.96) -- (.1,14.96) % Well
	decorate{arc [start angle=89.62, end angle = -269.45, radius=14.96cm]} % Earth’s surface
	-- ( -.1,13.96) -- cycle;
\draw[->] (0,0)--(82.8:14.93) node[fill=black!10, inner sep=2pt, sloped, pos=0.98, left]{Post in Alexandria} ; % Angled radius
\draw[->] (0,0) -- node[fill=black!10, inner sep=2pt, sloped, pos=0.98, left]{Well in Syene} (0,13.94); % Vertical radius
\draw (0,4) arc [start angle=90, end angle = 82.8, radius=4cm]; % Earth’s angle
\draw (2,15.13) arc [start angle=270, end angle = 262.8, radius=.7cm]; % Post’s angle
\draw[<->,postaction={decorate, decoration={markings,
mark=at position .5 with {\node[above=2pt, fill=white, inner sep=2pt, transform shape]{5000 stadia};}}}] (0,15.11) arc [start angle=90, end angle = 82.8, radius=15.11cm]; % distance
\draw[very thick] (82.8:14.96) -- (82.8:15.96) ; % Post
\draw[<->,rounded corners] (2.03,15.13) -| (2.33,7) node[fill=black!10,inner sep=2pt]{$\tfrac{1}{50}$ of a circle} |- (.52,3.972);
\draw[->,postaction={decorate, decoration={markings,
mark=at position .5 with {\node[above=2pt, fill=white, inner sep=2pt, transform shape]{North};}}}] (81:15.46) arc [start angle=81, end angle = 77.4, radius=15.46cm]; % North
\fill (0,0) circle[radius=1pt] node[right]{Earth’s Center}; % Vertical radius
\stopbuffer

Measuring the angle was the easy part. Measuring the distance from Alexandria to the well was more difficult.
Eratosthenes may have measured the distance by making the trip himself, or he may have used distance measurements made by others.
Either way, he found the distance to be about $5{,}000$ stadia, or about $900\units{km}$. Multiplying by $50$, he estimated Earth’s circumference to be about about $250{,}000$ stadia, or about $45{,}000\units{km}$.
Earth’s circumference is now known to be $40{,}000\units{km}$ (or $4.00\sci{7}\units{m}$). Eratosthenes’s estimate is about $12\%$ larger than the correct value, impressively accurate considering the difficulty of the measurement.
%If you measure the circumference around Earth’s equator, you get the 40,075 km figure I mentioned, But if you measure it from pole to pole, you get 40,007 km. [http://www.universetoday.com/26461/circumference-of-the-earth/]

\marginTikZ{}{EarthRadius}{Eratosthenes determined Earth’s size. The well’s depth and the post’s height are greatly exaggerated.}

Astronomical measurements improved significantly over the next few centuries. The Babylonians, in particular, became master astronomers. They had been recording daily observations of the celestial motions since since the middle of the seventh century \scaps{bc} and continued into the first century \scaps{ad}. This approximately 800 year record was essential for Greek astronomers attempting to refine their model of the world.

When Ptolemy attacked this problem, armed with the precise astronomical observations of the Babylonians, he produced a geocentric model with dozens of complicated, invisible wheels that moved the planets though their complex motions. Unlike his elegant system of intonation, his model of the world is insanely complex and arbitrary. He explained all of the details in the \booktitle{Almagest,} which was not lost. His geocentric model, the \keyterm{Ptolemaic system,} became the accepted model of the world through the middle ages.


During the Renaissance, when musicians were experimenting with their daring thirds, the Polish astronomer Nicholas Copernicus produced his own daring model of the world. Copernicus used the same Babylonian data as Ptolemy – it was still the best data in existance – to produce a heliocentric model. All of the details are in his 1543 book, \booktitle{On the Revolutions of the Heavenly Spheres.} Figure \in[fig:CopernicanSystem] is Copernicus' simplified sketch of his system. His full model, like Ptolemy’s, was also insanely complex, requiring dozens of spheres to account for all of the planets' motions. His heliocentric model is often called the \keyterm{Copernican system}.

Given the complexity of the Copernican system, few people were persuaded to abandon Ptolemy's geocentric model.  Finding the correct model of the world would require two things: a willingness to abandon failed ideas, and new observations.
\pagereference[AncientWorldModelsEnd]

%%%%%%%%%%%%%%%%%%%%%%%%%%%%%%%%%%%%%%%%%%%%%
\startsection[title=Galileo’s spyglass]
%%%%%%%%%%%%%%%%%%%%%%%%%%%%%%%%%%%%%%%%%%%%%

While Vincenzo Galilei was making waves in the music world, his equally confrontational son, Galileo Galilei\index{Galileo}, became a professor of mathematics, first at the University of Pisa (1589--1592) and then at the University of Padua (1592--1610).
Galileo understood both the geocentric and heliocentric models and taught the geocentric model to students. His primary interest was not astronomy, but rather motion, fluids, and structures. He wrote several papers on these subjects for his students, but the papers were not published or distributed widely.

Then, in 1609, he heard of a device which seized his attention completely. He tells the story of this device and the discoveries that followed in his 1610 booklet, \booktitle{The Sidereal Messenger}. (Sidereal means \quotation{of the stars}.)
\startblockquote
	About ten months ago a report reached my ears that a Dutchman had constructed a spyglass, by the aid of which visible objects, although at a great distance from the eye of the observer, were seen distinctly as if near....\autocite{p.~49}{Galileo1610}
\stopblockquote
Amazed by this report, he set out to understand and then construct his own spyglass, which we now call a telescope. He describes how he first built a telescope capable of magnifying objects by a factor of three, then eight, and finally, \quotation{sparing neither labor nor expense,} he constructed a telescope that magnified objects thirty times.
\startblockquote
	It would be altogether a waste of time to enumerate the number and importance of the benefits which this instrument may be expected to confer when used by land or sea. But without paying attention to its use for terrestrial objects, I betook myself to observations of the heavenly bodies.\autocite{p.~49--50}{Galileo1610}
\stopblockquote

Galileo shares his observations with delight and wonder. He presents his observations of the Moon in careful drawings, figure \in[fig:MoonSurface], that reveal its surface in greater detail than anyone had seen before.
% Margin image
\placefigure[margin][fig:MoonSurface]{The Moon’s hills and valleys are clearly visible in Galileo’s original watercolors.} {\externalfigure[MoonPhases6][width=\rightmarginwidth]}
\startblockquote
	...anyone may know with the certainty that is due to the use of our senses that the Moon certainly does not possess a smooth and polished surface, but one rough and uneven, and, just like the face of the earth itself, it is everywhere full of vast protuberances, deep chasms, and sinuosities.\autocite{p.~48}{Galileo1610}
\stopblockquote
%Just as Vincenzio Galilei’s had overturned Aristotle’s understanding of a string’s vibration,
This observation challenged Aristotle’s view that objects in the heavens were perfectly smooth and round, made of an immutable substance unlike anything on Earth.

% Margin image
\placefigure[margin][fig:Orion]{Galileo writes, \quotation{I had determined to depict the entire constellation of Orion, but I was overwhelmed by the vast quantity of stars.... For this reason I have selected the three stars in Orion’s Belt and the six in his Sword, which have long been well known groups, and I have added eighty other stars recently discovered in their vicinity.... The well-known or old stars, for the sake of distinction, I have depicted of larger size....}\autocite{p.~65}{Galileo1610}} {\externalfigure[orion][width=\rightmarginwidth]}

Galileo next shared his amazement when the telescope,
	\quotation{set distinctly before the eyes other stars in myriads, which have never been seen before, and which surpass the old, previously known, stars in number more than ten times.}\autocite{p.~48}{Galileo1610}\ He drew a small sample, figure \in[fig:Orion], to show how numerous the newly revealed stars are compared to those previously known.
This multitude of stars did not challenge any widely held view about the heavens, but it certainly demonstrated the telescope’s power.

Finally, Galileo relates his most amazing discovery.
\startblockquote
	There remains the matter that seems to me to deserve to be considered the most important in this work. That is, I should disclose and publish to the world the occasion of discovering and observing four planets never seen from the beginning of the world up to our own times, their positions, and the observations made during the last two months about their movements and their changes of magnitude [brightness]. And I summon all astronomers to apply themselves to examine and determine their periodic times, which it has not been permitted me to achieve up to this day owing to the restriction of my time.\autocite{p.~67--68}{Galileo1610}
\stopblockquote

Galileo discovered these planets in the vicinity of Jupiter, observing three of them for the first time on January 7, 1610. Over the next few nights he observed them moving quickly around Jupiter, so that on some nights they appeared just to the east of Jupiter, and on other nights just to the west.
These observations are reproduced in figure \in[fig:JupiterMoons].
%He reports that on January 11 the explanation for their motion became clear.
\startblockquote
	I therefore concluded, and decided unhesitatingly, that there were three stars in the heavens moving around Jupiter, like Venus and Mercury around the Sun. This was finally established as clear as daylight by numerous other subsequent observations. These observations also established that there are not only three, but four, wandering sidereal bodies performing their revolutions around Jupiter.\autocite{p.~69}{Galileo1610}
\stopblockquote
The four celestial objects that Galileo discovered, which he referred to as \quotation{planets,} \quotation{stars} and \quotation{wandering sidereal bodies} in the passages above, are now called moons of Jupiter, or jovian moons. Today the term \keyterm{star} is reserved for large, hot objects like the Sun that produce their own light. The term \keyterm{planet} is reserved for objects orbiting the Sun or another star. \emph{Moon}\index{moon} refers to an object orbiting a planet. Jupiter has many moons too small for Galileo  to have seen with his telescope.
Jupiter’s four large moons, often called the \keyterm{Galilean moons}, are huge, almost as large as the planet Mars.

% Margin image
\placefigure[margin][fig:JupiterMoons]{Galileo’s drawings of his first seven observations of Jupiter and its moons in January of 1610. He discovered four moons, but on many nights he saw fewer. Sometimes two of the moons were too close together to be seen separately, or moons were obscured by Jupiter. Clouds prevented observations on January 9 and 14.\autocite{p.~68--70}{Galileo1610}}
{\small
	\leftaligned{January 7, 1610}\\
	\externalfigure[Jupiter1][width=\rightmarginwidth]\\
	\leftaligned{January 8}\\
	\externalfigure[Jupiter2][width=\rightmarginwidth]\\
	\leftaligned{January 10}\\
	\externalfigure[Jupiter3][width=\rightmarginwidth]\\
	\leftaligned{January 11}\\
	\externalfigure[Jupiter4][width=\rightmarginwidth]\\
	\leftaligned{January 12}\\
	\externalfigure[Jupiter5][width=\rightmarginwidth]\\
	\leftaligned{January 13}\\
	\externalfigure[Jupiter6][width=\rightmarginwidth]\\
	\leftaligned{January 15}\\
	\externalfigure[Jupiter7][width=\rightmarginwidth]
	\blank[small]
}

Galileo’s  discovery of Jupiter’s moons removed one of the primary objections to the Copernican system.
\startblockquote
	Additionally, we have a notable and splendid argument to remove the scruple of those who can tolerate the revolution of the planets around the Sun in the Copernican system, but are so disturbed by the motion of one Moon around the earth (while both accomplish an orbit of a year’s length around the Sun) that they think this constitution of the universe must be rejected as impossible. For now we have not just one planet revolving around another while both traverse a vast orbit around the Sun, but four planets which our sense of sight presents to us circling around Jupiter (like the Moon around the earth) while the whole system travels over a mighty orbit around the Sun in the period of twelve years.\autocite{p.~83--84}{Galileo1610}
\stopblockquote

%\section{Period of orbits and rotations}

When Galileo says that Jupiter completes an orbit \quotation{in the period of twelve years,} he is using \quotation{period} just as it was used for vibrations in Chapter \in[ch:Music]. Period\index{period} is still the time required for one \keyterm{cycle}.
In the case of the vibrating string, a cycle is one complete back-and-forth motion. In the case of an orbit, like Galileo describes, a cycle is one complete revolution of the orbiting object around the central object. In the case of rotation, a cycle is one complete rotation about the axis. The period of Earth’s rotation about its axis is $T=1\units{day}$. %In the case of an orbit, a cycle is one complete revolution about the orbit’s center.
The period of Earth’s revolution around the Sun is $T=1\units{year}=365.24\units{days}$. %In all cases the period is the time required to complete a cycle.
Musical vibrations have periods of hundredths or thousandths of a second. Orbits can have periods of days, years, decades or longer. The same symbol $T$ represents periods of any duration.

\section{Period relations}
The orbits of planets can be described by their frequencies rather than their periods. For example, Mercury orbits the Sun with a period $T=88.0$ days. The frequency can be found using the frequency relations from Chapter \in[ch:Music]. We could find the frequency in Hertz, but since the period is months long, let us find the frequency in cycles per year.
\startformula
	f = \frac{1\units{cyc}}{T}	%\\
		= \frac{1\units{cyc}}{88.0\ucan{d}}
			\left(\frac{365.24\ucan{d}}{1\units{yr}}\right)
		= 4.15\units{cyc/yr}
\stopformula
In practice, frequencies are rarely used to describe orbits and rotations because the periods are so long. However, there are some cases of extremely fast rotations and revolution where frequency is used.
% Some of the are explored in the problems at the end of the chapter.

%\question
%What is the frequency of Jupiter’s orbit around the Sun?
%\begin{solution} Period is the reciprocal of frequency.
%	\begin{align*}
%		T &= \frac{1\units{cyc}}{f}	\\
%		f &= \frac{1\units{cyc}}{T}	%\\
%			= \frac{1\units{cyc}}{12\units{yr}}	%\\
%			%= \frac{1\units{\ucan{cyc}}}{440\units{\ucan{cyc}/s}}
%			= \answer{0.083\units{cyc/yr}}
%	\end{align*}
%Jupiter completes only 0.083 cycles of its orbit in one Earth year.
%We can convert this to Hertz.
%	\startformula
%		f = 0.083\frac{\units{cyc}}{\units{\ucan{yr}}}
%			\frac{1\units{\ucan{yr}}}{365\units{\ucan{days}}}
%			\frac{1\units{\ucan{day}}}{24\units{\ucan{hr}}}
%			\frac{1\units{\ucan{hr}}}{60\units{\ucan{min}}}
%			\frac{1\units{\ucan{min}}}{60\units{s}}
%		= \answer{2.6\sci{-9}\units{Hz}}
%	\stopformula
%\end{solution}

%Astronomers usually do not use frequency to describe orbits or rotations; they  use period. However, there are some exceptional events where frequency is useful. Two can be found in the problems at the end of this chapter.

%\section{Amplitude and radius}

Since period is the more common measure in astronomy, it is useful to solve the frequency relations for the period, $T$, which gives the \keyterm{period relations}.
\startformula
	T = \frac{1\units{cyc}\cdot \Delta t}{N} = \frac{1\units{cyc}}{f}
\stopformula
As with frequency, the most accurate way to measure the period is to measure the duration of many cycles and then divide by the number of cycles, as shown in the first fraction in the period relations. The factor $1\units{cyc}$ in the numerator will cancel the units of $N$ (also cycles) so that the period is in seconds, years, or any other time unit.




%[Salviati helps Simplicio draw the Solar System.]
\stopsection


%[Galileo uses the periods to make a stronger case for heliocentrism in the Dialogue. Larger orbits correspond to longer periods. [pp.206-7] Motionless stars are \quotation{so many suns.}[p.241] Note the 1:2:4 resonances of Jupiter’s inner three moons (octaves!). ]

\startsection[title={Solar System speeds}]

%[Galileo was not the only astronomer refining our model of the Solar System. Tycho precision observations and Kepler sophisticated modeling.... A scale diagram of the Solar System would be good here. Can I find a drawing of Kepler’s Solar System? Alternatively, we could talk about the older precision data and with a teaser about better data which was not available to Galileo, but will be discussed in chapter 7.]

A circular orbit’s radius is measured from the orbit’s center, just as amplitude is measured from the oscillating motion’s central point.
Planets and moons following circular orbits travel with a constant speed.
%\footnote{This footnote is only for savvy readers screaming \quotation{but the \emph{direction}!} Newton, in 1687, taught us that changes in the velocity’s direction are just as important as changes in the magnitude. In Newton’s view, the velocity \emph{is not} constant along a circular path because the direction changing. Lagrange responded, in 1790, that we can bend our coordinates to match our situation, and then consider velocities along the bent coordinates. After bending the coordinates to follow the circle, the velocity \emph{is} constant. Lagrange’s trick is both practical and powerful. We will take full advantage of it. The two methods are compared in Chapter \in[ch:Orbits].}
The speed can be calculated using \quotation{distance over duration.} During one cycle the distance is the circumference of the orbit, $2\pi R$, and the duration is the period, $T$.
%\highlightbox{
\startformula[eq:vT]\pagereference[eq:vT]
	\text{speed} = \frac{2\pi R}{T}.
\stopformula
%}
%We have measured the displacement along the circular path.%, even though the planet returns to its original position after one orbit.

\startexample[ex:EarthSurfaceSpeed]
Earth completes one full rotation every day, so a person near the equator will travel Earth’s circumference each day. The tower in Florence is almost halfway between the equator and the North Pole, so each day the tower completes a somewhat smaller circle, with a radius of $4600\units{km}$, as shown in \in{figure}[fig:FlorenceCircle1]. What is the speed of tower due to Earth’s rotation?
% Florence: 43°46′17″N 11°15′15″E
\startplacefigure[location=margin, reference=fig:FlorenceCircle1, title={Florence circles Earth's axis every day, following a circle of radius $R = 4600\units{km}$.}]
\startMPcode
  pickup pencircle scaled 0.8pt ;
  draw externalfigure "EarthNorthPole.png" scaled 0.498 shifted (-2.5cm,-2.5cm) ;
  path Equator, Circle, Radius ; pair Florence ;
  Equator := fullcircle scaled 5cm ;
  Florence := dir(-33.75)*1.805cm ;
  Circle := fullcircle scaled 3.61cm ;
  Radius := origin -- Florence ;
  draw Equator;
  label.bot  ("Equator", (0,-2.5cm)) ;
  dotlabel.top ("Earth's Axis", origin) ;
  draw Radius ;
  drawarrow Circle;
  label.urt  ("$R$", .5Florence) ;
  begingroup;
    interim labeloffset := 2mm ;
    dotlabel.lft  ("Florence", Florence) ;
  endgroup;
\stopMPcode
\stopplacefigure
\startsolution
The formula for the speed of circular motion will provide the answer, with the help of several unit conversions.
\startformula\startmathalignment
	\NC \text{speed}	\NC = \frac{2\pi R}{T}		\NR
	\NC				\NC = \frac{2\pi\cdot4600\ucan{km}}{24\units{\ucan{hr}}}
				\left(\frac{1000\units{m}}{1\units{\ucan{km}}}\right)
				%\frac{1\units{\ucan{day}}}{24\units{\ucan{hr}}}
				\left(\frac{1\units{\ucan{hr}}}{60\units{\ucan{min}}}\right)
				\left(\frac{1\units{\ucan{min}}}{60\units{s}}\right)	\NR
	\NC				\NC = \answer{330\units{m/s}}
				%\quad \text{or}\quad
				%\answer{460\units{m/s}}
\stopmathalignment\stopformula
	The tower’s speed is $330\units{m/s}$ in the direction of Earth’s rotation (west to east). That is fast!
\stopsolution
\stopexample

In 1609, one year before Galileo’s \booktitle{Sidereal Messenger}, Johannes Kepler announced that the planets’ orbits are not described by circles – they are actually slightly elliptical. Ellipses are simple shapes, but they did not fit well into the mechanical models of spheres and wheels that drove both the Ptolemaic and Copernican world systems. Kepler found a mathematical model for the world’s motions, but the forces that could drive these off-center, elliptical motions were mysterious. The perspective was shifting, from a mechanical model to a new physics of motion. Galileo, who did not take much notice of Kepler’s announcement, did understand motion, which he had been studying before his sudden interest in astronomy. He used the discoveries from his lab to understand the world he discovered through his spyglass.

\booktitle{The Sidereal Messenger}, published in March of 1610, made Galileo a celebrity. In April, the grand duke of Tuscany, Cosimo II de' Medici, appointed Galileo to be his \quotation{Philosopher and Chief Mathematician.} Galileo quit his job at the University of Padua and moved to Florence where he refined his telescopes and studied the heavens in greater detail. 

Galileo's telescope revealed that Venus goes through phases, much like the Moon (\in{fig.}[fig:Phases]). Astronomers knew that Venus becomes brighter and then dimmer as it moves, but only with a telescope can Venus's phases be seen. Galileo also observed the apparent size of Venus changing dramatically. Venus appears much larger and closer in its crescent phase; much smaller and more distant when fully lit.

% Margin image
\placefigure[margin][fig:Phases]{Galileo’s drawings show Venus’s phases and changes in apparent size, further supporting the heliocentric model.\autocite{p.~217}{Galileo1623}} {\externalfigure[VenusPhases][width=144pt]}

Venus's phases and changes in size are additional strong evidence for the heliocentric model. Sometimes Venus and Earth are on the same side of the Sun, with Venus passing between the Sun and Earth. During this pass, Venus gets relatively close to Earth and is lit by the Sun almost from behind, giving Venus a large, crescent appearance.  At other times Venus and Earth are on nearly opposite sides of the Sun. Then Venus is especially far away and fully lit, giving Venus a small, round appearance.

Galileo continued to write about his observations in letters and books, arguing in favor of the heliocentric model. He promoted observation as the best method for discerning truths about the world, just as his father had promoted experiments as the best method for discerning truths about sound and music.


\stopsection

\subject{Notes}
\blank
\startcolumns
%\placefootnotes[criterium=chapter]
\placenotes[endnote][criterium=chapter, method=local]
\stopcolumns

\subject{Bibliography}
        \placelistofpublications  [criterium=chapter, method=local]			% Citations for this chapter only

\stopchapter

\stoptext

\stopcomponent


\page
%%%%%%%%%%%%%%%% EXERCISES %%%%%%%%%%%%%%%%
\startsubject[title=Exercises]
%\setuplayout[
%	leftmargin=36pt, % 1/2 in
%	leftmargindistance=9pt, % 3/8 in
%	width=477pt, % 4 1/4 in
%	rightmargindistance=9pt, % 3/8 in
%	rightmargin=36pt,  % 2 in
%]
%\setupheadertexts[text][section][\pagenumber][\pagenumber][chapter]
%\setupheadertexts[margin][][][][]
%
%\blank
%\startcolumns[n=2, tolerance=verytolerant]
\startitemize[n,packed]
\question
Find the frequency of the pitch one octave above middle A ($440\units{Hz}$).  What is the period of this higher pitch? %(Here \quote{A} is the name of the pitch.  It has nothing to do with the vibration’s amplitude, $A$.)
\blank

\question
Find the frequency of the pitch one perfect fifth below middle A.
\blank
%\question
%What the frequency of the highest A on a grand piano? What subscript should be put on this A?

\question Starting with the duration formula and working only with variables\nowhitespace
\startitemize[a,joinedup]%,packed]
\item solve for $t_{\rm f}$,
\item solve for $t_{\rm i}$.
\stopitemize
\blank

\question Starting with the frequency formula and working only with variables
\startitemize[a,packed,joinedup]
\item solve for $N$,
\item solve for $\Delta t$.
\stopitemize
\blank

\question
How many cycles will the middle A ($440\units{Hz}$) string oscillate in $2.5\units{s}$?
\blank

\question
How long will it take for the  D string with a frequency of $293\units{Hz}$ to oscillate one thousand times?
\blank
%\question
%What is the frequency of C$_0$? What is its period?
%\blank

\question Starting with the frequency relations and working only with variables
\startitemize[a,packed]
\item find $T$ in terms of $f$,
\item find $T$ in terms of $N$ and $\Delta t$.
\stopitemize
\blank[big]


%\question The Moon completes one revolution around Earth in 29.3 days. Find the speed of the Moon in its orbit around Earth.

\question The inner two moons of Jupiter, Io and Europa, are in a 1:2 resonance, which means that the ratio of Io’s period to Europa’s period is one-to-two. Europa’s orbital period is 3.55 days. What is Io’s orbital period?
%\begin{solution}[3in]
%\begin{align*}
%	\frac{T\sub{E}}{T\sub{I}} &= \frac{2}{1}	\\
%	T\sub{I} &= \half T\sub{E}	\\
%		&= \half 3.55\units{d}	\\
%		&= \answer{1.78\units{d}}
%\end{align*}
%\end{solution}

\question Jupiters second and third moons, Europa and Ganymede, are also in a 1:2 resonance. What is the Ganymede’s orbital period?
%\begin{solution}[3in]
%\begin{align*}
%	\frac{T\sub{E}}{T\sub{I}} &= \frac{2}{1}	\\
%	T\sub{I} &= \half T\sub{E}	\\
%		&= \half 3.55\units{d}	\\
%		&= \answer{1.78\units{d}}
%\end{align*}
%\end{solution}


\question Galileo published \textit{The Starry Messenger} in 1610. The Galileo spacecraft, which studied Jupiter and its moons up close, became the first craft to orbit Jupiter in 1995. How many orbits did Jupiter make between these two events? Jupiter orbits the Sun once every 11.86 years.

\question The Galileo spacecraft was launched on October 18, 1989. Its successful mission as the first man-made satellite of Jupiter ended on September 21, 2003 with an intentional crash into Jupiter’s atmosphere. Find the length of the Galileo mission in jovian years.

%\question Saturn orbits the Sun once every 29.5 years. How many orbits has it made since Galileo published \textit{The Starry Messenger} in 1610?
%\begin{solution}[3in]
%	\startformula
%T = \frac{1\units{cyc}\cdot\Delta t}{N}
%\stopformula
%First, find $\Delta t$.
%\startformula
%	\Delta t = t\sf - t\si = 2017 - 1610 = 407\units{yr}
%\stopformula
%Then we can find the number of cycles (or orbits).
%	\begin{align*}
%		T &= \frac{1\units{cyc}\cdot\Delta t}{N}	\\
%		N &= \frac{1\units{cyc}\cdot\Delta t}{T}	\\
%			&= \frac{1\units{cyc}\cdot(407\units{\ucan{yr}})}{29.5\units{\ucan{yr}}}	\\
%			& = \answer{13.8\units{cyc}}
%				\quad\text{\emph{or}}\quad
%				\answer{13.8\units{orbits}}
%	\end{align*}
%\end{solution}

\question The Cassini spacecraft journey to Saturn began with its launch on October 15, 1997. Cassini orbited Saturn, studying the planet, its many moons, and its amazing rings until September 15, 2017. How long was Cassini’s mission in Saturn’s years? Saturn’s orbit takes 29.5 Earth years.

\question Earth’s Moon orbits Earth with a frequency of $13.4\units{cyc/yr}$. What is the Moon’s orbital period in days?
%\begin{solution}[3in]
%\begin{align*}
%	T &= \frac{1\units{cyc}}{f}	\\
%		&= \frac{1\units{cyc}}{13.4\units{cyc/yr}}	\\
%		&= 0.0746\units{\ucan{yr}}\left(\frac{365\units{d}}	{1\units{yr}}\right)	\\
%		&= \answer{27.2\units{d}}
%\end{align*}
%Accept answer in years is question does not specify days (2017).
%\end{solution}


\question In 1967, Jocelyn Bell Burnell and Antony Hewish discovered a star that flashed brightly every $1.33\units{s}$. Eventually, pulsing stars like this became known as pulsars. Pulsars are rapidly spinning neutron stars which shine bright beams of light out of their magnetic poles. We see a flash every time one of the spinning pulsar’s beams points in our direction. What is this pulsar’s frequency?

\question In 1968 another pulsar was discovered in the nearby Crab Nebula with a period of only $33\units{ms}$. What is this pulsar’s frequency?

\question On August 17, 2017, an extremely sensitive device known as \scaps{ligo} (Laser Interferometer
Gravitational-Wave Observatory) detected a tiny vibration of space-time. These vibrations were gravitational waves from a pair of orbiting neutron stars which revealed the stars' orbital frequency to be $20\units{Hz}$. What was the orbital period of these two stars? (This ferocious dance did not last long. Over the next thirty seconds the neutron stars spiraled into each other, producing an enormous explosion. The merged stars then collapsed to form a black hole. Luckily, this violent event happened very, very far from us.)

\stopitemize
%\stopcolumns
\stopsubject