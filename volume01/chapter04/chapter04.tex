% !TEX useConTeXtSyncParser
\startcomponent c_chapter04
\project project_world
\product prd_volume01

\doifmode{*product}{\setupexternalfigures[directory={chapter04/images}]}

\setupsynctex[state=start, method=min] % "method=max" or "min"

%%%%%%%%%%%%%%%%%%%%%%%%%%%%%
\startchapter[title=Vis Viva,reference=ch:VisViva]
%%%%%%%%%%%%%%%%%%%%%%%%%%%%%

% Margin text
\placefigure[margin,none]{}{\tfx
	\startalignment[flushleft]
You must early on accustom your mind to think, and to be self-sufficient. You will perceive at all the times in your life what resources and what consolations one finds in study, and you will see that it can even furnish pleasure and delight.
	\stopalignment
	\break
	\rightaligned{\it Foundations of Physics}
	\rightaligned{\sc Emilie Du Châtelet}
	\rightaligned{1706–1749}}

%[Findent=-.4em, Nindent=.9em]
\Initial{L}{eibniz} was not satisfied with Newton’s momentum because it failed to achieve the basic goal of Descartes’s quantity of motion, which was to quantify the total motion of a system with many moving parts.
%A system of many moving parts has a greater quantity of motion than a system in which all of the parts are at rest. Momentum completely fails to make this basic distinction.

\startbuffer[TikZ:restingballs]
\environment env_physics
\environment env_TikZ
\setupbodyfont [libertinus,11pt]
\setoldstyle % Old style numerals in text
\startTEXpage
\starttikzpicture% tikz code
\draw[shade, ball color = black] (2,0.25) circle[radius=.25cm]; % left Ball
\draw[shade, ball color = black] (3,0.25) circle[radius=.25cm]; % right Ball
\shade[top color=gray] (0,-.2) rectangle (5,0); % Ground
\draw (0,0)--(5,0);
\draw[white] (0,1.5)-- ++(5,0); % Sky to make height better
\stoptikzpicture
\stopTEXpage
\stopbuffer

\placefigure[margin][fig:restingballs] % location, label
{Two cannon balls resting harmlessly on the ground.} % caption text
{\noindent\typesetbuffer[TikZ:restingballs]} % figure contents


\placetable
    [margin]
    [T:restingballs]
    {Descartes ($m\abs{v}$), Newton ($mv$), and Leibniz ($mv^2$) all agree: there is no motion when the cannon balls are not moving.}
    {\vskip9pt\small%\hbox{
	\starttabulate[|l|c|c|c|]
\FL[2]%\toprule
\NC			\NC $m\abs{v}$	\		\NC $mv$ 			\NC $mv^2$		\NR
\HL
\NC Left Ball	\NC $0$				\NC $0$				\NC $0$			\NR
\NC Right Ball	\NC $0$				\NC $0$				\NC $0$			\NR
\HL
\NC Total		\NC $0$				\NC $0$				\NC $0$			\NR
\LL[2]%\bottomrule
    \stoptabulate}

Consider two cannon balls resting harmlessly on the ground and two others fired directly toward each other at great speed (\in{figure}[fig:restingballs] and \in{}[fig:firedballs]). Descartes and Newton would agree that the resting cannon balls have no motion (\in{table}[T:restingballs]). They would disagree about the moving cannon balls (\in{table}[T:firedballs]). Descartes’s quantity of motion is positive for each moving cannon ball. Therefore, according to Descartes, the total quantity of motion for the moving cannon balls is large. Newton, however, taught us to consider the direction of the motion.
Both moving cannon balls have momentum, but one’s momentum is positive while the other’s momentum is negative. The total momentum is \emph{zero} -- exactly the same as the total momentum of the two motionless cannon balls.
\placetable
    [margin]
    [T:firedballs]
    {Cannon balls fired directly toward each other have no total quantity of motion, according to Newton.  Descartes and Leibniz both say there is motion, but calculate different amounts (using $m=40\units{kg}$ and $v=\pm50\units{m/s}$).}
    {\vskip18pt\small%\hbox{
	\starttabulate[|l|c|c|c|]
\FL[2]%\toprule
\NC		\NC $m\abs{v}$		\NC $mv$ 			\NC $mv^2$			\NR
\NC		\NC $\unit{kg\.m/s}$	\NC $\unit{kg\.m/s}$		\NC $\unit{kg\.m^2\!/s^2}$	\NR
\HL
\NC Left	\NC \ $2000$		\NC $\phantom{-}2000$	\NC $1.0\sci{5}$		\NR
\NC Right	\NC \ $2000$		\NC $-2000$			\NC $1.0\sci{5}$		\NR
\HL
\NC Total	\NC $4000$		\NC $\phantom{-200}0$	\NC $2.0\sci{5}$		\NR
\LL[2]%\bottomrule
    \stoptabulate}

\startbuffer[TikZ:firedballs]
\environment env_physics
\environment env_TikZ
\setupbodyfont [libertinus,11pt]
\setoldstyle % Old style numerals in text
\startTEXpage
\starttikzpicture% tikz code
\shade[right color=gray,left color=white] (4,1) rectangle (7.5,1.5);
\draw[shade, ball color = black] (7.5,1.25) circle[radius=.25cm]; % left Ball
\shade[left color=gray] (9.2,1) rectangle (12.7,1.5);
\draw[shade, ball color = black] (9.2,1.25) circle[radius=.25cm]; % right Ball
\fill[fill=black!70] (4,0.75)-- ++(0,1)-- ++(-3,.25)-- ++(0,-1.5)-- cycle; % left Cannon Barrel
\fill[fill = black!70] (1,1.25) circle[radius=0.75cm]; % left Cannon Back
\draw[fill=gray] (0,0)-- ++(0,0.25)-- ++(1,1)-- ++(1,0)-- ++(1,-1)-- ++(0,-0.25)-- cycle; % left Cannon Base
\draw[fill = black!70] (1.5,1.25) circle[radius=0.25cm]; % left Cannon Pivot
\fill[fill=black!70] (12.7,0.75)-- ++(0,1)-- ++(3,.25)-- ++(0,-1.5)-- cycle; % right Cannon Barrel
\fill[fill = black!70] (15.7,1.25) circle[radius=0.75cm]; % right Cannon Back
\draw[fill=gray] (16.7,0)-- ++(0,0.25)-- ++(-1,1)-- ++(-1,0)-- ++(-1,-1)-- ++(0,-0.25)-- cycle; % right Cannon Base
\draw[fill = black!70] (15.2,1.25) circle[radius=0.25cm]; % right Cannon Pivot
\shade[top color=gray] (0,-.2) rectangle (16.7,0); % Ground
\draw (0,0)--(16.7,0);
\draw[white] (0,2.5)-- ++(16.7,0); % Sky to make height better
\stoptikzpicture
\stopTEXpage
\stopbuffer

\placefigure[margin][fig:firedballs] % location
{Two cannon balls fired directly at one another have lots of motion\dots or none.}	% caption text
{\vskip10pt\hbox{\starttikzpicture
	\draw[white] (0,0)-- ++(5,0); % Sky to make height better
\stoptikzpicture}}

\placewidefloat[bottom,none]
{This is its caption I need to fix.}
{\noindent\typesetbuffer[TikZ:firedballs]} % figure contents

Clearly, the moving cannon balls have \emph{something} which the resting cannon balls lack.
Descartes’s quantity of motion is a reasonable guess, but collision experiments contradicted the predictions based on Descartes’s quantity of motion.
Newton’s momentum gives correct predictions for collisions, but momentum fails to distinguish between the large motion of the moving cannon balls and the lack of motion in the resting cannon balls.

Like Newton, Leibniz sought inspiration in the laws of collision demonstrated with pendulum collisions. He was especially impressed by Huygens, who had given a detailed analysis of these collisions.
\startblockquote
%	Huygens, the first that I know who has adorned our age with splendid discoveries, seems to me in this argument also to have reached the pure and unadulterated truth and to have freed this doctrine from paralogisms, by certain rules formerly published. Wren also, Wallis and Mariotte, men excellent in these studies though diverse in method, demonstrated nearly these same rules.[New Understanding p. 676]
	Huygens, who illuminated our age with his excellent discoveries, seems to be the first person I know of to have arrived at the pure and clear truth on this matter.... %, and the first to have freed this subject from paralogisms through certain laws he once published.
	Wren, Wallis, and Mariotte, gentlemen excellent in these studies, though, granted, in different ways, all obtained virtually the same rules.\autocite{123}{Leibniz1695}%\footnote{\cite[righttext={ p.\nbsp 123}][Leibniz1695]}
\stopblockquote
Contrary to Newton, Leibniz adopted Huygen’s conserved quantity $mv^2$ as the replacement for Descartes’s quantity of motion. Huygens’s $mv^2$ is positive for any motion in any direction.
Two cannon balls sitting in the field will have a total $mv^2$ equal to zero, as all agree they must  (\in{table}[T:restingballs]).
The two cannon balls fired straight at each other will have tremendous total $mv^2$, as Leibniz thought they must due to their motion (\in{table}[T:firedballs]).

Leibniz viewed $mv^2$ as the true measure of motion and called it the \visviva, or \quote{living force.} He elevated conservation of \visviva\ to the level of a universal law, claiming that it was conserved not only in elastic collisions, as Huygens had shown, but in all interactions.
Leibniz first proposed the universal conservation law for $mv^2$ in 1686, the year before Newton published \booktitle{Principia}. Leibniz made \visviva\ one of the cornerstones of his comprehensive philosophy of the world, encompassing logic, physics, ethics, and even theology.

%Following the publication of Newton’s \booktitle{Principia},
Leibniz accepted  momentum as a conserved measure of motion and attempted to address the confusion among the Cartesians who mistakenly conflated Descartes’s quantity of motion and Newton’s momentum. %(described at the end of \in{Chapter}[ch:Momentum]). Since the term \quote{momentum} had not appeared, Leibniz referred to Newton’s momentum as the quantity of progress, distinguishing it from Descartes’s quantity of motion.
%His essay, published in the journal \booktitle{Acta Eruditorum}, opened with a title unlikely to win friends among the Cartesians: \quote{Brief Demonstration of a Notable Error of Descartes and Others Concerning a Natural Law, According to which God is Said Always to Conserve the Same Quantity of Motion; A Law Which They Also Misuse in Mechanics.}
Notice how Leibniz treats momentum as a directional quantity,  positive for motion in one direction and negative for motion in the other, just as we have done.
\startblockquote
I remark also another conservation, namely, that of the momentum, but neither is this the conservation of that which is absolute.
I call momentum the quantity of motion with which a body proceeds in a certain direction, so that if the body went in a contrary direction, this momentum would be a negative quantity.
Now if two or more bodies\dots
% are concurrent, we take the momentum from the direction whence proceeds their common centre of gravity, and if all these bodies
proceed from the same direction, then we must take the sum of the momentum of each for the total momentum; and it is plain that in this case the total momentum and the total quantity of motion of the bodies are the same thing. But if one of the bodies proceeded from a contrary direction, its momentum in the direction in question would be negative and consequently must be subtracted from the others in order to have the total momentum.
Thus if there are only two bodies, one of which proceeds in [a] direction\dots% of the common centre,
and the other in a contrary direction, from the quantity of motion of the first must be subtracted that of the second, and the remainder will be the total momentum. Now it will be found that the total momentum is conserved, or that there is as much momentum in the same direction before or after the impact. But it is also plain that this conservation does not correspond to that which is demanded of something absolute. For it may happen that the velocity, quantity of motion, and \visviva\ being very considerable, their momentum is null. This occurs when the two opposed bodies have their quantities of motion equal. In such case, according to the sense we have just given, there is no total momentum at all.\autocite{658. \quote{Quantity of progress} and \quote{progress} changed to \quote{momentum.} \quote{Force} changed to \textit{\quote{vis viva.}}}{Leibniz1691}
\stopblockquote

Leibniz did not suggest that momentum should be abandoned, but that the \visviva\ should also be considered. Let us look at the specific collision that Newton used as an example in his explanation of the conservation of momentum, shown in \in{figure}[fig:NewtonsExampleElastic]. In that one dimensional collision an object $A$ with a mass of $3$ and a velocity of $2$ was overtaken by a smaller object $B$ with a mass of $1$ and a speed of $10$. Newton found the total initial momentum was $6+10=16$. The total momentum was conserved in the collision, but Newton explained that this conservation could be achieved many different ways depending on how much momentum was transferred from ball $B$ to ball $A$ in the collision. Transferring $5$ of momentum would produce a final momentum of $15+1=16$. Transferring $8$ of momentum would produce a final momentum of $18-2=16$. Both of these outcomes, and many others, obey the law of conservation of momentum.

\startbuffer[TikZ:fig:NewtonsExampleElastic]
\environment env_physics
\environment env_TikZ
\setupbodyfont [libertinus,11pt]
\setoldstyle % Old style numerals in text
\startTEXpage\small
\starttikzpicture% tikz code
	\matrix{
	\startaxis[
		margin cart track,
		axis x line* = middle,
		xticklabels=\empty,
		xlabel={},
	   	extra x ticks={25,30},
	   	extra x tick labels=\empty,
   		extra x tick style={grid=major},
		ymin=-1.5, ymax = 25,
		footnotesize, %clip=false,
	]
	\fill [on layer={axis background}] (-0.5,0) rectangle (49.5,-1.5)[opacity=.1];
	\coordinate (B) at (25,249);
	\coordinate (A) at (30,249);
	\pic (first) at (B) [rotate=-4.68] {small pendulum=100};
	\pic (second) at (A) [rotate=-0.935] {pendulum=100};
	\path (6,6) node[above right]{$B$};
	\path (24,6) node[above left]{$A$};
    \stopaxis\\
	\startaxis[
		margin cart track,
		axis x line* = middle,
		xticklabels=\empty,
		xlabel={},
	   	extra x ticks={25,30},
	   	extra x tick labels=\empty,
   		extra x tick style={grid=major},
		ymin=-1.5, ymax = 10,
		footnotesize, %clip=false,
	]
	\fill [on layer={axis background}] (-0.5,0) rectangle (49.5,-1.5)[opacity=.1];
	\coordinate (B) at (25,249);
	\coordinate (A) at (30,249);
	\pic (third) at (B) {small pendulum=100}node[below right]{$B$};
	\pic (forth) at (A) {pendulum=100}node[below right]{$B$};
    \stopaxis\\
	\startaxis[
		margin cart track,
		axis x line* = middle,
	   	extra x ticks={25,30},
	   	extra x tick labels=\empty,
   		extra x tick style={grid=major},
		ymin=-1.5, ymax = 10,
		footnotesize, %clip=false,
	]
	\fill [on layer={axis background}] (-0.5,0) rectangle (49.5,-1.5)[opacity=.1];
	\coordinate (B) at (25,249);
	\coordinate (A) at (30,249);
	\pic (fifth) at (B) [rotate=-0.935] {small pendulum=100};
	\pic (sixth) at (A) [rotate=2.81] {pendulum=100};
    \stopaxis\\
	};
\stoptikzpicture
\stopTEXpage
\stopbuffer

\placefigure[margin][fig:NewtonsExampleElastic]
{In Newton’s example a small ball overtakes a larger ball, collides, and rebounds, transferring momentum to the larger ball.}
{\noindent\typesetbuffer[TikZ:fig:NewtonsExampleElastic]} % figure contents

\startbuffer[TikZ:PendulumCollisionGraph3]
\environment env_physics
\environment env_TikZ
\setupbodyfont [libertinus,11pt]
\setoldstyle % Old style numerals in text
\startTEXpage\small
\starttikzpicture% tikz code
	\startaxis[
		footnotesize,
		width=2.13in,%\marginparwidth,
		y={1mm},%x={2mm},
		xlabel={$t$ (s)},
		xmin=0, xmax=1.57,
		xtick={0,0.5,1.0,1.5},
		minor x tick num=4,
		ylabel={$x$ (cm)},
	  	%every axis y label/.style={at={(ticklabel cs:0.5)},rotate=90,anchor=center},
		ymin=0, ymax=50,
		minor y tick num=4,
	   	extra y ticks={25,30},
	   	extra y tick labels=\empty,
   		extra y tick style={grid=major},
		clip mode=individual,
		]
		\addplot[thick,smooth,domain=0:0.785,samples=101]{25-20*cos(2*deg(x))};
		\addplot[thick,smooth,domain=0:0.785,samples=101]{30-4*cos(2*deg(x))};
		\addplot[thick,smooth,domain=0.785:1.57,samples=101]{25+4*cos(2*deg(x))};
		\addplot[thick,smooth,domain=0.785:1.57,samples=101]{30-12*cos(2*deg(x))};
		\startscope[opacity=.3,transparency group]
		\fill[opacity=.3](0,5) circle[radius=.4mm];
		\draw[opacity=.3, shade, ball color = white] (0,5) circle[radius=2mm]node[above right=1mm]{$B$};
		\draw[opacity=.3, shade, ball color = white] (0,26) circle[radius=3mm]node[below right=1mm]{$A$};
		\fill[opacity=.3](0,26) circle[radius=.4mm];
%		\draw  [semithick](-.4,-4) -- (-.4,52);
%		\draw  [semithick](.4,-4) -- (.4,52);
		\draw[opacity=.3, shade, ball color = white] (0.785,25) circle[radius=2mm];
		\fill[opacity=.3](0.785,25) circle[radius=.4mm];
		\draw[opacity=.3, shade, ball color = white] (0.785,30) circle[radius=3mm];
		\fill[opacity=.3](0.785,30) circle[radius=.4mm];
%		\draw  [semithick](.6,-4) -- (.6,52);
%		\draw  [semithick](1.4,-4) -- (1.4,52);
		\draw[opacity=.3, shade, ball color = white] (1.57,21) circle[radius=2mm];
		\fill[opacity=.3](1.57,21) circle[radius=.4mm];
		\draw[opacity=.3, shade, ball color = white] (1.57,42) circle[radius=3mm];
		\fill[opacity=.3](1.57,42) circle[radius=.4mm];
%		\draw  [semithick](1.6,-4) -- (1.6,52);
%		\draw  [semithick](2.4,-4) -- (2.4,52);
		\stopscope
	\stopaxis
\stoptikzpicture
\stopTEXpage
\stopbuffer


\placefigure[margin][fig:PendulumCollisionGraph3] % location
{The position vs.~time graph for Newton’s collision.}	% caption text
{\noindent\typesetbuffer[TikZ:PendulumCollisionGraph3]} % figure contents

Perhaps Leibniz’s \visviva\ can help us make a more specific prediction for this collision. First, we should calculate the initial \visviva\ in Newton’s collision. We are actually going to do something that is logically equivalent. We will calculate the closely related kinetic energy.

\section{Kinetic energy}

Modern calculations do not use \visviva, but instead use \keyterm{kinetic energy}, which is half of the \visviva. The symbol for kinetic energy is $K$.
%Leibniz’s \visviva\ is correct, but many calculations will be easier if we use the \keyterm{kinetic energy}, represented by the symbol $K$, which is half of the \visviva.
%\highlightbox
The factor of one-half is of no consequence for Leibniz. If his \visviva\ quantifies the \quote{absolute} motion of the system, then one-half of the \visviva\ will work just as well. Since we will use kinetic energy later, it is best to get in the habit of including the factor of one-half.

\placetable[margin][T:NewtonCollisionBefore]
{The momentum and kinetic energy before Newton’s example collision.}
{\small
	\starttabulate[|l|c|c|c|]
		\FL[2]%\toprule
		\NC			\NC Momentum 	\NC Kinetic Energy	\NR
		\HL
		\NC Ball A		\NC $6$			\NC $6$		\NR
		\NC Ball B		\NC $10$			\NC $50$		\NR
		\HL
		\NC Total		\NC $16$			\NC $56$		\NR
		\LL[2]%\bottomrule
	\stoptabulate
}

In Newton’s collision example, ball $A$'s kinetic energy is
\startformula
K=\half mv^2 = \half\cdot3\cdot2^2 = 6,
\stopformula
and $B$'s is
\startformula
K=\half mv^2 = \half\cdot1\cdot10^2 = 50.
\stopformula
The total initial kinetic energy is $K\si=6+50=56$. The initial momenta, and kinetic energies for this collision are recorded in \in{table}[T:NewtonCollisionBefore].

Let us make a similar table showing the values after the collision. We will consider both of the outcomes mentioned above. First, consider the case with the larger momentum transfer, where ball $A$’s final momentum is $18$ and ball $B$’s is $-2$. Ball $A$’s final velocity is found from its momentum.
	\startformula\startmathalignment
		\NC p	\NC= mv \NR
		\NC v	\NC= \frac{p}{m} = \frac{18}{3}=6
	\stopmathalignment \stopformula
Ball $A$’s kinetic energy can then be found.
	\startformula
	K	= \half mv^2 = \half\cdot3\cdot6^2=54
	\stopformula
A similar calculation gives Ball $B$’s kinetic energy. Confirm the values in \in{table}[T:NewtonCollisionAfter4], which show that \emph{both} the momentum and the kinetic energy are conserved in this case. When ball $B$ transferred $8$ of momentum to ball $A$, it also transferred $48$ of kinetic energy, keeping both totals the same.


\placetable
    [margin]
    [T:NewtonCollisionAfter4]
    {The final momentum and kinetic energy in Newton’s example with a perfectly elastic bounce.}
    {\vskip9pt\small%\hbox{
	\starttabulate[|l|c|c|c|]
\FL[2]%\toprule
\NC			\NC Momentum 	\NC Kinetic Energy	\NR
\HL
\NC Ball A		\NC $18$				\NC $54$		\NR
\NC Ball B		\NC $-2$				\NC $2$		\NR
\HL
\NC Total		\NC $16$				\NC $56$		\NR
\LL[2]%\bottomrule
    \stoptabulate}


If the collision has a really good bounce, the kinetic energy after the collision will be the same as the kinetic energy before, and the collision is called \keyterm{elastic}.
Leibniz describes what happens during an elastic collision.
\placefigure[margin][fig:LeibnizCollision]{
	Leibniz describes how two colliding objects are deformed during a collision, from \booktitle{A Specimen of Dynamics}.\autocite{}{Newton1726}
}{\externalfigure[LeibnizCollision][width=144pt]}

\startblockquote
	If bodies A and B collide as in figure \in[fig:LeibnizCollision], and come from $\mathrm{A}_1$, and $\mathrm{B}_1$, to the place $\mathrm{A}_2$, $\mathrm{B}_2$, where they collide, they will, little by little, be compressed there, just like two inflated balls, and approach one another more and more, continually increasing the internal pressure. By that very circumstance the motion itself is weakened, the \visviva\ having been
	transformed into their elasticity, until they are altogether at rest. Then, finally, restoring themselves through their elasticity, they rebound from one another; having started a retrograde motion from rest, a motion that continually increases, in the end they move
	apart, having regained the same speed with which they originally approached one another, but directed oppositely, and they return to $\mathrm{A}_3$, and $\mathrm{B}_3$, which coincide with the places $\mathrm{A}_1$, and $\mathrm{B}_1$, if the bodies are assumed to be of the same size and the same velocity.\autocite{132. \quote{Figure 7} changed to \quote{\in{figure}[fig:LeibnizCollision].} \quote{Force} changed to \textit{\quote{vis viva.}}}{Leibniz1695}%
	%
	%	If the bodies A and B [figure \in[fig:LeibnizCollision]] meet and come from $A_1$, $B_1$ into the place of concourse $A_2$, $B_2$, that there they are
	%gradually compressed like two inflated balls, and more and more approach each other in turn by the continually increased pressure; that, moreover, by this thing the motion itself is weakened, the force itself of the effort being carried over into the elasticities (elastra) of the bodies, until at length they are reduced to rest; but then at length the elasticity of the bodies restoring them, they themselves rebound from each other in turn with a retrograde motion begun again from rest and continually increasing; at length with the same velocity with which they approached each other, regained but turned in the opposite direction, they recede in turn from each other and return into the positions $A_3$, $B_3$ which coincide with the positions $A_1$, $B_1$, if the bodies are supposed equal and of equal velocity.
	%	[New Understanding p. 686]
\stopblockquote
In Leibniz’s example, the speeds are the same after the collision as before, so the total kinetic energy also remains the same. In many elastic collisions each object does not retain the same velocity; one speeds up while the other slows down, as in Newton’s example. However, Huygens demonstrated the total $mv^2$ remains the same in all perfectly elastic collisions. Therefore, the total kinetic energy also remains the same.

However, most collisions are not perfectly elastic, and some kinetic energy is lost. In these \keyterm{inelastic} collisions the final total kinetic energy is less than the initial kinetic energy.
The effects of the inelastic collision can be seen by looking again at Newton’s example. In the case where ball $B$ only transfers $4$ of momentum to ball $A$, both balls are a little more sluggish after the collision due to the loss of kinetic energy, as shown in figures~\in[fig:NewtonsExampleInelastic] and \in[fig:PendulumCollisionGraph4]. (Compare to the bounces in fig.~\in[fig:NewtonsExampleElastic] and fig.~\in[fig:PendulumCollisionGraph3].) The kinetic energy lost in the collision can calculated by using the balls’ final momenta to find their velocities and then their kinetic energies. The intermediate step of finding the velocity is tedious, so it is worth finding the direct relationship between kinetic energy and momentum.
%To get the greatest benefit from these two ideas it is helpful to relate momentum and kinetic energy directly.
This is accomplished by replacing the velocity with momentum in the kinetic energy formula. Since $v= p/m$, this gives


\startbuffer[TikZ:NewtonsExampleInelastic]
\environment env_physics
\environment env_TikZ
\setupbodyfont [libertinus,11pt]
\setoldstyle % Old style numerals in text
\startTEXpage\small
\starttikzpicture% tikz code
	\matrix{
	\startaxis[
		margin cart track,
		axis x line* = middle,
		xticklabels=\empty,
		xlabel={},
	   	extra x ticks={25,30},
	   	extra x tick labels=\empty,
   		extra x tick style={grid=major},
		ymin=-1.5, ymax = 25,
		footnotesize, %clip=false,
	]
	\fill [on layer={axis background}] (-0.5,0) rectangle (49.5,-1.5)[opacity=.1];
	\coordinate (B) at (25,249);
	\coordinate (A) at (30,249);
	\pic (first) at (B) [rotate=-4.68] {small pendulum=100};
	\pic (second) at (A) [rotate=-0.935] {pendulum=100};
	\path (6,6) node[above right]{$B$};
	\path (24,6) node[above left]{$A$};
    \stopaxis\\
	\startaxis[
		margin cart track,
		axis x line* = middle,
		xticklabels=\empty,
		xlabel={},
	   	extra x ticks={25,30},
	   	extra x tick labels=\empty,
   		extra x tick style={grid=major},
		ymin=-1.5, ymax = 10,
		footnotesize, %clip=false,
	]
	\fill [on layer={axis background}] (-0.5,0) rectangle (49.5,-1.5)[opacity=.1];
	\coordinate (B) at (25,249);
	\coordinate (A) at (30,249);
	\pic (third) at (B) {small pendulum=100};
	\pic (forth) at (A) {pendulum=100};
    \stopaxis\\
	\startaxis[
		margin cart track,
		axis x line* = middle,
	   	extra x ticks={25,30},
	   	extra x tick labels=\empty,
   		extra x tick style={grid=major},
		ymin=-1.5, ymax = 10,
		footnotesize, %clip=false,
	]
	\fill [on layer={axis background}] (-0.5,0) rectangle (49.5,-1.5)[opacity=.1];
	\coordinate (B) at (25,249);
	\coordinate (A) at (30,249);
	\pic (fifth) at (B) [rotate=0.467] {small pendulum=100};
	\pic (sixth) at (A) [rotate=2.34] {pendulum=100};
    \stopaxis\\
	};
\stoptikzpicture
\stopTEXpage
\stopbuffer

\placefigure[margin][fig:NewtonsExampleInelastic]
{In Newton’s example a small ball overtakes a larger ball, collides, and rebounds, transferring less momentum to the larger ball than it did in the perfectly elastic collision.}
{\noindent\typesetbuffer[TikZ:NewtonsExampleInelastic]} % figure contents

\startbuffer[TikZ:PendulumCollisionGraph4]
\environment env_physics
\environment env_TikZ
\setupbodyfont [libertinus,11pt]
\setoldstyle % Old style numerals in text
\startTEXpage\small
\starttikzpicture% tikz code
	\startaxis[
		footnotesize,
		width=2.13in,%\marginparwidth,
		y={1mm},%x={2mm},
		xlabel={$t$ (s)},
		xmin=0, xmax=1.57,
		xtick={0,0.5,1.0,1.5},
		minor x tick num=4,
		ylabel={$x$ (cm)},
	  	%every axis y label/.style={at={(ticklabel cs:0.5)},rotate=90,anchor=center},
		ymin=0, ymax=50,
		minor y tick num=4,
	   	extra y ticks={25,30},
	   	extra y tick labels=\empty,
   		extra y tick style={grid=major},
		clip mode=individual,
		]
		\addplot[thick,smooth,domain=0:0.785,samples=101]{25-20*cos(2*deg(x))};
		\addplot[thick,smooth,domain=0:0.785,samples=101]{30-4*cos(2*deg(x))};
		\addplot[thick,smooth,domain=0.785:1.57,samples=101]{25-2*cos(2*deg(x))};
		\addplot[thick,smooth,domain=0.785:1.57,samples=101]{30-10*cos(2*deg(x))};
		\startscope[opacity=.3,transparency group]
		\fill[opacity=.3](0,5) circle[radius=.4mm];
		\draw[opacity=.3, shade, ball color = white] (0,5) circle[radius=2mm]node[above right=1mm]{$B$};
		\draw[opacity=.3, shade, ball color = white] (0,26) circle[radius=3mm]node[below right=1mm]{$A$};
		\fill[opacity=.3](0,26) circle[radius=.4mm];
%		\draw  [semithick](-.4,-4) -- (-.4,52);
%		\draw  [semithick](.4,-4) -- (.4,52);
		\draw[opacity=.3, shade, ball color = white] (0.785,25) circle[radius=2mm];
		\fill[opacity=.3](0.785,25) circle[radius=.4mm];
		\draw[opacity=.3, shade, ball color = white] (0.785,30) circle[radius=3mm];
		\fill[opacity=.3](0.785,30) circle[radius=.4mm];
%		\draw  [semithick](.6,-4) -- (.6,52);
%		\draw  [semithick](1.4,-4) -- (1.4,52);
		\draw[opacity=.3, shade, ball color = white] (1.57,27) circle[radius=2mm];
		\fill[opacity=.3](1.57,27) circle[radius=.4mm];
		\draw[opacity=.3, shade, ball color = white] (1.57,40) circle[radius=3mm];
		\fill[opacity=.3](1.57,40) circle[radius=.4mm];
%		\draw  [semithick](1.6,-4) -- (1.6,52);
%		\draw  [semithick](2.4,-4) -- (2.4,52);
		\stopscope
	\stopaxis
\stoptikzpicture
\stopTEXpage
\stopbuffer

\placefigure[margin][fig:PendulumCollisionGraph4] % location
{The position vs.~time graph for Newton’s less elastic collision.}	% caption text
{\noindent\typesetbuffer[TikZ:PendulumCollisionGraph4]} % figure contents

\startformula
	K = \half mv^2 = \half m \left(\frac{p}{m}\right)^2. %= \frac{p^2}{2m}
\stopformula
A little simplification gives the useful relationship
\highlightbox{
\startformula[eq:1DKp]
	K = \frac{p^2}{2m}. % = \half mv^2
\stopformula
}%\end{shaded}
%Momentum requires energy\index{energy}.
Because momentum and kinetic energy are central concepts in mechanics, this relationship is extremely useful. There are a couple of things to notice about this formula.
First, momentum and kinetic energy are closely related. If an object is not moving, then its momentum and kinetic energy are both zero. If it is moving, then neither is zero.
Second, momentum is directional, but kinetic energy is not. The momentum's sign indicates the momentum's direction. The kinetic energy is always positive, independent of the direction.
An object moving in the positive direction with a momentum of $10\units{kg\.m/s}$ has the same kinetic energy as it would moving the negative direction with a momentum of $-10\units{kg\.m/s}$.

\placetable
    [margin]
    [T:NewtonCollisionAfter1]
    {The final momentum and kinetic energy in Newton’s example with an imperfect bounce. Some kinetic energy is lost, but the momentum stays the same.}
    {\vskip9pt\small%\hbox{
	\starttabulate[|l|c|c|c|]
\FL[2]%\toprule
\NC			\NC Momentum		\NC Kinetic Energy	\NR
\HL
\NC Ball A		\NC $15$			\NC $37\onehalf$	\NR
\NC Ball B		\NC $1$			\NC $\onehalf$		\NR
\HL
\NC Total		\NC $16$			\NC $38$			\NR
\LL[2]%\bottomrule
    \stoptabulate}

Returning to Newton’s collision, $A$’s kinetic energy after the inelastic collision can be easily found with the new kinetic energy formula.
\startformula
	K = \frac{p^2}{2m} = \frac{15^2}{2\cdot3} = 37\onehalf
\stopformula
Confirm the remaining kinetic energies in \in{table}[T:NewtonCollisionAfter1]. After the inelastic collision the total kinetic energy is only 38, less than the 56 that the balls had before the collision. %Some of the energy has been \quotation{absorbed by the small parts which compose the mass} according to Leibniz.
\Visviva, which is twice the kinetic energy, is also lost, dropping from 112 before the inelastic collision to only 76 after.
However, notice that none of the momentum is lost. In an inelastic collision less momentum is transferred, but the total momentum stays the same. Inelastic collisions lose kinetic energy (and \visviva) but not momentum!

The loss of \visviva\ in inelastic collisions posed a mortal threat to Leibniz’s conservation of \visviva. Leibniz knew that Descartes’s quantity of motion was abandoned due to similar experimental failures. Like Descartes, Leibniz defended his theory anyway, arguing that \visviva\ is conserved even in experiments where it appeares to be lost. Leibniz did not invoke the unseen substances of Descartes, but instead sought to save \visviva within the objects themselves.
\startblockquote
	%But it is necessary to admit, that although bodies must be thus naturally elastic in the sense which I have just explained, nevertheless
	The elasticity often appears insufficient in the masses or bodies which we employ....
	%, even if these masses should be composed of elastic parts and should resemble a sack full of hard balls which would yield to a moderate impact, without leaving the sack, as we see in the case of soft bodies or those which yield without recovering themselves sufficiently.
	The reason is that the parts are not sufficiently united therein to transfer their change to the whole. Whence it comes that in the impact of such bodies a part of the \visviva\ is absorbed by the small parts which compose the mass.... %, without this force being given to the whole; and this must always happen when the pressed mass does not recover perfectly; although it also happens that a mass shows itself more or less elastic according to the different manner of the impact, witness the water itself which yields to a moderate impression, and makes a cannon-ball rebound.[New Understanding p. 669]
	Now when the parts of the bodies absorb the \visviva\ of the impact, as a whole, as when two pieces of rich earth or of clay come into collision, or in part, as when two wooden balls meet, which are much less elastic than two globes of jasper or tempered steel...%; when, I say, some \visviva\ is absorbed by the parts, it is as good as lost for the absolute force, and for the respective velocity, that is to say, for the third and the first equation, which do not succeed, since
	that which remains after the impact has become less than what it was before the impact, by reason of a part of the \visviva\ being turned else where.\autocite{669--670}{Leibniz1691}
\stopblockquote

Leibniz explanation is mostly correct. When the balls collide some of the balls’ \visviva\ is transferred to the atoms and molecules within each ball. These microscopic parts jiggle constantly, and their random motion caries considerable \visviva. The collision increases this microscopic random motion, increasing the internal \visviva.


Since some of the \visviva\ is transferred to microscopic motion, the visible \visviva\ will
%be evidently conserved in elastic collisions, but will
appear to decrease in inelastic collisions.
\startblockquote
	%But the momentum...%, or rather the second equation, is not concerned therein. And even the motion of this total progress
	%remains... %alone
%	When the two bodies proceed together after the impact with the velocity of their common centre, as do two balls of rich earth or clay. But in the semi-elastics, as two wooden balls, it happens still further that the bodies mutually depart after the impact, although with a weakening of the first equation, following this force of the impact which has not been absorbed. And in consequence of certain experiments touching the degree of the elasticity of this wood, we might predict what should happen to the balls which should be made of it in every kind of collision or impact.
%
	But this loss of the total \visviva...%, or this failure of the third equation,
	does not detract from the inviolable truth of the law of the conservation of the same \visviva\ in the world. For that which is absorbed by the minute parts is not absolutely lost for the universe....%, although it is lost for the total \visviva\ of the concurrent bodies.
	\autocite{670. \quote{Force} replaced by \quote{\visviva.}}{Leibniz1691}
\stopblockquote

While microscopic random motion can hide a large amount of \visviva, this motion cannot hide any momentum. %The microscopic motion is random and disorganized.
The random motion, because it is random, occurs equally in all directions, so its total momentum is zero.
%The total momentum of the parts is not exactly zero.
The total momentum -- due only to the atoms shared, nonrandom motion -- is visible in the ball’s motion. Since no momentum can be hidden in the microscopic parts \quote{the total momentum...remains,} even in inelastic collisions.\autocite{670. \quote{Progress} replaced by \quote{momentum.}}{Leibniz1691}

%%%%%%%%%%%%%%%%%%%%%%%%%%%%%%%%%%%%%%%%%%%%%%%%%%%

%%%%%%%%%%%%%%%%%%%%%%%%%%%%%%%%%%%%%%%%%%%%%%%%%%%
\startexample[ex:NewtonTotalInelastic]
	In Newton’s collision example, one possible result is ball $B$ sticking to ball $A$ when they collide. In this case, what is their final kinetic energy and how much kinetic energy is lost?

\startsolution
After the collision, since ball $B$ sticks to ball $A$, they can be treated as a single object with a total mass $M = m_A + m_B =4$ and momentum $p=16$. (See \at{p.}[NewtonInelasticSpeed] for the discussion of momentum in this collision. \in{Figure}[fig:NewtonTotalInelastic] shows this collision performed with pendulums. The position vs.\ time graph is shown in \in{figure} [fig:PendulumCollisionGraph5].)

The final kinetic energy can be calculated from either the final velocity or the final momentum. We use the final momentum.
\startformula
	K\sf	= \frac{p\sf^2}{2M}
		= \frac{16^2}{2\cdot4}
		= 32
\stopformula
The final kinetic energy is much less than the initial kinetic energy of 56. The kinetic energy lost is $K\si - K\sf = 56 - 32 = 24$. According to Leibniz, this kinetic energy is not lost to the universe but has gone into the balls’ constituent parts.
\stopsolution


\startbuffer[TikZ:NewtonTotalInelastic]
\environment env_physics
\environment env_TikZ
\setupbodyfont [libertinus,11pt]
\setoldstyle % Old style numerals in text
\startTEXpage\small
\starttikzpicture% tikz code
	\matrix{
	\startaxis[
		margin cart track,
		axis x line* = middle,
		xticklabels=\empty,
		xlabel={},
	   	extra x ticks={25,30},
	   	extra x tick labels=\empty,
   		extra x tick style={grid=major},
		ymin=-1.5, ymax = 25,
		footnotesize, %clip=false,
	]
	\fill [on layer={axis background}] (-0.5,0) rectangle (49.5,-1.5)[opacity=.1];
	\coordinate (B) at (25,249);
	\coordinate (A) at (30,249);
	\pic (first) at (B) [rotate=-4.68] {small pendulum=100};
	\pic (second) at (A) [rotate=-0.935] {pendulum=100};
	\path (6,6) node[above right]{$B$};
	\path (24,6) node[above left]{$A$};
    \stopaxis\\
	\startaxis[
		margin cart track,
		axis x line* = middle,
		xticklabels=\empty,
		xlabel={},
	   	extra x ticks={25,30},
	   	extra x tick labels=\empty,
   		extra x tick style={grid=major},
		ymin=-1.5, ymax = 10,
		footnotesize, %clip=false,
	]
	\fill [on layer={axis background}] (-0.5,0) rectangle (49.5,-1.5)[opacity=.1];
	\coordinate (B) at (25,249);
	\coordinate (A) at (30,249);
	\pic (third) at (B) {small pendulum=100}node[below right]{$B$};
	\pic (forth) at (A) {pendulum=100}node[below right]{$B$};
    \stopaxis\\
	\startaxis[
		margin cart track,
		axis x line* = middle,
	   	extra x ticks={25,30},
	   	extra x tick labels=\empty,
   		extra x tick style={grid=major},
		ymin=-1.5, ymax = 10,
		footnotesize, %clip=false,
	]
	\fill [on layer={axis background}] (-0.5,0) rectangle (49.5,-1.5)[opacity=.1];
	\coordinate (B) at (25,249);
	\coordinate (A) at (30,249);
	\pic (fifth) at (B) [rotate=1.84] {small pendulum=100};
	\pic (sixth) at (A) [rotate=1.84] {pendulum=100};
    \stopaxis\\
	};
\stoptikzpicture
\stopTEXpage
\stopbuffer

\placefigure[margin][fig:NewtonTotalInelastic]
{In Newton’s example of a totally inelastic collision, a small ball overtakes a larger ball, collides, and stick. After the collision they move together as a single object.}
{\noindent\typesetbuffer[TikZ:NewtonTotalInelastic]} % figure contents

\startbuffer[TikZ:PendulumCollisionGraph5]
\environment env_physics
\environment env_TikZ
\setupbodyfont [libertinus,11pt]
\setoldstyle % Old style numerals in text
\startTEXpage\small
\starttikzpicture% tikz code
	\startaxis[
		footnotesize,
		width=2.13in,%\marginparwidth,
		y={1mm},%x={2mm},
		xlabel={$t$ (s)},
		xmin=0, xmax=1.57,
		xtick={0,0.5,1.0,1.5},
		minor x tick num=4,
		ylabel={$x$ (cm)},
	  	%every axis y label/.style={at={(ticklabel cs:0.5)},rotate=90,anchor=center},
		ymin=0, ymax=50,
		minor y tick num=4,
	   	extra y ticks={25,30},
	   	extra y tick labels=\empty,
   		extra y tick style={grid=major},
		clip mode=individual,
		]
		\addplot[thick,smooth,domain=0:0.785,samples=101]{25-20*cos(2*deg(x))};
		\addplot[thick,smooth,domain=0:0.785,samples=101]{30-4*cos(2*deg(x))};
		\addplot[thick,smooth,domain=0.785:1.57,samples=101]{25-8*cos(2*deg(x))};
		\addplot[thick,smooth,domain=0.785:1.57,samples=101]{30-8*cos(2*deg(x))};
		\startscope[opacity=.3,transparency group]
		\fill[opacity=.3](0,5) circle[radius=.4mm];
		\draw[opacity=.3, shade, ball color = white] (0,5) circle[radius=2mm]node[above right=1mm]{$B$};
		\draw[opacity=.3, shade, ball color = white] (0,26) circle[radius=3mm]node[below right=1mm]{$A$};
		\fill[opacity=.3](0,26) circle[radius=.4mm];
%		\draw  [semithick](-.4,-4) -- (-.4,52);
%		\draw  [semithick](.4,-4) -- (.4,52);
		\draw[opacity=.3, shade, ball color = white] (0.785,25) circle[radius=2mm];
		\fill[opacity=.3](0.785,25) circle[radius=.4mm];
		\draw[opacity=.3, shade, ball color = white] (0.785,30) circle[radius=3mm];
		\fill[opacity=.3](0.785,30) circle[radius=.4mm];
%		\draw  [semithick](.6,-4) -- (.6,52);
%		\draw  [semithick](1.4,-4) -- (1.4,52);
		\draw[opacity=.3, shade, ball color = white] (1.57,33) circle[radius=2mm];
		\fill[opacity=.3](1.57,33) circle[radius=.4mm];
		\draw[opacity=.3, shade, ball color = white] (1.57,38) circle[radius=3mm];
		\fill[opacity=.3](1.57,38) circle[radius=.4mm];
%		\draw  [semithick](1.6,-4) -- (1.6,52);
%		\draw  [semithick](2.4,-4) -- (2.4,52);
		\stopscope
	\stopaxis
\stoptikzpicture
\stopTEXpage
\stopbuffer


\placefigure[margin][fig:PendulumCollisionGraph5] % location
{The position vs.~time graph for Newton’s totally inelastic collision.}	% caption text
{\noindent\typesetbuffer[TikZ:PendulumCollisionGraph5]} % figure contents

\stopexample

%%%%%%%%%%%%%%%%%%%%%%%%%%%%%%%%%%%%%%%%%%%%%%%%%%%

Leibniz recognized the truth of Newton’s momentum formula, even if he preferred his \visviva\ as an absolute measure of motion. His criticism of Descartes’s quantity of motion provoked a forceful defense from the Cartesians, who mistakenly pointed to Newton’s successful momentum as vindication. Over the next decade the \visviva\ debate became entangled with a bitter priority dispute between Newton and Leibniz over who had discovered calculus. Newton’s followers frequently viewed these debates as a question of loyalty rather than science.

Some scientists did not take sides purely out of loyalty, but instead approached the question with rigorous calculations and careful experiments, as scientists should. The famous Swiss mathematician Johann Bernoulli was persuaded to switch camps by Leibniz’s mathematical arguments. Bernoulli became a dedicated and vocal advocate for Leibniz and \visviva.

\placefigure[margin][fig:GravesandeDrop]
	{’s Gravesande dropped spheres into soft clay from various heights to measure their \visviva. This digram appears in his 1740 textbook.\autocite{plate 32}{Gravesande1747}}
	{\externalfigure[GravesandeDrop][width=144pt]}
\placefigure[margin][fig:GravesandeDropSpheres]
	{’s Gravesande used three spheres with the same radius for the above experiment. Two of them were hollow inside so that they had less mass than the solid one. Their masses were in the ratio 3:2:1.\autocite{plate 32}{Gravesande1747}}
	{\externalfigure[GravesandeDropSpheres][width=144pt]}

The Dutch scientist Willem ’s Gravesande designed experiments to compare impacts of different objects traveling at different speeds. %]
He did this first by dropping balls into clay. He worked with a set of balls identical in size but different in mass, shown in figures \in[fig:GravesandeDrop] and \in[fig:GravesandeDropSpheres]. When any of these balls was dropped onto the clay it would land with an inelastic thud.
’s Gravesande measured the size of the dent left in the clay to determine the impact’s effect. If a light, fast ball left the same size dent as a heavy, slow ball, then the two balls had produced the same effect. He thought the effect would be determined by the ball’s momentum $mv$, so that if the heavy ball’s mass is double the light ball’s mass, then the light ball’s velocity would have to be double the heavy ball’s velocity to make the same size dent. If the heavy ball was four times the mass of the light ball, then the light ball would need to be traveling four times as fast to produce the same effect.

When ’s Gravesande performed this experiment, the results surprised him. His experiments showed that if the heavy ball’s mass was four times the lighter ball’s mass, then the lighter ball would need to move only twice as fast to produce the same dent. This is  predicted by Leibniz’s formula $mv^2$. After confirming this for many different ratios of mass, ’s Gravesande concluded that he had been mistaken. He adopted \visviva\ as a measure of motion, and enthusiastically shared the discovery. His contribution arrived as the \visviva\ debate was at its peak in the 1720s.

%\startblockquote
%	The 1720s saw the height of the argument in the English, French, and Dutch learned journals. In 1722, two articles in the \booktitle{Journal litt\'eraire} described Willem ’s Gravesande’s version of an experiment with copper and ceramic balls---bodies of different masses---falling into a clay-filled tray. He compared the depression they made as a way of quantifying the \quote{force} of one body colliding with another and confirming Leibniz’s concept.\autocite{185}{Zinsser2006}
%\stopblockquote

While ’s Gravesande’s evidence won a few converts, many scientists remained loyal to the Cartesian quantity of motion, largely out of devotion to Newton. In 1728, French scientist Dortous de Mairan made a case for the Cartesian formula in his discourses on the \quote{Estimation and the Measure of Motor Forces of Bodies.} His forceful argument convinced most scientists on the continent that \visviva\ was a failed idea.
%\begin{marginfigure}%[2.5in]
%	\marginfig{GravesandeDropCraters}
%	\caption[Colliding Elastic Bodies]{Leibniz describes how two colliding objects are deformed during a collision, from \booktitle{A Specimen of Dynamics}.\autocite{}{Newton1726}}
%	\label{fig:GravesandeDropCraters}%
%\end{marginfigure}

\section{Work}
Engineers in the eighteenth century were not reading academic papers about \visviva, they were building increasingly powerful and complex machines – water wheels, windmills, pumps – to perform important, difficult tasks like grinding grain and lifting water out of mines. Engineers sought to quantify each machine's usefulness in order to answer economic questions: If a small team of horses could run a mill grinding grain, could a windmill  or a water wheel achieve the same result at a lower cost? If a water wheel in town could grind a certain amount of grain, how much water could be pumped out of a mine in the hills using a similar water wheel there? These questions are answered by determining the \keyterm{work} done by the machine.

I am sure you have done quite a bit of physical work and have some sense of what physical work is. 
Lifting a heavy box or pushing a heavy cart requires work. In all cases work requires a force $F$ exerted over a displacement $\Delta x$. The resulting amount of work $W$ is simply the product of the force and the displacement.
\startformula
	W = F\Delta x
\stopformula
%
%The argument for the work formula is straight forward. Sliding the refrigerator two meters requires twice as much work as sliding it one meter, because first you slide it one meter, and then you slide it one meter again, doing the same amount of work twice. Therefore, the work is proportional to the distance.
%Sliding two refrigerators one meter requires twice as much work as sliding a single refrigerator one meter. If you slide the refrigerators one at a time, the work is twice as much because you do the work of sliding one refrigerator twice. If you push them one meter at the \emph{same} time, then you must push with twice as much force. Pushing with this double force accomplishes double the work in only one meter. Therefore, the work is proportional to the force. We conclude, $W = F\Delta x$. This argument was made many times by engineers and scientists during the \visviva\ debate, although I am not sure who argued it first.
%During the \visviva\ debate, this was called effort, or the effect of the \visviva. We will use the current term: work.

Since work is force times displacement, the \scaps{si} units for work are $\unit{N\.m}$. These units are so common that they have been given their own name – \keyterm{joules}. The symbol for a joule is J (not to be confused with the symbol for impulse, $J$ ). % and it is related to the other standard SI units by the definition
\startformula
	1\units{J}=1\units{N\.m}
\stopformula
The following example shows how work is calculated, including the unit conversion to joules.

%%%%%%%%%%%%%%%%%%%%%%%%%%%%%%%%%%%%%%%%%%%%%%%%%%%
\startexample[ex:HorseW]
James Watt sold powerful engines across England in the late eighteenth century. For his marketing materials, he wanted to compare the work done by his engines to the work done by a horse. In a 1782 experiment, he found that in one hour a horse pulling with $800\units{N}$ of force could travel $3.3\units{km}$. How much work did the horse do in an hour?
%\startbuffer[TikZ:BlockW]
%\environment env_physics
%\environment env_TikZ
%\setupbodyfont [libertinus,11pt]
%\setoldstyle % Old style numerals in text
%\startTEXpage\small
%\starttikzpicture% tikz code
%	\fill [black!10] (0,0) rectangle (4.9,-.15);
%	\startaxis[
%		margin cart track,
%		ymax=10,
%	]
%		\pic at (10,0) {block}[opacity=.4];
%		\pic at (26,0) {block};
%   		\draw[-{Straight Barb}, thick] (10,2.5) --node[pos=.5,above=2.5mm]{$\Delta x=16.0\units{cm}$}(26,2.5);
%    \stopaxis
%\stoptikzpicture
%\stopTEXpage
%\stopbuffer
%
%\placefigure[margin][fig:BlockW] % location
%{A block slides on a track, slowing to a stop in example~\in[ex:BlockW].}	% caption text
%{\noindent\typesetbuffer[TikZ:BlockW]} % figure contents

\startsolution
The work done by the horse is found using the work formula.
%	\startformula\startmathalignment
%		\NC W	\NC = F\Delta x \NR
%		\NC		\NC = (0.30\units{N})(16.0\units{cm})	\NR
%%		\NC		\NC = (0.30\units{N})(0.16\units{m})	\NR
%		\NC		\NC = 0.048\units{N\.m}			\NR
%		\NC		\NC = \answer{0.048\units{J}}
%	\stopmathalignment \stopformula
\startformula
	W  = F\Delta x
		= (800\units{N})(3300\units{m})
		= 2\,640\,000\units{N\.m}
		= \answer{2.64\sci{6}\units{J}}
\stopformula
	In one hour, the horse did $2.64\sci{6}\units{J}$ of work.
\stopsolution
\stopexample
%%%%%%%%%%%%%%%%%%%%%%%%%%%%%%%%%%%%%%%%%%%%%%%%%%%

%%%%%%%%%%%%%%%%%%%%%%%%%%%%%%%%%%%%%%%%%%%%%%%%%%%%
%\startexample[ex:BlockW]
%	The block slides $16.0\units{cm}$ along the track against a friction force of $-0.30\units{N}$ (The force is negative, opposing the block’s positive motion). What is the work done by the block?
%\startbuffer[TikZ:BlockW]
%\environment env_physics
%\environment env_TikZ
%\setupbodyfont [libertinus,11pt]
%\setoldstyle % Old style numerals in text
%\startTEXpage\small
%\starttikzpicture% tikz code
%	\fill [black!10] (0,0) rectangle (4.9,-.15);
%	\startaxis[
%		margin cart track,
%		ymax=10,
%	]
%		\pic at (10,0) {block}[opacity=.4];
%		\pic at (26,0) {block};
%   		\draw[-{Straight Barb}, thick] (10,2.5) --node[pos=.5,above=2.5mm]{$\Delta x=16.0\units{cm}$}(26,2.5);
%    \stopaxis
%\stoptikzpicture
%\stopTEXpage
%\stopbuffer
%
%\placefigure[margin][fig:BlockW] % location
%{A block slides on a track, slowing to a stop in example~\in[ex:BlockW].}	% caption text
%{\noindent\typesetbuffer[TikZ:BlockW]} % figure contents
%
%\startsolution
%The work done by the block is found using the work formula, but take care to use the correct force between the block and the track. The work done by the block is calculated using the force \emph{of} the block, but we are given the force \emph{on} the block. Newton’s third law tells us that the force of the block on the track is equal and opposite to the force of the track on the block. Therefore, we will calculate the work using $F=0.30\units{N}$ for the force of the block on the track.
%%	\startformula\startmathalignment
%%		\NC W	\NC = F\Delta x \NR
%%		\NC		\NC = (0.30\units{N})(16.0\units{cm})	\NR
%%%		\NC		\NC = (0.30\units{N})(0.16\units{m})	\NR
%%		\NC		\NC = 0.048\units{N\.m}			\NR
%%		\NC		\NC = \answer{0.048\units{J}}
%%	\stopmathalignment \stopformula
%\startformula
%	W  = F\Delta x
%		= (0.30\units{N})(0.16\units{m})
%		= 0.048\units{N\.m}
%		= \answer{0.048\units{J}}
%\stopformula
%	The work done by the sliding block is $0.048\units{J}$ or $4.8\sci{-2}\units{J}$.
%\stopsolution
%\stopexample
%%%%%%%%%%%%%%%%%%%%%%%%%%%%%%%%%%%%%%%%%%%%%%%%%%%%

In the example above we used positive values for both the force $F$ and the displacement $\Delta x$. These are correct if the horse pulls in the positive direction. If the horse instead pulls in the negative direction, then the force $F$ would be negative. The horse's displacement $\Delta x$ would also be negative. The work done by the horse would be
\startformula
	W  = F\Delta x
		= (-800\units{N})(-3300\units{m})
		= \answer{2.64\sci{6}\units{J}},
\stopformula
exactly the same positive work done by the horse in the example. No matter which way the horse pulls, he does positive work! 

It is possible to do negative work. When the force and the displacement are in opposite directions – one is positive and the other is negative – the work formula correctly gives a negative amount of work, as we will soon see.


\section{Conservation of energy}

Engineers’ work and physicists’ \visviva\ were first linked by Daniel Bernoulli, son of the Swiss \visviva\ advocate Johann Bernoulli, mentioned earlier. D.~Bernoulli had undertaken a detailed study of fluid motion, and in the process he recognized that \visviva\ (or kinetic energy) is not the only type of energy. I am not going to list the other types of energy now – D.~Bernoulli will assist us with that in the next chapter. Here it is sufficient to know that there are other types of energy and only the total energy is conserved. D. Bernoulli used his conservation of energy principal to great effect in his groundbreaking  \booktitle{Hydrodynamica}, published in 1738.

The symbol for total energy is $H$. %, honoring William Rowan Hamilton, who we will meet in \in{Chapter}[ch:1Ddynamics].
The conservation law for total energy $H$ looks very similar to the conservation law for momentum.
\startformula[eq:1DEconserve]
	H\si + W + Q = H\sf %\sub{net} + Q\sub{net}
\stopformula
Energy can be added or removed from a system through work $W$ done by outside forces.
A person pushing a heavy cart does work, increasing the cart’s kinetic energy.

Energy can also be added as heat $Q$ by cooking the cart over a fire, which increases the random, microscopic energy of the cart’s atoms and molecules. D. Bernoulli had some deep insights into how $Q$ related to the motion of the microscopic parts, but he could not calculate $Q$. The correct measurement of heat became one of the greatest challenges for physicists during the nineteenth century. We will return to this topic in Volume II. Until then, we will faithfully include $Q$ when writing the conservation of energy equation, but immediately cross it out because the heat transfer $Q$ is zero in mechanics problems. In mechanics, any change in a system’s total energy is due to outside forces doing work on the system.

%Leibniz’s did not know that the balls’ temperature is a measure of the microscopic random motion – higher temperature indicating greater microscopic \visviva. The loss of the balls’ \visviva\ to the microscopic parts causes an increase in temperature, but the increase is extremely small, far too small for seventeenth century experimenters to measure. The idea that the temperature is connected to microscopic \visviva\ appeared several times in the decades that followed, but was not widely accepted until the middle of the nineteenth century. We will return to this idea in Volume II.

With any conservation law we must take care to define the system. With momentum conservation, we could chose which objects to include in the system. For example, in pendulum collisions we typically included both balls in our system, adding their momenta to find the system's total momentum. With conservation of energy we chose both the objects and the types of energy we wish to include in the system. In the pendulum examples we included the balls' kinetic energy, but not the energy of microscopic random motion. As we study more types of energy, we will carefully chose which types to include in the system. For the remainder of this chapter, only kinetic energy $K$ will be included in the system's total energy $H$.

%%%%%%%%%%%%%%%%%%%%%%%%%%%%%%%%%%%%%%%%%%%%%%%%%%%
\startexample[ex:CartWork1]
You need to push a $75\units{kg}$ cart full of books down a straight, level hallway. The cart's large wheels roll easily on the smooth, hard floor, so almost no force is required to push the cart once it is moving. Considerable force is required to get the cart moving due to the large mass of books. You apply $27\units{N}$ of force to get the cart moving, and continue to apply $27\units{N}$ for the first $2.0\units{m}$ of movement. What is the moving cart's new kinetic energy and speed?
\startsolution
The conservation of energy equation is always a safe way to start. We will use conservation of energy to find the change in the cart's kinetic energy during the first $2\units{m}$, when you are doing work by pushing the cart.
\startformula
	H\si + W + \cancel{Q} = H\sf
\stopformula
The total energy $H$ is the cart's kinetic energy $K$. Since the cart is not moving initially, the initial kinetic energy is zero, and $H\si = K\si = 0\units{J}$.
\startformula\startmathalignment
	\NC	\cancel{H\si} + W	 \NC = K\sf	\NR
	\NC	 F\Delta x			 \NC = K\sf	\NR
\stopmathalignment \stopformula
The cart's final kinetic energy is equal to the work you do pushing the cart for the first $2\units{m}$.
\startformula
	K\sf	= F\Delta x
		= (27\units{N})(2.0\units{m})
		= 54\units{J}
\stopformula
The cart's speed can be found from the kinetic energy formula.
\startformula\startmathalignment
	\NC K		\NC = \half mv^2	\NR
	\NC v		\NC = \sqrt{\frac{2K}{m}}
		 = \sqrt{\frac{2(54\units{J})}{75\units{kg}}}
		 = \sqrt{\frac{2(54\units{N\.m})}{75\units{kg}}}	\NR
\stopmathalignment\stopformula
The units are tricky. I made the substitution $1\units{J} = 1\units{N\.m}$ above, but I also need to substitute $1\units{N} = 1\units{kg\.m/s^2}$. (See p.~\at[eq:NetwonUnit].) Then the units simplify correctly.
\startformula
	v = \sqrt{\frac{2(54\units{kg\.m^2/s^2})}{75\units{kg}}}
		= \sqrt{1.44\units{m^2/s^2}}
		= 1.2\units{m/s}
\stopformula
Your work gives the cart a speed of $1.2\units{m/s}$.
\stopsolution\stopexample

The example above required a new unit conversion for the cart's kinetic energy.
\startformula
	K\sf = 54\units{J} = 54\units{N\.m} = 54\units{(kg\.m/s^2)\.m} = 54\units{kg\.m^2/s^2}
\stopformula
Here we see the useful energy unit conversions:
\startformula
	1\units{J} = 1\units{N\.m} = 1\units{kg\.m^2/s^2}
\stopformula
%Joules are the usual {\sc si} units for energy, while $\units{N\.m}$ are the units one expects from the work formula $W=F\Delta x$ and $\units{kg\.m^2/s^2}$ are the units one expects from the kinetic energy formula $K=\onehalf mv^2$.

This is an appropriate place for me to make a frank confession: I have a terrible memory. I cannot remember these energy unit conversions. However, I do remember the work formula and the kinetic energy formula. They are short, and I have used each about one million times. Whenever I need energy unit conversions, I just remember the formulas $W=F\Delta x$ and $K=\onehalf mv^2$. The left sides have the energy unit $\units{J}$ while the right sides have the units $\units{N\.m}$ and $\units{kg\.m^2/s^2}$, respectively. When I am working with non-{\sc si} units, as I often do, these tricks give me the conversions in whatever crazy units I am using. Tricks like this made physics attractive to me at an early age! (A professor at The University of Chicago, where I earned my Ph.D., observed that I  passed the physics department's grueling candidacy exam without actually knowing much of anything. I took this as a vindication of my figure-it-out survival strategy.) 

%Energy will come in many different forms, with many different formulas, but the units can always be converted to joules.

The rolling cart of the previous example, which you have given $54\units{J}$ of kinetic energy, will continue to roll at $1.2\units{m/s}$ until it is acted on by another force. Let us say that your destination is $8\units{m}$ from where you originally started pushing the cart. You will need to start slowing the cart some distance before reaching the destination. You will do this by pulling backward on the cart, opposing its forward motion. As the cart slows your force and the displacement will be in opposite directions. You will do negative work on the cart, reducing its kinetic energy to zero, as described in the next example.


%%%%%%%%%%%%%%%%%%%%%%%%%%%%%%%%%%%%%%%%%%%%%%%%%%%
\startexample[ex:CartWork2]
The $75\units{kg}$ cart from \in{example}[ex:CartWork1] is coasting along the hallway with a $54\units{J}$ of energy. As you approach your destination, you begin pulling backward on the cart to slow it. You pull with a backward force of $18\units{N}$. How far does the cart roll before coming to a stop?

\startsolution
We again use conservation of energy, this time as you slow the cart to a stop ($H\sf = K\sf = 0\units{J}$).
\startformula\startmathalignment
	\NC H\si + W + \cancel{Q}	\NC = \cancel{H\sf}	\NR
	\NC	K\si + F\Delta x	 		\NC = 0	\NR
	\NC	 F\Delta x 				\NC = -K\si	\NR
\stopmathalignment \stopformula
The cart's initial kinetic energy is positive, so the work must be negative. The force and the displacement must be in opposite directions. We will call the direction of motion positive. The force, which is opposing the motion, will be negative, $F = -18\units{N}$. Solving for $\Delta x$,
\startformula
	\Delta x = \frac{-K\si}{F}
		= \frac{-54\units{J}}{-18\units{N}}
		= \frac{-54\units{N\.m}}{-18\units{N}}
		= 3.0\units{m}
\stopformula
The cart rolls $3\units{m}$ before coming to a complete stop.
\stopsolution\stopexample


%%%%%%%%%%%%%%%%%%%%%%%%%%%%%%%%%%%%%%%%%%%%%%%%%%%%
%\startexample[ex:BlockSlide]
%The cart and block, together weighing $550\units{g}$, slide in the positive direction against a force of friction equal to $-0.40\units{N}$. Initially, they have a velocity of $2.5\units{m/s}$. How far will they slide before coming to rest?
%\startbuffer[TikZ:BlockSlide]
%\environment env_physics
%\environment env_TikZ
%\setupbodyfont [libertinus,11pt]
%\setoldstyle % Old style numerals in text
%\startTEXpage\small
%\starttikzpicture% tikz code
%	\fill [black!10] (0,0) rectangle (4.9,-.15);
%	\startaxis[
%		margin cart track,
%		ymax=10,
%		legend style={draw=none, at={(1,1)}, anchor=south east},%yshift = 1ex},
%	]
%	%\pic at (40,0) {cart};
%	\pic at (10,0) {cart};
%	\pic at (19,0) {block};
%	\draw[->, thick] (10,2.5) --node[above=2.5mm]{$v=2.5\units{m/s}$} (35,2.5);
%    %\draw[->, thick] (16,7.5) -- (20,7.5);% node[above=2.5mm]{$p=mv$}
%		% Legend
%   		\addlegendimage{
%	  	legend image code/.code={\draw[thick,|-|](-0.5cm,0cm)--(0cm,0cm);}
% 		};
%		\addlegendentry{$=50\units{cm/s}$}
%    \stopaxis
%\stoptikzpicture
%\stopTEXpage
%\stopbuffer
%
%\placefigure[margin][fig:BlockSlide] % location
%{A cart pushes a block on a track, slowing to a stop in example~\in[ex:BlockSlide].}	% caption text
%{\noindent\typesetbuffer[TikZ:BlockSlide]} % figure contents
%\startsolution
%	The conservation of energy equation is always a safe way to start.
%	\startformula[eq:1DEconserve]
%		H\si + W + \cancel{Q} = H\sf
%	\stopformula
%	The only contribution to the initial total energy $H\si$ is the cart and block’s kinetic energy $K\si = \half mv^2$, where $m$ is the total mass of the cart and block.
%\startformula
%	H\si	= K\si
%		= \half mv^2
%		= \half (550\units{g})(2.5\units{m/s})^2
%		= 1.72\units{kg\.m^2/s^2}
%\stopformula
%The units are quite a mess, but they clean up nicely using the fact that $1\units{N} = 1\units{kg\.m/s^2}$. Putting this in with care gives.
%\startformula
%	K\si = 1.72\units{N\.m}
%		= 1.72\units{J}
%\stopformula
%The units of kinetic energy are joules, just like work!
%
%The work $W$ done on the cart and block by the force of friction is $F\Delta x$, where $F$ is the force on the block, not the force exerted by the block on the track. The force on the block is negative, so the work done on the block will be negative, decreasing the total energy. In fact, it decreases the total energy to zero ($H\sf=0$) because the cart and block stop.
%
%Put these into the conservation of energy equation and solve for the displacement.
%\startformula\startmathalignment
%	\NC	K\si + F\Delta x	 \NC = 0\NR
%	%\NC	F\Delta x \NC = K\si	\NR
%	\NC	\Delta x \NC = \frac{K\si}{F}
%			= \frac{1.72\units{J}}{-0.30\units{N}}
%			= \frac{1.72\units{\ucan{N}\.m}}{-0.30\,\ucan{N}}
%			= 5.7\units{m}
%\stopmathalignment \stopformula
%This cart and block start quickly with a small force resisting their motion, so they are able to slide a considerable distance, $5.7\units{m}$, before coming to a stop.
%\stopsolution\stopexample

%\startbuffer[TikZ:BlockSlideK] % This is replaced by the MetaPost version below.
%\environment env_physics
%\environment env_TikZ
%\setupbodyfont [libertinus,11pt]
%\setoldstyle % Old style numerals in text
%\startTEXpage\small%
%\starttikzpicture% tikz code
%	\startaxis[
%		footnotesize,
%		%width=5.1cm,%\marginparwidth,
%		y={1cm}, x={3.67mm},
%		xlabel={$x$ (m)},
%		xmin=0, xmax=10,
%		minor x tick num=3,
%		ylabel={Energy (J)},
%	  	%every axis y label/.style={at={(ticklabel cs:0.5)},rotate=90,anchor=center},
%		ymin=0, ymax=2,
%		minor y tick num=4,
%		]
%	  	\addplot[thick, samples=121, domain=0:10,]
%		    {1.72-(0.2)*x}node[above right,pos=.4] {$H=K$}
%		;
%	\stopaxis
%\stoptikzpicture
%\stopTEXpage
%\stopbuffer
%
%\placefigure[margin][fig:BlockSlideK]
%{The block’s energy as a function of position for \in{example}[ex:BlockSlide].}
%{\noindent\typesetbuffer[TikZ:BlockSlideK]} % figure contents

\startuseMPgraphic{graph::CartWork} % I'd like to add minor ticks, 0.667mm long.
	path p;
	p := (0,0) -- (2,54) -- (5,54) -- (8,0);
	picture myarrow;
	myarrow := image(
		drawarrow (-2mm,0) -- origin withpen pencircle scaled 0.8pt;
	);
draw begingraph(4cm,3.5cm);
	setrange(origin, 8,70);
	graph_template.itick := origin--(1mm,0) ;
	for y=auto.y:
		itick.lft(formatted("$@g$", y), y);
		itick.lft(formatted("@s", ""), y) withcolor "middlegray";
		itick.rt(formatted("@s", ""), y) withcolor "middlegray";
	endfor
	for x=auto.x:
		itick.bot(formatted("$@g$", x), x);
		itick.bot(formatted("@s", ""), x) withcolor "middlegray";
		itick.top(formatted("@s", ""), x) withcolor "middlegray";
	endfor
	glabel.lft(textext("Energy (J)") rotated 90,OUT) ;
	glabel.bot(textext("$x$ (m)"), OUT);
	gdraw(p) withpen pencircle scaled 0.8pt;
	glabel.top("$H=K$",1.6);
	glabel.bot("$F=27\units{N}$",0.5) rotatedaround(point 0.5 of p, angle (1,2.7));
	glabel.bot("$F=0\units{N}$",1.5);
	glabel.bot("$F=-18\units{N}$",2.5) rotatedaround(point 2.5 of p, angle (1,-1.8));
	glabel(myarrow, 0.5) rotatedaround(point 0.5 of p, angle (1,2.7));
	glabel(myarrow, 1.5);
	glabel(myarrow, 2.5) rotatedaround(point 2.5 of p, angle (1,-1.8));
endgraph;
\stopuseMPgraphic

\startplacefigure[location=margin, reference=fig:CartWork, title={The cart's energy as a function of position for \in{examples}[ex:CartWork1] and \in[ex:CartWork2]. The graph's slope is the force applied to the cart.}]
\small
\reuseMPgraphic{graph::CartWork}
\stopplacefigure

\noindent
The plot in \in{figure}[fig:CartWork] shows the cart's energy as a function of position as you roll it down the hallway, a total distance of $8\units{m}$. During the first $2\units{m}$, your force steadily increases the cart's kinetic energy from $0\units{J}$ to $54\units{J}$. In fact, the force you exert is the slope of the energy graph.
\startformula
	F = \frac{\Delta H}{\Delta x}
		= \frac{54\units{J}}{2\units{m}} 
		= \frac{54\units{N\.m}}{2\units{m}} 
		= 27\units{N} 
\stopformula
The cart then coasts for the next $3\units{m}$. During this part of the journey you exert no force and the cart's kinetic energy remains constant. Again, your force, $0\units{N}$, is the energy graph's slope. Finally, during the last $3\units{m}$, you steadily reduce the cart's kinetic energy by exerting a force that opposes the cart's motion. This negative force, $-18\units{N}$, is the energy graph's slope during the final $3\units{m}$.

\startuseMPgraphic{graph::CartPosition} % I'd like to add minor ticks, 0.667mm long.
	path p;
	p := (0,0) ..controls (0,1.11) and (0.66,2.22).. (2,3.33)
		-- (5,5.83) ..controls (7,7.5) and (8,9.17).. (8,10.83);
	picture myarrow;
	myarrow := image(
		drawarrow (-2mm,0) -- origin withpen pencircle scaled 0.8pt;
	);
draw begingraph(4cm,4cm);
	setcoords(linear, -linear);
	setrange(origin, 8,12);
	graph_template.itick := origin--(1mm,0) ;
	for y=auto.y:
		itick.lft(formatted("$@g$", y), y);
		itick.lft(formatted("@s", ""), y) withcolor "middlegray";
		itick.rt(formatted("@s", ""), y) withcolor "middlegray";
	endfor
	for x=auto.x:
		itick.top(formatted("$@g$", x), x);
		itick.bot(formatted("@s", ""), x) withcolor "middlegray";
		itick.top(formatted("@s", ""), x) withcolor "middlegray";
	endfor
	glabel.lft(textext("$t$ (s)") rotated 90,OUT) ;
	glabel.top(textext("$x$ (m)"), OUT);
	gdraw(p) withpen pencircle scaled 0.8pt;
endgraph;
\stopuseMPgraphic

\startplacefigure[location=margin, reference=fig:CartPosition, title={The cart's position as a function of time for \in{examples}[ex:CartWork1] and \in[ex:CartWork2]. The graph is rotated $90\degree$ clockwise so the position axis aligns with the position axis in \in{figure}[fig:CartWork]. As time increases (downward on the graph), the cart moves from $x=0\units{m}$ to $x=8\units{m}$, accelerating for the first few seconds, then coasting at a speed of $1.2\units{m/s}$, and finally decelerating over the last few seconds.}]
\small
\reuseMPgraphic{graph::CartPosition}
\stopplacefigure

Energy vs.\ position graphs, like the one in \in{figure}[fig:CartWork], are extremely useful, and they will become more useful as we learn about more types of energy. Energy graphs work well with the position vs.\ time graphs we have seen so frequently. \in{Figure}[fig:CartPosition] shows the position vs. time graph for your trip pushing the cart of books. I have rotated the graph $90°$ clockwise to align the position axis with the position axis of the energy graph in \in{figure}[fig:CartWork] above. 
(You may want to rotate the page for a minute so you can recognize the time and position axes.) The position vs.\ time graph shows the cart accelerating during the first few seconds as you push the cart the first $2\units{m}$. Then the position graph's slope is a constant $1.2\units{m/s}$ as the cart coasts for $3\units{m}$. Finally, the cart's velocity decreases as you bring it to a stop.

On position vs.\ time graphs the story is always told from left to right – or from top to bottom if the graph is rotated. The story goes from earlier time to later time. Time never goes backwards!

On energy graphs the story can go either direction. In \in{examples}[ex:CartWork1] and \in[ex:CartWork2], you pushed the cart from $x=0\units{m}$ to $x=8\units{m}$, left to right. You could just as easily push the cart from $x=8\units{m}$ to $x=0\units{m}$, right to left. \in{Figure}[fig:CartGraphs] shows the graphs for such a return trip, this time with a much lighter empty cart ($9\units{kg}$). Follow the story on the graphs.

At $t=0\units{s}$ you are at $x=8\units{m}$. You push the empty cart in the negative direction with a gentle force $F=-9N$. You push for $2\units{m}$ in the negative direction, a displacement of $\Delta x = -2\units{m}$. The energy graph's slope is negative here. Since the displacement is also negative, you do positive work on the cart, and the cart's kinetic energy increases to $K = 18\units{J}$. 

The cart then coasts in the negative direction with kinetic energy $K = 18\units{J}$. This kinetic energy is positive even though the cart is moving in the negative direction. Kinetic energy is \emph{always} positive!
The empty cart's kinetic energy is much less than the full cart's $54\units{J}$ of energy in your earlier trip. However, the empty cart's speed is $2\units{m/s}$, greater than the heavy cart's, because the empty cart's mass is so much smaller. The empty cart's velocity is $-2\units{m/s}$ because it is moving in the negative direction. This is visible as a slope of $-2\units{m/s}$ on the position vs.\ time graph. (Rotate if you need to, so you can see the negative slope.)

Finally, you stop the cart with a force $F=18\units{N}$ in the positive direction. This force is the same magnitude as the force you used to stop the full cart. Since the empty cart is much lighter, you are able to stop it is a much shorter distance, brining it to a stop at $x=0\units{m}$. The return trip with the empty cart takes less energy and less time than the first trip with the full, heavy cart.

\startuseMPgraphic{graph::CartGraphs} % I'd like to add minor ticks, 0.667mm long.
	path H;
	H := (8,0) -- (6,18) -- (1,18) -- (0,0);
	picture myarrow;
	myarrow := image(
		drawarrow (-2mm,0) -- origin withpen pencircle scaled 0.8pt;
	);
draw begingraph(4cm,1.5cm);
	setrange(origin, 8,30);
	graph_template.itick := origin--(1mm,0) ;
	for y=auto.y:
		itick.lft(formatted("$@g$", y), y);
		itick.lft(formatted("@s", ""), y) withcolor "middlegray";
		itick.rt(formatted("@s", ""), y) withcolor "middlegray";
	endfor
	for x=auto.x:
		itick.bot(formatted("$@g$", x), x);
		itick.bot(formatted("@s", ""), x) withcolor "middlegray";
		itick.top(formatted("@s", ""), x) withcolor "middlegray";
	endfor
	glabel.lft(textext("Energy (J)") rotated 90,OUT) ;
	glabel.bot(textext("$x$ (m)"), OUT) shifted (0,-1mm);
	gdraw(H) withpen pencircle scaled 0.8pt;
	glabel.top("$H=K$",1.4);
	glabel.bot("$-9\units{N}$",0.6) rotatedaround(point 0.6 of H, angle (1,-0.9));
	glabel.bot("$F=0\units{N}$",1.5);
	glabel.bot("$18\units{N}$",2.4) rotatedaround(point 2.4 of H, angle (1,1.8));
	glabel(myarrow, 0.5) rotatedaround(point 0.5 of H, angle (-1,0.9));
	glabel(myarrow, 1.5) rotatedaround(point 1.5 of H, 180);
	glabel(myarrow, 2.5) rotatedaround(point 2.5 of H, angle (-1,-1.8));
endgraph;
	path p;
	p := (8,0) ..controls (8,0.66) and (7.33,1.33).. (6,2)
		-- (1,4.5) ..controls (0.33,4.83) and (0,5.17).. (0,5.5);
	picture myarrow;
	myarrow := image(
		drawarrow (-2mm,0) -- origin withpen pencircle scaled 0.8pt;
	);
draw begingraph(4cm,2cm);
	setcoords(linear, -linear);
	setrange(origin, 8,6);
	graph_template.itick := origin--(1mm,0) ;
	for y=auto.y:
		itick.lft(formatted("$@g$", y), y);
		itick.lft(formatted("@s", ""), y) withcolor "middlegray";
		itick.rt(formatted("@s", ""), y) withcolor "middlegray";
	endfor
	for x=auto.x:
		%itick.top(formatted("$@g$", x), x);
		itick.bot(formatted("@s", ""), x) withcolor "middlegray";
		itick.top(formatted("@s", ""), x) withcolor "middlegray";
	endfor
	glabel.lft(textext("$t$ (s)") rotated 90,OUT) ;
	%glabel.top(textext("$x$ (m)"), OUT);
	gdraw(p) withpen pencircle scaled 0.8pt;
endgraph shifted (0cm, -2.4cm);
\stopuseMPgraphic

\startplacefigure[location=margin, reference=fig:CartGraphs, title={The cart's energy as a function of position and position as a function of time as the empty cart is returned to $x=0$. These graphs are drawn to the same scale as \in{figures}[fig:CartWork] and \in[fig:CartPosition].}]
\small
\reuseMPgraphic{graph::CartGraphs}
\stopplacefigure

’s Gravesande’s experiment provides another good example of positive and negative work, this time in the $y$-direction. A dropped ball is pulled downward by the gravitational force. The downward force is negative, and the downward displacement is also negative, so work done by the gravitational force is positive, increasing the ball’s kinetic energy.

When the ball hits the clay, the clay exerts an upward force much greater than the gravitational force. The ball continues down a bit, making a dent in the clay, so there is a small negative displacement even as the clay is exerting a large positive force. The work done on the ball as it stops is the product of the positive force and the negative displacement. This negative work eliminates the ball's kinetic energy, brining it to a stop.

The size of the dent in the clay was proportional to the work done by the ball during the impact. Dents are harder to measure than displacements, but the idea is the same. His data showed that the work done was proportional to the ball's \visviva, not the ball's momentum.

\section{Kinetic energy and work in three dimensions}

Energy is not directional, so all types of energy are represented by scalars, not by vectors. Kinetic energy depends on an object’s speed, but not its direction. When working in two or three dimensions, $v$, is the speed.
\startformula
v = \sqrt{v_x^2+v_y^2+v_z^2},
\stopformula
Putting this speed into the kinetic formula gives the kinetic energy formula in three dimensions.
\startformula
	K= \half mv^2 = \half m\left(v_x^2+v_y^2+v_z^2\right)
\stopformula
Momentum plays such a central role in motion that it is usually better to work with momentum rather than velocity. A simple calculation gives
\startformula
	K = \frac{p^2}{2m} = \frac{p_x^2+p_y^2+p_z^2}{2m}.
\stopformula
Since each component of the momentum contributes to the kinetic energy, it is often useful to consider these contributions individually.
\startformula
	K = K_x + K_y + K_z,
\stopformula
where the individual contributions are
\startformula\
	K_x = \half mv_x^2 = \frac{p_x^2}{2m},\qquad
	K_y = \half mv_y^2 = \frac{p_y^2}{2m},\qquad
	K_z = \half mv_z^2 = \frac{p_z^2}{2m}.
\stopformula
Each contribution to the kinetic energy represents the energy required to give the object momentum in the corresponding direction. $K_x$ is the energy required to give the object momentum $p_x$ in the $x$-direction, while $K_y$ is the additional energy required to give the object momentum $p_y$ in the $y$-direction, and likewise for $K_z$.

The contributions to the kinetic energy are \emph{not} components of a kinetic energy vector. It is useful to divide up the kinetic energy into three contributions, but I fear that you may be tempted to treat these as components of a vector. Do not do it! You will never have reason to write the vector $\vec K = \components{K_x,K_y,K_z}$, because kinetic energy is not a vector. Instead write $K= K_x + K_y + K_z$ to get a number. For a vector, like momentum, you would never add the components, $p=p_x+p_y+p_z$, because the result is a number, not the vector. Instead write $\vec{p}=\components{p_x,p_y,p_z}$.

The work formula in three dimensions is also a total of the work in each direction.
\startformula
	W = F_x\Delta x + F_y\Delta y + F_z\Delta z
\stopformula
We have already discussed how force in the same direction as the displacement does positive work, while force opposite the displacement does negative work. The work formula in three dimensions shows that force perpendicular to the displacement does no work at all, neither positive nor negative.

If I push on the back of a rolling cart I do positive work, increasing its kinetic energy. If I push backward on the front of a rolling cart I do negative work, decreasing its kinetic energy. If I push on the side of a rolling cart, so that I neither speed it up nor slow it down, then I do no work on the cart.

\startexample
You are pushing a refrigerator with a force of $\components{25,10}\unit{N}$. Your friend is also pushing, but in another direction.  When you both finish pushing, the refrigerator’s displacement is $\Delta x=5.0\units{m}$ and $\Delta y=-3.0\units{m}$.  What is the work that you do on the refrigerator?

\startsolution
We don’t need your friend’s force to find the work you do. Your force is enough.
\startformula\startmathalignment
\NC	W \NC= F_x\Delta x + F_y\Delta y \NR
\NC		\NC= (25\units{N})(5.0\units{m}) + (10\units{N})(-3.0\units{m}) \NR
\NC		\NC= 125\units{N\.m} - 30\units{N\.m} \NR
\NC		\NC= \answer{95\units{J}}
\stopmathalignment\stopformula
You do $95\units{J}$ of work on the refrigerator. There is no way to know from this information if your friend is working at all.
\stopsolution
\stopexample

Daniel Bernoulli linked engineers' work to physicists' \visviva\ through conservation of energy, and he demonstrated that this profound link could solve important and difficult problems in both physics and engineering. These ideas and others in his revolutionary \booktitle{Hydrodynamica} were accepted slowly by the physics community, and were not fully embraced until many year's after \booktitle{Hydrodynamica's} publication in 1738. Perhaps Bernoulli's ideas were too far ahead of his time, or perhaps the humble Bernoulli lacked the necessary gifts of self promotion.
One idea, the conservation of \visviva, did gain rapid acceptance – but not primarily due to a Bernoulli's work.

\section{Vis viva revived}
Leibnitz’s \visviva\ was given new life by a bold genius, Emilie Du Châtelet. %, who investigated both sides of the debate in great depth.
Emilie was born in 1706. Her wealthy father used his connections in the court of King Louis XIV to arrange Emilie’s marriage to a high ranking military officer with a distinguished noble lineage. The couple did not know each other before their wedding, but they knew the benefits of their union. At the age of eighteen, she became a noblewoman, Madame la marquise Du Châtelet. Her husband, twelve years older, gained financial resources and connections to advance his military carer. Shortly after the birth of their third child in 1733, Emilie Du Châtelet took an interest in mathematics and geometry, reading many books and hiring an excellent tutor. She met the famous playwright Voltaire that same year and began collaborating with him intimately on many projects, including the study of Newtonian Physics.

In 1740, at the age of thirty four, Du Châtelet wrote a physics book, \booktitle{Institutions de Physique} or \booktitle{Foundations of Physics}. She had been unable to find an introductory physics text for her son, so she wrote \booktitle{Foundations} for him. She explained Newtonian mechanics with great clarity and conviction. No reader could doubt her mastery and enthusiasm for Newton’s theory of motion.

In the final chapter, Du Châtelet explains \visviva. She reminds her son of \quotation{the memoir that M.~de Mairan gave in 1728 to the Academy against \visviva\ that we read together, and in which the famous proceeding is explained with much clarity and eloquence.} She continues,
\startblockquote
	As this work appears to me to be the most ingenious that has been produced against \visviva, I will pause to take the time to remind you here of some passages, and to refute them.\autocite{188}{DuChatelet1740}
\stopblockquote
Even as she refuted the Cartesian position, her mastery of Newtonian physics was on full display. In many  examples she first used Newtonian physics to predict what would happen. She then showed, based on this Newtonian analysis, that \visviva\ was also conserved, but Descartes’ quantity of motion was not.

As her final argument of this type, Du Châtelet calls on the authority of Isaac Newton himself! In the forth and final edition of \booktitle{Opticks}, published in 1730, Newton included an essay on forces in nature. This famous essay became know by its unhelpful title: \booktitle{Query 31}. In it, Newton argues that the quantity of motion is \emph{not} conserved.
\quotation{For from the various Composition of two Motions, ’tis very certain that there is not always the same quantity of Motion in the World.}
He provides an example in which \quotation{two Globes joined by a slender Rod, revolve about their common Center of Gravity with an uniform Motion, while that Center moves on uniformly in a right Line\dots,} as shown in \in{figure}[fig:spinningmoving].\autocite{373}{Newton1730} (The center of gravity is an idea we will return to later in this chapter. We will call it the center of mass.) These combined motions, once they are started, will continue uniformly without any external forces. In this situation,
\startbuffer[TikZ:spinning]
\environment env_physics
\environment env_TikZ
\setupbodyfont [libertinus,11pt]
\setoldstyle % Old style numerals in text
\startTEXpage
\def\angles{72,54,...,-90}
\starttikzpicture% tikz code
	\clip (-2.5,-1.75) rectangle (2.5,1.75);% Clipping Rectangle
	\foreach \T in \angles {% Draw dumbbells
		\pic[rotate={\T},opacity={.5-(\T/180)}]{dumbbell};
	}
	\foreach \T in \angles {% Draw path segments
		\draw[-{Straight Barb[scale length=.5]}] ({\T+18}:1.5cm)
			arc [start angle={\T+18}, end angle={\T}, radius=1.5cm];
		\draw[-{Straight Barb[scale length=.5]}] ({\T+198}:1.5cm)
			arc [start angle={\T+198}, end angle={\T+180}, radius=1.5cm];
	}
\stoptikzpicture
\stopTEXpage
\stopbuffer

\placefigure[margin][fig:spinning] % location
{Two globes joined by a slender rod revolve about their common center with a uniform motion.}	% caption
{\noindent\typesetbuffer[TikZ:spinning]} % figure contents


\startbuffer[TikZ:spinningmoving]
\environment env_physics
\environment env_TikZ
\setupbodyfont [libertinus,11pt]
\setoldstyle % Old style numerals in text
\startTEXpage
\def\angles{522,504,...,-90}
\starttikzpicture% tikz code
	\clip (-14,-1.75) rectangle (2.5,1.75);% Clipping Rectangle
	\foreach \T in \angles {% Dumbbells
		\draw[-{Straight Barb[scale length=.5]}] ({-3.14*(\T+18)/120},0) -- ({-3.14*\T/120},0)pic[{}
		-{},rotate={\T},opacity={.5-(\T/1200)}]{dumbbell};
	}
\stoptikzpicture
\stopTEXpage
\stopbuffer

%\placefigure[margin][fig:spinningmoving] % location
%{As the joined globes revolve, their center moves uniformly in a straight line.}	% caption text
%{\noindent\typesetbuffer[TikZ:spinningmoving]} % figure contents

\placefigure[margin][fig:spinningmoving] % location
{As the joined globes revolve, their center moves uniformly in a straight line.}	% caption text
{\vskip4.4in\hbox{\starttikzpicture
	\draw[white] (0,0)-- ++(5,0); % Sky to make height better
\stoptikzpicture}}

\placewidefloat[bottom,none]
{This is its caption I need to fix.}
{\hbox{\noindent\typesetbuffer[TikZ:spinningmoving]}} % figure contents


\startblockquote
the Sum of the Motions of the two Globes, as often as the Globes are in the right Line described by their common Center of Gravity, will be bigger than the Sum of their Motions, when they are in a Line perpendicular to that right Line. By this Instance it appears that Motion may be got or lost.\autocite{373}{Newton1730}
\stopblockquote
Du Châtelet confirms this with specific masses and velocities. She gives a mass $m=1$ to each globe. She sets the rod spinning such that the globes move with speeds $v=1$ about the center, while that center moves with a speed $v=1$ to the right. She first considers the moment when the rod is perpendicular to the line, as in \in{figure}[fig:spinningmovingperpendicular]. At this moment globe B’s velocity due to the spinning is in the same direction as the rod’s central motion, so the velocities add to give a speed $v\sub{B} = 2$. Globe A’s velocity due to the spinning is exactly opposite the rod’s central velocity, so the velocities add to give a speed $v\sub{A}=0$. The total quantity of motion is then
\startbuffer[TikZ:spinningmovingperpendicular]
\environment env_physics
\environment env_TikZ
\setupbodyfont [libertinus,11pt]
\setoldstyle % Old style numerals in text
\startTEXpage
\def\angles{72,54,...,-90}
\starttikzpicture% tikz code
	\clip (-2.5,-2.1) rectangle (2.5,2);% Clipping Rectangle
	\foreach \T in \angles {% Dumbbells
		\draw[-{Straight Barb[scale length=.5]}] ({-3.14*(\T+18)/120},0) -- ({-3.14*\T/120},0)pic[{}
		-{},rotate={\T+90},opacity={.1}]{dumbbell};
	}
	\pic[at={(0,0)},rotate={90}]{dumbbell};
	\draw[->, thick](0,1.5) --node[above]{$v_B=2$} ++(2,0);
	\draw(0,-1.5)node[below=1mm]{$v_A=0$};
\stoptikzpicture
\stopTEXpage
\stopbuffer
\placefigure[margin][fig:spinningmovingperpendicular] % location
{The globes' velocities when the rod is perpendicular to the line of motion.}	% caption text
{\noindent\typesetbuffer[TikZ:spinningmovingperpendicular]} % figure contents
\startformula
mv\sub{B}+mv\sub{A} = 1\cdot 2 + 1\cdot 0 = 2.
\stopformula
\startbuffer[TikZ:spinningmovingperparallel]
\environment env_physics
\environment env_TikZ
\setupbodyfont [libertinus,11pt]
\setoldstyle % Old style numerals in text
\startTEXpage
\def\angles{72,54,...,-90}
\starttikzpicture% tikz code
	\clip (-2.5,-2.1) rectangle (2.5,2);% Clipping Rectangle
	\foreach \T in \angles {% Dumbbells
		\draw[-{Straight Barb[scale length=.5]}] ({-3.14*(\T+18)/120},0) -- ({-3.14*\T/120},0)pic[{}
		-{},rotate={\T+180},opacity={.1}]{dumbbell};
	}
	\pic[at={(0,0)},rotate={180}]{dumbbell};
	\draw[->, thick](-1.5,0) --node[above left]{$v_B=\sqrt{2}$} ++(1,1);
	\draw[->, thick](1.5,0) --node[below left]{$v_A=\sqrt{2}$} ++(1,-1);
\stoptikzpicture
\stopTEXpage
\stopbuffer
\placefigure[margin][fig:spinningmovingperparallel] % location
{The globes' velocities when the rod is aligned with the line of motion.}	% caption text
{\noindent\typesetbuffer[TikZ:spinningmovingperparallel]} % figure contents
She then considers the moment when the rod aligns with the line, as in \in{figure}[fig:spinningmovingperparallel]. At this moment B’s velocity due to the spinning is upward, perpendicular to the rod’s central velocity, so the velocities add (using the tip-to-tail method) to give a speed $v\sub{A}=\sqrt 2$. Similarly, A’s velocity due to the rod’s rotation is downward, giving body A a speed $v\sub{A}=\sqrt 2$. The total quantity of motion is then
\startformula
mv\sub{B}+mv\sub{A} = 1\cdot \sqrt 2 + 1\cdot \sqrt 2 = 2 \sqrt 2.
\stopformula
\startbuffer[TikZ:spinningmovingmomentum]
\environment env_physics
\environment env_TikZ
\setupbodyfont [libertinus,11pt]
\setoldstyle % Old style numerals in text
\startTEXpage
\def\angles{72,54,...,-90}
\starttikzpicture% tikz code
	\clip (-2.5,-.5) rectangle (2.5,1.1);% Clipping Rectangle
	\draw[->, thick](-1,0) --node[above left]{$p_B=\sqrt{2}$} ++(1,1);
	\draw[->, thick](0,1) --node[above right]{$p_A=\sqrt{2}$} ++(1,-1);
	\draw[->, thick](-1,0) --node[below]{$p=2$} ++(2,0);
\stoptikzpicture
\stopTEXpage
\stopbuffer
\placefigure[margin][fig:spinningmovingmomentum] % location
{The globes' momentum when the rod is aligned with the line of motion.}	% caption text
{\noindent\typesetbuffer[TikZ:spinningmovingmomentum]} % figure contents
This quantity of motion is bigger than the quantity of motion in the perpendicular orientation, just as Newton said. Clearly, the quantity of motion is not constant as the globes spin. However, the total momentum $\vec p = \vec p\sub{A} + \vec p\sub{B}$ is constant, always to the right with a magnitude $p = 2$, as shown \in{figure}[fig:spinningmovingmomentum]. What about the \visviva?
The following example gives us a clue.

%%%%%%%% EXAMPLE %%%%%%%%%%%%%%%%%%%%%%%%%%%%%%%%%%%%%%%%
\startexample[ex:NewtonGlobes]
Find the total kinetic energy of Newton’s connected globes when the connecting rod is perpendicular to the center’s line of motion and also when the rod is aligned with the center’s line of motion. Use Du Châtelet’s masses and speeds (with units). Each globe has a mass of $1\units{kg}$. The speeds are $1\units{m/s}$ due to the rotation about the center of mass and $1\units{m/s}$ due to the center of mass’s motion.
\startsolution
When the rod is perpendicular to the center’s line of motion the, total kinetic energy is
\startformula \startmathalignment
\NC K	\NC = K\sub{A} + K\sub{B} 				\NR
\NC		\NC = \half mv\sub{A}^2 + \half mv\sub{B}^2	\NR
\NC		\NC = \half(1\units{kg})\left(0\units{m/s}\right)^2 + \half(1\units{kg})\left(2\units{m/s}\right)^2	\NR
\NC		\NC = 0\units{J} + 2\units{J}	\NR
\NC		\NC = 2\units{J}	.\NR
\stopmathalignment \stopformula
When the rod is aligned with the center’s line of motion, the kinetic energy is
\startformula
K = \half(1\units{kg})\left(\sqrt{2}\units{m/s}\right)^2 + \half(1\units{kg})\left(\sqrt{2}\units{m/s}\right)^2 = 2\units{J}.
\stopformula
The kinetic energy is the same for both orientations, $K=2\units{J}$. With some additional work is it possible to show that the energy remains constant for all orientations.
\stopsolution
\stopexample
%%%%%%%%%%%%%%%%%%%%%%%%%%%%%
Du Châtelet computed the \visviva, which is double the kinetic energy, and reached the same conclusion.
\startblockquote
It is easy to prove that the \visviva\ always remains the same, although the quantity of motion varies.\autocite{188}{DuChatelet1740}
\stopblockquote
You would be wise to wonder if Newton’s example or Du Châtelet’s choice of values are somehow special. Perhaps the symmetry of two masses or the match between the rotational and linear speeds causes the kinetic energy to balance in some way, keeping it constant. In fact, this example is not special.
\startblockquote
In all cases, and especially in that which I have just cited from M. Newton, the \visviva\ stay invariable. \autocite{199}{DuChatelet1740}
\stopblockquote
Any velocities, any masses, and any rigid shape will have a constant kinetic energy unless external forces do work on the object to change its energy. If the masses are different or if the shape is asymmetrical, then the center of mass is not necessarily in the geometric center, but the object still rotates naturally about its center of mass and kinetic energy is constant. (We will learn more about theses situations near the end of the chapter when we look more closely and rotational motion.) Du Châtelet knew these complicated cases worked as well, but the simple example above made her point quite clearly.


\section{Angular motion and rotational kinetic energy}
While freely moving objects travel along straight lines, interactions cause an abundance of circular motions, from the orbits that dominate the solar system down to the everyday rotations of tops, wheels and even Earth itself. So far, we have studied circular motions using the same tools we use for linear motions, $\vec p = m\vec v$ for momentum and $K = \half mv^2$ for kinetic energy.
%For orbits $\vec v$ was the instantaneous velocity of the orbiting planet or moon. For the connected globes in Newton’s Query 31, $\vec v$ was the velocity of each globe, found by adding the velocities due to rotation and central motion.

\startbuffer[TikZ:rolling]
\environment env_physics
\environment env_TikZ
\setupbodyfont [libertinus,11pt]
\setoldstyle % Old style numerals in text
\startTEXpage
\def\angles{72,54,...,-90}
\starttikzpicture% tikz code
	\fill [black!10] (0,0) rectangle (5,-.15);
		\draw[fill=black!10] (2,1.6) circle[radius=1.6];
		\draw[->](0.4,1.6) -- ++(.4,.4);
		\draw[->](0.8,1.2) -- ++(.3,.3);
		\draw[->](0.8,2.0) -- ++(.5,.3);
		\draw[->](1.2,0.8) -- ++(.2,.2);
		\draw[->](1.2,1.6) -- ++(.4,.2);
		\draw[->](1.2,2.4) -- ++(.6,.2);
		\draw[->](1.6,.4) -- ++(.1,.1);
		\draw[->](1.6,1.2) -- ++(.3,.1);
		\draw[->](1.6,2.0) -- ++(.5,.1);
		\draw[->](1.6,2.8) -- ++(.7,.1);
		\draw[->](2,0.8) -- ++(.2,0);
		\draw[->](2,1.6) -- ++(.4,0);
		\draw[->](2,2.4) -- ++(.6,0);
		\draw[->](2,3.2) -- ++(.8,0);
		\draw[->](2.4,.4) -- ++(.1,-.1);
		\draw[->](2.4,1.2) -- ++(.3,-.1);
		\draw[->](2.4,2.0) -- ++(.5,-.1);
		\draw[->](2.4,2.8) -- ++(.7,-.1);
		\draw[->](2.8,0.8) -- ++(.2,-.2);
		\draw[->](2.8,1.6) -- ++(.4,-.2);
		\draw[->](2.8,2.4) -- ++(.6,-.2);
		\draw[->](3.2,1.2) -- ++(.3,-.3);
		\draw[->](3.2,2.0) -- ++(.5,-.3);
		\draw[->](3.6,1.6) -- ++(.4,-.4);
		\draw[thin] (0,0)--(5,0);
\stoptikzpicture
\stopTEXpage
\stopbuffer

\placefigure[margin][fig:rolling] % location
{Different parts of a rolling disk move with many different velocities.}	% caption
{\noindent\typesetbuffer[TikZ:rolling]} % figure contents


Unfortunately, we are about to run into serious trouble because nearly all rotating objects have parts that are moving with many different velocities. Parts near the center of rotation complete small circles while parts far away  complete much larger circles in the same time. Figure~\in[fig:rolling] shows a typical example – a rolling disk. Parts near the top move faster, while parts near the ground are slower. Parts on the leading side are moving down toward the ground, while parts on the trailing side are moving upward. If you are skilled with vector calculus then you know just what to do, and you can skip the rest of this section. If you don’t have those skills, you will need some tools to handle momentum and energy in common situations like rolling. Specifically, we need a method for finding the total momentum and total kinetic energy of an object that is rotating and possibly moving at the same time.

The trick for momentum is quite simple: ignore the rotation and look only at the motion of the center. The total momentum of the wheel in figure~\in[fig:rolling] is $\vec p = m\vec v$, where $m$ is the wheel’s mass and $\vec v$ is the velocity of the wheel’s center, shown in figure~\in[fig:rollingcenter]. The rotation contributes nothing to the total momentum.

\startbuffer[TikZ:rollingcenter]
\environment env_physics
\environment env_TikZ
\setupbodyfont [libertinus,11pt]
\setoldstyle % Old style numerals in text
\startTEXpage\small
\def\angles{72,54,...,-90}
\starttikzpicture% tikz code
	\fill [black!10] (0,0) rectangle (5,-.15);
		\draw[fill=black!10] (2,1.6) circle[radius=1.6];
		\fill(2,1.6) circle[radius=.4mm];
		\draw[->,thick](2,1.6) --node[above]{$v$} ++(2,0);
		\draw[thin] (0,0)--(5,0);
\stoptikzpicture
\stopTEXpage
\stopbuffer

\placefigure[margin][fig:rollingcenter] % location
{The rolling disk’s total momentum depends only on the velocity $v$ of the disk’s center.}	% caption
{\noindent\typesetbuffer[TikZ:rollingcenter]} % figure contents

Newton’s example with the connected globes illustrates how this trick works. The rotation contributes to each globe's velocity, but the contributions are exactly equal and opposite. If the rotation increases one globe's velocity, it decreases or reverses the other's. If it gives an upward component one globe's velocity it gives a downward component to the other's.  These equal and opposite contributions cancel in the total momentum. The situation is the same for any symmetrical object – like a disk, ball, or rod – where every bit of momentum on one side is canceled by an equal and opposite momentum on the opposite side. For example, every part of a wheel has a matching part on the opposite side, ensuring their momenta cancel.

If the object is not symmetrical we force the issue by finding a point where the cancelation happens anyway. The simplest asymmetric case, and the only case you need to be able to do on you own, is two unequal globes connected by a rigid rod. In the spirit of Du Châtelet, we will consider simple masses $m\sub{A} = 3\units{kg}$ and $m\sub{B} = 1\units{kg}$, shown in \in{figure}[fig:spinning31]. Rotation about the rod’s midpoint gives equal and opposite velocities to the globes, but not equal and opposite momenta. The the heavy globe's momentum will be three times larger than the  lighter globe’s. These will not cancel in the total momentum.

\startbuffer[TikZ:spinning31]
\environment env_physics
\environment env_TikZ
\setupbodyfont [libertinus,11pt]
\setoldstyle % Old style numerals in text
\startTEXpage
\def\angles{72,54,...,-90}
\starttikzpicture% tikz code
	\clip (-2.5,-2.1) rectangle (2.5,2.1);% Clipping Rectangle
	\foreach \T in \angles {% Draw dumbbells
		\pic[rotate={\T},opacity={.5-(\T/180)}]{dumbbell31};
	}
	\foreach \T in \angles {% Draw path segments
		\draw[-{Straight Barb[scale length=.5]}] ({\T+18}:1.6cm)
			arc [start angle={\T+18}, end angle={\T}, radius=1.6cm];
		\draw[-{Straight Barb[scale length=.5]}] ({\T+198}:1.6cm)
			arc [start angle={\T+198}, end angle={\T+180}, radius=1.6cm];
	}
	\draw[->, thick](0,1.6) --node[above]{$p_A$} ++(1.8,0);
	\draw[->, thick](0,-1.6) --node[below]{$p_B$} ++(-0.6,0);
\stoptikzpicture
\stopTEXpage
\stopbuffer

\placefigure[margin][fig:spinning31] % location
{Two globes of different masses joined by a slender rod revolve about their geometric center. The globes have the same speed, but more massive top globe has a greater momentum.}	% caption
{\noindent\typesetbuffer[TikZ:spinning31]} % figure contents

\startbuffer[TikZ:spinning31CoM]
\environment env_physics
\environment env_TikZ
\setupbodyfont [libertinus,11pt]
\setoldstyle % Old style numerals in text
\startTEXpage
\def\angles{72,54,...,-90}
\starttikzpicture% tikz code
	\clip (-2.5,-2.9) rectangle (2.5,2.7);% Clipping Rectangle
	\foreach \T in \angles {% Draw dumbbells
		\pic[rotate={\T},opacity={.5-(\T/180)}]{dumbbell31CoM};
	}
	\foreach \T in \angles {% Draw path segments
		\draw[-{Straight Barb[scale length=.5]}] ({\T+18}:2.4cm)
			arc [start angle={\T+18}, end angle={\T}, radius=2.4cm];
		\draw[-{Straight Barb[scale length=.5]}] ({\T+198}:0.8cm)
			arc [start angle={\T+198}, end angle={\T+180}, radius=0.8cm];
	}
	\draw[->, thick](0,0.8) --node[above]{$p_A$} ++(0.9,0);
	\draw[->, thick](0,-2.4) --node[below]{$p_B$} ++(-0.9,0);
\stoptikzpicture
\stopTEXpage
\stopbuffer

\placefigure[margin][fig:spinning31CoM] % location
{Two globes of different masses joined by a slender rod revolve about their center of mass. The smaller globes has a greater speed than the larger mass, but the globes’ momenta have the same magnitude. The total momentum is zero.}	% caption
{\noindent\typesetbuffer[TikZ:spinning31CoM]} % figure contents

Let’s abandon the midpoint and instead rotate about a point that is closer to the heavy globe, as shown in figure~\in[fig:spinning31CoM]. This rotation gives a smaller velocity to the heavy globe while giving a larger velocity to the lighter globe. If we pick the point so that the velocities are in the ratio of one-to-three, then multiplying by the masses, which are in the ratio of three-to-one, will give the globes equal and opposite the momenta. This special rotation point is the \emph{center of mass.} The larger the difference in the masses, the farther the center of mass is shifted toward the heavier mass.

Every object has a center of mass, and will spin about this center without any net momentum. Every object’s total momentum is $\vec p = m \vec v$ where $m$ is the object’s mass and $\vec v$ is the velocity of its center of mass. The total momentum always obeys conservation of momentum,
\startformula
	\vec p\si + \vec F\sn\Delta t = \vec p\sf,
\stopformula
where $\vec F\sn$ is the total of all external forces acting on any part of the object. The rotation can be totally ignored in momentum calculations.

The rotation cannot be ignored in kinetic energy calculations. Rotation always increases the total kinetic energy. Even so, there is a trick that makes the calculation rather direct: simply add the kinetic energy due to the rotation to the kinetic energy of the center of mass to find the total kinetic energy.
\startformula
	K = K\sub{CoM} + K\sub{rot}
\stopformula
The kinetic energy of the center of mass and the kinetic energy due to rotation about the center of mass make independent contributions to the total kinetic energy. This is another special property of the center of mass.

The center of mass’s kinetic energy can be found using the center of mass’s velocity $\vec v$ or the total momentum $\vec p$.
\startformula
	K\sub{CoM} = \half mv^2 = \frac{p^2}{2m},
\stopformula
where $m$ is the total mass, $v$ is the center of mass’s speed, and $p$ is the magnitude of the total momentum. The formula using the magnitude of the total momentum is especially convenient because you don’t need to know anything about where the center of mass is or how it is moving.
%%%%%%%% EXAMPLE %%%%%%%%%%%%%%%%%%%%%%%%%%%%%%%%%%%%%%%%
\startexample[ex:NewtonGlobes]
Find the center of mass’s kinetic energy for Newton’s connected globes. Use Du Châtelet’s masses and speeds (with units). Each globe has a mass of $1\units{kg}$. The globes’ velocities get a $1\units{m/s}$ contribution from the rotation about the center of mass and a $1\units{m/s}$ contribution from the center of mass’s motion.
\startsolution
When finding the center of mass’s kinetic energy, the motion due to rotation can be ignored. We only need the speed due to the center of mass’s motion, which is $v = 1\units{m/s}$.
\startformula
K = \half mv^2 = \half(2\units{kg})(1\units{m/s})^2 = 1\units{J}.
\stopformula
Here I used the total mass $m = 2\units{kg}$ to compute the center of mass’s kinetic energy.
\stopsolution
\stopexample
%%%%%%%%%%%%%%%%%%%%%%%%%%%%%
The only challenge remaining is to find the kinetic energy due to the rotation about the center of mass. This requires adding all of the parts’ kinetic energies, using their speeds due to the rotation. For Newton’s connected globes, the speeds are the same, making this fairly simple.
%%%%%%%% EXAMPLE %%%%%%%%%%%%%%%%%%%%%%%%%%%%%%%%%%%%%%%%
\startexample[ex:NewtonGlobes]
Find total kinetic energy of Newton’s connected globes using the center of mass’s kinetic energy and the kinetic energy due to the rotation. Use Du Châtelet’s masses and speeds (with units). Each globe has a mass of $1\units{kg}$. The globes’ velocities get a $1\units{m/s}$ contribution from the rotation about the center of mass and a $1\units{m/s}$ contribution from the center of mass’s motion.
\startsolution
We found the center of mass’s kinetic energy in the previous example, but we still need to find the kinetic energy due to the rotation. The rotation contributes a velocity of magnitude $v = 1\units{m/s}$ to each of the globes. Here we use the mass of each globe, $m=1\units{kg}$, to find the individual kinetic energies and then add to find the rotational kinetic energy of the connected pair.
\startformula\startmathalignment
\NC K\sub{rot} \NC = \half mv^2 + \half mv^2 \NR
\NC		\NC= \half(1\units{kg})(1\units{m/s})^2 + \half(1\units{kg})(1\units{m/s})^2 \NR
\NC		\NC= \half\units{J} + \half\units{J} \NR
\NC		\NC= 1\units{J}
\stopmathalignment\stopformula
Find the total kinetic energy by adding this rotational kinetic energy to the center of mass’s kinetic energy, found in the previous example.
\startformula
	K = K\sub{CoM} + K\sub{rot}
		= 1\units{J} + 1\units{J}
		= 2\units{J}
\stopformula
The total kinetic energy is $2\units{J}$.
\stopsolution
\stopexample
%%%%%%%%%%%%%%%%%%%%%%%%%%%%%
The total kinetic energy calculated using $K = K\sub{CoM} + K\sub{rot}$ in this example matches the total energy calculated in example~\in[ex:NewtonGlobes] when we used the method $K = K\sub{A} + K\sub{B}$. This is a good sign!
These two methods should give the same answer, but there is an important difference in the calculation. In \in{example}[ex:NewtonGlobes], we found the total kinetic energy in two specific orientations. Other orientations are more challenging.
Using $K = K\sub{CoM} + K\sub{rot}$, we have replaced the complicated motion of the two globes with a very simple type of compound motion: uniform motion of the center of mass compounded with uniform circular motion about the center of mass. Each of type of uniform motion makes an unchanging contribution to the total kinetic energy, confirming Du Châtelet's assertion that the total \visviva\ always remains the same.

%Using $K = K\sub{CoM} + K\sub{rot}$, the center of mass kinetic energy $K\sub{CoM}$ and the kinetic energy due to rotation $K\sub{rot}$ are both independent of the orientation. We find them once, and they stay the same throughout the motion

In addition to the Newtonian arguments, Du Châtelet cited specific experiments, like ’s Gravesande’s, providing further evidence for momentum and \visviva\ – and evidence against Descartes’ quantity of motion. Du Châtelet understood the difference between Newton’s successful momentum and Descartes’ failed quantity of motion, and she drove a wedge between the two as she advanced he argument for \visviva.
%Du Châtlet’s arguments must have confounded the Cartesians terribly. They viewed themselves as Newton’s defenders, but challenging her arguments for Leibniz’s \visviva\ would mean abandoning Newton as well!
Fifty-three years after Newton’s \booktitle{Principia}, this reckoning was long overdue.


Du Châtelet immediately faced two attacks. Even before \booktitle{Foundations} was published, her former tutor, Samuel König, falsely claimed the work was his and that Du Châtelet had simply taken his dictation to produce \booktitle{Foundations}. %König expected his claim was taken seriously because few intellectuals of the time could not imagine a woman producing a work of such sophistication.
Historian Judith P.\ Zinsser describes the second challenge faced by Du Châtelet.
\startblockquote
	The second attack in the spring of 1741 came from Dortous de Mairan, newly appointed the secretary of the Academy of Sciences. Initially, he had thanked Du Châtelet for having sent him a copy of her book. Reading her final chapter, the explicit attack on his work, must have been an unwelcome surprise. Angered by the veiled sarcasm in which she had presented his views, he saw himself as the champion, not only of the Cartesian formula he had defended in 1728 against Bernoulli but also the Academy of Sciences, now challenged by a young noblewoman. He quickly formulated a pamphlet that he distributed throughout the Republic of Letters that portrayed Du Châtelet as a silly coquette, seduced by Leibnizian ideas when she should have accepted those of men wiser and better than she. Du Châtelet surprised him again, however, and answered his pamphlet point by point in a style contemporaries acknowledged was his equal in wit and daring. Probably at the insistence of his friends, Dortous de Mairan let the matter drop.\autocite{107--108}{Zinsser2009}
\stopblockquote

Du Châtelet composed her detailed response in only three weeks, demonstrating her brilliance and thereby removing any doubt that she was the true author of \booktitle{Foundations of Physics}.

\startblockquote
	The controversy, and the approval and acclaim for the \booktitle{Foundations} from mathematicians and physicists in England, France and Germany\dots secured Du Châtelet’s reputation as a member of the elite, intellectual world of the Republic of Letters.\autocite{108}{Zinsser2009}
\stopblockquote

Newton was already widely revered. Du Châtelet’s \booktitle{Foundations of Physics} made his physics widely understood. \booktitle{Foundations} cured the persistent confusion underlying the \visviva\ controversy and spread Leibniz’s insight to many who had dismissed him. Kinetic energy became a central concept in physics.

Calculating rotational kinetic energy $K\sub{rot}$ for more complicated shapes is, of course, more complicated. The object’s different parts move at different speed depending on their distance from the axis of rotation. 
As we learned in \in{Chapter}[ch:Motion], all of these different speeds are related to the object's angular velocity $\omega$. (\at{p.}[eq:angularvelocity])
\startformula
	v = r\omega
\stopformula
Here, $v$ is the speed of one part, and $r$ is that part's distance from the axis of rotation. Knowing each part’s speed allows us to calculate each part’s rotational kinetic energy. Adding these together for all the parts gives the object’s rotational kinetic energy. This rotational kinetic energy can be added to the center of mass’s kinetic energy to give the rotating object’s total kinetic energy.

The next example shows how the many part's different speeds are replaced by the one angular velocity to find the total kinetic energy.

%%%%%%%%%%%%%%%%%%%%%%%%%%%%%%%%%%%%%%%%%%%%%%%%%%%
\startexample[ex:spinning31CoMv]
Two globes, with masses $m\sub{A} = 3.0\units{kg}$ and $m\sub{B} = 1.0\units{kg}$, are connected by a light, rigid, $1.0\units{m}$ rod, as shown. The whole thing is moving with a speed of $1.0\units{m/s}$ while it rotates with an angular velocity of $2.0\units{rad/s}$. Find the magnitude of the total momentum and find the total kinetic energy.
\startbuffer[TikZ:spinning31CoMv]
\environment env_physics
\environment env_TikZ
\setupbodyfont [libertinus,11pt]
\setoldstyle % Old style numerals in text
\startTEXpage
\def\angles{72,54,...,-90}
\starttikzpicture% tikz code
	\clip (-2.5,-2.9) rectangle (2.5,2.7);% Clipping Rectangle
	\foreach \T in \angles {% Draw dumbbells
		\pic[rotate={\T},opacity={.5-(\T/180)}]{dumbbell31CoM};
	}
	\foreach \T in \angles {% Draw path segments
		\draw[-{Straight Barb[scale length=.5]}] ({\T+18}:2.4cm)
			arc [start angle={\T+18}, end angle={\T}, radius=2.4cm];
		\draw[-{Straight Barb[scale length=.5]}] ({\T+198}:0.8cm)
			arc [start angle={\T+198}, end angle={\T+180}, radius=0.8cm];
	}
	\draw[->, thick](0,0) --node[above]{$v$} ++(1.5,0);
		\path ({tan(57.3)},1) coordinate (A) -- (0,0) coordinate (B) -- ({-tan(57.3)},1) coordinate (C)
pic [draw, <-, thick, angle eccentricity=1.2, pic text=$\omega$, angle radius=12mm] {angle};
\stoptikzpicture
\stopTEXpage
\stopbuffer

\placefigure[margin][fig:spinning31CoMv] % location
{Two globes of different masses joined by a slender rod revolve about their center of mass as they move, in example~\in[ex:spinning31CoMv].}	% caption
{\noindent\typesetbuffer[TikZ:spinning31CoMv]} % figure contents

\startsolution
The total momentum is $\vec p = m\vec v$. The rotation has no effect.
\startformula
	p = mv = 4.0\units{kg}\cdot 1.0\units{m/s} = 4.0\units{kg\.m/s}.
\stopformula
The center of mass’s kinetic energy is
\startformula
	K\sub{CoM} = \frac{p^2}{2m} = \frac{(4.0\units{kg\.m/s})^2}{2(4.0\units{kg})} = 2.0\units{J}.
\stopformula
The rotational kinetic energy is the hard part. First, we must find this object’s center of mass, since that will be the center of rotation. The masses are in the ratio of three-to-one, so the distances $r\sub{A}$ and $r\sub{B}$ must be in the ratio of one-to-three. For a $1.0\units{m}$ rod this is achieved by $r\sub{A} = 0.25\units{m}$ and $r\sub{B} = 0.75\units{m}$. These distances allow us to find the individual globes’ speeds due to the rotation using $v = r\omega$. The rotational kinetic energy is
\startformula\startmathalignment
\NC 	K\sub{rot}	\NC = \half m\sub{A}^{} v\sub{A}^2	+ \half m\sub{B}^{} v\sub{B}^2	\NR
\NC 			\NC = \half m\sub{A}^{} r\sub{A}^2\omega^2 + \half m\sub{B}^{} r\sub{B}^2\omega^2	\NR
\NC 			\NC = \half \left(m\sub{A}^{} r\sub{A}^2 + m\sub{B}^{} r\sub{B}^2\right)\omega^2		\NR
\NC 			\NC = \half I\omega^2		\NR
\stopmathalignment\stopformula
where $I = \left(m\sub{A}^{}r\sub{A}^2 + m\sub{B}^{}r\sub{B}^2\right)$. Next we calculate $I$ and put that value in the simple formula for $K\sub{rot}= \half I\omega^2$ above.
\startformula\startmathalignment
\NC	I	\NC = m\sub{A}^{}r\sub{A}^2 + m\sub{B}^{}r\sub{B}^2						\NR
\NC 		\NC = (3.0\units{kg^2})(0.12\units{m})^2 + (3.0\units{kg^2})(0.12\units{m})^2	\NR
\NC 		\NC = 0.75\units{kg^2\.m^2}										\NR
\stopmathalignment\stopformula
Using this in the formula for rotational kinetic energy gives
\startformula
\NC	K\sub{rot}	\NC = \half I\omega^2						\NR
\NC			\NC = \half (0.75\units{kg^2\.m^2})(2.0\units{r/s})^2	\NR
\NC			\NC = 1.5\units{J}							\NR
\stopformula
Finally, we add the center of mass kinetic energy to the rotational kinetic energy to get the total kinetic energy.
\startformula
K = K\sub{CoM} + K\sub{rot} = 2.0\units{J} + 1.5\units{J} = 3.5\units{J}
\stopformula
The magnitude of the total momentum is $4.0\units{kg\.m/s}$ and the total kinetic energy is $3.5\units{J}$.
\stopsolution
\stopexample
%%%%%%%%%%%%%%%%%%%%%%%%%%%%%%%%%%%%%%%%%%%%%%%%%%%

In the middle of the above example we found a simple formula for the rotational kinetic energy
\startformula
	K\sub{rot} = \half I\omega^2
\stopformula
This formula works for all shapes. The rotational kinetic energy is a sum of $\onehalf mv^2$ for all of the parts. The parts have different masses and distances from the center of mass, but they all share the same angular velocity. No matter how many parts the object has, the angular velocity $\omega$ will factor out of the sum (as will the half). This leaves the total $mr^2$, which is called the \emph{moment of inertia.} The moment of inertial depends only on how the object is built – the masses and distances – not on how it is moving. The moments of inertia for some common spinning shapes are shown in table~\in[T:MoIShapes].

To quickly review: An object’s total momentum is the momentum of its center of mass, $\vec p = m\vec v$. An object’s total kinetic energy is the sum of the kinetic energy of its center of mass and its rotational kinetic energy.
\startformula
	K = K\sub{CoM} + K\sub{rot}
	\qquad
	K\sub{CoM}  = \half mv^2 = \frac{p^2}{2m}
	\qquad
	K\sub{rot} = \half I\omega^2
\stopformula

\placetable[margin][T:MoIShapes] % Label
    {The moment of inertia for some common shapes. In all cases the density of the object must be constant. Planets and moons are not constant density, but the formula for a solid ball is a good to within several percent.} % Caption
    {\vskip9pt\small\starttabulate[|l|l|]
\FL[2]%\toprule
%\NC 	Shape		\NC  Moment of Inertia	\NR
%\HL
\NC 	Thin Hoop \NC  $I = mR^2$	\NR
\NC 	Solid Disk \NC  $I=\half mR^2$	\NR
%\NC 	Hollow Sphere \NC  $I=\half mR^2$	\NR
\NC 	Solid Ball \NC  $I=\textfrac{2}{5}mR^2$	\NR
\LL[2]%\bottomrule
\stoptabulate}

\noindent We will conclude this section by returning to the rolling disk first shown in figure~\in[fig:rolling]. To find the rolling disk’s kinetic energy we do not need to find the many different velocities of its various parts. We only need to look at its center of mass motion and its rotational motion, shown in figure~\in[fig:rollingomega]. First, consider the rotational motion, using the moment of inertia for a disk $I=\onehalf\,mR^2$.

\startbuffer[TikZ:rollingomega]
\environment env_physics
\environment env_TikZ
\setupbodyfont [libertinus,11pt]
\setoldstyle % Old style numerals in text
\startTEXpage\small
\def\angles{72,54,...,-90}
\starttikzpicture% tikz code
	\fill [black!10] (0,0) rectangle (5,-.15);
		\draw[fill=black!10] (2,1.6) circle[radius=1.6];
		\fill(2,1.6) circle[radius=.4mm];
		\draw[->, thick](2,1.6) --node[above]{$v$} ++(2,0);
		\path ({2+tan(35.8)},2.6) coordinate (A) -- (2,1.6) coordinate (B) -- ({2-tan(35.8)},2.6) coordinate (C)
pic [draw, <-, thick, angle eccentricity=1.2, pic text=$\omega$, angle radius=8mm] {angle};
		\draw[thin] (0,0)--(5,0);
\stoptikzpicture
\stopTEXpage
\stopbuffer

\placefigure[margin][fig:rollingomega] % location
{The rolling disk’s total kinetic energy depends on the velocity $v$ of the disk’s center and the rotational velocity $\omega$. Because it is rolling, these are related by $v=R\omega$.}	% caption
{\noindent\typesetbuffer[TikZ:rollingomega]} % figure contents

\startformula
	K\sub{rot}
		= \textfrac{1}{2} I\omega^2
		= \textfrac{1}{2} \left(\textfrac{1}{2} mR^2\right)\omega^2
		= \textfrac{1}{4} m(R\omega)^2.
\stopformula
As the disk rolls, the point in contact with the ground does not slide – it is the one point on the disk that is not moving. For a rolling object of radius $R$ the center of mass velocity and angular velocity are linked by the rolling constraint:
\startformula
	v = R\omega.
\stopformula
This allows further simplification of the rotational kinetic energy.
\startformula
	K\sub{rot} = \textfrac{1}{4} m(R\omega)^2
		= \textfrac{1}{4} mv^2
\stopformula
The rolling disk’s total kinetic energy is
\startformula
	K = K\sub{CoM} + K\sub{rot}
		= \textfrac{1}{2}mv^2 + \textfrac{1}{4}mv^2
		= \textfrac{3}{4}mv^2.
\stopformula
A rolling object’s total kinetic energy is always larger than $\onehalf mv^2$, but how much larger depends on the object’s shape. An objects with the mass far from the center, like a hoop, will have a larger moment of inertia $I$ and will have more kinetic energy when rolling. An object with the mass distributed more toward the axis of rotation will have a smaller moment of inertia, and a smaller total kinetic energy when rolling, as seen in the next example.

%%%%%%%%%%%%%%%%%%%%%%%%%%%%%%%%%%%%%%%%%%%%%%%%%%%
\startexample[ex:RollingCannonBall]
A $40\units{kg}$ cannon ball is rolling on a smooth surface at a speed of $2.0\units{m/s}$, as shown in figure~\in[fig:RollingCannonBall]. What is its total kinetic energy?
\startbuffer[TikZ:RollingCannonBall]
\environment env_physics
\environment env_TikZ
\setupbodyfont [libertinus,11pt]
\setoldstyle % Old style numerals in text
\startTEXpage\small
\def\angles{72,54,...,-90}
\starttikzpicture% tikz code
	\fill [black!10] (0,0) rectangle (5,-.15);
		\draw[shade, ball color=black!50] (2,1) circle[radius=1];
		\fill(2,1) circle[radius=.4mm];
		\draw[->, thick](2,1) --node[above, pos=0.7]{$v$} ++(2,0);
		\path ({2+tan(57.3)},2) coordinate (A) -- (2,1) coordinate (B) -- ({2-tan(57.3)},2) coordinate (C)
pic [draw, <-, thick, angle eccentricity=1.2, pic text=$\omega$, angle radius=12mm] {angle};
		\draw[thin] (0,0)--(5,0);
\stoptikzpicture
\stopTEXpage
\stopbuffer

\placefigure[margin][fig:RollingCannonBall] % location
{The rolling cannon ball in example~\in[ex:RollingCannonBall].}	% caption
{\noindent\typesetbuffer[TikZ:RollingCannonBall]} % figure contents
\startsolution
We find the rotational kinetic energy first, using the moment of inertia for a solid ball and the rolling constraint, $v=R\omega$.
\startformula
	K\sub{rot}
		= \textfrac{1}{2} I\omega^2
		= \textfrac{1}{2} \left(\textfrac{2}{5} mR^2\right)\omega^2
		= \textfrac{1}{5} m(R\omega)^2.
		= \textfrac{1}{5} mv^2
\stopformula
The rolling cannon ball’s total kinetic energy is
\startformula
	K = K\sub{CoM} + K\sub{rot}
		= \textfrac{1}{2}mv^2 + \textfrac{1}{5}mv^2
		%= \textfrac{5}{10}mv^2 + \textfrac{2}{10}mv^2
		= \textfrac{7}{10}mv^2.
\stopformula
This formula is true for any rolling solid ball. Plugging in the mass and speed for this specific example gives
\startformula
	K = \textfrac{7}{10}mv^2.
		= \textfrac{7}{10}(40\units{kg})(2.0\units{m/s})^2.
		= 110\units{kg\.m^2/s^2}
		= 110\units{J}.
\stopformula
The total kinetic energy is $110\units{J}$.
\stopsolution
\stopexample
%%%%%%%%%%%%%%%%%%%%%%%%%%%%%%%%%%%%%%%%%%%%%%%%%%%

\section{Vis viva’s legacy}

Du Châtelet continued to contribute to the physics community by producing a complete translation of Newton’s \booktitle{Principia} into French. In addition to translating the text, she translated the mathematics from Newton’s geometric calculus to Leibniz’s more powerful algebraic notation, making the text much more useful.  She also produced a commentary and updated all of the astronomical data and calculations. This monumental intellectual achievement was also a physical feat, as she worked seventeen hours a day – while pregnant – to complete what she called \quotation{my Newton.} Genius, nobility, wealth, and fame could not reduce the perils of pregnancy and childbirth for an older woman in the eighteenth century. At the age of 42, she feared that her work would be left unfinished if she let it slip past her due date.

She was working after midnight when she went into labor. The birth was relatively easy. She retired to bed for a few days but continued to work, completing corrections on the printer's proofs and then sending all of her working papers  to the royal librarian. Then, on the sixth day, she developed a sudden headache and had trouble breathing. Perhaps a blood clot, formed during her time in bed, had broken free and lodged in her lungs. The royal physician and the most prominent doctors of the region were called, but they could offer only opiates for some relief. She died that night, September 10, 1749.

Du Chatlâtet's translation and comentary contributed greatly to the advancement of physics on the continent, but was largely ignored in England where Newton was so revered that all of the universities stubbornly continued using his cumbersome geometric methods.

On the continent, energy continued to play a central roll in the advancing physics, as we will see, and it continues to be central today. High-energy experiments done in the twentieth century revealed that the formula $K=\onehalf mv^2$ is not correct for speeds close to the speed of light ($3\sci{8}\units{m/s}$), just like the formula for momentum. Also like the formula for momentum, this kinetic energy formula is \emph{extremely} accurate for lower speeds (anything less than $10^7\units{m/s}$). For higher speeds we will need the updated formulas in Volume II.

Conservation of energy in its many forms is a bit more complicated that Leibniz’s original idea for conservation of \visviva, but added complexity also makes energy extremely useful. Food, fuel, and batteries all contain energy stored in other forms that can be converted into kinetic energy.

The conservation laws for momentum and energy have survived every experimental test and are thought to be among the most fundamental laws of nature. They can be used in every situation, from the interactions of quarks inside a proton to the stretching of space and time over the vast scales of the cosmos. Learn them and use them. They are treasures.

\subject{Notes}
%\placefootnotes[criterium=chapter]
\placenotes[endnote][criterium=chapter]

%\subject{Notes I want}
%\startitemize[n, packed]
%\item Galileo.
%\item Newton.
%\item Du Châtelet.
%\stopitemize

\blank[.5in]
\startblockquote\it
Lastly, let us think of fostering a taste for study, a taste which makes our happiness depend only on ourselves. Let us preserve ourselves from ambition, and, above all, let us be certain of what we want to be; let us choose for ourselves our path in life, and let us try to strew that path with flowers.\\
	\rightaligned{\it Discourse on Happiness}\\% final lines
	\rightaligned{\sc Emilie Du Châtelet}
	%\rightaligned{1706–1749}}
\stopblockquote

\stopchapter
\stopcomponent
%%%%%%%%%%%%%%%%%%%%%%%%%%%%%%%%%%%%%%%%%%%%%%%%%%%
%%%%%%%%%%%%%%%%%%%%%%%%%%%%%%%%%%%%%%%%%%%%%%%%%%%

%We can see how Du Châtelet’s Newtonian arguments work by looking again at the case of the cart pushing the block. The goal of this calculation is to find the work work done by the cart and block when they start with initial velocity $v\si$. We may find that work is proportoinal to their quantity of motion ($m\abs{v}$), their \visviva\ ($mv^2$), or something else entirely. A clear result should resolve the \visviva\ debate, at least for this case. I warn you that this discussion is the most advanced so far in this book. You may need to read it several times. I recommend ignoring all of the minus signs initially. Then, when you feel you understand the course of the argument, read it again paying attention to the valuable directional information provided by these signs.
%
%To calculate the work done during the slide we need the force exerted by the block on the track and the displacement of the block and cart from the beginning of the slide until the end. We do not know the force, but we can calculate the displacement $\Delta x$. We will start with displacement and hope for the best.
%
%The final displacement at the end of the slide is the product of average velocity $v\sub{ave.}$ during the slide and the duration of the slide $\Delta t$.
%\startformula
%	\Delta x = v\sub{ave.} \Delta t % = v\sub{ave.}\frac{mv\si}{F}
%\stopformula
%Since the cart and block slow uniformly from their initial velocity to a complete stop, their \emph{average} velocity during this time is half of the initial velocity, $v\sub{ave.} = \half v\si$.
%\startformula
%	\Delta x = \half v\si \Delta t
%\stopformula
%
%The time required for the block to slow to a complete stop can be found using Newton’s conservation of momentum.
%\startformula\startmathalignment
%	p\sf &= p\si + F\Delta t		\\
%	F\Delta t &= \cancel{p\sf} - p\si		\\
%	\Delta t &= -\frac{mv\si}{F}
%\stopmathalignment \stopformula
%Do not be alarmed by the minus sign. It will not give a negative time! The force is always opposite the motion. If the initial velocity positive, then the force is negative and the change in time is positive, as it should be.
%
%%We saw in Chapter \in[ch:Momentum] that the time required for the cart and block to reach a stop is
%%\startformula
%%	\Delta t = \frac{p\si}{F} = \frac{mv\si}{F}
%%\stopformula
%%Recall when we studied this problem in Chapter \in[ch:Momentum], we chose the velocity to be in the positive direction, so the momentum is positive as well. We also chose $F$ to be the force that the block exerts on the track, so that it is also positive.
%
%This change in time can now be used to find the displacement as the block slows to a stop.
%\startformula
%	\Delta x = \half v\si \left(-\frac{mv\si}{F}\right) = -\frac{mv\si^2}{2F}
%\stopformula
%Here we see that the displacement $\Delta x$ is proportional to the \visviva, not the momentum $mv$. (The minus sign ensures the displacement and the force are in opposite directions, as they should be.) This displacement similar to the effect that ’s Gravesande saw when he dropped balls into clay, also confirming \visviva\ rather than momentum as the correct predictor of the motion’s effect.
%
%
%%, just as we saw in the data for the cart pushing the block. In fact, we can use the slope of the line on the graph to determine the force.
%%\begin{gather*}
%%	\text{slope}=\frac{\text{rise}}{\text{run}}= \frac{mv^2}{\Delta x} = 2F \\
%%	F = \half \text{slope}
%%\end{gather*}
%%The force $F$ is half of the slope of the best fit line through the points on the graph of $mv^2$ vs.\ $\Delta x$.
%
%%’s Gravesande originally measured the impact craters in the clay because their size revealed how much work was done by the impacting sphere, work that came from the sphere’s \visviva. We measured the displacement of the sliding block because it likewise revealed the work done by the block. Since we have calculated the relationship between the block’s displacement and the \visviva, we can finish making the connection by finding the work done.
%Finally, we can use this displacement to find the work done by the block. Since $F$ is the force \emph{on} the block, the work done \emph{by} the block will be $W = -F\Delta x$. This work will be positive because the force on the block $F$ is opposite the displacement $\Delta x$.
%\startformula
%	W = -F\Delta x = -\cancel{F}\left(-\frac{mv\si^2}{2\cancel{F}}\right) = \half mv^2
%\stopformula
%%The force cancels out!
%The work done by the sliding block is half of the system’s initial \visviva. Stated the other way, \visviva\ is a measure of the work that can be done, which each unit of work requiring two units of \visviva. Leibniz is correct, as show by the careful use of Newtonian physics!
%
%The cancelation of the force $F$ deserves a bit more comment. If the force is big, then the cart and block only slide a short distance. If the force is small, then they slide farther, doing the same amount of work over a larger distance. The work done depends only on the initial \visviva\ of the cart and block. The force determines the distance over which that work is done.

%Today we always use half of the \visviva, $\half mv^2$, which is called the kinetic energy. The work $W$ done by the cart and block is equal to their initial kinetic energy $\half mv^2$. I will have more to say about kinetic energy in a moment. First, I wish to finish the story of Emily Du Châtelet and the \visviva\ debate.


%%%%%%%%%%%%%%%%%%%%%%%%%%%%%%%%%%%%%%%%%%%%%%%%%%%
Templates
%%%%%%%%%%%%%%%%%%%%%%%%%%%%%%%%%%%%%%%%%%%%%%%%%%%

% Margin image
\placefigure[margin][fig:LeibnizCollision]
	{Leibniz describes how two colliding objects are deformed during a collision, from \booktitle{A Specimen of Dynamics}.\autocite{}{Newton1726}}
	{\externalfigure[LeibnizCollision][width=144pt]}

% Margin diagram
\placefigure[margin] % location
{Diagram caption}	% caption text
{\starttikzpicture	% tikz code
	\draw (-1.5,0) -- (1.5,0);
	\draw (0,-1.5) -- (0,1.5);
\stoptikzpicture.}

% Single line formula
\startformula
\text{formula}
\stopformula

% Multi-line formula with = aligned
\startformula \startmathalignment
\NC \frac{L_{\text{low}}}{L_{\text{high}}} \NC= \frac{2}{1} \NR
\NC \frac{L_{\text{low}}}{L_{\text{high}}} \NC= \frac{1}{2} \NR
\NC L_{\text{high}} \NC= \frac{1}{2}L_{\text{low}} \NR
\stopmathalignment \stopformula

Text $\text{Text}$.Text

%%%%%%%%%%%%%%%%%%%%%%%%%%%%%%%%%
\section{Section Title}
%%%%%%%%%%%%%%%%%%%%%%%%%%%%%%%%%

Text.

\startformula \startmathalignment
\NC \frac{f_{\text{high}}}{f_{\text{low}}} \NC= \frac{2}{1} \NR
\NC f_{\text{high}} \NC= 2\,f_{\text{low}} = 2(27.5\units{Hz}) = 55\units{Hz} \NR
\stopmathalignment \stopformula

The frequency of A$_1$ is $55\units{Hz}$.
\bigskip

% Margin image
\placefigure[margin,none] {} {\externalfigure[Dummy][width=144pt]}

% Margin diagram
\placefigure[margin] % location
{Diagram caption}	% caption text
{\starttikzpicture	% tikz code
	\draw (-1.5,0) -- (1.5,0);
	\draw (0,-1.5) -- (0,1.5);
\stoptikzpicture.}

% textwidth figure
\placefigure {Caption for textwidth figure.} {\externalfigure[Dummy][width=\textwidth]}

% pagewidth figure. Horizontal position needs to be fixed.
\placefigure {Caption for pagewidth figure.} {\externalfigure[Dummy][width=377pt]}

% Single line formula
\startformula
\text{formula}
\stopformula

% Multi-line formula with = aligned
\startformula \startmathalignment
\NC \frac{L_{\text{low}}}{L_{\text{high}}} \NC= \frac{2}{1} \NR
\NC \frac{L_{\text{low}}}{L_{\text{high}}} \NC= \frac{1}{2} \NR
\NC L_{\text{high}} \NC= \frac{1}{2}L_{\text{low}} \NR
\stopmathalignment \stopformula

%%%%%%%%%%%%%%%%%%%%%%%%%%%%%%%%%%%%%%%%%%%%%%%%%%%
\startexample[ex:cartK]
	The block slides $16.0\units{cm}$ along the track against a friction force of $-0.50\units{N}$ (The force is negative, opposing the block’s positive motion). What is the work done by the block?
\startsolution
The work done by the block is found using the work formula, but take care to use the correct force between the block and the track. The work done by the block is calculated using the force \emph{of} the block, but we are given the force \emph{on} the block. Newton’s third law tells us that the force of the block on the track is equal and opposite to the force of the track on the block. Therefore, we will calculate the work using $F=0.50\units{N}$ for the force of the block on the track.
	\startformula\startmathalignment
		\NC W	\NC = F\Delta x \NR
		\NC		\NC = (0.50\units{N})(16.0\units{cm}) \NR
		\NC		\NC = \answer{8.0\units{N\.cm}}
	\stopmathalignment \stopformula
	The work done by the sliding block is $8.0\units{N\.cm}$ or $0.080\units{N\.m}$.
\stopsolution
\stopexample
%%%%%%%%%%%%%%%%%%%%%%%%%%%%%%%%%%%%%%%%%%%%%%%%%%%
