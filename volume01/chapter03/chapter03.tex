% !TEX useConTeXtSyncParser
\startcomponent c_chapter03
\project project_world
\product prd_volume01

\doifmode{*product}{\setupexternalfigures[directory={chapter03/images}]}

\setupsynctex[state=start,method=max] % "method=max" or "min"

%%%%%%%%%%%%%%%%%%%%%%%%%%%%%
\startchapter[title=The Quantity of Motion, reference=ch:Momentum]
%%%%%%%%%%%%%%%%%%%%%%%%%%%%%

\placefigure[margin,none]{}{\small
	\startalignment[flushleft]
I earnestly ask that everything be read with an open mind and that the defects in a subject so difficult may be not so much reprehended as investigated, and kindly supplemented, by new endeavors of my readers.
	\stopalignment
	\startalignment[flushright]
	{\booktitle{Philosophi\ae\ Naturalis\\Principia Mathematica}}\\
	{\sc Isaac Newton}\\
	1642–1727
	\stopalignment
}

%\sidebar{\epigraph{The things that so often vexed the minds of the ancient phillosophers\\And fruitlessly disturb the school with noisy debate\\We see right before our eyes, since mathematics drives away the cloud.\\Error and doubt no longer encumber us with mist;\\For the Keenness of a sublime intelligence has made it possible for us to enter\\The dewellings of the gods above and to climb the heights of heaven.}
%{{\scshape Edm. Halley}\\? – ?}}

%shifting focus to a more pressing question. the realization that first question is not why the stick continues to move after it leaves you hand, because it is perfectly natural for the stick to continue to move. Rather, it is necessary to ask why the stick eventually stops, for this requires some interaction that interferes with the sticks natural motion.

%Having learned from Galileo that uniform motion requires no explanation, natural philosophers turned their attention to nonuniform motion.
%Specifically, they sought to explain why the motion changes by the amount that it does?
%Motion can change gradually or abruptly in response to many types of forces – the gentle force of the air, the relentless pull of gravity, or the sudden blow from a collision. %Starting from Galileo’s qualitative understanding of why motion changes, we will try to predict how much motion will change in response to outside forces.

\Initial{G}{alileo’s correct reasoning about motion was this:} to get an object moving, you must give it some impetus.
The object will continue moving with that impetus until it interacts with something else and either loses some portion of its impetus or gains some new impetus. Galileo did not have a formula for impetus, so this reasoning is purely qualitative. %\footnote{Recall Salviati’s question on page \pageref{SalvImpetus}: \quotation{So, if it were propelled by some impetus forcibly impressed on it, what would its motion be and how long would it last?}}
To predict how much an object’s motion will change, the impetus idea needed to be replaced with something quantitative.

In 1644, six years after Galileo’s \booktitle{Two New Sciences}, Ren\'e Descartes presented his own ideas about motion in \booktitle{Principia Philosophi\ae }. As the title suggests, this book is philosophical and not based on experiments like Galileo’s. Descartes proposed a \quotation{quantity of motion} to replace impetus.
A moving object’s quantity of motion, according to Descartes, is equal to the object’s mass times its speed. He gave an example of two objects, one small but fast and another large but slow, which have the same quantity of motion.

\startblockquote
When one part of matter is moved twice as fast as the other, and the other is twice as big as the former, the smaller has just as much motion as the greater.\autocite{p.~53. Translation by Paige Moore}{Descartes1644}
\stopblockquote
In modern mathematical language, Descartes’s quantity of motion is $m\vabs{\vec{v}}$.
Since mass and speed are positive, Descartes’s quantity of motion is alway positive for a moving object, regardless of the motion’s direction. The quantity of motion is zero only if the object is at rest. %The total quantity of motion for a collection of objects is positive if any of them are moving and zero only if all of them are at rest.

Descartes, as was his style, did not write out the formula for his quantity of motion. Even if he had, he certainly would not have written $m\vabs{\vec{v}}$, because vectors had not been invented! He would have written $mv$ where $v$ is the speed. We are going to adopt this cleaner notation for speed while keeping the notation $\vec v$ for the velocity vector and $v_x$, $v_y$, and $v_z$ for the vector’s components.

In his \booktitle{Principia}, Descartes presented several applications of his quantity of motion, but he had stated the most important laws much earlier in \booktitle{The World}, completed in 1633 but not published in his lifetime. His first law is a concise restatement of Galileo’s observation that motion naturally continues.
\startblockquote
	The first [law] is that each individual part of matter always continues to remain in the same state unless collision with others constrains it to change that state. That is to say, if the part\dots%has some size, it will never become smaller unless others divide it; if it is round or square, it will never change that shape without others forcing it to do so; if it
	is stopped in some place, it will never depart from that place unless others chase it away; and if it has once begun to move, it will always continue with an equal force until others stop or retard it.\autocite{p.~61.}{Descartes1633}
\stopblockquote
Descartes then offers a second law that is quite new.
\startblockquote
	I suppose as a second rule that, when one of these bodies pushes another, it cannot give the other any motion except by losing as much of its own at the same time; nor can it take away from the other body’s motion unless its own is increased by as much. This rule, joined to the preceding, agrees quite well with all experiences in which we see one body begin or cease to move because it is pushed or stopped by some other.\autocite{p.~65.}{Descartes1633}
\stopblockquote
Motion is not created or destroyed, but passed from one object to another. The motion of individual objects change, \quotation{nevertheless,} Descartes declared, the total motion in the universe \quotation{has a certain fixed quantity} that never changes. Objects gain and lose motion, but in total, \quotation{motion is conserved.}\autocite{p.~53–4. Translation by Paige Moore}{Descartes1644}

Descartes applied his quantity of motion to the simplest interaction: a collision between two objects.
In this case the amount that one object speeds up could be determined based on how much the other object slows down.
He considered collisions between two identical objects and two different objects, objects with the same speed and with different speeds, objects traveling in the same direction and or in opposite directions. For each case Descartes used the constancy of total motion to predict the objects’ subsequent motion.
Several of Descartes’s predictions conflicted with easily observed actual collisions, but Descartes dismissed these failures. Descartes explained, wrongly, that his predictions were correct for perfectly hard bodies isolated from any outside interactions. He further claimed that experiments are always affected by unseen substances filling the space between objects. These substances, invented to excuse Descartes’s experimental failures, do not actually exist.

Not everyone accepted Descartes’s defense of his failed predictions.
Christiaan Huygens, who was well versed in Cartesian philosophy, produced a different set of principles which he presented in \booktitle{The Motion of Colliding Bodies}, published in 1656.
% De Motu Corporum Ex Percussione
Huygens explained that the \emph{direction} of the quantity of motion must be taken into account. He also found the quantity $mv^2$ is conserved in collisions with a good bounce (called \emph{elastic collisions}). The quantity $mv^2$ is not conserved in collisions with a poor bounce (called \emph{inelastic collisions}).

About that time, the recently established Royal Society in London put out a request for papers on the topic of collisions. Isaac Newton relates the results of that request, accompanied by the diagram shown in figure \ref{fig:PrincipiaPendula}.

\placefigure[margin][fig:PrincipiaPendula] % Location, Label
{This diagram from Newton’s \booktitle{Principia} shows how simple pendulums were used to study collisions.\autocite{p.~22.}{Newton1726}} % Caption
{\externalfigure[PrincipiaPendulaCollision][width=144pt]} % File

\startblockquote
	Sir Christopher Wren, Dr. John Wallis, and Mr. Christiaan Huygens, easily the foremost geometers of the previous generation, independently found the rules of the collisions and reflections of hard bodies, and communicated them to the Royal Society at nearly the same time, entirely agreeing with one another (as to these rules); and Wallis was indeed the first to publish what had been found, followed by Wren and Huygens. But Wren additionally proved the truth of these rules before the Royal Society by means of an experiment with pendulums, which the eminent Mariotte soon after thought worthy to be made the subject of a whole book.\autocite{p.~424–425.}{Newton1999}
\stopblockquote

%The experiment with pendulums, performed in 1668, revealed how the quantity of motion is transferred from one object to another in a collision. Perhaps one ball is released so that it strikes a second ball at rest. The first ball’s quantity of motion gets redistributed in the collision, causing the second ball to move while the first ball slows, stops, or even rebounds in the opposite direction.
%Measuring the ball’s masses and their velocities before and after the collision allowed the experimenters to test different formulas for the quantity of motion.
%
%While masses are easy to measure, accurately measuring velocity is more difficult. Wren’s demonstration solves the velocity problem through clever use of the pendulum properties we learned from Galileo in the last section of Chapter \ref{ch:Motion}.
Wren's demonstration was so wonderful that we are going to look at exactly how it worked. You could easily build your own pendulums, as I have, to perform these experiments yourself.

\startbuffer[PendulumCollisionWren]
	\matrix{
	\startaxis[
		margin cart track,
		axis x line* = middle,
		xticklabels=\empty,
		xlabel={},
	   	extra x ticks={20,25},
	   	extra x tick labels=\empty,
   		extra x tick style={grid=major},
		ymin=-1.5, ymax = 25,
		footnotesize, %clip=false,
	]
	\fill [on layer={axis background}] (-0.5,0) rectangle (49.5,-1.5)[opacity=.1];
	\coordinate (A) at (20,249);
	\coordinate (B) at (25,249);
	\pic (second) at (A) [rotate=deg(-15/245)] {pendulum=100};
	\pic (first) at (B) {small pendulum=100};
    \stopaxis\\
	\startaxis[
		margin cart track,
		axis x line* = middle,
		xticklabels=\empty,
		xlabel={},
	   	extra x ticks={20,25},
	   	extra x tick labels=\empty,
   		extra x tick style={grid=major},
		ymin=-1.5, ymax = 10,
		footnotesize, %clip=false,
	]
	\fill [on layer={axis background}] (-0.5,0) rectangle (49.5,-1.5)[opacity=.1];
	\coordinate (A) at (20,249);
	\coordinate (B) at (25,249);
	\pic (forth) at (A) {pendulum=100};
	\pic (third) at (B) {small pendulum=100};
    \stopaxis\\
	\startaxis[
		margin cart track,
		axis x line* = middle,
	   	extra x ticks={20,25},
	   	extra x tick labels=\empty,
   		extra x tick style={grid=major},
		ymin=-1.5, ymax = 10,
		footnotesize, %clip=false,
	]
	\fill [on layer={axis background}] (-0.5,0) rectangle (49.5,-1.5)[opacity=.1];
	\coordinate (A) at (20,249);
	\coordinate (B) at (25,249);
	\pic (sixth) at (A) [rotate=deg(8/245)] {pendulum=100};
	%\node at (sixth-center) {A};
	\pic (fifth) at (B) [rotate=deg(21/245)] {small pendulum=100};
	%\node at (fifth-center) {B};
    \stopaxis\\
	};
\stopbuffer

\marginTikZ{}{PendulumCollisionWren}{{The larger ball is pulled to the side and released (top). It swings to the center, striking the smaller ball (middle). After the collision, the balls slow to a stop (bottom).}} % vskip, name, caption

Two balls are hung from identical long strings so that they are barely touching, as shown in Newton’s illustration (fig.~\ref{fig:PrincipiaPendula}). One of the balls is pulled to the side and released (fig.~\ref{fig:PendulumCollisionWren}). The location where it is released determines the speed with which it strikes the other ball. Until the collision, the motion of the ball is simple harmonic, though only for one quarter-period.  The quarter-period motion can be seen in figure~\ref{fig:PendulumCollisionWrenGraph}. The maximum speed formula for simple harmonic motion (p.~\pageref{eq:vmaxT}) gives the speed of the incoming ball as it arrives at the collision.

\startbuffer[PendulumCollisionWrenGraph]
	\startaxis[
		footnotesize,
		width=2.13in,%\marginparwidth,
		y={1mm},%x={2mm},
		xlabel={$t$ (s)},
		xmin=0, xmax=1.57,
		xtick={0,0.5,1.0,1.5},
		minor x tick num=4,
		ylabel={$x$ (cm)},
	  	%every axis y label/.style={at={(ticklabel cs:0.5)},rotate=90,anchor=center},
		ymin=0, ymax=50,
		minor y tick num=4,
	   	extra y ticks={20,25},
	   	extra y tick labels=\empty,
   		extra y tick style={grid=major},
		clip mode=individual,
		]
		\addplot[thick,smooth,domain=0:0.785,samples=101]{20-15*cos(2*deg(x))};
		\addplot[thick,smooth,domain=0:0.785,samples=2]{25};
		\addplot[thick,smooth,domain=0.785:1.57,samples=101]{20-8*cos(2*deg(x))};
		\addplot[thick,smooth,domain=0.785:1.57,samples=101]{25-21*cos(2*deg(x))};
		%\startscope[opacity=.3,transparency group]
		\fill[opacity=.3](0,25) circle[radius=.4mm];
		\draw[shade, opacity=.3, ball color = white] (0,25) circle[radius=2mm];
		\draw[ball color = white, shade, opacity=.3] (0,5) circle[radius=3mm];
		\fill[opacity=.3](0,5) circle[radius=.4mm];
%		\draw  [semithick](-.4,-4) -- (-.4,52);
%		\draw  [semithick](.4,-4) -- (.4,52);
		\draw[shade, opacity=.3, ball color = white] (0.785,20) circle[radius=3mm];
		\fill[opacity=.3](0.785,20) circle[radius=.4mm];
		\draw[shade, opacity=.3, ball color = white] (0.785,25) circle[radius=2mm];
		\fill[opacity=.3](0.785,25) circle[radius=.4mm];
%		\draw  [semithick](.6,-4) -- (.6,52);
%		\draw  [semithick](1.4,-4) -- (1.4,52);
		\draw[shade, opacity=.3, ball color = white] (1.57,28) circle[radius=3mm];
		\fill[opacity=.3](1.57,28) circle[radius=.4mm];
		\draw[shade, opacity=.3, ball color = white] (1.57,46) circle[radius=2mm];
		\fill[opacity=.3](1.57,46) circle[radius=.4mm];
%		\draw  [semithick](1.6,-4) -- (1.6,52);
%		\draw  [semithick](2.4,-4) -- (2.4,52);
		%\stopscope
	\stopaxis
\stopbuffer

\marginTikZ{}{PendulumCollisionWrenGraph}{The position vs.~time graph for the collision experiment in Examples \ref{ex:PendulumCollisionWren1} and \ref{ex:PendulumCollisionWren2}.} % vskip, name, caption

\startexample[ex:PendulumCollisionWren1]
	Suppose Wren’s demonstration was done using pendulums with periods of 3.14 seconds. %The larger, left ball has a mass of 75 grams, while the smaller, right ball has a mass of 25 grams.
	The larger ball is pulled back a distance of 15.0 centimeters and released to strike the smaller ball, as shown in figure~\ref{fig:PendulumCollisionWren}. What is the larger ball’s velocity as it arrives at the collision? 	\startsolution
To find the speed at the collision point, we use the formula for the maximum speed of simple harmonic motion. The amplitude is the ball’s initial distance from its equilibrium point, $A=15.0\units{cm}$. (Vector notation is not required for one dimensional problems. For these I will continue to use $\abs{v}$ for speed and $v$ for velocity.)
	\startformula
		\abs{v}\sub{max} = \frac{2\pi A}{T}
			= \frac{2\cancel{\pi}\cdot15.0\units{cm}}{\cancel{3.14}\units{s}}
			= 30.0\units{cm/s}
	\stopformula
The speed does not tell us the velocity’s direction, but we can see that the ball is moving in the positive direction, so the velocity is $30.0\units{cm/s}$.
\stopsolution
\stopexample

The balls’ velocities immediately after the collision can be determined from the distances they travel after the collision. Their motions after the collision are again simple harmonic, with amplitudes equal to the distances the balls swing after the collision.

\startexample[ex:PendulumCollisionWren2]
	Use the data in Example~\ref{ex:PendulumCollisionWren1} and the distances in figure~\ref{fig:PendulumCollisionWren} or \ref{fig:PendulumCollisionWrenGraph} to find the velocity of each ball after the collision.
\startsolution
After the collision the large ball swings from its equilibrium point at $x=20.0\units{cm}$ to its stopping point at $x=28.0\units{cm}$. The amplitude of the motion is
	\startformula
		A = \Delta x = x\sf-x\si = 28.0\units{cm} - 20.0\units{cm} = 8\units{cm}
	\stopformula
This can be used to find the velocity of the large ball.
	\startformula
		\abs{v}\sub{max} = \frac{2\pi A}{T}
			= \frac{2\cancel{\pi}\cdot8.0\units{cm}}{\cancel{3.14}\units{s}}
			= 16.0\units{cm/s}
	\stopformula
The same procedure gives the velocity of the small ball, which swings from its equilibrium position at $x=25\units{cm}$ to a stop at $x=46\units{cm}$.
	\startformula
		\abs{v}\sub{max} = \frac{2\pi A}{T}
			= \frac{2\pi\,\Delta x}{T}
			= \frac{2\cancel{\pi}(46.0\units{cm}-25.0\units{cm})}{\cancel{3.14}\units{s}}
			= 42.0\units{cm/s}
	\stopformula
Both of the velocities are to the right, which is positive. The velocity of the large ball is $16.0\units{cm/s}$ and the velocity of the small ball is $42.0\units{cm/s}$.
\stopsolution
\stopexample

\noindent Since pendulums will be colliding frequently in this chapter, we will save work later by incorporating $A=\Delta x$ into the speed formula to get a formula for the pendulum collision velocities.
\startformula
	v = \frac{2\pi\,\Delta x}{T}
\stopformula
The velocity before the collision is determined from the displacement before the collision, which would be from the release point to the equilibium point. The velocity after the collision is determined from the displacement after the collision, which would be from the equilibrium point to the stopping point. This specialized formula is only for pendulum collisions.

To give a full demonstration, Wren would have performed collisions with both balls moving before the collision, not just one. This presents the problem of how to time the swings so that the balls reach their equilibrium positions at the same time. However, Galileo taught us that pendulums’ periods are independent of the mass and the amplitude. The balls will each take the same amount of time, one-quarter period, to reach the central collision point, even if they are different masses and are pulled back different distances. To time the collision correctly, simply release the balls at the same time!

\startbuffer[PendulumCollision3]
	\matrix{
	\startaxis[
		margin cart track,
		axis x line* = middle,
		xticklabels=\empty,
		xlabel={},
	   	extra x ticks={25,30},
	   	extra x tick labels=\empty,
   		extra x tick style={grid=major},
		ymin=-1.5, ymax = 25,
		footnotesize, %clip=false,
	]
	\fill [on layer={axis background}] (-0.5,0) rectangle (49.5,-1.5)[opacity=.1];
	\coordinate (B) at (25,249);
	\coordinate (A) at (30,249);
	\pic (first) at (B) [rotate=-4.68] {small pendulum=100};
	\pic (second) at (A) [rotate=-0.935] {pendulum=100};
    \stopaxis\\
	\startaxis[
		margin cart track,
		axis x line* = middle,
		xticklabels=\empty,
		xlabel={},
	   	extra x ticks={25,30},
	   	extra x tick labels=\empty,
   		extra x tick style={grid=major},
		ymin=-1.5, ymax = 10,
		footnotesize, %clip=false,
	]
	\fill [on layer={axis background}] (-0.5,0) rectangle (49.5,-1.5)[opacity=.1];
	\coordinate (B) at (25,249);
	\coordinate (A) at (30,249);
	\pic (third) at (B) {small pendulum=100};
	\pic (forth) at (A) {pendulum=100};
    \stopaxis\\
	\startaxis[
		margin cart track,
		axis x line* = middle,
	   	extra x ticks={25,30},
	   	extra x tick labels=\empty,
   		extra x tick style={grid=major},
		ymin=-1.5, ymax = 10,
		footnotesize, %clip=false,
	]
	\fill [on layer={axis background}] (-0.5,0) rectangle (49.5,-1.5)[opacity=.1];
	\coordinate (B) at (25,249);
	\coordinate (A) at (30,249);
	\pic (fifth) at (B) [rotate=-0.935] {small pendulum=100};
	\pic (sixth) at (A) [rotate=2.81] {pendulum=100};
    \stopaxis\\
	};
\stopbuffer

\marginTikZ{}{PendulumCollision3}{In this collision experiment, the balls arrive at the center at the same time, but with different velocities.} % vskip, name, caption

\startexample[ex:PendulumCollisionWren3]
	Wren continues his demonstration with the balls reversed. This time both balls are pulled to the side and released. The small ball catches up with the large ball and bounces off of it, as shown in figure \ref{fig:PendulumCollision3}. Find all of the velocities.
\startsolution
Before the collision, the small ball moves from $x=5.0\units{cm}$ to $x=25.0\units{cm}$.
	\startformula
		v = \frac{2\pi\,\Delta x}{T}
			= \frac{2\cancel{\pi}(25.0\units{cm}-5.0\units{cm})}{\cancel{3.14}\units{s}}
			= 40.0\units{cm/s}
	\stopformula
The large ball moves from $x=26.0\units{cm}$ to $x=30.0\units{cm}$.
	\startformula
		v = \frac{2\pi\,\Delta x}{T}
			= \frac{2\cancel{\pi}(30.0\units{cm}-26.0\units{cm})}{\cancel{3.14}\units{s}}
			= 8.0\units{cm/s}
	\stopformula
After the collision, the small ball moves from $x=25.0\units{cm}$ to $x=21.0\units{cm}$.
	\startformula
		v = \frac{2\pi\,\Delta x}{T}
			= \frac{2\cancel{\pi}(21.0\units{cm}-25.0\units{cm})}{\cancel{3.14}\units{s}}
			= -8.0\units{cm/s}
	\stopformula
The large ball moves from $x=30.0\units{cm}$ to $x=42.0\units{cm}$.
	\startformula
		v = \frac{2\pi\,\Delta x}{T}
			= \frac{2\cancel{\pi}(42.0\units{cm}-30.0\units{cm})}{\cancel{3.14}\units{s}}
			= 24.0\units{cm/s}
	\stopformula
All of the velocities above have the correct direction.
\stopsolution
\stopexample


\startbuffer[PendulumCollisionGraph3]
	\startaxis[
		footnotesize,
		width=2.13in,%\marginparwidth,
		y={1mm},%x={2mm},
		xlabel={$t$ (s)},
		xmin=0, xmax=1.57,
		xtick={0,0.5,1.0,1.5},
		minor x tick num=4,
		ylabel={$x$ (cm)},
	  	%every axis y label/.style={at={(ticklabel cs:0.5)},rotate=90,anchor=center},
		ymin=0, ymax=50,
		minor y tick num=4,
	   	extra y ticks={25,30},
	   	extra y tick labels=\empty,
   		extra y tick style={grid=major},
		clip mode=individual,
		]
		\addplot[thick,smooth,domain=0:0.785,samples=101]{25-20*cos(2*deg(x))};
		\addplot[thick,smooth,domain=0:0.785,samples=101]{30-4*cos(2*deg(x))};
		\addplot[thick,smooth,domain=0.785:1.57,samples=101]{25+4*cos(2*deg(x))};
		\addplot[thick,smooth,domain=0.785:1.57,samples=101]{30-12*cos(2*deg(x))};
		\startscope[opacity=.3,transparency group]
		\fill[opacity=.3](0,5) circle[radius=.4mm];
		\draw[opacity=.3, shade, ball color = white] (0,5) circle[radius=2mm];
		\draw[opacity=.3, shade, ball color = white] (0,26) circle[radius=3mm];
		\fill[opacity=.3](0,26) circle[radius=.4mm];
%		\draw  [semithick](-.4,-4) -- (-.4,52);
%		\draw  [semithick](.4,-4) -- (.4,52);
		\draw[opacity=.3, shade, ball color = white] (0.785,25) circle[radius=2mm];
		\fill[opacity=.3](0.785,25) circle[radius=.4mm];
		\draw[opacity=.3, shade, ball color = white] (0.785,30) circle[radius=3mm];
		\fill[opacity=.3](0.785,30) circle[radius=.4mm];
%		\draw  [semithick](.6,-4) -- (.6,52);
%		\draw  [semithick](1.4,-4) -- (1.4,52);
		\draw[opacity=.3, shade, ball color = white] (1.57,21) circle[radius=2mm];
		\fill[opacity=.3](1.57,21) circle[radius=.4mm];
		\draw[opacity=.3, shade, ball color = white] (1.57,42) circle[radius=3mm];
		\fill[opacity=.3](1.57,42) circle[radius=.4mm];
%		\draw  [semithick](1.6,-4) -- (1.6,52);
%		\draw  [semithick](2.4,-4) -- (2.4,52);
		\stopscope
	\stopaxis
\stopbuffer

\marginTikZ{}{PendulumCollisionGraph3}{The position vs.~time graph for the collision experiment in Example \ref{ex:PendulumCollisionWren3}.} % vskip, name, caption

\noindent
The direction of the pendulum’s velocity $v$ is the direction of the pendulum’s displacement $\Delta x$, so the pendulum collision formula gives the pendulum’s velocity -- including direction -- not just the speed. If the displacement is in the positive direction then the velocity is positive. If the displacement is in the negative direction then the velocity is negative, as is the case for the small ball after the collision in the Example \ref{ex:PendulumCollisionWren3} above. The full position vs.\ time graph for this collision, showing the rebound of the small ball, is in figure \ref{fig:PendulumCollisionGraph3}.


%\placefigure[margin][] % location
%{}	% caption text
%	\marginfig{PrincipiaSmall}
%	\caption[Newton’s \booktitle{Principia}]{The title page of Newton’s \booktitle{Principia} \autocite{Newton1687}}
%


The rules of collision required close attention to the direction of motion. Newton took note of this in his most important work, \booktitle{Philosophi\ae\ Naturalis Principia Mathematica}, where he made direction central to the study of all motion. Newton introduced his own quantity of motion $m\vec{v}$. This is a directional quantity. In one dimension, Newton’s quantity of motion is found by multiplying the mass, which is always positive, by the velocity, which can be positive or negative depending on the direction of motion. The resulting quantity of motion is positive for objects moving in the positive direction and negative for objects moving in the negative direction, just like velocity. This directional quantity of motion is the cornerstone of Newton’s physics.

Like Descartes, Newton did not write a formula for his quantity of motion. If he had, he would have written $mv$, just like Descartes, because vectors had not been invented. This ambiguity between speed and velocity caused tremendous confusion for decades. We are going to avoid that confusion by using vector notation. Before we dive into the mathematics of vectors, let us take a moment to note another promising replacement for Descartes’s disappointing $mv$.

%We have been using one dimensional language, but Newton understood this to work just as well in two or three dimensions. We will continue to use one dimensional language because all of the important ideas and many experiments require only one dimension. We will return to two and three dimensional language toward the end of the chapter.

Gottfried Leibniz also admired the contributions of Huygens, Wallis, and Wren, and he began developing his own system of mechanics. Unlike Newton, Leibniz adopted Huygens’s formula $mv^2$ as the important quantity. Leibniz called $mv^2$ the \textit{vis viva} or \quotation{living force} and made it the basis of his own theory of motion. The \textit{vis viva} is much closer in spirit to Descartes’s original quantity of motion because $mv^2$ is positive for any motion and zero only for systems completely at rest, just like Descartes’s $mv$.

The science community quickly accepted Newton’s theory of motion, although very few natural philosophers actually understood it. Newton’s many successful predictions and experimental verifications demonstrated the power of his methods, even to those who could not understand them.

The science community was rather skeptical of Leibniz’s \textit{vis viva}, which did not have the same level of experimental support, at least not in the mechanical laboratory experiments that scientists preferred.
%The \textit{quantitas motus} was the cornerstone of Newton’s comprehensive formulation of mechanics, put forth in the \booktitle{Philosophi\ae\ Naturalis Principia Mathematica}. The \booktitle{Principia} does not mention $mv^2$, so, they reasoned, the \textit{vis viva} must be incorrect. Given the tremendous stature of Newton, Leibniz’s $mv^2$ was largely dismissed in favor of Newton’s $mv$.
Even without the endorsement of the scientific community, Leibniz’s formula for \textit{vis viva} made useful predictions for many practical applications. Engineers and chemists adopted his formula and tested it in many experiments involving friction, fluids, and heat – things that scientists tried to avoid in motion experiments. The practical value of \textit{vis viva} suggested that it was both correct and important.

A rift developed between Leibniz’s supporters, who used $mv^2$, and Newton’s supporters, who used $mv$. Newton’s supporters were called the Cartesians because of the similarity between Newton’s formula $m\vec{v}$ and Descartes’s $mv$. Fittingly, they sometimes included direction, as Newton taught them, and sometimes they did not, reverting to Descartes’ directionless quantity of motion. It was a mess.

% Was Leibniz describing something new, or was he simply using the wrong formula for the quantity of motion?

This disagreement was a perfectly reasonable scientific dispute that could have been resolved with careful experiments and detailed observations.
Unfortunately, the \textit{vis viva} debate instead devolved into a battle of egos, carried out with great animosity, tainted by nationalism. Fueled by other philosophical and political disagreements, the conflict raged for almost half a century, continuing long after Newton’s and Leibniz’s deaths.
This unproductive turn was, in part, due to human nature, but conflicting experimental results also contributed to the confusion. Some experiments supported the formula $mv$ while others supported the formula $mv^2$.
In the long run, the acrimonious debate did lead to ingenious experiments, careful analysis, and eventually to profound insights.
The unexpected resolution became the foundation for all physics that followed.
\placefigure[margin][fig:GravesandeCollider] % Location, Label
{An apparatus for collision experiments described in Willem ’s Gravesande’s \booktitle{Elements of Mathematical Physics}, published in 1720. (The ’s preceding his last name is not a typo.)\autocite{p.~83.}{Gravesande1720}} % Caption
{\externalfigure[GravesandeCollider][width=144pt]} % File

%\placefigure[margin][fig:GravesandeData] % location
%{Data collected using the apparatus above.\autocite{p.~90.}{Gravesande1720}}	% caption text
%	\marginfig{GravesandeData}

In this chapter, we will study Newton’s directional quantity of motion, which is now called momentum.
His \booktitle{Principia} will be our guide.
In Chapter \ref{ch:VisViva}, we will return to Liebniz’s \textit{vis viva}.

\section{Mass and momentum}
Newton’s \booktitle{Principia} begins with two introductory chapters. The first chapter contains eight important definitions. The second chapter presents his famous three laws of motion. This introduction is followed by three books in which the laws are applied to everything from swinging pendulums and sinking stones to the orbits of planets.
%The first book describes motion that is free of resistance. The second Book describes motions involving resistance and fluids, as well as pendulums. The third and final book applies the lessons of book 1 to the motion of heavenly bodies.
The three books are filled with amazing analyses of the physical world and brilliant mathematics. I am tempted to share many of these gems with you, but to do so would lead us too far from our intended path. We will confine our attention to the two introductory chapters, where Newton presents his most enduring ideas.

The first chapter begins by defining the \quotation{quantity of matter,} which we call \keyterm{mass}. Compressing, stretching, or even melting an object does not change its mass, although it might change its volume. An object’s mass can only be changed by adding matter or taking matter away.

The mass of an object can be found by adding the masses of all of its parts. The mass of a cup and saucer is the mass of the cup plus the mass of the saucer.
Quantities that add in this manner are called \keyterm{extensive}. We will discover several other extensive quantities, but be aware that many quantities are \emph{not} extensive and should \emph{not} be added up. For example, temperature is not extensive. The temperature of a cup and saucer is not the temperature of the cup plus the temperature of the saucer. Extensive quantities are special and useful, allowing us to understand complicated systems in terms of simpler parts.

\placefigure[margin][] % Location, Label
{The astronaut has the same mass on the moon as on Earth. The scale, which measures the force required to hold up the astronaut, reads a lower \emph{weight} due to the weaker gravity on the moon} % Caption
{\externalfigure[astro][width=144pt]} % File

An object’s mass is not the same as the object’s weight, which is a measure of force required to support the object against the pull of gravity. It is convenient to weigh an object to find its mass, since an object with greater mass requires a proportionally greater supporting force. Metric scales convert the weight to mass automatically, displaying the mass in units of grams or kilograms. Scales can be fooled if they are used in a place where there are other forces acting (such as under water or in a moving vehicle) or where the gravitational pull is different (such as in space or on other planets). Scuba divers, sky divers, and astronauts do not  have less mass when they are diving, falling, or orbiting, but scales cannot accurately measure their mass during those activities. Please do all weighing in stationary, Earth-bound settings. We will discuss the relationship between weight and mass when we introduce gravitational force later in this chapter. For now, we will be concerned only with mass.

%\section{Momentum and impulse}
Newton’s second definition establishes the \textit{quantitas motus}, now called \keyterm{momentum}.
(From here on, I will translate the Latin \textit{quantitas motus}, and where appropriate, \textit{motus}, as \emph{momentum.})


\placefigure[margin,none]{}{\small
	\startalignment[flushleft]
\textit{{\sc Definitio II.} Quantitas motus est mensura ejusdem orta ex velocitate et quantitate materi\ae\ conjunctim.}\autocite{p.~1.}{Newton1726}
	\stopalignment
}

\startblockquote
	{\sc Definition 2 }
	\textit{Momentum is a measure of motion that arises from the velocity and the mass jointly.}\autocite{p.~404. \quotation{Quantity of motion} changed to \quotation{momentum,} \quotation{quantity of matter} to \quotation{mass.}}{Newton1999}
\stopblockquote
The symbol $\vec{p}$ represents momentum. Using our mathematical language, Newton's definition of momentum is%
%\highlightbox{%
\startformula
	\vec{p} = m\vec{v}.	%\tag*{\scaps{momentum}}
	%\label{eq:1Dp}
\stopformula
%}

This is a vector equation, which means that both sides of the equation are directional.
In one dimension, the momentum is found by multiplying the velocity by the mass. Since the mass is always positive, the momentum will be positive if the velocity is positive, and the momentum will be negative is the velocity is negative. The momentum always points in the same direction as the velocity.

\startbuffer[CartVelotictyToMomentumEx]
	\fill [black!10] (0,0) rectangle (4.9,-.15);
	\startaxis[
		margin cart track,
		ymax=14,
		legend style={draw=none, at={(1,1)}, anchor=south east},%yshift = 1ex},
	]
	\pic at (40,0) {cart};
	%\pic at (16,0) {cart};
    %\draw[->, thick] (0,15) --node[right=2.5mm]{$=100\units{g\.m/s}$} (5,15);
    \draw[->, thick] (40,2.5) --node[above=2.5mm]{$v=-60\units{cm/s}$} ++(-30,0);
		% Legend
   		\addlegendimage{
	  	legend image code/.code={\draw[thick,|-|](-0.5cm,0cm)--(0cm,0cm);}
 		};
		\addlegendentry{$=10\units{cm/s}$}
    \stopaxis
\stopbuffer

\marginTikZ{}{CartVelotictyToMomentumEx}{The cart’s velocity in \in{example}[ex:CartMomentum].} % vskip, name, caption

\startbuffer[CartMomentumEx]
	\fill [black!10] (0,0) rectangle (4.9,-.15);
	\startaxis[
		margin cart track,
		ymax=14,
		legend style={draw=none, at={(1,1)}, anchor=south east},%yshift = 1ex},
	]
	\pic at (40,0) {cart};
	%\pic at (16,0) {cart};
    %\draw[->, thick] (0,15) --node[right=2.5mm]{$=100\units{g\.m/s}$} (5,15);
    \draw[->, thick] (40,2.5) --node[above=2.5mm]{$p=-480\units{g\.m/s}$} (16,2.5);
		% Legend
   		\addlegendimage{
	  	legend image code/.code={\draw[thick,|-|](-0.5cm,0cm)--(0cm,0cm);}
 		};
		\addlegendentry{$=100\units{g\.m/s}$}
    \stopaxis
\stopbuffer

\marginTikZ{}{CartMomentumEx}{The cart’s momentum in \in{example}[ex:CartMomentum].} % vskip, name, caption

\startexample[ex:CartMomentum]
	A cart, whose mass is $800\units{g}$, is rolling on the track with a velocity of $-60\units{cm/s}$, as shown in \in{figure}[fig:CartVelotictyToMomentumEx]. Calculate the cart’s momentum.
	\startsolution
		Although it is fine to work with centimeters, we prefer meters. Convert the velocity to meters per second using the fact that the \quotation{c} in \quotation{cm} stands for $10^{-2}$.
	\startformula
		v = -60\units{cm/s} = -60\sci{-2}\units{m/s} = -0.60\units{m/s}
	\stopformula
	To calculate the momentum, use the momentum formula.
 	\startformula\startmathalignment
	\NC	p \NC = mv \NR
	\NC		\NC = (800\units{g})(-0.60\units{m/s}) \NR
	\NC		\NC = \answer{-480\units{g\.m/s}}
	\stopmathalignment\stopformula
	The momentum of the cart is $-480\units{g\.m/s}$, as shown in \in{figure}[fig:CartMomentumEx].
	The negative answer indicates that the momentum is in the negative direction. Other acceptable answers are $-48000\units{g\.cm/s}$ and $-0.48\units{kg\.m/s}$. Do not mix scientific notation and SI prefixes (for example $-4.8\sci{4}\units{g\.cm/s}$), since it is needlessly confusing.
	%KK-Do the students know what SI prefixes are?
\stopsolution
\stopexample

The momentum formula is even more useful in two or three dimensions, where there are more directions. However, all of Newton’s laws can be understood using one-dimensional examples, which we will do before conquering additional dimensions.

After defining momentum, Newton explains that momentum is an extensive quantity, like mass.
\startblockquote
	The momentum of a whole is the sum of the momenta of the individual parts, and thus if a body is twice as large as another and has equal velocity there is twice as much momentum, and if it has twice the velocity there is four times as much momentum. \autocite{p.~44. \quotation{Quantity of motion,} \quotation{motion} changed to \quotation{momentum.}}{Newton1999}
\stopblockquote

\startbuffer[CartBlockVelocity]
	\fill [black!10] (0,0) rectangle (4.9,-.15);
	\startaxis[
		margin cart track,
		ymax=11,
		legend style={draw=none, at={(1,1)}, anchor=south east},%yshift = 1ex},
	]
	%\pic at (40,0) {cart};
	\pic at (16,0) {cart};
    \draw[->, thick] (16,2.5) -- (30,2.5);% node[above=2.5mm]{$p=mv$}
	\pic at (16,5) {block};
    \draw[->, thick] (16,7.5) -- (30,7.5);% node[above=2.5mm]{$p=mv$}
		% Legend
   		\addlegendimage{
	  	legend image code/.code={\draw[thick,|-|](-0.5cm,0cm)--(0cm,0cm);}
 		};
		\addlegendentry{$=10\units{cm/s}$}
    \stopaxis
\stopbuffer

\marginTikZ{}{CartBlockVelocity}{Velocity vectors are shown for the cart and the block. The block is riding atop the cart, so the block and the cart have the same velocity.} % vskip, name, caption

\noindent
Taken to the extreme, the momentum of an object is the sum of the momenta of all of the atoms that make up the object.
When a car moves, every one of its atoms has some momentum. The total momentum of the car is the sum of the momenta of all of the atoms making up the car.
A locomotive traveling at the same velocity has far more atoms, so it also has far more momentum.
Displacement and velocity are \textit{not} extensive. If every atom in a cart is displaced by $30\units{cm}$ then the whole cart is displaced by $30\units{cm}$. You do not add the displacements of all of the atoms to get the displacement of the cart. Likewise, if every atom in the locomotive has a velocity of $-10\units{m/s}$, then the whole locomotive has a velocity of $-10\units{m/s}$. % The momentum in each atom of the cart or locomotive is quite small. They must be added to give object’s total momentum.
Objects moving together have the same velocity (fig. \ref{fig:CartBlockVelocity}), but can have different momenta (fig. \ref{fig:CartMomentum}).
The extensive and directional nature of Newton’s momentum makes it an extremely powerful tool for understanding motion.

\startbuffer[CartMomentum]
	\fill [black!10] (0,0) rectangle (4.9,-.15);
	\startaxis[
		margin cart track,
		ymax=11,
		legend style={draw=none, at={(1,1)}, anchor=south east},%yshift = 1ex},
	]
	%\pic at (40,0) {cart};
	\pic at (16,0) {cart};
    \draw[->, thick] (16,2.5) -- (40,2.5);% node[above=2.5mm]{$p=mv$}
	\pic at (16,5) {block};
    \draw[->, thick] (16,7.5) -- (20,7.5);% node[above=2.5mm]{$p=mv$}
		% Legend
   		\addlegendimage{
	  	legend image code/.code={\draw[thick,|-|](-0.5cm,0cm)--(0cm,0cm);}
 		};
		\addlegendentry{$=100\units{g\.m/s}$}
    \stopaxis
\stopbuffer

\marginTikZ{}{CartMomentum}{Momentum vectors are shown for the cart and the block. The cart’s momentum is greater than the block’s because the cart has a greater mass.} % vskip, name, caption

Newton’s remaining six definitions establish the many ways that he will use the term \textit{vis} or \quotation{force} in the \booktitle{Principia}.
%In the seventeenth and eighteenth century \quotation{force} was used to describe almost anything associated with a change in an object’s motion, or anything preventing such a change. There is only one
%It would be confusing to go through all of them since m
Most relate to specific situations, and some have been superseded by more fundamental concepts. We will only need the definition of the \textit{vis impressa}, or \quotation{impressed force,}  which we call \keyterm{impulse}.
(I will translate the Latin \textit{vis impressa}, and where appropriate \textit{vis}, as \emph{impulse.})

\placefigure[margin,none]{}{\small
	\startalignment[flushleft]
\textit{{\sc Definitio IV.} Vis impressa est actio in corpus exercita, ad mutandum ejus statum vel quiescendi vel movendi uniformiter in directum.}\autocite{p.~2.}{Newton1726}
	\stopalignment
}

\startblockquote
	{\sc Definition 4 }
	\textit{Impulse is the action exerted on a body to change its state either of resting or of moving uniformly straight forward.}\autocite{p.~405. \quotation{Impressed force} changed to \quotation{impulse,} \quotation{quantity of matter} to \quotation{mass.}}{Newton1999}
\stopblockquote
Impulse is always exerted by another object; impulse cannot originate within the object itself. Impulse is a directional quantity represented by the vector $\vec{J}$. In one dimension, the impulse’s sign determins whether it changes the motion of the object toward the positive direction or toward the negative direction.

\section{Newton’s first and second laws of motion}
With these definitions established, Newton presents his revolutionary three laws of motion.
The first law is a precise restatement of Galileo’s insight.
\placefigure[margin,none]{}{\small
	\startalignment[flushleft]
\textit{{\sc Lex I.} Corpus omne perseverare in statu suo quiescendi vel movendi uniformiter in directum, nisi quatenus illud a viribus impressis cogitur statum suum mutare.}\autocite{p.~13.}{Newton1726}% Third edition wording.
	\stopalignment
}
\startblockquote
	{\sc Law 1 }
	\textit{Every body perseveres in its state of being at rest or of moving uniformly straight forward, except insofar as it is compelled by impulse to change its state.}\autocite{p.~416. \quotation{Force impressed}  changed to \quotation{impulse.}}{Newton1999}
		%Projectile persevere in their motions, except insofar as they are retarded by the resistance of the air and are impelled downward by the force of gravity.... And larger bodies – planets and comets – preserve for a longer time both p%\autocite{Newton1999}%\footnote{ Isaac Newton, The Principia, A new translation by I.B. Cohen and A. Whitman, University of California press, Berkeley 1999.}
\stopblockquote
In modern language, the first law states that an object’s velocity is unchanged except when it is acted upon by an impulse from another object. Only  interactions with other objects can change the speed or direction of an object’s motion. These interactions often happen through direct contact – friction stopping a sliding block or wind blowing a leaf.  But other interactions happen at a distance – Earth’s gravity pulling an apple downward or bending the moon’s path into a circular orbit.

Constant momentum means constant speed and unchanging direction. In the absence of an outside force, objects move in straight lines. Uniform circular motion, which Galileo thought was perfectly natural, is \emph{not} natural in Newton’s view. The changing direction requires some outside force. All of the solar system’s motions – the circular orbits of the planets and their moons – require constant interactions.

Newton’s second law describes how the object’s momentum is altered by an impulse from another object.

\placefigure[margin,none]{}{\small
	\startalignment[flushleft]
\textit{{\sc Lex II.} Mutationem motus proportionalem esse vi motrici impress\ae, \& fieri secundum lineam rectam qua vis illa imprimitur.}\autocite{p.~13.}{Newton1726}
	\stopalignment
}
\startblockquote
	{\sc Law 2 }
	\textit{A change in momentum is\dots the impulse and takes place along the straight line in which that impulse is impressed.}\autocite{p.~416. \quotation{Motion} changed to \quotation{momentum,} \quotation{motive force impressed} to \quotation{impulse.}}{Newton1999}
\stopblockquote
This law originated in collision experiments like those described at the beginning of the chapter. If one ball is released so that it strikes another ball at rest, then the ball at rest receives an impulse from the incoming ball. The ball at rest gains momentum due to the impulse, impelling it into motion.
Greater impulse delivered by the incoming ball gives the ball at rest a proportionally greater momentum.

The incoming ball also receives an impulse from the ball at rest. This impulse changes the incoming ball’s momentum. Since the direction of this impulse is opposite the direction of the incoming momentum, the impulse will decrease or even reverse the momentum of the incoming ball.

The delivered impulse $\vec{J}$ is added to the ball’s initial momentum to produce the final momentum.
%\highlightbox{
\startformula
	\vec{p}\si + \vec{J} = \vec{p}\sf
	%\label{eq:1DpconserveJ}
\stopformula
%}%\end{shaded}
Impulse, like the momentum, is directional. If the impulse is in the same direction as the initial momentum (either positive or negative), then they will have the same sign and the impulse will increase the momentum’s magnitude, (making it either \emph{more} positive or \emph{more} negative). If the impulse is in the direction opposite the momentum, they will have opposite signs and the impulse will decrease or even reverse the momentum.

\startbuffer[PendulumCollisionWrenDup]
	\matrix{
	\startaxis[
		margin cart track,
		axis x line* = middle,
		xticklabels=\empty,
		xlabel={},
	   	extra x ticks={20,25},
	   	extra x tick labels=\empty,
   		extra x tick style={grid=major},
		ymin=-1.5, ymax = 25,
		footnotesize, %clip=false,
	]
	\fill [on layer={axis background}] (-0.5,0) rectangle (49.5,-1.5)[opacity=.1];
	\coordinate (A) at (20,249);
	\coordinate (B) at (25,249);
	\pic (second) at (A) [rotate=deg(-15/245)] {pendulum=100};
	\pic (first) at (B) {small pendulum=100};
    \stopaxis\\
	\startaxis[
		margin cart track,
		axis x line* = middle,
		xticklabels=\empty,
		xlabel={},
	   	extra x ticks={20,25},
	   	extra x tick labels=\empty,
   		extra x tick style={grid=major},
		ymin=-1.5, ymax = 10,
		footnotesize, %clip=false,
	]
	\fill [on layer={axis background}] (-0.5,0) rectangle (49.5,-1.5)[opacity=.1];
	\coordinate (A) at (20,249);
	\coordinate (B) at (25,249);
	\pic (forth) at (A) {pendulum=100};
	\pic (third) at (B) {small pendulum=100};
    \stopaxis\\
	\startaxis[
		margin cart track,
		axis x line* = middle,
	   	extra x ticks={20,25},
	   	extra x tick labels=\empty,
   		extra x tick style={grid=major},
		ymin=-1.5, ymax = 10,
		footnotesize, %clip=false,
	]
	\fill [on layer={axis background}] (-0.5,0) rectangle (49.5,-1.5)[opacity=.1];
	\coordinate (A) at (20,249);
	\coordinate (B) at (25,249);
	\pic (sixth) at (A) [rotate=deg(8/245)] {pendulum=100};
	%\node at (sixth-center) {A};
	\pic (fifth) at (B) [rotate=deg(21/245)] {small pendulum=100};
	%\node at (fifth-center) {B};
    \stopaxis\\
	};
\stopbuffer

\marginTikZ{}{PendulumCollisionWrenDup}{The larger ball is pulled to the side and released (top). It swings to the center, striking the smaller ball (middle). After the collision, the balls slow to a stop (bottom).} % vskip, name, caption

\startexample[ex:PendulumImpulse]
Return to the collision described in Examples \ref{ex:PendulumCollisionWren1} and \ref{ex:PendulumCollisionWren2}, where a large ball strikes a stationary, smaller ball. (The diagram showing the collision is reproduced here as figure \ref{fig:PendulumCollisionWrenDup}.) Use the results of those two examples to find the impulse delivered to the large ball in the collision. The large ball has a mass of $165\units{g}$.
	\startsolution
	In Examples \ref{ex:PendulumCollisionWren1} and \ref{ex:PendulumCollisionWren2} the velocity of the large ball was found both just before and just after the collision. To find the impulse we will start with Newton’s second law, inserting the formula for momentum. In one dimension Newton’s second law does not require vector notation.
 	\startformula\startmathalignment
	\NC	p\si + J \NC = p\sf \NR
	\NC	J \NC = p\sf - p\si \NR
	\NC		\NC = mv\sf - mv\si \NR
	\NC		\NC = m(v\sf - v\si) \NR
	\NC		\NC = (165\units{g})(0.160\units{m/s}-0.300\units{m/s}) \NR
	\NC		\NC = \answer{-23.1\units{g\.m/s}}
	\stopmathalignment\stopformula
	The large ball received an impulse of $-23.1\units{g\.m/s}$.
	While the large ball continued to move in the positive direction, it received an impulse in the negative direction, opposite its motion. This impulse slowed the large ball, but was not enough change its direction.
\stopsolution
\stopexample

Newton makes the important observation that the impulse need not be delivered all at once.
\startblockquote
	If some impulse generates any momentum, then twice the impulse will generate twice the momentum, and three times the impulse will generate three times the momentum, whether the impulse is all at once or successively by degrees.\autocite{p.~416. \quotation{Force} changed to \quotation{impulse,} \quotation{motion} to \quotation{momentum.}}{Newton1999}
\stopblockquote
If a large ball is struck by three smaller balls, it does not matter if the three smaller balls strike at the same time or in some sequence. The change in the momentum of the larger ball will be the total impulse delivered, whether delivered all at once or in succession.
To calculate the final momentum after the three impacts, we can simply add all three impulses (represented by $\vec{J}_1$, $\vec{J}_2$, and $\vec{J}_3$) to the initial momentum.
\startformula
	\vec{p}\si + \vec{J}_1 + \vec{J}_2 + \vec{J}_3 = \vec{p}\sf
\stopformula
Objects in our everyday lives are constantly interacting with other objects, so there are often many impulses that must be added. We will represent total of all of these impulses as $\vec{J}\sn$, with the \quotation{net} reminding us to catch \emph{all} the impulses and add them when finding the change in momentum.
\startformula
	\vec{p}\si + \vec{J}\sn = \vec{p}\sf
\stopformula
As problems get more complicated, this reminder will become quite valuable.

\startbuffer[BlockSlide]
	\matrix{
	\startaxis[
		margin cart track,
		axis x line* = middle,
		xticklabels=\empty,
		xlabel={},
		ymin=-1.5, ymax = 8,
		footnotesize, %clip=false,
%		legend style={draw=none, at={(1,1)}, anchor=south east},%yshift = 1ex},
	]
	\fill [on layer={axis background}] (-0.5,0) rectangle (49.5,-1.5)[opacity=.1];
	\pic at (10,0) {block};
    \draw[->, thick] (10,2.5) --node[above]{$p\si$} (25,2.5);
    \draw[->, thick] (10,0.25) --node[above, pos=.6]{$F$} (0,0.25);
%		% Legend
%   		\addlegendimage{
%	  	legend image code/.code={\draw[thick,|-|](-0.5cm,0cm)--(0cm,0cm);}
% 		};
%		\addlegendentry{$=100\units{g\.m/s}$}
    \stopaxis\\
	\startaxis[
		margin cart track,
		axis x line* = middle,
		xticklabels=\empty,
		xlabel={},
		ymin=-1.5, ymax = 8,
		footnotesize, %clip=false,
	]
	\fill [on layer={axis background}] (-0.5,0) rectangle (49.5,-1.5)[opacity=.1];
	\pic at (30,0) {block};
    \draw[->, thick] (30,2.5) --node[above, pos=.6]{$p$} (40,2.5);
    \draw[->, thick] (30,0.25) --node[above, pos=.6]{$F$} (20,0);
    \stopaxis\\
	\startaxis[
		margin cart track,
		axis x line* = middle,
		xticklabels=\empty,
		xlabel={},
		ymin=-1.5, ymax = 8,
		footnotesize, %clip=false,
	]
	\fill [on layer={axis background}] (-0.5,0) rectangle (49.5,-1.5)[opacity=.1];
	\pic at (42,0) {block};
    \draw[->, thick] (42,2.5) -- (47,2.5)node[above]{$p$};
    \draw[->, thick] (42,0.25) --node[above, pos=.6]{$F$} (32,0);
    \stopaxis\\
	\startaxis[
		margin cart track,
		axis x line* = middle,
		ymin=-1.5, ymax = 8,
		footnotesize, %clip=false,
	]
	\fill [on layer={axis background}] (-0.5,0) rectangle (49.5,-1.5)[opacity=.1];
	\pic at (46,0) {block};
    \node[left] at (43,3) {$p\sf = 0$};
    \stopaxis\\
	};
\stopbuffer

\marginTikZ{}{BlockSlide}{A block with initial momentum $p\si$ slides to a stop. The friction between the block and the track exerts a force $F$ on the block opposing the block's motion. This force gradually reduces the block's momentum until the block slows to a stop. As the momentum gets smaller, the displacements also get smaller.} % vskip, name, caption
In many cases, objects will receive an impulse over an extended time, like the sliding block in \in{figure}[fig:BlockSlide].
The block has initial positive momentum $p\si$, and is sliding in the positive direction. Friction between the block and the track produces a force in the negative direction, opposing the block's motion. This force gradually reduces the block's momentum, slowing the block to a stop. During the slide, the force of friction delivers and impulse that is equal and opposite to the block's initial momentum – just enough impulse to bring the block's final momentum to zero.
This impulse does not arrive in a brief strike, but is delivered gradually over the duration of the slide. \keyterm{Force} is the rate at which impulse is delivered. The total impulse delivered is the force $\vec{F}$ multiplied by its duration $\Delta t$.
%\highlightbox{
\startformula
	\vec{J} = \vec{F}\Delta t	%\tag*{\emph{impulse}}
	%\label{eq:J}
\stopformula
%}

To illustrate, consider two identical drinking glasses dropped onto the floor, one from a considerable height and one from close to the floor, as shown in \in{figure}[fig:Glass1].
%\footnote{You should already be thinking, \quotation{science fair project!}}
Both glasses  will experience the same force of gravity as they fall. However, the glass dropped from a greater height will fall for a longer time, receive a greater impulse, and therefore arrive at the floor with a greater momentum than the glass dropped from a lower height.

\placefigure[margin][fig:Glass1] % Location, Label
{A drinking glass falling from a greater height is more likely to be broken by the floor’s stopping force.} % Caption
{\externalfigure[glass1][width=144pt]} % File

The floor will exert an impulse to stop both glasses.
% This impulse is delivered in a much shorter time.
The glass with the greater momentum (the one dropped from higher) will require a greater impulse to stop. The impulses will be delivered in approximately the same time for both glasses, so the one receiving the greater impulse (again, the one dropped from higher) will also experience greater force and is more likely to break.

Consider again two glasses, this time falling from the same height, but one landing on tile and one on carpet, as shown in \in{figure}[fig:Glass2]. In this case, they both experience the same force of gravity for the same duration of fall, so they have the same momentum as they arrive at the floor. Both must be stopped, so they each receive the same impulse from the floor. However, the tile floor will stop the glass in a very short time, while the carpet will compress, stoping the glass over a longer time. For the tile to deliver the same impulse in a shorter time requires a greater force. The glass landing on the tile experiences this greater force and is more likely to break.

\placefigure[margin][fig:Glass2] % Location, Label
{A drinking glass falling onto a harder floor is more likely to be broken by the floor’s stopping force.} % Caption
{\externalfigure[glass2][width=144pt]} % File

When finding a change in momentum, we will almost always need to compute the impulse from the force and the time. When there are multiple forces acting, the total impulse $\vec{J}\sn$ is due to the total force $\vec{F}\sn$.
Thus we have the useful \keyterm{momentum update formula}.
\startformula\pagereference[eq:1Dpconserve]
	\vec{p}\si + \vec{F}\sn \Delta t = \vec{p}\sf
\stopformula

An object’s initial momentum, plus the total force acting on the object multiplied by the forces’ duration, is the object’s final momentum.

The standard metric unit of force is a newton, represented by $\unit{N}$.
\startformula\pagereference[eq:NetwonUnit]
	1\units{N} = 1\units{kg\.m/s^2}
\stopformula
The following example shows how to convert newtons in a typical momentum update problem.

\startbuffer[BlockSlideEx]
	\startaxis[
		margin cart track,
		axis x line* = middle,
		ymin=-1.5, ymax = 8,
		legend style={draw=none, at={(1,1)}, anchor=south east},%yshift = 1ex},
	]
	\fill [on layer={axis background}] (-0.5,0) rectangle (49.5,-1.5)[opacity=.1];
	\pic at (10,0) {block};
    \draw[->, thick] (10,2.5) --node[above]{$p\si$} (25,2.5);
    \draw[->, thick] (10,0.25) --node[above, pos = .6]{$F$} (0,0.25);
		% Legend
   		\addlegendimage{
	  	legend image code/.code={\draw[thick,|-|](-0.5cm,0cm)--(0cm,0cm);}
 		};
		\addlegendentry{$=100\units{g\.m/s}$}
    \stopaxis
\stopbuffer

\marginTikZ{}{BlockSlideEx}{A block with initial momentum $p\si$ slides to a stop.} % vskip, name, caption

\startexample[ex:BlockSlide]
A block with an initial momentum of $p\si=150\units{g\.m/s}$ slides on a track, as shown in \in{figure}[fig:BlockSlideEx]. The friction with the track produces a gentle backwards force of $-0.30\units{N}$. (Negative because it opposes the block’s positive velocity.) How long will it take for the block to come to a stop?
\startsolution
This one-dimensional problem can be solved with the momentum update formula. There is only one force, $F\sn=-0.30\units{N}$. The block comes to rest so the final momentum is $p\sf=0$. Solve to find the duration of the slide, $\Delta t$.
	\startformula\startmathalignment
	\NC	p\si + F\sn\Delta t \NC = p\sf \NR
	\NC	F\Delta t \NC = \cancel{p\sf} - p\si \NR
	\NC	\Delta t \NC = \frac{-p\si}{F} \NR
	\NC		\NC = \frac{-p\si}{F} \NR
	\NC		\NC = \frac{-150\units{\ucan{g}\.\ucan{m}/\ucan{s}}}{-0.30\units{\ucan{kg}\.\ucan{m}/s^{\ucan{2}}}}
				\left(\frac{1\units{\cancel{kg}}}{1000\units{\cancel{g}}}\right) \NR
	\NC		\NC = \answer{0.50\units{s}}
	\stopmathalignment\stopformula
	The block slides for $0.50\units{s}$ before friction brings it to a stop.
\stopsolution
\stopexample

Newton gives credit to Galileo for the first and second laws. This attribution is probably too generous. Galileo viewed circular motion as perfectly natural. Descartes recognized that without an outside force an object will always move in a straight line, not a circle. Descartes’s straight line motion became Newton’s first law. Galileo also did not quantify the role of mass in his impetus, so he was only able to analyze motions that are independent of mass: free fall and projectile motion. Descartes quantified motion, but incorrectly. Newton’s directional momentum made his second law truly original.

\section{Newton’s third law and conservation of momentum}

Newton’s third law is entirely original.
The third law does not relate directly to an object’s motion, but instead describes the reciprocity between the impulses or forces that two objects exert on each other. The word \textit{actioni}, or \quotation{action,} in the third law refers equally well to either impulse or force.
\placefigure[margin,none]{}{\small
	\startalignment[flushleft]
\textit{{\sc Lex III.} Actioni contrariam semper et \ae qualem esse reactionem: sive corporum duorum actiones in se mutuo semper esse \ae quales et in partes contrarias dirigi.}\autocite{p.~14.}{Newton1726}
	\stopalignment
}
\startblockquote
	{\sc Law 3 }
	\textit{To any action there is always an equal and opposite reaction; in other words, the actions of two bodies upon each other are always equal and always opposite in directions.}\autocite{p.~417.}{Newton1999}
\stopblockquote
Newton immediately illustrates his third law with some examples.
\startblockquote
	Whatever presses or draws something else is pressed or drawn just as much by it. If anyone presses a stone with a finger, the finger is also pressed by the stone. If a horse draws a stone tied to a rope, the horse will (so to speak) also be drawn back equally toward the stone, for the rope, stretched out at both ends, will urge the horse toward the stone and the stone toward the horse by one and the same endeavor to go slack and will impede the forward motion of the one as much as it promotes the forward motion of the other. If some body impinging upon another body changes the momentum of that body in any way by its own force, then, by the force of the other body (because of the equality of their mutual pressure), it also will in turn undergo the same change in its own momentum in the opposite direction. By means of these actions, equal changes occur in the momenta, not in the velocities – that is, of course, if the bodies are not impeded by anything else. For the changes in velocities that likewise occur in opposite directions are inversely proportional to the masses because the momenta are changed equally. This law is valid also for attractions\dots.\autocite{p.~417. \quotation{Motion}  changed to \quotation{momentum,} \quotation{motions} to \quotation{momenta.}}{Newton1999}%, as will be proved in the next scholium.
\stopblockquote

\placefigure[margin][] % Location, Label
{} % Caption
{\externalfigure[finger][width=144pt]} % File
\placefigure[margin][] % Location, Label
{} % Caption
{\externalfigure[horse][width=144pt]} % File

\noindent
Newton provides these examples because the third law is quite surprising and easily misunderstood. It is reasonable to think that you throw a stone by exerting a greater force on the stone than the stone exerts on you. Newton informs us that this intuition is \emph{wrong}. The forces are \emph{equal} and opposite.
%The stone does not move because you exert a greater force on the stone than the stone exerts on you.
The stone moves because you exert a force on the stone, and there is no opposing force \emph{on the stone} to stop the stone’s motion. When the only force on the stone is the throwing force, the stone will be thrown.

The stone exerts an equal and opposite force on you, but you are standing on the ground. The ground, through friction, exerts a force \emph{on you} opposing any motion. The force exerted by the ground \emph{on you} cancel’s the force of the stone \emph{on you}, so you do not move.

Throw a stone while floating in space, and you will find yourself drifting backwards with a momentum equal and opposite to the stone’s forward momentum. In this case, you have thrown the stone, and the stone has also thrown you.

%Newton’s third law is a crucial piece of the motion puzzle.

%\section{Conservation of momentum}

Before applying the laws of motion to specific physical questions, Newton proves a set of corollaries, or logical consequences, of the laws.
Corollary 3 expresses the important principle of \keyterm{conservation of momentum}, stating that the total momentum of a collection of objects is not changed by the forces that those objects exert on each other.
\placefigure[margin,none]{}{\small
	\startalignment[flushleft]
	\textit{{\sc Corollarium III.} Quantitas motus qu\ae\ colligitur capiendo summam motuum factorum ad eandum partem, \& differentiam factorum ad contrarias, non mutatur ab actione corporum inter se.}\autocite{p.~17.}{Newton1726}
	\stopalignment
}
\startblockquote
	{\sc Corollary 3 }
	\textit{The momentum, which is determined by adding the momenta in one direction and subtracting the momenta in the opposite direction, is not changed by the action of bodies on one another.}\medskip

	 For an action and the reaction opposite to it are equal by law 3, and thus by law 2 the changes which they produce in momenta are equal and in opposite directions. Therefore, \dots% if motions are in the same direction,
	whatever is added to the momentum of the first body %[\textit{lit.} the fleeing body]
	will be subtracted from the momentum of the second body %[\textit{lit.} the pursuing body]
	in such a way that the sum remains the same as before.\autocite{p.~420. \quotation{Quantity of motion} changed to \quotation{momentum,} \quotation{motions} to \quotation{momenta.}}{Newton1999} %But if the bodies meet head-on, the quantity subtracted from each of the motions will be the same, and thus the difference of the motions made in opposite directions will remain the same.
\stopblockquote
The total momentum of a collection of objects cannot be changed by forces exerted by those objects on each other.
The total momentum of the collection can only be changed by forces exerted by other objects \emph{outside} of the collection.

Corollary 3 does not give a new equation, but it does give a new interpretation to % equation \eqref{eq:1Dpconserve},
the momentum update formula. We will also call this the law of \keyterm{conservation of momentum}.
\highlightbox{
\startformula
	\vec{p}\si + \vec{F}\sn\Delta t = \vec{p}\sf %\tag{\ref{eq:1Dpconserve}}
\stopformula
}
In the conservation of momentum equation, the momentum can refer to the momentum of a single object, or to the total momentum of a collection of objects. The collection of objects is called a \keyterm{system}. The initial and final momenta are the \emph{total} initial and final momenta of all of the parts of system. The force on the left-hand side of the equation is the total force exerted on the system by objects outside of the system.

Choosing the system wisely can make a problem much easier. In a collision between two balls, you can use one ball as your system, so that you only have one initial momentum and one final momentum. In this case, the second ball will provide the external force that alters the first ball’s momentum.
Alternatively, you could include both balls in the system. Then you must include the momentum of \emph{both} balls in the total initial momentum and in the total final momentum. In this case the external force is zero because there are no external forces on the two balls. (Actually, for a pendulum experiment, strings exert an upward force on the balls and gravity pulls down, but these forces are perpendicular to the direction along which the collision is occurring. Perpendicular forces like these can be ignored if the path is fixed.)

The wise choice is the system including both balls, which allows us to ignore the force between the balls. This force and its duration are difficult to determine, and ultimately unimportant for determining what happens in the collision. Newton illustrates with a simple example.

Before turning to Newton’s example, I would like you to notice one thing that may have seemed a bit odd in Corollary 3. Newton determines the total momentum by \quotation{adding the momenta in one direction and subtracting the momenta in the other direction.} We have not been doing that. We add all of the momenta, but the momenta in one direction are positive, and the momenta in the other direction are negative. Adding $-5$ to something is the same as subtracting $5$, so we get the same answer as Newton. However, things get complicated for Newton when there is a second object, because the two objects might be moving in the same direction or opposite directions, which means special cases to consider. Using positive and negative to represent direction avoids all of that. We just add, no matter what direction things are going, and let the signs take care of themselves. %I suspect that Newton did not use this approach because vectors, the basis for our method, were not invented until the second half of the nineteenth century, almost two hundred years after the publication of the \booktitle{Principia}.
I removed these extra cases from the proof of Corollary 3 above, but in the example below you will see him subtracting a momentum rather than adding a negative momentum. It should not be too difficult to follow.

Newton also does not use any specific units, just \quotation{parts} for everything. When he says that body $A$ \quotation{has two parts of velocity} and body $B$ has \quotation{ten parts of velocity} he is saying that $B$ is moving five times faster than $A$. A part of mass times a part of velocity is a part of momentum. (\quotation{Parts} are no longer accepted units.)% The \scaps{si} units which we use had not been established in Newton’s time.)

Newton's example can be demonstrated with pendulums, as shown in \in{figure}[fig:NewtonCollision].

\startbuffer[NewtonCollision]
	\matrix{
	\startaxis[% Release
		margin cart track,
		axis x line* = middle,
		xticklabels=\empty,
		xlabel={},
	   	extra x ticks={25,30},
	   	extra x tick labels=\empty,
   		extra x tick style={grid=major},
		ymin=-1.5, ymax = 25,
		footnotesize, %clip=false,
	]
	\fill [on layer={axis background}] (-0.5,0) rectangle (49.5,-1.5)[opacity=.1];
	\coordinate (B) at (25,249);
	\coordinate (A) at (30,249);
	\pic (first) at (B) [rotate=-4.68] {small pendulum=100};
	\pic (second) at (A) [rotate=-0.935] {pendulum=100};
	\path (6,6) node[above right]{$B$};
	\path (24,6) node[above left]{$A$};
    \stopaxis\\
	\startaxis[% Collision
		margin cart track,
		axis x line* = middle,
		xticklabels=\empty,
		xlabel={},
	   	extra x ticks={25,30},
	   	extra x tick labels=\empty,
   		extra x tick style={grid=major},
		ymin=-1.5, ymax = 10,
		footnotesize, %clip=false,
	]
	\fill [on layer={axis background}] (-0.5,0) rectangle (49.5,-1.5)[opacity=.1];
	\coordinate (B) at (25,249);
	\coordinate (A) at (30,249);
	\pic (third) at (B) {small pendulum=100}node[below right]{$B$};
	\pic (forth) at (A) {pendulum=100}node[below right]{$B$};
    \stopaxis\\
	\startaxis[% Final 1
		margin cart track,
		axis x line* = middle,
		xticklabels=\empty,
		xlabel={},
	   	extra x ticks={25,30},
	   	extra x tick labels=\empty,
   		extra x tick style={grid=major},
		ymin=-1.5, ymax = 10,
		footnotesize, %clip=false,
	]
	\fill [on layer={axis background}] (-0.5,0) rectangle (49.5,-1.5)[opacity=.1];
	\node[right, align = left] at (-2,5) {$\Delta p_A = 9$\\$\Delta p_B = -9$};
	\coordinate (B) at (25,249);
	\coordinate (A) at (30,249);
	\pic (fifth) at (B) [rotate=0.467] {small pendulum=100};
	\pic (sixth) at (A) [rotate=2.34] {pendulum=100};
    \stopaxis\\
	\startaxis[
		margin cart track,
		axis x line* = middle,
		xticklabels=\empty,
		xlabel={},
	   	extra x ticks={25,30},
	   	extra x tick labels=\empty,
   		extra x tick style={grid=major},
		ymin=-1.5, ymax = 10,
		footnotesize, %clip=false,
	]
	\fill [on layer={axis background}] (-0.5,0) rectangle (49.5,-1.5)[opacity=.1];
	\node[right, align = left] at (-2,5) {$\Delta p_A = 10$\\$\Delta p_B = -10$};
	\coordinate (B) at (25,249);
	\coordinate (A) at (30,249);
	\pic (fifth) at (B) [rotate=0] {small pendulum=100};
	\pic (sixth) at (A) [rotate=2.497] {pendulum=100};
    \stopaxis\\
	\startaxis[
		margin cart track,
		axis x line* = middle,
		xticklabels=\empty,
		xlabel={},
	   	extra x ticks={25,30},
	   	extra x tick labels=\empty,
   		extra x tick style={grid=major},
		ymin=-1.5, ymax = 10,
		footnotesize, %clip=false,
	]
	\fill [on layer={axis background}] (-0.5,0) rectangle (49.5,-1.5)[opacity=.1];
	\node[right, align = left] at (-2,5) {$\Delta p_A = 11$\\$\Delta p_B = -11$};
	\coordinate (B) at (25,249);
	\coordinate (A) at (30,249);
	\pic (fifth) at (B) [rotate=-0.467] {small pendulum=100};
	\pic (sixth) at (A) [rotate=2.653] {pendulum=100};
    \stopaxis\\
	\startaxis[
		margin cart track,
		axis x line* = middle,
	   	extra x ticks={25,30},
	   	extra x tick labels=\empty,
   		extra x tick style={grid=major},
		ymin=-1.5, ymax = 10,
		footnotesize, %clip=false,
	]
	\fill [on layer={axis background}] (-0.5,0) rectangle (49.5,-1.5)[opacity=.1];
	\node[right, align = left] at (-2,5) {$\Delta p_A = 12$\\$\Delta p_B = -12$};
	\coordinate (B) at (25,249);
	\coordinate (A) at (30,249);
	\pic (fifth) at (B) [rotate=-0.935] {small pendulum=100};
	\pic (sixth) at (A) [rotate=2.81] {pendulum=100};
    \stopaxis\\
	};
\stopbuffer

\marginTikZ{}{NewtonCollision}{To demonstrate Newton’s example with pendulums, both balls must be pulled to the left and released (top diagram). Both balls move in the positive direction and have positive momentum when they collide (second diagram). If nine units of momentum are transferred, then ball $B$ still has one unit of momentum and continues in the positive direction (top diagram). If ball $B$ transfers all ten units of its momentum to ball $A$, then ball $B$ will remain stoped at its equilibrium position after the collision ($\Delta p_A = 10$). Ball $B$ can transfer more momentum than it has, causing it to rebound in the negative direction ($\Delta p_A = 11$ and $\Delta p_A = 12$).} % vskip, name, caption

\startblockquote
	For example, suppose a spherical body $A$ is three times as large as a spherical body $B$ and has two parts of velocity, and let $B$ follow $A$ in the same straight line with ten parts of velocity; then the momentum of $A$ is to the momentum of $B$ as six to ten. Suppose that their momenta are of six parts and ten parts respectively; the sum will be sixteen parts.
\stopblockquote

The masses are $m_A=3$ and $m_B=1$. The initial velocities are $v\sub{$A$i}=2$ and $v\sub{$B$i}=10$. Therefore, the total initial momentum is
\startformula\startmathalignment
\NC	p\si \NC = p\sub{$A$i} + p\sub{$B$i}		\NR
\NC		\NC = m_A v\sub{$A$i} + m_B v\sub{$B$i}		\NR
\NC		\NC = 3 \cdot 2 + 1\cdot 10	\NR
\NC		\NC = 6 + 10	\NR
\NC		\NC = 16.		\NR
\stopmathalignment\stopformula
Initially, the bodies have a total of sixteen parts of momentum. When the collision occurs, some of body $B$’s momentum will be transferred to body $A$. Since there are no external forces, this transfer will not result in any change in the total momentum.
\startblockquote
	When the bodies collide, therefore, if body $A$ gains...% three or four or five parts of momentum, body $B$ will lose just as many parts of momentum and thus after reflection body $A$ will continue with nine or ten or eleven parts of motion and $B$ with seven or six or five parts of momentum, the sum being always, as originally, sixteen parts of momentum. Suppose body $A$ gains
	nine or ten or eleven or twelve parts of momentum and so moves forward with fifteen or sixteen or seventeen or eighteen parts of momentum after meeting body $B$; then body $B$, by losing as many parts of momentum as $A$ gains, will either move forward with one part, having lost nine parts of momentum, or will be at rest, having lost its forward momentum of ten parts, or will move backward with one part of momentum, having lost its momentum and (if I may say) one part more, or will move backward with two parts of momentum because a forward motion of twelve parts has been subtracted. And thus the sums, $15+1$ or $16+0$, of the motion in the same direction and the differences, $17-1$ and $18-2$ of the momenta in opposite directions will always be sixteen parts of momentum, just as before the bodies met and were reflected.\autocite{pp.~420–421. \quotation{Motion} changed to \quotation{momentum,} \quotation{motions} to \quotation{momenta.}}{Newton1999}
\stopblockquote

%\startblockquote
%	And since the momenta with which the bodies will continue to move after the reflections are known, the velocity of each will be found, on the supposition that it is to the velocity before the reflection as the momentum after the reflection is to the momentum before the reflection.
%	For example, in the last case, where the motion of body $A$ was six parts before reflection and eighteen parts afterward, and its velocity was two parts before reflection, its velocity will be found to be six parts after reflection on the basis of the following statement: as six parts of momentum before reflection is to eighteen parts of momentum afterward, so two parts of velocity before reflection is to six parts of velocity afterward.
%\stopblockquote

To restate Newton’s conclusion in our notation, the total final momentum is
\startformula
	p\sf = p\sub{$A$f} + p\sub{$B$f},
\stopformula
which will still be sixteen. Depending on the amount of momentum transferred, the total of sixteen can be achieved in a variety of ways. In Newton’s last two cases, body $B$ transfers so much momentum to $A$ that $B$’s final momentum is negative, meaning that $B$ has changed direction in the collision and is traveling in the negative direction. The complete position vs.\ time graphs for Newton's examples are shown in \in{figure}[fig:NewtonCollisionGraph].

\startbuffer[NewtonCollisionGraph]
	\startaxis[
		footnotesize,
		width=2.13in,%\marginparwidth,
		y={1mm},%x={2mm},
		xlabel={$t$ (s)},
		xmin=0, xmax=1.57,
		xtick={0,0.5,1.0,1.5},
		minor x tick num=4,
		ylabel={$x$ (cm)},
	  	%every axis y label/.style={at={(ticklabel cs:0.5)},rotate=90,anchor=center},
		ymin=0, ymax=50,
		minor y tick num=4,
	   	extra y ticks={25,30},
	   	extra y tick labels=\empty,
   		extra y tick style={grid=major},
		clip mode=individual,
		]
		\addplot[thick,smooth,domain=0:0.785,samples=101]{25-20*cos(2*deg(x))};
		\addplot[thick,smooth,domain=0:0.785,samples=101]{30-4*cos(2*deg(x))};
		\addplot[thick,smooth,domain=0.785:1.57,samples=101]{25-2*cos(2*deg(x))};
		\addplot[thick,smooth,domain=0.785:1.57,samples=101]{30-10*cos(2*deg(x))};
		\addplot[thick,smooth,domain=0.785:1.57,samples=101]{25};
		\addplot[thick,smooth,domain=0.785:1.57,samples=101]{30-10.67*cos(2*deg(x))};
		\addplot[thick,smooth,domain=0.785:1.57,samples=101]{25+2*cos(2*deg(x))};
		\addplot[thick,smooth,domain=0.785:1.57,samples=101]{30-11.33*cos(2*deg(x))};
		\addplot[thick,smooth,domain=0.785:1.57,samples=101]{25+4*cos(2*deg(x))};
		\addplot[thick,smooth,domain=0.785:1.57,samples=101]{30-12*cos(2*deg(x))};
		\startscope[opacity=.3,transparency group]
		\fill[opacity=.3](0,5) circle[radius=.4mm];
		\draw[opacity=.3, shade, ball color = white] (0,5) circle[radius=2mm]node[above right=1mm]{$B$};
		\draw[opacity=.3, shade, ball color = white] (0,26) circle[radius=3mm]node[below right=1mm]{$A$};
		\fill[opacity=.3](0,26) circle[radius=.4mm];
%		\draw  [semithick](-.4,-4) -- (-.4,52);
%		\draw  [semithick](.4,-4) -- (.4,52);
		\draw[opacity=.3, shade, ball color = white] (0.785,25) circle[radius=2mm];
		\fill[opacity=.3](0.785,25) circle[radius=.4mm];
		\draw[opacity=.3, shade, ball color = white] (0.785,30) circle[radius=3mm];
		\fill[opacity=.3](0.785,30) circle[radius=.4mm];
%		\draw  [semithick](.6,-4) -- (.6,52);
%		\draw  [semithick](1.4,-4) -- (1.4,52);
		\draw[opacity=.3, shade, ball color = white] (1.57,27) circle[radius=2mm];
		\fill[opacity=.3](1.57,27) circle[radius=.4mm];
		\draw[opacity=.3, shade, ball color = white] (1.57,40) circle[radius=3mm];
		\fill[opacity=.3](1.57,40) circle[radius=.4mm];
		\draw[opacity=.3, shade, ball color = white] (1.57,25) circle[radius=2mm];
		\fill[opacity=.3](1.57,25) circle[radius=.4mm];
		\draw[opacity=.3, shade, ball color = white] (1.57,40.67) circle[radius=3mm];
		\fill[opacity=.3](1.57,40.67) circle[radius=.4mm];
		\draw[opacity=.3, shade, ball color = white] (1.57,23) circle[radius=2mm];
		\fill[opacity=.3](1.57,23) circle[radius=.4mm];
		\draw[opacity=.3, shade, ball color = white] (1.57,41.33) circle[radius=3mm];
		\fill[opacity=.3](1.57,41.33) circle[radius=.4mm];
		\draw[opacity=.3, shade, ball color = white] (1.57,21) circle[radius=2mm];
		\fill[opacity=.3](1.57,21) circle[radius=.4mm];
		\draw[opacity=.3, shade, ball color = white] (1.57,42) circle[radius=3mm];
		\fill[opacity=.3](1.57,42) circle[radius=.4mm];
%		\draw  [semithick](1.6,-4) -- (1.6,52);
%		\draw  [semithick](2.4,-4) -- (2.4,52);
		\stopscope
	\stopaxis
\stopbuffer

\marginTikZ{}{NewtonCollisionGraph}{The position vs.~time graph for Newton’s collision. Four different results are shown, each for a different amount of momentum transferred from $B$ to $A$.} % vskip, name, caption

Conservation of momentum does not determine how much momentum is transferred. Conservation of momentum only tells us that the transfer does not change the total momentum. Some objects bounce better than others, which causes a greater transfer of momentum. We will return to this issue in Chapter \ref{ch:VisViva}, where we will find the limits on how much momentum can be transferred.

If the objects stick together, then we can calculate the final velocity.
The two bodies stuck together act as one body with a mass that is the total mass. In Newton’s example, the total mass of $A$ and $B$ is $m\sub{total} = m_A + m_B = 1+3 = 4$. We can use this mass to find their velocity after the collision.
\startformula\startmathalignment
\NC	p\sf \NC = m\sub{total} v\sf	\NR
\NC	v\sf \NC = \frac{p\sf}{m\sub{total}}	\NR
\NC	v\sf \NC = \frac{16}{4}	\NR
\NC	v\sf \NC = 4			\NR %\answer{4}
\stopmathalignment\stopformula
\pagereference[NewtonInelasticSpeed]

A collision where the objects stick together is called \emph{totally inelastic}. Newton's example could have resulted in a totally inelastic collision, as shown in \in{figure}[fig:NewtonInelasticCollision]. The position vs.\ time graph for this inelastic collision (\in{fig.}[fig:NewtonInelasticCollisionGraph]) shows the two balls moving together after the collision.

\startbuffer[NewtonInelasticCollision]
	\matrix{
	\startaxis[
		margin cart track,
		axis x line* = middle,
		xticklabels=\empty,
		xlabel={},
	   	extra x ticks={25,30},
	   	extra x tick labels=\empty,
   		extra x tick style={grid=major},
		ymin=-1.5, ymax = 25,
		footnotesize, %clip=false,
	]
	\fill [on layer={axis background}] (-0.5,0) rectangle (49.5,-1.5)[opacity=.1];
	\coordinate (B) at (25,249);
	\coordinate (A) at (30,249);
	\pic (first) at (B) [rotate=-4.68] {small pendulum=100};
	\pic (second) at (A) [rotate=-0.935] {pendulum=100};
	\path (6,6) node[above right]{$B$};
	\path (24,6) node[above left]{$A$};
    \stopaxis\\
	\startaxis[
		margin cart track,
		axis x line* = middle,
		xticklabels=\empty,
		xlabel={},
	   	extra x ticks={25,30},
	   	extra x tick labels=\empty,
   		extra x tick style={grid=major},
		ymin=-1.5, ymax = 10,
		footnotesize, %clip=false,
	]
	\fill [on layer={axis background}] (-0.5,0) rectangle (49.5,-1.5)[opacity=.1];
	\coordinate (B) at (25,249);
	\coordinate (A) at (30,249);
	\pic (third) at (B) {small pendulum=100}node[below right]{$B$};
	\pic (forth) at (A) {pendulum=100}node[below right]{$B$};
    \stopaxis\\
	\startaxis[
		margin cart track,
		axis x line* = middle,
	   	extra x ticks={25,30},
	   	extra x tick labels=\empty,
   		extra x tick style={grid=major},
		ymin=-1.5, ymax = 10,
		footnotesize, %clip=false,
	]
	\fill [on layer={axis background}] (-0.5,0) rectangle (49.5,-1.5)[opacity=.1];
	\coordinate (B) at (25,249);
	\coordinate (A) at (30,249);
	\pic (fifth) at (B) [rotate=1.84] {small pendulum=100};
	\pic (sixth) at (A) [rotate=1.84] {pendulum=100};
    \stopaxis\\
	};
\stopbuffer

\marginTikZ{}{NewtonInelasticCollision}{In Newton’s example of a totally inelastic collision, a small ball overtakes a larger ball, collides, and stick. After the collision they move together as a single object.} % vskip, name, caption

\startbuffer[NewtonInelasticCollisionGraph]
	\startaxis[
		footnotesize,
		width=2.13in,%\marginparwidth,
		y={1mm},%x={2mm},
		xlabel={$t$ (s)},
		xmin=0, xmax=1.57,
		xtick={0,0.5,1.0,1.5},
		minor x tick num=4,
		ylabel={$x$ (cm)},
	  	%every axis y label/.style={at={(ticklabel cs:0.5)},rotate=90,anchor=center},
		ymin=0, ymax=50,
		minor y tick num=4,
	   	extra y ticks={25,30},
	   	extra y tick labels=\empty,
   		extra y tick style={grid=major},
		clip mode=individual,
		]
		\addplot[thick,smooth,domain=0:0.785,samples=101]{25-20*cos(2*deg(x))};
		\addplot[thick,smooth,domain=0:0.785,samples=101]{30-4*cos(2*deg(x))};
		\addplot[thick,smooth,domain=0.785:1.57,samples=101]{25-8*cos(2*deg(x))};
		\addplot[thick,smooth,domain=0.785:1.57,samples=101]{30-8*cos(2*deg(x))};
		\startscope[opacity=.3,transparency group]
		\fill[opacity=.3](0,5) circle[radius=.4mm];
		\draw[opacity=.3, shade, ball color = white] (0,5) circle[radius=2mm]node[above right=1mm]{$B$};
		\draw[opacity=.3, shade, ball color = white] (0,26) circle[radius=3mm]node[below right=1mm]{$A$};
		\fill[opacity=.3](0,26) circle[radius=.4mm];
%		\draw  [semithick](-.4,-4) -- (-.4,52);
%		\draw  [semithick](.4,-4) -- (.4,52);
		\draw[opacity=.3, shade, ball color = white] (0.785,25) circle[radius=2mm];
		\fill[opacity=.3](0.785,25) circle[radius=.4mm];
		\draw[opacity=.3, shade, ball color = white] (0.785,30) circle[radius=3mm];
		\fill[opacity=.3](0.785,30) circle[radius=.4mm];
%		\draw  [semithick](.6,-4) -- (.6,52);
%		\draw  [semithick](1.4,-4) -- (1.4,52);
		\draw[opacity=.3, shade, ball color = white] (1.57,33) circle[radius=2mm];
		\fill[opacity=.3](1.57,33) circle[radius=.4mm];
		\draw[opacity=.3, shade, ball color = white] (1.57,38) circle[radius=3mm];
		\fill[opacity=.3](1.57,38) circle[radius=.4mm];
%		\draw  [semithick](1.6,-4) -- (1.6,52);
%		\draw  [semithick](2.4,-4) -- (2.4,52);
		\stopscope
	\stopaxis
\stopbuffer

\marginTikZ{}{NewtonInelasticCollisionGraph}{The position vs.~time graph for Newton’s totally inelastic collision.} % vskip, name, caption

\startexample[]
	A cart, whose mass is $800\units{g}$, is rolling with a velocity of $-60\units{cm/s}$ on a track. The moving cart collides with a stationary cart whose mass is $1600\units{g}$. They stick together and continue rolling.
	Calculate the carts’ velocity $v\sf$ after the collision.
	\startsolution
	The first part of this problem can be solved using conservation of momentum to find $p\sub{f}$. Since there are no external forces changing the momentum of the carts, $F\sn\Delta t=0$.
	\startformula\startmathalignment
	\NC	p\si + \cancel{F\sn\Delta t} \NC = p\sf\NR
	\NC	p\sub{$A$i} + \cancel{p\sub{$B$i}} \NC = p\sub{$A$f} + p\sub{$B$f}
	\stopmathalignment\stopformula
	Initially only one cart is moving. I have called that cart $A$, with initial momentum $p\sub{$A$i}$. The other cart, cart $B$, is not moving initially so its initial momentum is zero. The final momentum is the total final momentum of both carts together.

	Next, insert the momentum formula, $p=mv$, for each momentum. Cart $A$ has mass $m_A=800\units{g}$ and cart $B$ mass $m_B=1600\units{g}$. The initial velocity of cart $A$ is $v\si=-60\units{cm/s}$. Both carts have the same final velocity $v\sf$, which we are trying to find.
	\startformula\startmathalignment
	\NC		m_A v\si \NC = m_A v\sf + m_B v\sf	\NR
	\NC		m_A v\si \NC = (m_A + m_B) v\sf	\NR
	%\NC	(m_A + m_B) v\sf \NC = m_A v\si	\NR
	\NC		v\sf  \NC =  \frac{m_A}{(m_A + m_B)} v\si	\NR
	\NC			\NC =  \frac{800\units{g}}{(800\units{g} + 1600\units{g})}(-0.60\units{m/s})	\NR
	\NC			\NC = -0.20\units{m/s}	\NR
	\stopmathalignment\stopformula
	The final velocity of the stuck-together carts is $-0.20\units{m/s}$, which can also be expressed as $-20\units{cm/s}$. This is one third of the initial velocity $v\sub{i}=-60\units{cm/s}$ because the initial momentum is now being used to move three times as much mass. The minus sign tells us the final velocity, like the initial velocity, is in the negative direction.
\stopsolution
\stopexample

\section{Gravitational force and the motion of projectiles}

Galileo, recognizing the tremendous power of his new telescopes, immediately turned them skyward to study the heavenly bodies and observe their motions. Newton, recognizing the tremendous power of his new laws of motion, immediately applied them to the heavenly bodies, explaining the motions that Galileo observed.

These heavenly motions are quite unlike the collisions that inspired Newton’s laws. In a collision momentum changes suddenly when the objects make contact. A planet’s momentum changes gradually due to the constant gravitational force of the Sun. Without this gravitational force, every planet would travel in a straight line with a constant speed, leaving the Sun far behind. Instead, the Sun’s gravitational force constantly changes the direction of each planet’s momentum, bending the planet’s path and holding the planet in perpetual orbit about the Sun.

The moons of Jupiter are held in orbit by Jupiter’s gravitational force, while our moon is held in orbit by Earth’s gravitational force. Earth’s gravitational force is ever present not only for the moon, but for everything near Earth, including each of us. We will start our study of gravity here, on Earth’s surface. Once we mastered the tremendous power of vectors, we too will turn our attention to the heavens.

Let’s look first at Galileo’s example of the ball rolling off an edge and falling due to the force of gravity. Figure
\ref{fig:BallOffTableVel2} shows the velocity vectors from Chapter \ref{ch:Motion}. To apply Newton’s laws to these velocities requires some basic mathematical operations with vectors. In particular, we must be able to multiply a vector by a number, as in the momentum formula $\vec{p}=m\vec{v}$, and we must be able to add vectors, as in the law of conservation of momentum $\vec p\si + \vec F\Delta t = \vec p\sf$.

\startbuffer[BallOffTableVel2]
	\startaxis[
		footnotesize,
		%width=\marginparwidth, scale only axis,
		x={1mm},y={1mm},
		xlabel={$x$ (cm)},
		xmin=0,xmax=39.5,
		ylabel={$y$ (cm)},
		ymin=0,ymax=45,
		minor x tick num=4,
		axis x line=bottom,
		minor y tick num=4,
		axis y line=left,
		tick align=outside,
		x axis line style={-},
		y axis line style={-},
		clip = false,
		legend style={draw=none,at={(1,1)},anchor=south east,yshift = 1ex},
		]
		%\fill(10,19) circle[radius=.4mm]node[below]{$\scriptstyle $\coordinates{1.0,1.9}\unit{cm}};
		%\fill(9,13) circle[radius=.4mm]node[below]$\scriptstyle \coordinates{9,13}\unit{cm}$;
		%\draw[shade, ball color = white] (-10,47) circle[radius=3][opacity=.2];
		%\fill(-10,47) circle[radius=.4mm][opacity=.2];
		\draw[shade, ball color = white] (0,48) circle[radius=3][opacity=.2];
		\fill(0,48) circle[radius=.4mm][opacity=.2];
		%\fill[color = white] (7.5,3) circle[radius=3];
		\draw[shade, ball color = white] (12,43.1) circle[radius=3][opacity=.4];
		\fill(12,43.1) circle[radius=.4mm][opacity=.4];
		\draw[shade, ball color = white] (24,28.4) circle[radius=3][opacity=.6];
		\fill(24,28.4) circle[radius=.4mm][opacity=.6];
		\draw[shade, ball color = white] (36,3.9) circle[radius=3];
		\fill(36,3.9) circle[radius=.4mm];
		\fill [on layer={axis background}] (-1.5,0) rectangle (39.5,-1.5)[opacity=.1];
		\fill [on layer={axis background}] (-1.5,0) rectangle (0,45)[opacity=.1];
		\fill [on layer={axis background}] (-1.5,45) rectangle (-7.5,43.5)[opacity=.1];
		\draw[thin] (-2,45)--(0,45);
		\draw[thin] (-7.5,45)--(-5.5,45);
		% Velocity at t=0s
   		\draw[thick,->] (0,48) -- (6,48);%node[above right,pos=.3]$\scriptstyle 1.2\units{m/s}$
		% Velocity at .1s
		\draw[semithick, densely dotted,-{>[scale=.5]}] (12,43.1) --(12,38.1);%node[right]{$\scriptstyle -1.0\units{m/s}$}
		\draw[semithick, densely dotted,-{>[scale=.5]}] (12,38.1) -- (18,38.1);%node[above right,pos=.3]{$\scriptstyle 1.2\units{m/s}$}
		\draw[thin] (12,40.1) -- (14,40.1)-- (14,38.1);
   		\draw[thick,->] (12,43.1) -- (18,38.1);%node[below left]{$\scriptstyle \components{1.2,-2.0}\unit{m/s}$}
   		% Velocity at .2s
		\draw[semithick, densely dotted,-{>[scale=.5]}] (24,28.4) --node[left,pos=.55]{$\scriptstyle -2.0\units{m/s}$} (24,18.42);
		\draw[semithick, densely dotted,-{>[scale=.5]}] (24,18.42) --node[below]{$\scriptstyle 1.2\units{m/s}$} (30,18.42);
		\draw[thin] (24,20.42) -- (26,20.42)-- (26,18.42);
   		\draw[thick,->] (24,28.4) --node[above right, pos=.6]{$\scriptstyle 2.3\units{m/s}$} (30,18.42);%node[below left]$\scriptstyle \components{1.2,-2.0}\unit{m/s}$$
   		% Velocity at .3s
		%\startpgfinterruptboundingbox
		\draw[semithick, densely dotted,-{>[scale=.5]}] (36,3.9) --(36,-10.8);%node[left=-1em,fill=white,pos=.7]{$\scriptstyle -2.9\units{m/s}$}
		\draw[semithick, densely dotted,-{>[scale=.5]}] (36,-10.8) --(42,-10.8);%node[below]{$\scriptstyle 1.2\units{m/s}$}
		\draw[thin] (36,-8.8) -- (38,-8.8)-- (38,-10.8);
   		\draw[thick,->] (36,3.9) -- (42,-10.8);%node[below left]
		%\stoppgfinterruptboundingbox
		% Legend
   		\addlegendimage{
	  	legend image code/.code={\draw[thick,|-|](-0.5cm,0cm)--(0cm,0cm);}
 		};
		\addlegendentry{$=1\units{m/s}$}
	\stopaxis
\stopbuffer

\marginTikZ{}{BallOffTableVel2}{Velocity vectors and their components reproduced from figures \ref{fig:BallOffTableVelMag} and \ref{fig:BallOffTableVelComp}.
} % vskip, name, caption

Multiplying vectors by numbers works in a very natural way. For example, multiplying a vector by $2$ doubles the vector’s length while keeping the vector’s direction the same. If vector $\vec v$ is the vector with magnitude of $2.3\units{m/s}$ in figure~\ref{fig:BallOffTableVel2}, then the vector $2\vec v$ would have a magnitude of $4.6\units{m/s}$ and would point in the same direction as $\vec v$. The vector’s components would double as well. Since $\vec v = \components{1.2,-2.0}\unit{m/s}$, doubling the vector gives $2\vec v = \components{2.4,-2.0}\unit{m/s}$.

Finding momentum is not quite as easy as doubling. The ball in figure~\ref{fig:BallOffTableVel2} has mass $m=150\units{g}$. The momentum $\vec p = m \vec v$ has a magnitude of $(150\units{g})(2.3\units{m/s}) = 345\units{g\.m/s}$ and points in the same direction as $\vec v$, as shown in figure~\ref{fig:BallOffTableMomentum}. The figure also shows the momentum vector’s components.
\startformula\startmathalignment[m=2,distance=2em]
\NC	p_x \NC = mv_x \NC  p_y \NC = mv_y	\NR
\NC		\NC = (150\units{g})(1.2\units{m/s})
				\NC 	\NC = (150\units{g})(-2.0\units{m/s})	\NR
\NC		\NC = 180\units{g\.m/s}
				\NC   \NC = -300\units{g\.m/s}
\stopmathalignment\stopformula
\startformula
	\vec p = \components{180,-300}\unit{g\.m/s}
\stopformula

\startbuffer[BallOffTableMomentum]
	\startaxis[
		footnotesize,
		%width=\marginparwidth, scale only axis,
		x={1mm},y={1mm},
		xlabel={$x$ (cm)},
		xmin=0,xmax=39.5,
		ylabel={$y\,/\units{cm}$},
		ymin=0,ymax=45,
		minor x tick num=4,
		axis x line=bottom,
		minor y tick num=4,
		axis y line=left,
		tick align=outside,
		x axis line style={-},
		y axis line style={-},
		clip = false,
		legend style={draw=none,at={(1,1)},anchor=south east,yshift = 1ex},
		]
		%\fill(10,19) circle[radius=.4mm]node[below]{$\scriptstyle \coordinates{1.0,1.9}\unit{cm}$};
		%\fill(9,13) circle[radius=.4mm]node[below]{$\scriptstyle \coordinates{9,13}\unit{cm}$};
		%\draw[shade, ball color = white] (-10,47) circle[radius=3][opacity=.2];
		%\fill(-10,47) circle[radius=.4mm][opacity=.2];
		\draw[shade, ball color = white] (0,48) circle[radius=3][opacity=.2];
		\fill(0,48) circle[radius=.4mm][opacity=.2];
		%\fill[color = white] (7.5,3) circle[radius=3];
		\draw[shade, ball color = white] (12,43.1) circle[radius=3][opacity=.4];
		\fill(12,43.1) circle[radius=.4mm][opacity=.4];
		\draw[shade, ball color = white] (24,28.4) circle[radius=3][opacity=.6];
		\fill(24,28.4) circle[radius=.4mm][opacity=.6];
		\draw[shade, ball color = white] (36,3.9) circle[radius=3];
		\fill(36,3.9) circle[radius=.4mm];
		\fill [on layer={axis background}] (-1.5,0) rectangle (39.5,-1.5)[opacity=.1];
		\fill [on layer={axis background}] (-1.5,0) rectangle (0,45)[opacity=.1];
		\fill [on layer={axis background}] (-1.5,45) rectangle (-7.5,43.5)[opacity=.1];
		\draw[thin] (-2,45)--(0,45);
		\draw[thin] (-7.5,45)--(-5.5,45);
		% Velocity at t=0s
   		\draw[thick,->] (0,48) -- (9,48);%node[above right,pos=.3]{$\scriptstyle 1.2\units{m/s}$}
		% Velocity at .1s
		\draw[semithick, densely dotted,-{>[scale=.5]}] (12,43.1) --(12,35.6);%node[right]{$\scriptstyle -1.0\units{m/s}$}
		\draw[semithick, densely dotted,-{>[scale=.5]}] (12,35.6) -- (21,35.6);%node[above right,pos=.3]{$\scriptstyle 1.2\units{m/s}$}
		\draw[thin] (12,37.6) -- (14,37.6)-- (14,35.6);
   		\draw[thick,->] (12,43.1) -- (21,35.6);%node[below left]{$\scriptstyle \components{1.2,-2.0}\unit{m/s}$}
   		% Velocity at .2s
		\draw[semithick, densely dotted,-{>[scale=.5]}] (24,28.4) --node[left,pos=.55]{$\scriptstyle -300\units{g\cdot m/s}$} (24,13.43);
		\draw[semithick, densely dotted,-{>[scale=.5]}] (24,13.43) --node[below]{$\scriptstyle 180\units{g\cdot m/s}$} (33,13.43);
		\draw[thin] (24,15.43) -- (26,15.43)-- (26,13.43);
   		\draw[thick,->] (24,28.4) --node[above, sloped, pos=.6]{$\scriptstyle 345\units{g\cdot m/s}$} (33,13.43);%node[above right]{$\scriptstyle \components{180,-300}\unit{m/s}$}
   		% Velocity at .3s
		%\startpgfinterruptboundingbox
		\draw[semithick, densely dotted,-{>[scale=.5]}] (36,3.9) -- +(0,-22.05);%node[left=-1em,fill=white,pos=.7]{$\scriptstyle -2.9\units{m/s}$}
		\draw[semithick, densely dotted,-{>[scale=.5]}] (36,-18.15) -- +(9,0);%node[below]{$\scriptstyle 1.2\units{m/s}$}
		\draw[thin] (36,-16.15) -- ++(2,0)-- ++(0,-2);
   		\draw[thick,->] (36,3.9) -- +(9,-22.05);%node[below left]
		%\stoppgfinterruptboundingbox
		% Legend
   		\addlegendimage{
	  	legend image code/.code={\draw[thick,|-|](-0.5cm,0cm)--(0cm,0cm);}
 		};
		\addlegendentry{$=100\units{g\.m/s}$}
	\stopaxis
\stopbuffer

\marginTikZ{}{BallOffTableMomentum}{Momentum vectors point in the direction of motion, like the velocity vectors in figure \ref{fig:BallOffTableVel2}. %The momentum’s magnitude and components are the mass times the speed and the velocity’s components. The momentum arrows require a new scale.
} % vskip, name, caption


\noindent
Momentum vectors and velocity vectors have different units, so they need different scales. Our diagrams will have either velocity vectors or momentum vectors, but not both, so there will only be one vector scale on each diagram.

\startexample[ex:cannonpv]
	A cannon fires a $40\units{kg}$ cannon ball at an upward angle so that its initial momentum is $\vec p\si = \components{1600, 1200}\unit{kg\.m/s}$, as shown in figure~\in[fig:cannonp]. Find the cannon ball’s initial velocity and speed.

\startbuffer[cannonp]
%\draw[->,ultra thick] (12,1.5) -- node[above, pos=.6]{$p\si$}(13,1.5);
%\shade[right color=gray,left color=white] (.6,.24) rectangle (0.9,.26);
\draw[shade, ball color = black] (0.78,.61) circle[radius=.05cm]; % Ball
\fill[fill=black!70] (0.76,0.47)-- ++(-.12,.16)-- ++(-.51,-.32)-- ++(0.18,-0.24)-- cycle; % Cannon Barrel
\fill[fill = black!70] (0.22,0.19) circle[radius=0.15cm]; % Cannon Back
\draw[fill=gray] (0,0)-- ++(0,0.05)-- ++(.2,.2)-- ++(.2,0)-- ++(.2,-.2)-- ++(0,-0.05)-- cycle; % Cannon Base
\draw[fill = black!70] (0.3,0.25) circle[radius=0.05cm]; % Cannon Pivot
\shade[top color=gray] (0,-.04) rectangle (5,0); % Ground
\draw (0,0)--(5,0);
\draw[thick,->] (0.78,0.61)--node[above, sloped]{$\scriptstyle \vec p\si = \components{1600, 1200}\unit{kg\cdot m/s}$} ++(2.4,1.8); % Momentum
\stopbuffer

\marginTikZ{}{cannonp}{A cannon ball is fired with the momentum shown (ex.~\in[ex:cannonpv]).} % vskip, name, caption

\startsolution
Start with the momentum formula $\vec p = m\vec v$ and immediately break it into component equations. Solve each component equation for the components of $\vec v$.
\startformula
	\vec p = m\vec v
\stopformula
\startformula\startmathalignment[m=2,distance=2em]
\NC	p_x	\NC = mv_x
			\NC	p_y	\NC = mv_y							\NR
\NC	v_x	\NC = \frac{p_x}{m}
			\NC	v_y	\NC = \frac{p_y}{m}						\NR
\NC		\NC = \frac{1600\units{kg\.m/s}}{40\units{kg}}
			\NC		\NC = \frac{1200\units{kg\.m/s}}{40\units{kg}}	\NR
\NC		\NC = 40\units{m/s}
			\NC		\NC = 30\units{m/s}						\NR
\stopmathalignment\stopformula
Combine the components into the velocity vector $\vec v$.
\startformula
	\vec v = \components{40,30}\unit{m/s}
\stopformula
The cannon ball’s speed $v$ is the magnitude of its velocity $\vec v$.
\startformula
	v = \vabs{\vec{v}}
		= \sqrt{v_x^2 + v_y^2}
		= \sqrt{(40\units{m/s})^2 + (30\units{m/s})^2}
%		= \sqrt{6400\unit{m^2/s^2} + 3600\unit{m^2/s^2}}
%		= \sqrt{10000\unit{m^2/s^2}}
		= 50\units{m/s}
\stopformula
The cannon ball’s velocity is $\components{40,30}\unit{m/s}$. Its speed is $50\units{m/s}$.

\startbuffer[cannonv]
%\draw[->,ultra thick] (12,1.5) -- node[above, pos=.6]{$p\si$}(13,1.5);
%\shade[right color=gray,left color=white] (.6,.24) rectangle (0.9,.26);
\draw[shade, ball color = black] (0.78,.61) circle[radius=.05cm]; % Ball
\fill[fill=black!70] (0.76,0.47)-- ++(-.12,.16)-- ++(-.51,-.32)-- ++(0.18,-0.24)-- cycle; % Cannon Barrel
\fill[fill = black!70] (0.22,0.19) circle[radius=0.15cm]; % Cannon Back
\draw[fill=gray] (0,0)-- ++(0,0.05)-- ++(.2,.2)-- ++(.2,0)-- ++(.2,-.2)-- ++(0,-0.05)-- cycle; % Cannon Base
\draw[fill = black!70] (0.3,0.25) circle[radius=0.05cm]; % Cannon Pivot
\shade[top color=gray] (0,-.04) rectangle (5,0); % Ground
\draw (0,0)--(5,0);
\draw[thick,->] (0.78,0.61)--node[above, sloped]{$\scriptstyle \vec v = \components{40, 30}\unit{m/s}$} ++(2.4,1.8); % Momentum
\stopbuffer

\marginTikZ{}{cannonv}{The cannon ball’s velocity found in example~\in[ex:cannonpv].} % vskip, name, caption

\stopsolution
\stopexample

The conservation of momentum equation includes vector multiplication. The net force vector $\vec F\sn$ is multiplied by the duration $\Delta t$. The resulting impulse vector is added to the initial momentum vector $\vec p_i$ to get the final momentum vector $\vec p\sf$.
\startformula
	\vec{p}\si +\vec{F}\sn\Delta t = \vec{p}\sf
\stopformula
When Newton introduces his second law he provides a helpful diagram, which I have reproduced and adorned with vector notation in figure~\ref{fig:AddVecParallogram}. This diagram shows an object’s initial momentum $\vec p\si$ horizontal and to the right. The object receives an impulse $\vec{F}\sn\Delta t$ which is down and slightly to the right. The initial momentum and the impulse are added to produce the final momentum $\vec{p}\sf$ which is down and far to the right, as shown. Newton combines the vectors by producing a parallelogram from the initial momentum and the impulse. In this parallelogram method, the final momentum is the diagonal of the parallelogram.

\startbuffer[AddVecParallogram]
		x={1cm},y={1cm},
   		\draw[thick,->] (1,2) --node[above]{$\vec p\si$} (4,2);
   		\draw[thick,->] (1,2) --node[below left]{$\vec{F}\sn\Delta t$} (2,0);
   		\draw[thick,->] (1,2) --node[above]{$\vec p\sf$} (5,0);
   		\draw[thin] (4,2) -- (5,0);
   		\draw[thin] (2,0) -- (5,0);
\stopbuffer

\marginTikZ{}{AddVecParallogram}{Newton’s parallelogram method for adding vectors.} % vskip, name, caption

Another method for adding the vectors is to arrange them tip-to-tail as shown in figure~\ref{fig:AddVecTipTail}. Here the added vectors are drawn in sequence. First, draw the initial momentum vector, $\vec p\si$, going to the right. Then, starting at that vector's tip, draw the impulse, $\vec J$, going down and slightly farther to the right. The final momentum vector is the straight vector that starts where the initial momentum starts and ends where the impulse vector ends. This tip-to-tail method gets its name from the fact that the tip of the initial momentum vector is where we place the tail of the impulse vector.

\startbuffer[AddVecTipTail]
		x={1cm},y={1cm},
   		\draw[thick,->] (1,2) --node[above]{$\vec p\si$} (4,2);
   		\draw[thick,->] (4,2) --node[above right]{$\vec{F}\sn\Delta t$} (5,0);
   		\draw[thick,->] (1,2) --node[below]{$\vec p\sf$} (5,0);
%   		\draw[thin] (4,2) -- (5,0);
%   		\draw[thin] (2,0) -- (5,0);
\stopbuffer

\marginTikZ{}{AddVecTipTail}{The tip-to-tail method for adding vectors.} % vskip, name, caption

In the tip-to-tail method it is easy to see that the $x$-component of the initial momentum added to the $x$-component of the impulse gives the $x$-component of the final momentum, and likewise for the $y$-components. When working with components it will be useful to break Newton’s second law into component equations (dropping \quotation{net} to reduce clutter).
\startformula
		\vec p\si + \vec{F}\Delta t = \vec p\sf
\stopformula
\startformula
	p\sub{i,$x$} + F_x\Delta t = p\sub{i,$x$} \qquad p\sub{i,$y$} + F_y\Delta t = p\sub{f,$y$}
\stopformula

%\section{Gravitational force}
%Once the ball has rolled off the edge, the force acting on it is the Earth’s gravitational force.
The rock dropped from the tower, the ball rolling off the edge, and the cannon ball are all projectiles. No longer in contact with anything else, their motion is bent downward by the constant gravitational force exerted by Earth below.
 Earth does not exert the same force on everything.
Massive things are heavy because Earth exerts a large force on them. Objects with little mass are light because Earth exerts a small force on them. The magnitude of the gravitational force is $mg$, where $g=9.8\units{m/s^2}$. The force is downward, which is the negative $y$-direction, so the gravitational force vector is
\startformula
	\vec F = \components{0,-mg}.
\stopformula
\startexample[ex:cannonballFg]
	Find the gravitational force on a $40.\units{kg}$ cannon ball.
	\startsolution
	The the magnitude of the gravitational force is $mg$.
	\startformula
		mg = (40.\units{kg})(9.8\units{m/s^2}) = 392\units{kg\.m/s^2} = 390\units{N}
	\stopformula
	The gravitational force is downward, in the negative $y$-direction.
\startformula
	\vec F = \components{0, mg} = \components{0,-390}\unit{N}
\stopformula
\stopsolution
\stopexample
The gravitational force is the only significant force acting on the falling rock, the ball rolling off the edge, and the cannon ball. For all of these objects the net force $\vec F\sn$ is simply the gravitational force.
%An object traveling through the air with great speed, like the cannon ball, will experience a significant force from the air resisting the object’s motion. The example below includes this resisting force, which must be added to the gravitational force to find the net force. %The forces must be added as vectors to find the net force vector. The net force is then used in the law of conservation of momentum, requiring another vector addition to find the final momentum.

\startexample[ex:cannonpf]
% Text image
\placetextfloat[bottom][fig:cannonpf]
{The cannon ball’s momentum changes due to the force of gravity and the air’s resisting force in example~\in[ex:cannonpf].
}	% caption text
{\noindent\small\starttikzpicture
%\draw[->,ultra thick] (12,1.5) -- node[above, pos=.6]{$p\si$}(13,1.5);
%\shade[right color=gray,left color=white] (.6,.24) rectangle (0.9,.26);
\draw[shade, ball color = black] (0.78,.61) circle[radius=.05cm]; % Ball initial
\fill[fill=black!70] (0.76,0.47)-- ++(-.12,.16)-- ++(-.51,-.32)-- ++(0.18,-0.24)-- cycle; % Cannon Barrel
\fill[fill = black!70] (0.22,0.19) circle[radius=0.15cm]; % Cannon Back
\draw[fill=gray] (0,0)-- ++(0,0.05)-- ++(.2,.2)-- ++(.2,0)-- ++(.2,-.2)-- ++(0,-0.05)-- cycle; % Cannon Base
\draw[fill = black!70] (0.3,0.25) circle[radius=0.05cm]; % Cannon Pivot
\shade[top color=gray] (0,-.2) rectangle (10.8,0); % Ground
\draw (0,0)--(10.8,0);
\draw[thick,->] (0.78,0.61)--node[above, sloped]{$\vec p\si = \components{1600, 1200}\unit{kg\cdot m/s}$} ++(2.4,1.8); % Momentum
\draw (0.78,0.61) parabola [bend pos=0] bend +(6.75,2.53) +(6.75,2.53); % Trajectory
\filldraw[white] (5.28,2.86) rectangle (7.53,3.14); % Trajectory
\draw[shade, ball color = black] (5.28,2.86) circle[radius=.05cm]; % Ball final
\draw[thick,->] (5.28,2.86)--node[above, sloped]{$\vec p\sf = ?$} ++(2.4,0.6); % Momentum
\stoptikzpicture}
	A cannon fires a $40\units{kg}$ cannon ball with a speed of $50\units{m/s}$. The ball is fired at an upward angle so that its initial momentum is $\vec p\si = \components{1600, 1200}\unit{kg\.m/s}$, as shown in figure~\in[fig:cannonpf]. While it is in the air gravity exerts a downward force $\vec F = \components{0, -390}\unit{N}$. %The ball encounters some resistance to moving through the air, which exerts a backward force of approximately $F\sub{air} = \components{-40, -30}\unit{N}$.
What is the cannon ball’s momentum $2.0\units{s}$ after it is fired? (Ignore air resistance.)
\startsolution
%We will solve this problem in two steps. First, we will find the net force acting on the cannon ball, $\vec F\sn = \vec F_g + \vec F\sub{air}$. Then we will use conservation of momentum to find the final momentum.
%Compute the net force by breaking the vector equation into component equations.
%\startformula
%	\vec F\sn = \vec F_g + \vec F\sub{air}
%\stopformula
%\startformula\startmathalignment[m=2,distance=2em]
%\NC	F\sub{net,$x$}	\NC = F_{g,x} + F\sub{air,$x$}
%		\NC	F\sub{net,$y$}	\NC = F_{g,y} + F\sub{air,$y$}	\NR
%\NC				\NC = 0 + (-40\units{N})
%		\NC				\NC = (-390\units{N}) - (-30\units{N})			\NR
%\NC				\NC = -40\units{N}
%		\NC				\NC = -420\units{N}
%\stopmathalignment\stopformula
%Combine the components into the vector $\vec F\sn$.
%\startformula
%	\vec F\sn = \components{-40,-420}\unit{m/s}
%\stopformula
We will solve this problem using the conservation of momentum equation, broken into components equations. (Here we drop the \quotation{net.}) First the $x$-component:
\startformula\startmathalignment
\NC	p\sub{i,$x$} + F_x\Delta t
	\NC = 1600\units{kg\.m/s} + (0\units{N})(2\units{s})			\NR
\NC	\NC = 1600\units{kg\.m/s}	= p\sub{f,$x$}
%	\NC = 1600\units{kg\.m/s} + (-40\units{N})(2.0\units{s})		\NR
%\NC	\NC = 1600\units{kg\.m/s} + (-40\units{kg\.m/s^2})(2.0\units{s})	\NR
%NC	\NC = 1520\units{kg\.m/s}	= p\sub{f,$x$}
\stopmathalignment\stopformula
Since the gravitational force’s $x$-component is zero, the $x$-component of the momentum is unchanged. The $y$-component does change due to the downward force of gravity.
\startformula\startmathalignment
\NC	p\sub{i,$y$} + F_y\Delta t
	\NC = 1200\units{kg\.m/s} + (-390\units{N})(2.0\units{s})		\NR
\NC	\NC = 420\units{kg\.m/s}	= p\sub{f,$y$}
\stopmathalignment\stopformula
Combine the components to make the final momentum vector (and round to two significant figures).
\startformula
	\vec p\sf = \components{1600,420}\unit{kg\.m/s}
\stopformula
Two seconds after leaving the cannon, the ball’s momentum is $\vec p\sf = \components{1600,420}\unit{kg\.m/s}$, as shown in \in{figure}[fig:cannonconservepf], where $\vec p\si$ and $\vec J$ are added by the tip-to-tail method.
The cannon ball's motion is shown in \in{figure}[fig:cannonpfanswer].
\stopsolution

\startbuffer[cannonFgFa]
\draw[shade, ball color = black!60] (0,0) circle[radius=.25cm]; % Ball
\filldraw (0,0) circle[radius=.02cm]; % Ball cm
\draw[thick,->] (0,0)--node[above]{$\vec p\si$} ++(2.4,1.8); % Momentum
\draw[thick,->] (0,0)--node[right]{$\vec F = \components{0, -390}\unit{N}$} ++(0,-1.96); % Fg
%\draw[thick,->] (0,0)--node[above left]{$\scriptstyle \vec F\sub{air} = \components{-40, -30}\unit{N}$} ++(-0.4,-0.3); % Fa
\stopbuffer

\marginTikZ{\vskip1.5in}{cannonFgFa}{The force of gravity and the resisting force of air on the cannon ball in example~\in[ex:cannonpf]. The air’s resisting force is opposite the ball’s motion} % vskip, name, caption

%\placefigure[margin][fig:cannonFnet] % location
%{To find the net force, add the force of gravity and the resisting force of air using the tip-to-tail method. The net force is almost directly downward.}	% caption text
%{\noindent\starttikzpicture
%\draw[shade, ball color = black!60] (0,0) circle[radius=.25cm]; % Ball
%\filldraw (0,0) circle[radius=.02cm]; % Ball cm
%%\draw[thick,->] (0.78,0.61)--node[above]{$\scriptstyle \vec p\si$} ++(2.4,1.8); % Momentum
%\draw[thick,->] (0,0)--node[right]{$\scriptstyle \vec F_g = \components{0, -390}\unit{N}$} ++(0,-3.92); % Fg
%\draw[thick,->] (0,-3.92)--node[below right]{$\scriptstyle \vec F\sub{air} = \components{-40, -30}\unit{N}$} ++(-0.4,-0.3); % Fa
%\draw[thick,->] (0,0)--node[left]{$\scriptstyle \vec F\sn = \components{-40, -420}\unit{N}$} ++(-0.4,-4.22); % Fn
%\stoptikzpicture}

\startbuffer[cannonconservepf]
\draw[shade, ball color = black!60] (0,0) circle[radius=.25cm]; % Ball
\filldraw (0,0) circle[radius=.02cm]; % Ball cm
\draw[thick,->] (0,0)--node[above, sloped, pos=.6]{$\scriptstyle \vec p\si = \components{1600, 1200}\unit{kg\cdot m/s}$} ++(2.4,1.8); % Momentum initial
\draw[thick,->] (2.4,1.8)--node[right]{$\vec F\sn \Delta t$} +(0,-1.17); % Fnet
\draw[thick,->] (0,0)--node[below, sloped, pos=.7]{$\scriptstyle \vec p\sf = \components{1600, 420}\unit{kg\cdot m/s}$} ++(2.4,0.63); % pf
\stopbuffer

\marginTikZ{}{cannonconservepf}{Using conservation of momentum to find the final momentum of the cannon ball in example~\in[ex:cannonpf].} % vskip, name, caption

\placetextfloat[bottom][fig:cannonpfanswer]
{The cannon ball’s initial momentum and its momentum two seconds later in example~\in[ex:cannonpf]
}	% caption text
{\noindent\small\starttikzpicture
%\draw[->,ultra thick] (12,1.5) -- node[above, pos=.6]{$p\si$}(13,1.5);
%\shade[right color=gray,left color=white] (.6,.24) rectangle (0.9,.26);
\draw[shade, ball color = black] (0.78,.61) circle[radius=.05cm]; % Ball initial
\fill[fill=black!70] (0.76,0.47)-- ++(-.12,.16)-- ++(-.51,-.32)-- ++(0.18,-0.24)-- cycle; % Cannon Barrel
\fill[fill = black!70] (0.22,0.19) circle[radius=0.15cm]; % Cannon Back
\draw[fill=gray] (0,0)-- ++(0,0.05)-- ++(.2,.2)-- ++(.2,0)-- ++(.2,-.2)-- ++(0,-0.05)-- cycle; % Cannon Base
\draw[fill = black!70] (0.3,0.25) circle[radius=0.05cm]; % Cannon Pivot
\shade[top color=gray] (0,-.2) rectangle (10.8,0); % Ground
\draw (0,0)--(10.8,0);
\draw[thick,->] (0.78,0.61)--node[above, sloped]{$\vec p\si = \components{1600, 1200}\unit{kg\cdot m/s}$} ++(2.4,1.8); % Momentum
\draw (0.78,0.61) parabola [bend pos=0] bend +(6.75,2.53) +(6.75,2.53); % Trajectory
\filldraw[white] (5.28,2.86) rectangle (7.53,3.14); % Trajectory
\draw[shade, ball color = black] (5.28,2.86) circle[radius=.05cm]; % Ball final
\draw[thick,->] (5.28,2.86)--node[above, sloped]{$\vec p\sf = \components{1500, 360}\unit{kg\cdot m/s}$} ++(2.4,0.6); % Momentum
\stoptikzpicture}
\stopexample
%The previous example multiple forces which had to be added as vectors to find the net force vector. Similarly, for a systems of multiple objects add the momenta as vectors to find the total momentum vector. For example, in a collision between objects moving in different directions the initial momentum of the system is the vector sum of all of the objects’ momentum vectors.

%Galileo persistently urges us to consider situation without accidental impediments like air resistance. Even in the case of the fast cannon ball, the effect of the air resistance was small. For an object moving through the air slowly, like Galileo’s ball rolling off the edge, the air resistance can be ignored. In the example of the rock dropped the tower, the rock moves with great speed due to the Earth’s rotation, but air resistance can again be ignored because the air also moves with Earth’s rotation, right along with the tower and the rock.

Galileo argued that the rock’s tendency to fall in no way diminishes its horizontal motion, which continues naturally, allowing it to keep up with the moving tower even as it falls. Newton’s law of conservation of momentum gives us a new insight into Galileo’s argument. According to Newton, the rock has horizontal momentum before it is dropped because it is being carried along with the tower by Earth’s rotation. When the rock is released it falls due to the downward gravitational force. This force gives the rock an increasing downward component to the rock’s momentum, but it does not alter the horizontal component, just as we saw for the cannon ball in the previous example.

To see the gradual affect of the gravitational force, let’s look at the cannon ball’s motion in small steps. Figure~\in[fig:cannonds] shows the cannon ball’s momentum and the downward gravitational force acting on it during its flight. Initially the momentum is to the right and upward. During a short duration $dt$ the momentum changes by a small amount $\vec F dt$. This continues, with the momentum changing an additional amount $\vec F dt$ during each short duration $dt$, as shown in figure~\in[fig:cannondp]. The changing momentum causes the cannon ball’s path to bend downward, as shown in figure~\in[fig:cannonds]. Newton’s momentum and the law of conservation of momentum give an explanation for the compound motion of projectiles described by Galileo.
%\placefigure[margin][fig:cannonFg] % location
%{The force of gravity acting downward on the cannon ball. The air’s resisting force is small enough to ignore.
%}	% caption text
%{\noindent\vskip.75in\starttikzpicture
%\draw[shade, ball color = black!60] (0,0) circle[radius=.25cm]; % Ball
%\filldraw (0,0) circle[radius=.02cm]; % Ball cm
%\draw[thick,->] (0,0)--node[above]{$\vec p\si$} ++(2.4,1.8); % Momentum
%\draw[thick,->] (0,0)--node[right]{$\vec F\sn = \vec F_g$} ++(0,-1.96); % Fg
%\stoptikzpicture}


\placefigure[margin][fig:cannonds] % location
{The curving path of a cannon ball, initially rising and eventually falling due to the downward gravitational force. The horizontal motion continues, unaffected by gravity, as Galileo observed.}	% caption text
{\vskip2in\hbox{\starttikzpicture
	\draw[white] (0,0)-- ++(5,0); % Sky to make height better
\stoptikzpicture}}

\placewidefloat
  [top,none]
  {This is its caption I need to fix.}
{\hbox{\small\starttikzpicture	% tikz code
%\draw[->,ultra thick] (12,1.5) -- node[above, pos=.6]{$p\si$}(13,1.5);
%\shade[right color=gray,left color=white] (.6,.24) rectangle (0.9,.26);
\draw[shade, ball color = black][opacity=0.25] (0.78,.61) circle[radius=.05cm]; % Ball initial
\fill[fill=black!70] (0.76,0.47)-- ++(-.12,.16)-- ++(-.51,-.32)-- ++(0.18,-0.24)-- cycle; % Cannon Barrel
\fill[fill = black!70] (0.22,0.19) circle[radius=0.15cm]; % Cannon Back
\draw[fill=gray] (0,0)-- ++(0,0.05)-- ++(.2,.2)-- ++(.2,0)-- ++(.2,-.2)-- ++(0,-0.05)-- cycle; % Cannon Base
\draw[fill = black!70] (0.3,0.25) circle[radius=0.05cm]; % Cannon Pivot
\shade[top color=gray] (0,-.2) rectangle (16.7,0); % Ground
\draw (0,0)--(16.7,0);
\draw[thick,->][opacity={0.25}] (0.78,0.61)-- ++(2.4,1.8); % Momentum
\draw[thick,->][opacity=0.25] (0.78,0.61)-- ++(0,-1.96); % Fg
\draw (0.78,0.61) parabola [bend pos=0] bend +(6.75,2.53) +(13.5,0); % Trajectory
\foreach \T in {1,2,3,4,5}{% Balls
	\draw[shade, ball color = black][opacity={(5+\T)/20}] ({0.78+\T*2.25},{0.61+\T*1.6875-\T*\T*0.28125}) circle[radius=.05cm]; % Balls
	\draw[thick,->][opacity={(5+\T)/20}] ({0.78+\T*2.25},{0.61+\T*1.6875-\T*\T*0.28125})-- ++(2.4,{1.8-\T*0.6}); % Momentum
	\draw[thick,->][opacity={(5+\T)/20}] ({0.78+\T*2.25},{0.61+\T*1.6875-\T*\T*0.28125})-- ++(0,-1.96); % Fg
}
\draw[shade, ball color = black] (14.28,0.616) circle[radius=.05cm]; % Ball final
\draw[thick,->] (14.28,0.61)--node[below left]{$\scriptstyle \vec p$} ++(2.4,-1.8); % Momentum
\draw[thick,->] (14.28,0.61)--node[left]{$\vec F$} ++(0,-1.96); % Fg
\stoptikzpicture}}

\startbuffer[cannondp]
\draw[shade, ball color = black!60] (0,0) circle[radius=.25cm]; % Ball
\filldraw (0,0) circle[radius=.02cm]; % Ball cm
\draw[thick,->][opacity=0.25] (0,0)-- ++(2.4,1.8); % Momentum initial
\foreach \T in {1,2,3,4,5}{% Balls
	\draw[thick,->][opacity={(5+\T)/20}] (0,0)-- ++(2.4,{1.8-\T*0.6}); % Momentum
	\draw[thick,->][opacity={(5+\T)/20}] (2.4,{2.4-\T*0.6})-- +(0,-0.6); % Fnet
}
\draw[thick,->] (2.4,-1.2)--node[right]{$\vec F\sn dt$} +(0,-0.6); % Fnet
\draw[thick,->] (0,0)--node[below left]{$\vec p$} ++(2.4,-1.8); % pf
\stopbuffer

\marginTikZ{}{cannondp}{Using conservation of momentum to find the final momentum of the cannon ball in example~\in[ex:cannonpf].} % vskip, name, caption

%\startexample[]
%	The $150\units{g}$ ball rolls off of the table with an initial momentum  $\vec{p}\si = \components{180,0}\unit{g\.m/s}$. What is the ball’s momentum $0.20\units{s}$ later?
%	\startsolution
%	We will find this using Newton’s second law, which we will more commonly call the conservation of momentum equation. The impulse is due to the gravitational force $\vec F$, so we will use all of the tools from this section.
%	\startformula
%	\NC	\vec p\si + \vec F\Delta t = \vec p\sf
%	\stopformula
%Break this equation into components and calculate. First, the $x$-direction which has no force.
%\startformula
%	p\sub{i,$x$} + F_x \Delta t
%		= 180\units{g\.m/s} + 0
%		= 180\units{g\.m/s}
%		= p\sub{f,$x$}
%\stopformula
%Second, the $y$-direction which has no initial momentum.
%\startformula\startmathalignment
%\NC	p\sub{i,$y$} + F_y\Delta t
%	\NC = 0 + (-mg)\Delta t	\NR
%\NC	\NC = -(150\units{g})(9.8\units{m/s^{2}})(0.20\units{s})	\NR
%\NC	\NC = -(294\units{g\.m/s})
%	= p\sub{f,$y$}
%\stopmathalignment\stopformula
%Combine the components to get the final momentum vector $\vec p\sub{f}$.
%\startformula
%	\vec p\sf = \components{180,-290}\unit{g\.m/s}
%\stopformula
%This is approximately the momentum shown in figure~\ref{fig:BallOffTableMomentum} for the ball $0.20\units{s}$. (Rounding in calculations can cause small differences in the last significant digit, as seen  in the difference between $290\units{g\.m/s}$ above and  $300\units{g\.m/s}$ in figure~\ref{fig:BallOffTableMomentum}. Both answers are acceptable.)
%\stopsolution
%\stopexample


\section{Circular motion and centripetal force}
Newton confirmed Galileo’s description of projectile motion, but offered a dramatically different view of circular motion. Galileo viewed circular motion as natural. Newton, following Descartes, recognized that only straight motion is natural – a force is required to bend an object’s path into a circle. A bent path is always evidence that some force is acting.

As an example, the moon does not travel in a straight line, but instead orbits Earth. The motion is approximately uniform circular, as shown in figure~\in[fig:MoonMomentum1]. Since the speed is constant, the momentum’s magnitude is also constant. The momentum’s direction is changing, always tangent to the circular path. Figure~\in[fig:MoonMomentum2] shows the changing momentum $p$ and the small changes $dp$. These momentum changes must be due to a force acting in the direction of the changes, as shown in figure~\in[fig:MoonMomentum3]. The direction of the force is changing, always staying perpendicular to the momentum. This bends the Moon’s path without changing its speed. Since the force is perpendicular to the path, it is aways directed toward the center of the circle, as shown it figure~\in[fig:MoonMomentum4]

The circular motion does not cause a force toward the center. Rather, the force causes the circular motion. Any circular motion we observe, such as the motions of moons around planets or planets around the Sun, must be due to a force pulling the orbiting object toward the center of its orbit.

\startbuffer[MoonMomentum1]
	\clip (-0.5,-5) rectangle (4.5,4.1);
	\draw[shade, ball color = white] (0,0) circle[radius=.4cm]node[above right=3mm]{Earth};% Earth
	\foreach \T in {-10,-20,-30}{% Earlier Moons
		\draw[shade, ball color = white][opacity={(20-\T)/100}] (\T:4cm) circle[radius=.2cm];
		\fill(\T:4cm) circle[radius=.4mm][opacity={(20-\T)/100}];
	}
	\draw[shade, ball color = white] (-40:4cm) circle[radius=.2cm]node[left=2mm]{Moon};% Final Moon
	\fill(-40:4cm) circle[radius=.4mm];
	\foreach \T in {180,170,...,-170}{% Draw path segments
		\draw[-{Straight Barb[scale length=.5]}] (\T:4cm)
		 arc [start angle=\T, end angle={\T-10}, radius=4cm];
	}
		\draw[-{Straight Barb[scale length=.5]},ultra thick] (-30:4cm)% Label last segment
		 arc [start angle=-30, end angle={-40}, radius=4cm]node[above=1.5mm]{$ds$};
	\foreach \T in {-10,-20,-30}{% Draw vectors
		\draw[][opacity={(20-\T)/100}] (0,0) -- (\T:4cm);
		\draw[thick,->][opacity={(20-\T)/100}] (\T:4cm) -- ++({\T-90}:3cm);
	}
	\draw (0,0) --node[below left]{$R$} (-40:4cm);% Final vectors
	\draw[thick,->] (-40:4cm) --node[left=1mm, pos=.7]{$p$} ++({-130}:3cm);
		\path (-50:4cm) coordinate (A) -- (0,0) coordinate (B) -- (4cm,0) coordinate (C)
pic [draw, <-, thick, angle eccentricity=1.2, pic text=$\omega$, angle radius=1.5cm] {angle};
\stopbuffer

\marginTikZ{}{MoonMomentum1}{The Moon orbits Earth with a constant speed. The moon’s momentum $\vec p$, always tangent to the path, has a constant magnitude $p$ but a changing direction.} % vskip, name, caption

\startbuffer[MoonMomentum2]
	\clip (-3.5,-3.1) rectangle (1.5,0.3);
	\draw[shade, ball color = white] (0,0) circle[radius=.2cm]node[left=2mm]{Moon};% Moon
	\fill(0,0) circle[radius=.4mm];
	\foreach \T in {-10,-20,-30}{% Draw vectors
		\draw[thick,->][opacity={(20-\T)/100}] (0,0) -- ({\T-90}:3cm);
		\draw[thick,->][opacity={(20-\T)/100}] ({\T-90}:3cm) -- ({\T-100}:3cm);
	}
	%\draw (0,0) --node[below left]{$R$} (-40:4cm);
	\draw[thick,->] (0,0) --node[left=2mm, pos=.6]{$p$} ({-130}:3cm);
	\draw[thick,->] ({-120}:3cm) --node[below left]{$dp$} ({-130}:3cm);
		\path (-140:4cm) coordinate (A) -- (0,0) coordinate (B) -- (0,-4cm) coordinate (C)
pic [draw, <-, thick, angle eccentricity=1.2, pic text=$\omega$, angle radius=1.5cm] {angle};
\stopbuffer

\marginTikZ{}{MoonMomentum2}{Placing the momentum vectors at the same place show’s the arc produced by the small momentum changes $d\vec p$.} % vskip, name, caption

\startbuffer[MoonMomentum3]
	\clip (-3.5,-3) rectangle (1.5,0.3);
	\draw[shade, ball color = white] (0,0) circle[radius=.2cm]node[left=2mm]{Moon};% Moon
	\fill(0,0) circle[radius=.4mm];
	\foreach \T in {-10,-20,-30}{% Draw vectors
		%\draw[][opacity={(20-\T)/100}] (0,0) -- (\T:4cm);
		\draw[thick,->][opacity={(20-\T)/100}] (0,0) -- ({\T-90}:3cm);
		\draw[thick,->][opacity={(20-\T)/100}] ({\T-90}:3cm) -- ++({\T+180}:2cm);
	}
	%\draw (0,0) --node[below left]{$R$} (-40:4cm);
	\draw[thick,->] (0,0) --node[left=2mm]{$p$} ({-130}:3cm);
	\draw[thick,->] ({-130}:3cm) --node[above right]{$F$} ++({140}:2cm);
\stopbuffer

\marginTikZ{}{MoonMomentum3}{The small momentum changes $d\vec p$ are produced by the force $\vec F$, which is tangent to the arc and therefore perpendicular to the momentum $\vec p$.} % vskip, name, caption

\startbuffer[MoonMomentum4]
	\clip (-0.5,-5) rectangle (4.5,4.1);
	\draw[shade, ball color = white] (0,0) circle[radius=.4cm]node[above right=3mm]{Earth};% Earth
	\foreach \T in {-10,-20,-30}{% Earlier Moons
		\draw[shade, ball color = white][opacity={(20-\T)/100}] (\T:4cm) circle[radius=.2cm];
		\fill(\T:4cm) circle[radius=.4mm][opacity={(20-\T)/100}];
	}
	\draw[shade, ball color = white] (-40:4cm) circle[radius=.2cm]node[left=2mm]{Moon};% Final Moon
	\fill(-40:4cm) circle[radius=.4mm];
	\foreach \T in {180,170,...,-170}{% Draw path segments
		\draw[-{Straight Barb[scale length=.5]}] (\T:4cm)
		 arc [start angle=\T, end angle={\T-10}, radius=4cm];
	}
	\foreach \T in {-10,-20,-30}{% Draw vectors
		%\draw[][opacity={(20-\T)/100}] (0,0) -- (\T:4cm);
		\draw[thick,->][opacity={(20-\T)/100}] (\T:4cm) -- ++({\T-90}:3cm);
		\draw[thick,->][opacity={(20-\T)/100}] (\T:4cm) -- ++({\T+180}:2cm);
	}
	%\draw (0,0) --node[below left]{$R$} (-40:4cm);% Final vectors
	\draw[thick,->] (-40:4cm) --node[left=1mm, pos=.7]{$p$} ++({-130}:3cm);
	\draw[thick,->] (-40:4cm) --node[left=3mm]{$F$} ++({140}:2cm);
\stopbuffer

\marginTikZ{}{MoonMomentum4}{As the Moon circles Earth, the force on the Moon is always toward Earth, perpendicular to the momentum, changing the momentum’s direction but not the speed.} % vskip, name, caption

The size and speed of the orbit also reveal the magnitude of the force.
Angular velocity is helpful here, because the radius vector $R$ (\in{fig.}[fig:MoonMomentum1]) and the momentum vector $p$ (\in{fig.}[fig:MoonMomentum2]) both rotate with the same angular velocity $\omega$. Recall that the speed of an object in uniform circular motion is the radius of the circle times the angular velocity (\at{p.}[eq:angularvelocity]).
\startformula
	v = R\omega
\stopformula
The tip of the momentum vector in \in{figure}[fig:MoonMomentum2] is also moving in a circle. The \quotation{radius} of this circle is the momentum's magnitude $p$. Force is defined as the momentum's rate of change, which is the \quotation{speed} along this momentum circle. This force is therefore the momentum times the angular velocity.
\startformula
	F = p\omega = \frac{pv}{R}
\stopformula
The last term in this extremely useful \emph{centripetal force formula} was found by the replacement $\omega = v/R$. Whenever you encounter an object traveling along a circular path, you should use the centripetal force formula to find the force directed toward the circle’s center. This is not a vector equation; it only relates the magnitude of the force $F$ to the magnitude of the momentum $p$. The correct directions are shown in \in{figure}[fig:MoonMomentum4].

Newton used the centripetal force formula to understand the orbits of moons and planets, leading to his discovery of universal law for gravitational forces. We will return to this law in Chapter~\in[ch:PotentialEnergy].

\section{Momentum in modern physics}
Newton’s three laws of motion and conservation of momentum have proven to be among the most enduring ideas in physics. At the time of the \textit{vis viva} debates, no one could have foreseen the physics revolutions of the twentieth century – relativity, quantum mechanics, and quantum field theory. Yet Newton’s laws have survived to this day.

Experiments with fast particles in the twentieth century have shown that
Newton’s formula for momentum, $m\vec{v}$, is not accurate for particles traveling close to the speed of light, $3\sci{8}\units{m/s}$. For lower speeds, this equation is extremely accurate, even at $3\sci{7}\units{m/s}$, one tenth of the speed of light, Newton’s formula has an error of less than $0.5\%$. Do not hesitate to use Newton’s $m\vec{v}$ unless you are working with particles traveling faster than $10^7\units{m/s}$. Einstein discovered the correct formula for all speeds. We will learn his momentum formula when we study high speed particles in Volume II.

You may be wondering what happened to the famous $\vec{F}=m\vec{a}$, which is usually called Newton’s second law. We did not introduce $\vec{F}=m\vec{a}$ for three reasons. First, that is not what Newton actually said. Second, $\vec{F}=m\vec{a}$ is not actually correct. To get from Newton’s second law to $\vec{F}=m\vec{a}$ you need to use the momentum formula, $\vec{p}=m\vec{v}$. As was just mentioned, $\vec{p}=m\vec{v}$ fails for very high speeds. Newton’s second law, \emph{as he originally stated it,} is correct for all speeds, even the speed of light. When we study interactions between fast particles in Volume II %Part \ref{part:Particles}
 we will still use Newton’s version of his second law. Third, in my experience students find momentum much easier to understand than acceleration. We will keep our focus on momentum rather than acceleration.

\section{Newton’s momentum and the Cartesian’s quantity of motion}
%Newton’s achievements were celebrated as revolutionary in his own time, and  three centuries of hindsight only makes his triumph more impressive.
The Cartesians celebrated Newton’s achievements, but very few actually understood the \booktitle{Principia}. In fact, most thought Newton had confirmed Descartes’ quantity of motion when he had actually replaced it! This shocking confusion is somewhat more understandable because of the geometric methods used by Newton and the Cartesians. To prevent you from falling into the same confusion, I have been using the modern term \emph{momentum} for Newton’s quantity of motion, even though he called it the \textit{quantitas motus}, the same words used by Descartes. They also used the same words for velocity and speed. Newton thought of his definition of \textit{quantitas motus} as a formula for the magnitude of the momentum. %, which would then be given direction when it was drawn on paper to do the geometric computations.
The magnitude of Newton’s \textit{quantitas motus} is, of course, exactly the same as Descarte’s quantity of motion. It is only when adding the momenta that Newton took direction into account, adding momenta in one direction, but subtracting momenta in the other (or by adding them as vectors using the paralelelogram method). It is easy to see why the Cartesians were confused. Do not become a twenty-first century Cartesian! Remember the difference between the vector and its magnitude.

One notable contemporary of Newton did \emph{not} confuse momentum with the quantity of motion – Gottfried Leibniz. He used algebraic techniques like ours, and he recognized the importance of the sign indicating the direction of velocity and momentum. Still, he chose to focus his attention on the directionless quantity $mv^2$, which he called the \visviva.
Recognizing Newton’s triumph, it would be easy for us to join the Cartesians in dismissing Leibniz. Before we do, it is worth understanding Leibniz’s purpose in proposing the conservation of \textit{vis viva}.


\subject{Notes}
\blank
%\startcolumns
%\placefootnotes[criterium=chapter]
\placenotes[endnote][criterium=chapter]

%\subject{Bibliography}
%        \placelistofpublications

%\stopcolumns
\stopchapter
\stopcomponent


%Scalar multiplication is a big deal in this section and we have vectors with different units. Multiply $\vec v$ by the scalar $\Delta t$ to get the displacement $\Delta \vec r$. There should be a good discussion of \emph{not} dividing by vectors. It is possible in some cases to solve for $\Delta t$ given $\Delta \vec{r}$ and $\vec{v}$, but often not. Show what happens in either case. Specifically, the equation is really three component equations, but if you solve for one unknown if may work in only one or two of the equations. This can also be used to show the dangers of just working with magnitudes.


%%Rather than following Galileo’s brilliant and lengthy refutation, we will confront Aristotle’s argument using conservation of momentum and the mathematical language of vectors.%, which will greatly reduce our effort.
%%The momentum of the rock is changed by the force of gravity.
%%\startformula
%%	\vec{p}\sf = \vec{p}\si + \vec{F}\Delta t
%%\stopformula
%%The force in this case is the force of gravity, which is directed downward and has magnitude $mg$.
%%\startformula
%%	\vec{F} = \coordinates{0,-mg,0}
%%\stopformula
%
%The constant force of gravity changes the rock’s momentum. However, we have learned that the $y$-component of the force will only affect the $y$-component of the momentum. Since the force has no $x$-component, the momentum’s $x$-component is unchanged by the force.
%
%If Earth’s surface were stationary, as Aristotle believed, the a rock at held at the top of the tower would have no horizontal momentum, $p\sub{i}x$=0}. When released, it continues to have no horizontal momentum, falling directly downward and landing at the base of the tower. Since this is what does happen, we seem to have strong evidence for Aristotle’s view.
%
%However, if the tower is moving, due to Earth’s rotation, then the rock held at the top of the tower would have significant horizontal momentum in order to move along with the tower, $p\sub{i}x$=mv\sub{tower}}. When released that stone would keep this horizontal momentum, falling toward the ground but simultaneously moving horizontally along with the tower. As rock falls, the tower moves $1.5\units{km}$ to the east, and the rock also travels exactly the same distance, striking the ground right at the base of the tower, exactly as we expect.

%In modern language we refer to them as different components of the momentum.


%\startblockquote
%%\paragraph{II. Plane equation}, which expresses
%[T]he conservation of the common or total progress of the two bodies.
%\startformula
%	av + by = ax + bz.
%\stopformula
%I call \emph{progress} here the quantity of motion which proceeds from the side of the centre of gravity, so that if the body $b$, for example, should proceed in the contrary direction before the impact, and thus its conspiring velocity $y$ be negative or be expressed by $-\abs{y}$, %understanding by $\abs{y}...$ mass (molem), or that which is positive in $y$,
%then the progress of $a$ will be $av$, the progress of $b$ will be $-b\abs{y}$. And the total progress will be $av - b\abs{y}$, which is the difference of the quantities of motion of the two bodies. If the bodies $a$ and $b$ proceed from one and the same side before and after the impact, these letters, $v$, $y$, $x$, $z$, signify only conspiring velocities real or affirmative, and consequently in this case it appears by this equation that the same quantity of motion will be conserved after and before the impact. But if the bodies $a$ and $b$ should proceed in a contrary direction before the impact and in the same direction after the impact, the difference of the quantity of motion before the impact would be equal to the sum of the quantity of motion after the impact. And there will be other similar variations according to the variation of the signs of the letters $y$, $x$, $z$.
%\stopblockquote


%\section{What about $F=ma$?}\label{sec:Fma}
%The formula for impulse uses the duration of the force’s action and the second involving the displacement.
%\startformula\startmathalignment
%\NC	p\sf \NC = p\si + F\Delta t	\NR
%\NC	F \NC = \frac{p\sf - p\si}{\Delta t}	\NR
%\NC		\NC = \frac{mv\sf - mv\si}{\Delta t}	\NR
%\NC		\NC = m\frac{v\sf - v\si}{\Delta t}	\NR
%\NC		\NC = m\frac{\Delta v}{\Delta t}	\NR
%\NC		\NC = ma	\NR
%\stopmathalignment\stopformula
%
%\highlightbox{
%\begin{equation}
%	F = \frac{dp}{dt}
%	\label{eq:Ns2nd}
%\end{equation}
%}%\end{shaded}




%There are three operations on vectors: addition, scalar multiplication, and finding a magnitude. There is also a special vector called the zero vector.
%
%Show that displacement vectors are true vectors by giving the rules for vectors and checking some of them off.\footnote{\booktitle{Linear and Geometric Algebra}, by Alan Macdonald is a great reference for this section.\autocite{Macdonald2010}}
%
%\begin{enumerate}
%	\item $a\vec v \in V$, $\vec u + \vec{u} \in V$
%	\item $\vec{u} + \vec{v} = \vec{v} + \vec{u}$
%	\item $(\vec{u} + \vec{v}) + \vec{w} = \vec{u} + (\vec{v} + \vec{w})$
%	\item $\vec{v} + \vec{0} = \vec{v}$
%	\item $0\vec{v} = \vec{0}$
%	\item $1\vec{v} = \vec{v}$
%	\item $a(\vec{u} + \vec{v}) = a\vec{u} + a\vec{v}$
%	\item $(a+b)\vec{v} = a\vec{v} + b\vec{v}$
%\end{enumerate}
%

% Templates:

% Margin image
\placefigure[margin][] % Location, Label
{} % Caption
{\externalfigure[][width=144pt]} % File

% Margin Figure
\placefigure[margin][] % location
{}	% caption text
{\noindent\starttikzpicture	% tikz code
\stoptikzpicture}
