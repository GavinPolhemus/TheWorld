% !TEX TS-program = ConTeXt Suite
% !TEX useOldSyncParser
\startcomponent c_chapter08
\project project_world
\product prd_volume01

\setupsynctex[state=start,method=max] % "method=max" or "min"

%%%%%%%%%%%%%%%%%%%%%%%%%%%%%
\startchapter[title={Waves, Sound and Music}, reference=ch:Waves]
%%%%%%%%%%%%%%%%%%%%%%%%%%%%%

\placefigure[margin,none]{}{\small
	\startalignment[flushleft]
It always struck me as a wonderful and peculiarly interesting mystery, that in the theory of musical sounds, in the physical and technical foundations of music, which above all other arts seems in its action on the mind as the most immaterial, evanescent, and tender creator of incalculable and indescribable states of consciousness, that here in especial the science of purest and strictest thought – mathematics – should prove preeminently fertile.\autocite{p.46}{Helmholtz1857}
	\stopalignment
	\startalignment[flushright]
	{\it On the Physiological Causes\\
		of Harmony in Music}\\
	{\sc Hermann Helmholtz}\\
	1821–1894
	\stopalignment
}

\Initial{I}{n the winter of 1857,} physicist and physician Herman Helmholtz gave a public lecture on the physics and physiology of music, a subject of his intense study for several years.

\startblockquote
The short space of time at my disposal obliges me to confine my attention to one particular point; but I shall select the most important of all, which will best show you the significance and results of scientific investigation in this field; I mean the foundation of concord. It is an acknowledged fact that the numbers of the vibrations of concordant tones bear to each other ratios expressible by small whole numbers. But why? What have the ratios of small whole numbers to do with concord? This is an old riddle, propounded by Pythagoras, and hitherto unsolved. Let us see whether the means at the command of modern science will furnish the answer.\autocite{p.~47}{Helmholtz1857}
\stopblockquote

Helmholtz had been interested in physics since childhood. However, \quotation{physics was at that time looked upon as an art by which a living could not be made,}\autocite{p.~384}{Helmholtz1891} so he turned his attention to living nature, studying medicine at \quotation{our Army Medical School, the Friedrich Willhelms Institut, which very materially helped the poorer students through their medical course.}\autocite{p.~385}{Helmholtz1891} Having benefited from this assistance, Helmholtz entered military service as a surgeon in 1843.

He still approached his work with a physicist’s perspective and became curious about how he might apply conservation of energy to living organisms. He devoted considerable attention to working every detail he could, and this developed into a wide ranging manifesto for conservation of energy across all areas of scientific inquiry. This bold and insightful work, published in 1847, gained him entry into the academic world, where he frequently worked at the intersection of physiology and physics – studying electrical impulses in nerves, heat generated by muscles, light and vision, and sound and hearing. His study of sound and hearing began in 1852 and developed into a meticulous and thorough experimental program, drawing on the latest scientific advances. Yet the ancient and simple Pythagorean model of music – simple ratios produce consonant sounds – presented the most profound challenge.

After introducing this riddle to his audience, Helmholtz translates ancient musical ideas in modern language. (You may recall some of this from Chapter \in[ch:Music].)

%\section{Musical tones are produced by rapid, regular vibrations}
\startblockquote
First of all, what is a musical tone? Common experience teaches us that all sounding bodies are in a state of vibration. This vibration can be seen and felt; and in the case of loud sounds we feel the trembling of the air even without touching the sounding bodies. Physical science has ascertained that any series of impulses which produce a vibration of the air will, if repeated with sufficient rapidity, generate sound.

This sound becomes a \emph{musical} tone, when such rapid impulses recur with perfect regularity and in precisely equal times. Irregular agitation of the air generates only noise. The \emph{pitch} of a musical tone depends on the number of impulses which take place in a given time; the more there are in the same time the higher or sharper is the tone. And, as before remarked, there is found to be a close relationship between the well-known harmonious musical intervals and the number of the vibrations of the air. If twice as many vibrations are performed in the same time for one tone as for another, the first is the octave above the second. If the numbers of vibrations in the same time are as 2 to 3, the two tones form a fifth; if they are as 4 to 5, the two tones form a major third.

The musical pitch of a tone depends entirely on the number of vibrations of the air in a second, and not at all upon the mode in which they are produced. It is quite, indifferent whether they are generated by the vibrating strings of a piano or violin, the vocal chords of the human larynx, the metal tongues of the harmonium, the reeds of the clarionet, oboe and bassoon, the trembling lips of the trumpeter, or the air cut by a sharp edge in organ pipes and flutes.

A tone of the same number of vibrations has always the same pitch, by whichever one of these instruments it is produced. That which distinguishes the note A of a piano for example, from the equally high A of the violin, flute, clarionet, or trumpet, is called the \emph{quality of the tone,} and to this we shall have to recur presently.
\stopblockquote

Musical tones, as Helmholtz explains, are produced by rapid periodic motion. This is great news, because you have become a periodic motion expert! Periodic motion of pendulums (and also masses on springs) have been a recurring theme since Galileo’s earliest pendulum observations in Chapter \in[ch:Motion]. Everything you have leaned about periodic motion applies to musical motion as well: period and frequency are related by \m{f=1\units{cyc}/T}, period and frequency are independent of amplitude, and the total energy remains constant during the motion, changing between purely potential energy at the turning points and purely kinetic energy when passing the equilibrium point.

\section{Motion of a musical string}
Helmholtz explains the periodic motion responsible for musical tones using the example of a single string, like the musical strings described in Chapter \in[ch:Music]. There are two important differences between the motion of a pendulum and the motion of a musical string. The first is simply a mater of scale. The pendulum’s motion is large and slow, making it easy to see and measure. The string’s motion is much smaller and quicker, making it difficult to see and very tricky to measure. However, the string’s motion is much easier to hear! The pendulum’s periodic motion is too slow to produce an audible pitch, but the string produces a clearly audible musical tone. The rapid string motion is usually described in terms of frequency, while the pendulum’s slower motion is described in terms of period, but this is purely a matter of convenience since the two are interchangeable through \m{f=1\units{cyc}/T}.

The second difference is more subtle, yet very important. While a pendulum can only swing back and forth, a string is capable of achieving an almost endless variety of shapes during the course of its motion. Helmholtz describes the simplest of these shapes, called \emph{pure waves.} (He greatly exaggerates the displacements in the diagram so that they can be seen. On a real instrument the displacements are hardly visible.)

%\section{Strings produce the simple tones of the harmonic series.}
\startblockquote
\placetextfloat[bottom][fig:HelmholtzStringModes]{Pure waves on a string vibrate with different frequencies. The lowest frequency is called the fundamental frequency of the string. The other motions’ frequencies are multiples of the fundamental frequency.} {\externalfigure[chapter08/HelmholtzStringModes][width=\textwidth]}
Fig.~\in[fig:HelmholtzStringModes], gives a number of forms of vibration of a string, corresponding to simple tones. The continuous line shows the extreme displacement of the string in one direction, and the dotted line in the other. The string labeled a produces its fundamental tone, the deepest simple tone it can produce, vibrating in its whole length, first on one side and then on the other. The string labeled b falls into two vibrating sections, separated by a single stationary point β, called a \emph{node} (knot). The tone is an octave higher, the same as each of the two sections would separately produce, and it performs twice as many vibrations in a second as the fundamental tone. The string labeled c has two nodes, \m{γ_1} and \m{\gamma_2} , and three vibrating sections, each vibrating three times as fast as the fundamental tone and hence giving its twelfth. The string labeled d has three nodes, \m{δ_1}, \m{δ_2}, \m{δ_3}, and four vibrating sections, each vibrating four times as quickly as the fundamental tone, and giving the second octave above it.

In the same way forms of vibration may occur with 5, 6, 7, etc., vibrating sections, each performing respectively, 5, 6, 7, etc. times as many vibrations in a second as the fundamental tone.%, and all other vibrational forms of the string may be conceived as compounded of a sum of such simple vibrational forms.
\stopblockquote

Hamilton’s energy methods are ideally suited to these pure waves. To achieve the curved shapes in figure \in[fig:HelmholtzStringModes] the string must stretch a bit, which stores potential energy in the string. The potential energy decreases as the string returns to its straight, equilibrium shape and then increases again as the string continues into its oppositely distorted shape (the dotted shapes in figure \in[fig:HelmholtzStringModes]). It is a simple matter to calculate frequencies from the potential energies, with the help of some calculus tricks. The result is exactly as Helmholtz states – pure waves with 2, 3, 4, etc.\ sections on the string have frequencies 2, 3, 4, etc.\ times the fundamental frequency.

Not only can a string vibrate in all of the ways shown in figure \in[fig:HelmholtzStringModes], the string can perform all of these periodic motions \emph{simultaneously}. In fact, any time a string is plucked, hammered, or bowed all of these different vibrations are put into motion. Marin Mersenne reported hearing the simultaneous vibrations, up to the fifth, on a monochord in 1637. Helmholtz explains how you may learn to hear these simultaneous vibrations as well.

\startblockquote
The vibrational forms with stationary points or nodes may be produced, by gently touching the string at one of these points, either with the finger or a rod, and rubbing the string with a violin bow, plucking it with the finger, or striking it with a pianoforte hammer. The bell-like harmonics or flageolet-tones of strings, so much used in violin playing, are thus produced.

Now suppose that a string has been excited, and after its tone has been allowed to continue for a moment, it is touched gently at its middle point \m{β}, Fig.~\in[fig:HelmholtzStringModes] b, or \m{δ_2}, Fig.~\in[fig:HelmholtzStringModes] d. The vibrational forms a and c, for which this point is in motion, will be immediately checked and destroyed; but the vibrational forms b and d, for which this point is at rest, will not be disturbed, and the tones due to them will continue to be heard. In this way we can readily discover whether certain members of the series of simple tones are contained in the compound tone of a string when excited in any given way, and the ear can be rendered sensible of their existence.

When once these simple tones in the sound of a string have been thus rendered audible, the ear will readily be able to observe them in the untouched string, after a little accurate attention.

The series of tones which are thus made to combine with a given fundamental tone, is perfectly determinate. They are tones which perform twice, thrice, four times, etc., as many vibrations in a second as the fundamental tone. They are called the \emph{upper partials,} or harmonic overtones, of the fundamental tone.
\stopblockquote

Mersenne was understandably perplexed. \quotation{How is it possible that a string can make many sounds at the same time?} Many physicists and mathematicians attempted to answer this question, but it eluded them until the late eighteenth century. It turns out that this is another form of compound motion. Galileo explained that a projectile moves horizontally with constant velocity while it falls with constant acceleration. The projectile’s motion in one direction has no effect on its motion in the other. Likewise, the pure waves on the string each move independently of the other pure waves.

Hamilton’s energy methods are again ideally suited to the problem. The potential and kinetic energies of the string due to one of the pure waves is not affected by the presence of other pure waves. This calculation requires some calculus tricks that were not known until the eighteenth century, but result is quite simple: the total energy of the string is just the energies of these waves added together. Using Hamilton’s methods to find the motion, the wave

When many pure waves are active on the string at the same time, an endless variety of shapes cane be produced. Helmholtz provides some examples of pure waves combining to form more complex wave shapes.  The top two frames in figure~\in[fig:StringSuperposition1] show two pure waves. When both of these waves a present on the string simultaneously, they add together to form the off-center bump in the bottom frame. The motion of the off-center bump will be somewhat complicated because the two pure waves oscillate with different frequencies. The off-center bump does not produce a new frequency. The pitch will still be that of the fundamental, but the quality of the tone will change due to the simultaneous sound of the frequency one octave higher.

\startbuffer[TikZ:StringSuperposition1]
\environment env_physics
\environment env_TikZ
\setupbodyfont [libertinus,11pt]
\setoldstyle % Old style numerals in text
\startTEXpage\small
\starttikzpicture% tikz code
\tikzmath{
function firstwave(\x){return sin(deg(((3.1415)*(x)));};
function secondwave(\x){return -0.5*sin(deg(((2*3.1415)*(x)));};
}
\startgroupplot[
  group style={
    % group name=my plots,
    group size=1 by 3, % columns by rows
    x descriptions at=edge top,
    % y descriptions at=edge right,
    % horizontal sep=0.5cm,
    vertical sep=0.5cm,
  },
   x={4cm},
   y={.5cm},
   xlabel={Horizontal position (m)},
   xmin=0, xmax=1,
   minor x tick num=1,
   ymin=-1.2, ymax=1.2,
   minor y tick num=3,
  grid=both,
]
\nextgroupplot
 []
  \addplot
  [
   semithick,
   domain=0:1,
   samples=300
  ]
    {firstwave(x)};
\nextgroupplot
 [   ylabel={Displacement (mm)},
]
  \addplot
  [
   semithick,
   domain=0:1,
   samples=300
  ]
    {secondwave(x)};
\nextgroupplot
 [   ymin=-2, ymax=2,
]
  \addplot
  [
   semithick, dotted,
   domain=0:1,
   samples=300
  ]
    {firstwave(x)};
  \addplot
  [
   semithick, dotted,
   domain=0:1,
   samples=300
  ]
    {secondwave(x)};
  \addplot
  [
   semithick, %red,
   domain=0:1,
   samples=300
  ]
    {firstwave(x)+secondwave(x)};
\stopgroupplot
\stoptikzpicture
\stopTEXpage
\stopbuffer

\placefigure[margin][fig:StringSuperposition1] % Location, Label
{Two pure waves combine to form a more complicated wave shape, in this case an off-center bump.}	 % caption text
{\noindent\typesetbuffer[TikZ:StringSuperposition1]}

In his next example to two pure waves in the top two frames of  figure~\in[fig:StringSuperposition2] combine to make a bump with a flat top. In this case the pure waves produce the fundamental frequency and the frequency three times higher. This will have the same pitch as the fundamental alone, but will have a different quality due to the simultaneous higher frequency.

\startbuffer[TikZ:StringSuperposition2]
\environment env_physics
\environment env_TikZ
\setupbodyfont [libertinus,11pt]
\setoldstyle % Old style numerals in text
\startTEXpage\small
\starttikzpicture% tikz code
\tikzmath{
function firstwave(\x){return sin(deg(((3.1415)*(x)));};
function secondwave(\x){return 0.167*sin(deg(((3*3.1415)*(x)));};
}
\startgroupplot[
  group style={
    % group name=my plots,
    group size=1 by 3, % columns by rows
    x descriptions at=edge top,
    % y descriptions at=edge right,
    % horizontal sep=0.5cm,
    vertical sep=0.5cm,
  },
   x={4cm},
   y={.5cm},
   xlabel={Horizontal position (m)},
   xmin=0, xmax=1,
   minor x tick num=1,
   ymin=-1.2, ymax=1.2,
   minor y tick num=3,
  grid=both,
]
\nextgroupplot
 []
  \addplot
  [
   semithick,
   domain=0:1,
   samples=300
  ]
    {firstwave(x)};
\nextgroupplot
 [   ylabel={Displacement (mm)},
]
  \addplot
  [
   semithick,
   domain=0:1,
   samples=300
  ]
    {secondwave(x)};
\nextgroupplot
 [
]
  \addplot
  [
   semithick, dotted,
   domain=0:1,
   samples=300
  ]
    {firstwave(x)};
  \addplot
  [
   semithick, dotted,
   domain=0:1,
   samples=300
  ]
    {secondwave(x)};
  \addplot
  [
   semithick, %red,
   domain=0:1,
   samples=300
  ]
    {firstwave(x)+secondwave(x)};
\stopgroupplot
\stoptikzpicture
\stopTEXpage
\stopbuffer

\placefigure[margin][fig:StringSuperposition2] % Location, Label
{A combination of pure waves to produce a flat-top bump. Any wave shape can be made by combining many pure waves.}	 % caption text
{\noindent\typesetbuffer[TikZ:StringSuperposition2]}

\startblockquote
These extremely simple examples will suffice to give a conception of the great multiplicity of forms resulting from this method of composition. Supposing that instead of two, several simple waves were selected, with heights %and initial points
arbitrarily chosen, an endless variety of changes could be effected, and, in point of fact, any given form of wave could be reproduced.%1
\stopblockquote
%1 Of course the waves could not overhang, but waves of such a form would have no possible analogue in waves of sound [which the reader will recollect are not actually in the forms here drawn, but have only condensations and rarefactions, conveniently replaced by these forms, p. 5].

This last point is important, and was not proven until 1822. Any complex wave shape on the string will exist as a sum of pure waves.

\startblockquote
The French mathematician Fourier has established a celebrated and important theorem which may be translated from mathematical into ordinary language thus: \emph{Any form of wave whatever can be compounded of a number of simple waves of different lengths.} The longest of these simple waves has the same length as that of the given form of wave, the others have lengths one-half, one-third, one-fourth, etc. as great.

By the different modes of uniting the crests and hollows of these simple waves, an endless multiplicity of wave-forms may be produced.
\stopblockquote

The frequencies heard by Mersenne, which are produced by the pure waves, are the only frequencies that the string can produce. The string can take on an endless variety of shapes, but any shape of the string is created as the sum of many pure waves. There are many ways to excite a musical string – guitar strings are plucked, violin strings are bowed, and piano strings are struck. Each of these produces a different shape  wave on the string. These different shapes contain different amounts of the various pure waves, giving each method of playing a distinctive quality of tone.

\section{Sound is a wave.}
For us to hear the strings tone, the vibrations of the string must travel through the air to our ears. These vibrations are carried through the air as sound waves, which are in many ways similar to the waves on the string itself. Sound waves are not confined to the string, however, and can spread out in all directions. For this reason, Helmholtz compares them to waves on the ocean, which are free to spread in any direction across the surface.

\startblockquote
I must now describe the propagation of sound through the atmosphere. The motion of a mass of air through which a tone passes, belongs to the so-called wave motions – a class of motions of great importance in physics. Light, as well as sound, is one of these motions.

The name is derived from the analogy of waves on the surface of water, and these will best illustrate the peculiarity of this description of motion.

When a point in a surface of still water is agitated – as by throwing in a stone – the motion thus caused is propagated in the form of waves, which spread in rings over the surface of the water. The circles of waves continue to increase even after rest has been restored at the point first affected. At the same time the waves become continually lower, the further they are removed from the centre of motion, and gradually disappear. On each wave-ring we distinguish ridges or crests, and hollows or troughs.

Crest and trough together form a wave, and we measure its length from one crest to the next.
\stopblockquote

Figure~\in[fig:Wavelenth] shows the crest-to-crest measure of a wavelength. The symbol for wavelength is a greek lambda: \m{\lambda}. Notice that Figure~\in[fig:Wavelenth] is not a position vs.\ time graph. The horizontal axis is the position along the wave, and the vertical axis is the displacement caused by the wave.
\startbuffer[TikZ:Wavelenth]
\environment env_physics
\environment env_TikZ
\setupbodyfont [libertinus,11pt]
\setoldstyle % Old style numerals in text
\startTEXpage\small
\starttikzpicture% tikz code
	\startaxis[
		footnotesize,
		width=2.25in,%\marginparwidth,
		y={10mm},%x={2mm},
		%xlabel={\m{t} (s)},
		xlabel={position (cm)},
		xmin=0, xmax=20,
		%xticklabels=\empty,
		minor x tick num=4,
		ylabel={displacement (mm)},
	  	%every axis y label/.style={at={(ticklabel cs:0.5)},rotate=90,anchor=center},
		ymin=-1.5, ymax=1.5,
		minor y tick num=4,
	   	extra y ticks={25},
	   	extra y tick labels=\empty,
   		extra y tick style={grid=major},
		clip mode=individual,
		]
		%\node[below left] at (axis description cs:1,1) {Lead};
		\addplot[thick,smooth,domain=0:20,samples=801]{cos(2*deg(x))};
  		\draw [{<[scale=0.7]}-{>[scale=0.7]}](3.14, 1.2) -- node[fill=white,inner sep=1pt]{\m{\lambda}}  (6.29, 1.2);
  		\draw [](3.14, 1) -- (3.14, 1.3);
  		\draw [](6.29, 1) -- (6.29, 1.3);
	\stopaxis
\stoptikzpicture
\stopTEXpage
\stopbuffer

\placefigure[margin][fig:Wavelenth] % Location, Label
{A wave and its wavelength \m{\lambda}.}	 % caption text
{\noindent\typesetbuffer[TikZ:Wavelenth]}

\startblockquote
While the wave passes over the surface of the fluid, the particles of the water which form it do not move on with it. This is easily seen, by floating a chip of straw on the water. When the waves reach the chip, they raise or depress it, but when they have passed over it, the position of the chip is not perceptibly changed.

Now a light floating chip has no motion different from that of the adjacent particles of water. Hence we conclude that these particles do not follow the wave, but, after some pitching up and down, remain in their original position. That which really advances as a wave is, consequently, not the particles of water themselves, but only a superficial form, which continues to be built up by fresh particles of water. The paths of the separate particles of water are more nearly vertical circles, in which they revolve with a tolerably uniform velocity, as long as the waves pass over them.
\stopblockquote

Sound waves are compression waves, consisting of regions where the air is alternately compressed together and then spread out, as shown in figure~\in[fig:CompressionWave1]. The horizontal displacement of the air particles is shown in the lower graph. Notice that where the graph is increasing the air is less dense. Particles just to the left of the diagram’s center are displaced to the left, while those just to the right of center are displaced to the right, leaving a low density region in the center. In regions where the graph is decreasing the particles are crowded together.

\startbuffer[TikZ:CompressionWave1]
\environment env_physics
\environment env_TikZ
\setupbodyfont [libertinus,11pt]
\setoldstyle % Old style numerals in text
\startTEXpage\small
\starttikzpicture% tikz code
\tikzmath{
function wave(\x){return cos(540*x);};
}
\startgroupplot[
  group style={
    % group name=my plots,
    group size=1 by 3, % columns by rows
    x descriptions at=edge bottom,
    % y descriptions at=edge right,
    % horizontal sep=0.5cm,
    vertical sep=0.5cm,
  },
   x={10cm},
   y={1cm},
   ytick=\empty,
   xmin=0, xmax=1,
   ymin=-1, ymax=1,
]
%\nextgroupplot
% [
%  axis y line = none,
%]
% \foreach \i in {1,2,...,3000}{
%  \fill let \n1={0.5 + 0.6*rand}
%        in ({\n1-0.09*cos(540*\n1)},rand) circle[radius=.4pt];
% }
\nextgroupplot
 [  axis y line = none,
]
 \foreach \i in {1,2,...,3000}{
  \fill let \n1={0.5 + 0.6*rand}
        in ({\n1+0.09*cos(540*\n1)},rand) circle[radius=.4pt];
 }
\nextgroupplot
 [ylabel={Displacement},
   xlabel={Horizontal position (m)},
   extra y ticks={0},
   extra y tick style={grid=major},
   extra y tick labels={\m{0}},
   minor x tick num=1,
]
%  \addplot
%  [
%   semithick, %red,
%   domain=0:1,
%   samples=300
%  ]
%    {wave(x)};
  \addplot
  [
   thick, %dotted, %red,
   domain=0:1,
   samples=300
  ]
    {wave(x)};
\stopgroupplot
\stoptikzpicture
\stopTEXpage
\stopbuffer

\placetextfloat[bottom][fig:CompressionWave1] % Location, Label
{A sound wave consists of regions of compressed and rarified air. The displacement of the air particles is shown in graph. Positive displacement represents particles shifted slightly to the right, while negative displacement represents particles shifted slightly to the left. Because particles are shifted different amounts, there are regions where the particles are crowded together, and regions where they are spread out.}	 % caption text
{\noindent\typesetbuffer[TikZ:CompressionWave1]}

\startblockquote
To return from waves of water to waves of sound. Imagine an elastic fluid like air to replace the water, %the ridges are replaced by condensed strata of air, and the hollows by rarefied strata. Now further imagine that these compressed waves are propagated by the same law as before,
and that also the vertical circular orbits of the several particles of water are compressed into horizontal straight lines. Then the waves of sound will retain the peculiarity of having the particles of air only oscillating backwards and forwards in a straight line, while the wave itself remains merely a progressive form of motion, continually composed of fresh particles of air. The immediate result then would be waves of sound spreading out horizontally from their origin.

But the expansion of waves of sound is not limited, like those of water, to a horizontal surface. They can spread out in any direction whatsoever. Suppose the circles generated by a stone thrown into the water to extend in all directions of space, and you will have the spherical waves of air by which sound is propagated.

Hence we can continue to illustrate the peculiarities of the motion of sound, by the well-known visible motions of waves of water.

The length of a wave of water, measured from crest to crest, is extremely different. A falling drop, or a breath of air, gently curls the surface of the water. The waves in the wake of a steamboat toss the swimmer or skiff severely. But the waves of a stormy ocean can find room in their hollows for the keel of a ship of the line, and their ridges can scarcely be overlooked from the masthead. The waves of sound present similar differences. The little curls of water with short lengths of wave correspond to high tones, the giant ocean billows to deep tones. Thus the contrabass C has a wave thirty-five feet long, its higher octave a wave of half the length, while the highest tones of a piano have waves of only three inches in length.
%1The exact lengths of waves corresponding to certain notes, or symbols of a tone, depend upon the standard pitch assigned to one particular note, and this differs in different countries. Hence the figures of the author have been left unreduced. They are sufficiently near to those usually adopted in England, to occasion no difficulty to the reader in these general remarks. – Tr.

%You perceive that the pitch of the tone corresponds to the length of the wave. To this we should add that the height of the ridges, or, transferred to air, the degree of alternate condensation and rarefaction, corresponds to the loudness and intensity of the tone. But waves of the same height may have different forms. The crest of the ridge, for example, may be rounded off or pointed. Corresponding varieties also occur in waves of sound of the same pitch and loudness. The so-called \emph{timbre} or quality of tone is what corresponds to the \emph{form} of the waves of water. In this sense then we can continue to speak of the form of waves of sound, and can represent it geometrically. We make the curve rise where the pressure, and hence density, increases, and fall where it diminishes.

%Among the forms of waves of sound is one of great importance, here termed the \emph{simple} or \emph{pure} wave-form, and represented in Fig.~\in[fig:HelmholtzSimpleTone].

%\placetextfloat[here][fig:HelmholtzSimpleTone]{A simple or pure waveform.} {\externalfigure[chapter08/HelmholtzSimpleTone]}%[width=\textwidth]

%It can be seen in waves of water only when their height is small in comparison with their length, and they run over a smooth surface without external disturbance, or without any action of wind. Ridge and hollow are gently rounded off, equally broad and symmetrical, so that, if we inverted the curve, the ridges would exactly fit into the hollows, and conversely. This form of wave would be more precisely defined by saying that the particles of water describe exactly circular orbits of small diameters, with exactly uniform velocities. To this simple wave-form corresponds a peculiar species of tone, which, from reasons to be hereafter assigned, depending upon its relation to quality, we will term a simple tone. The tone of tuneful human voices, singing the vowel oo in too, in the middle positions of their register, appears not to differ materially from this form of wave.

Finally, I would direct your attention to an instructive spectacle, which I have never been able to view without a certain degree of physico-scientific delight, because it displays to the bodily eye, on the surface of water, what otherwise could only be recognised by the mind’s eye of the mathematical thinker in a mass of air traversed in all directions by waves of sound. I allude to the composition of many different systems of waves, as they pass over one another, each undisturbedly pursuing its own path. We can watch it from the parapet of any bridge spanning a river, but it is most complete and sublime when viewed from a cliff beside the sea. It is then rare not to see innumerable systems of waves, of various length, propagated in various directions. The longest come from the deep sea and dash against the shore. Where the boiling breakers burst shorter waves arise, and run back again towards the sea. Perhaps a bird of prey darting after a fish gives rise to a system of circular waves, which, rocking over the undulating surface, are propagated with the same regularity as on the mirror of an inland lake. And thus, from the distant horizon, where white lines of foam on the steel-blue surface betray the coming trains of wave, down to the sand beneath our feet, where the impression of their arcs remains, there is unfolded before our eyes a sublime image of immeasurable power and unceasing variety, which, as the eye at once recognises its pervading order and law, enchains and exalts without confusing the mind.

Now, just in the same way you must conceive the air of a concert-hall or ballroom traversed in every direction, and not merely on the surface, by a variegated crowd of intersecting wave-systems. From the mouths of the male singers proceed waves of six to twelve feet in length; from the lips of the songstresses dart shorter waves, from eighteen to thirty-six inches long. The rustling of silken skirts excites little curls in the air, each instrument in the orchestra emits its peculiar waves, and all these systems expand spherically from their respective centres, dart through each other, are reflected from the walls of the room, and thus rush backwards and forwards, until they succumb to the greater force of newly generated tones.

Although this spectacle is veiled from the material eye, we have another bodily organ, the ear, specially adapted to reveal it to us. This analyses the interdigitation of the waves, which in such cases would be far more confused than the intersection of the water undulations, separates the several tones which compose it, and distinguishes the voices of men and women – nay, even of individuals – the peculiar qualities of tone given out by each instrument, the rustling of the dresses, the footfalls of the walkers, and so on.

%It is necessary to examine the circumstances with greater minuteness. When a bird of prey dips into the sea, rings of waves arise, which are propagated as slowly and regularly upon the moving surface as upon a surface at rest. These rings are cut into the curved surface of the waves in precisely the same way as they would have been into the still surface of a lake. The form of the external surface of the water is determined in this, as in other more complicated cases, by taking the height of each point to be the height of all of the waves which coincide at this point at one time. \emph{The height of every point of the surface of the water is equal to the sum of all the waves which at that moment there concur.}

%It is the same with the waves of sound. They, too, are added together at every point of the mass of air, as well as in contact with the listener’s ear. For them also the degree of condensation and the velocity of the particles of air in the passages of the organ of hearing are equal to the sums of the separate degrees of condensation and of the velocities of the waves of sound, considered apart.

When various simple waves concur on the surface of water, the compound wave-form has only a momentary existence, because the longer waves move faster than the shorter, and consequently the two kinds of wave immediately separate, giving the eye an opportunity of recognising the presence of several systems of waves. But when waves of sound are similarly compounded, they never separate again, because long and short waves traverse air with the same velocity. Hence the compound wave is permanent, and continues its course unchanged, so that when it strikes the ear, there is nothing to indicate whether it originally left a musical instrument in this form, or whether it had been compounded on the way, out of two or more undulations.
\stopblockquote

\section{The ear senses pitch}
\startblockquote
This single motion of the air produced by the simultaneous action of various sounding bodies, has now to be analysed by the ear into the separate parts which correspond to their separate effects. For doing this the ear is much more unfavourably situated than the eye. The latter surveys the whole undulating surface at a glance. But the ear can, of course, only perceive the motion of the particles of air which impinge upon it. And yet the ear solves its problem with the greatest exactness, certainty, and determinacy. This power of the ear is of supreme importance for hearing. Were it not present it would be impossible to distinguish different tones.

Some recent anatomical discoveries appear to give a clue to the explanation of this important power of the ear.
\stopblockquote

Helmholtz relates some remarkable discoveries about the way in which the ear is able to determine pitch. However, more detailed study in the twentieth century has shown that the issue is more complicated than he describes. The ear does perceive pitch, but only to an accuracy of about \m{\pm 15\%}, which corresponds to roughly a minor third. The nervous system then further analyzes the signal from the ear and is able to determine the frequency with far greater accuracy. Although Helmholtz’s interpretation of the ear’s anatomy was too simplistic, his conclusion is essentially correct.

\startblockquote
Experience then shows us that the ear really possesses the power of analysing waves of air into their elementary forms.
%By compound motions of the air, we have hitherto meant such as have been caused by the simultaneous vibration of several elastic bodies. Now, since the forms of the waves of sound of different musical instruments are different, there is room to suppose that the kind of vibration excited in the passages of the ear by one such tone will be exactly the same as the kind of vibration which in another case is there excited by two or more instruments sounded together. If the ear analyses the motion into its elements in the latter case, it cannot well avoid doing so in the former, where the tone is due to a single source. And this is found to be really the case.
\stopblockquote

The final section of Helmholtz’s lecture draws on all of the ideas set out in this chapter – the multitude of frequencies on a musical string, the combination of sound waves in the air, and the ear’s perception of pitch – to explain the Pythagoreans’ ancient model of consonance. Each of the topics in this chapter is challenging, so you will have an opportunity to master them before reading the final section of the lecture, which will be provided separately.

\stopchapter
\stopcomponent

%%%%% END

\section{What this means for music}
\startblockquote
If this last be c, the series may be written as follows in musical notation, [it being understood that, on account of the temperament of a piano, these are not precisely the fundamental tones of the corresponding strings on that instrument, and that in particular the upper partial, \m{b''\!\flat}, is necessarily much flatter than the fundamental tone of the corresponding note on the piano].

\startalignment[middle]\dontleavehmode\externalfigure[chapter08/HelmholtzHarmonicSeries][]\stopalignment

Not only strings, but almost all kinds of musical instruments, produce waves of sound which are more or less different from those of simple tones, and are therefore capable of being compounded out of a greater or less number of simple waves. The ear analyses them all by means of Fourier’s theorem better than the best mathematician, and on paying sufficient attention can distinguish the separate simple tones due to the corresponding simple waves.

Hence a simple tone is one excited by a succession of simple wave-forms. All other wave-forms, such as those produced by the greater number of musical instruments, excite sensations of a variety of simple tones.

Consequently, all the tones of musical instruments must in strict language, so far as the sensation of musical tone is concerned, be regarded as chords with a predominant fundamental tone.

The whole of this theory of upper partials or harmonic overtones will perhaps seem new and singular. Probably few or none of those present, however frequently they may have heard or performed music, and however fine may be their musical ear, have hitherto perceived the existence of any such tones, although, according to my representations, they must be always and continuously present. In fact, a peculiar act of attention is requisite in order to hear them, and unless we know how to perform this act, the tones remain concealed. As you are aware, no perceptions obtained by the senses are merely sensations impressed on our nervous systems. A peculiar intellectual activity is required to pass from a nervous sensation to the conception of an external object, which the sensation has aroused. The sensations of our nerves of sense are mere symbols indicating certain external objects, and it is usually only after considerable practice that we acquire the power of drawing correct conclusions from our sensations respecting the corresponding objects. Now it is a universal law of the perceptions obtained through the senses, that we pay only so much attention to the sensations actually experienced, as is sufficient for us to recognise external objects. In this respect we are very one-sided and inconsiderate partisans of practical utility; far more so indeed than we suspect. All sensations which have no direct reference to external objects, we are accustomed, as a matter of course, entirely to ignore, and we do not become aware of them till we make a scientific investigation of the action of the senses, or have our attention directed by illness to the phenomena of our own bodies. Thus we often find patients, when suffering under a slight inflammation of the eyes, become for the first time aware of those beads and fibres known as \emph{mouches volantes} swimming about within the vitreous humour of the eye, and then they often hypochondriacally imagine all sorts of coming evils, because they fancy that these appearances are new, whereas they have generally existed all their lives.

To this class of phenomena belong the upper partial tones. It is not enough for the auditory nerve to have a sensation. The intellect must reflect upon it. Hence my former distinction of a material and a spiritual ear.

We always hear the tone of a string accompanied by a certain combination of upper partial tones. A different combination of such tones belongs to the tone of a flute, or of the human voice, or of a dog’s howl. Whether a violin or a flute, a man or a dog is close by us is a matter of interest for us to know, and our ear takes care to distinguish the peculiarities of their tones with accuracy. The means by which we can distinguish them, however, is a matter of perfect indifference.

Whether the cry of the dog contains the higher octave or the twelfth of the fundamental tone, has no practical interest for us, and never occupies our attention. The upper partials are consequently thrown into that unanalysed mass of peculiarities of a tone which we call its \emph{quality.} Now as the existence of upper partial tones depends on the \emph{waveform,} we see, as I was able to state previously (p. 4), that the \emph{quality of tone} corresponds to the form of wave.
\stopblockquote

\subject{Notes}
%\placefootnotes[criterium=chapter]
\placenotes[endnote][criterium=chapter]

%\subject{Bibliography}
%        \placelistofpublications

%\stopchapter
%\stopcomponent

% Templates:

% Margin image
\placefigure[margin][] % Location, Label
{} % Caption
{\externalfigure[chapter03/][width=144pt]} % File

% Margin Figure
\startbuffer[TikZ:NAME]
\environment env_physics
\environment env_TikZ
\setupbodyfont [libertinus,11pt]
\setoldstyle % Old style numerals in text
\startTEXpage\small
\starttikzpicture% tikz code
\stoptikzpicture
\stopTEXpage
\stopbuffer

\placefigure[margin][fig:NAME] % Location, Label
{}	 % caption text
{\noindent\typesetbuffer[TikZ:NAME]}

% Aligned equation
\startformula\startmathalignment
\stopmathalignment\stopformula

% Aligned Equations
\startformula\startmathalignment[m=2,distance=2em]
\stopmathalignment\stopformula
