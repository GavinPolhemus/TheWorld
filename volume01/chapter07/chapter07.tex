% !TEX useAlternatePath
% !TEX useConTeXtSyncParser

\startcomponent c_chapter07
\project project_world
\product prd_volume01

\doifmode{*product}{\setupexternalfigures[directory={chapter07/images}]}

\setupsynctex[state=start,method=max] % "method=max" or "min"

%%%%%%%%%%%%%%%%%%%%%%%%%%%%%
\startchapter[title=Kepler's New Astronomy, reference=ch:Rotation]
%%%%%%%%%%%%%%%%%%%%%%%%%%%%%
% This chapter only deals with orbital angular momentum in 2D and orbits.

\placefigure[margin,none]{}{\small
	\startalignment[flushleft]%
What matters to me is not merely to impart to the reader what I have to say, but above all to convey to him the reasons, subterfuges, and lucky hazard which led me to my discoveries. When Christopher Columbus, Magellan, and the Portuguese relate how they went astray on their journeys, we not only forgive them, but would regret to miss their narration because without it the whole, grand entertainment would be lost. Hence I shall not be blamed if, prompted by the same affection for the reader, I follow the same method.
%\autocite[p. 318]{Koestler, Sleepwalkers}
	\stopalignment
	\startalignment[flushright]
	{\it New Astronomy}\\
	{\sc Johannes Kepler}\\
	1571--1630
	\stopalignment
}

\Initial{J}{ohannes Kepler,} at the age of 23, became interested in astronomy for a simple reason: it was his job. He had been studying for the priesthood, but he also loved mathematics. Before is graduation in the spring 1594, he received an unexpected invitation to teach mathematics and astronomy at the school in Graz, Austria.
Although Kepler had never thought of becoming an astronomer, he had taken the required astronomy courses at the University of Tübingen. He understood the Ptolemaic and Copernican systems (\at{pp.}[AncientWorldModels]-\at[AncientWorldModelsEnd]). After a bit of hesitation, he accepted the invitation.
 
%Since the Moon, Sun, and Earth are all circular – the most perfect shape – the Pythagoreans believed that all all celestial bodies circle Earth, a model that ran into conflicts with observations immediately. To explain the planet's erratic motions across the sky, ancient astronomers introduced compound motion into their models, so that the planets moved on circles which were themselves being carried by other circles.

\placefigure[margin][fig:Kepler1596system]{Woodcut from Kepler's \booktitle{Mysterium Cosmographicum} showing his early model of the solar system based on the five platonic solids.} {\externalfigure[Kepler1596system][width=\rightmarginwidth]}

During his second summer in Graz, Kepler was struck by a novel idea: perhaps the planets' orbits were exactly the right sizes to fit inside nested Platonic solids, as shown in \in{figure}[fig:Kepler1596system]. This idea is wrong, but given the close connection between geometry and astronomy, it was not an especially outlandish proposal in the sixteenth century. More controversial was the fact that he nested these polygons with the Sun at the center and Earth in orbit (in a spherical shell between the model's icosahedron and dodecahedron).
Kepler did not think that these spheres and shapes were physical objects in the solar system. Rather, he viewed them as a mathematical plan for the solar system.
In 1597, he published \booktitle{Mysterium Cosmographicum}, a small book exuberantly announcing his heliocentric model.


\startbuffer[U3D]
\startaxis[
 	  %axis line shift=1cm,
	   %axis lines*=left,
   hide x axis,
    hide y axis,
    hide z axis,
        axis lines=center,
        axis on top,
	 view={0}{45},
        width = 7cm,
    %z post scale = {1},
        clip mode = individual,
]
%    \addplot3 [
%        mesh, color = middlegray,
%        z buffer=sort,
%        samples=9,
%        domain=0.1:1,
%        y domain=0:2*pi,
%](
%{x * cos(deg(y))}, {x * sin(deg(y))}, {-10}
%    );
    \addplot3 [
        surf, faceted color = middlegray, color = gray,
        z buffer=sort,
        samples=6,
        domain=2.5:15,
        y domain=0:2*pi,
        samples y=25,
](
{x * cos(deg(y))}, {x * sin(deg(y))}, {-13.27/x}
    );
    \node at (-8,-6,-2.5) {$U$};
    \draw[->] (0,0,0) --node[below, pos = 0.98]{$r$} (16,0,0);
    \draw[shade, ball color = white] (0,0,0) circle[radius=.6mm]node[above=0.8mm] {\Sun};
    \draw[shade, ball color = darkgray] (0.5546,0,0) circle[radius=0.2mm]node[below] {\Mercury};
    \draw[shade, ball color = darkgray] (1.082,0,0) circle[radius=0.3mm]node[above] {\Venus};
    %\draw[] (1.5,0,0) -- (1.5,0,-8.874);
    \draw[shade, ball color = darkgray] (1.496,0,0) circle[radius=0.3mm]node[below=0.4mm] {\Earth};
    \draw[shade, ball color = darkgray] (2.259,0,0) circle[radius=0.25mm]node[above=0.7mm] {\Mars};
    \draw[shade, ball color = darkgray] (7.76,0,0) circle[radius=0.4mm]node[above] {\Jupiter};
    \draw[shade, ball color = darkgray] (14.23,0,0) circle[radius=0.4mm]node[above] {\Saturn};
\stopaxis
\stopbuffer

\marginTikZ{}{U3D}{A planet's gravitational potential energy depends on its distance $r$ from the Sun. A planet released from rest would accelerate toward lower potential energy, crashing into the Sun!} % vskip, name, caption




In \booktitle{Mysterium Cosmographicum}, Kepler argued that the Sun is the driver of the Solar System's motions.
Indeed, this insight was far ahead of its time.
We now know that gravitational potential energy, due to the Sun's enormous mass, produces the forces that keep the planets in their orbits.
Kepler's \quotation{drinking cup}\autocite{Described in a letter to Friedrich, Duke of Wuerttemberg, 27 Feb.~1596. v.~\convertnumber{KR}{13} p.~50.}{KeplerGW} model in \in{figure}[fig:Kepler1596system] even looks like our energy graph in \in{figure}[fig:U3D].
In our energy graph, the Sun produces the huge potential energy well that holds the much lighter planets in their orbits. (The Sun is about one-thousand times more massive that Jupiter, the largest planet.)

The energy graph in \in{figure}[fig:U3D] shows the first six planets at the correct distances along the $r$-axis. Notice how these distances approximately match the distances in Kepler's model. The  four inner planets – Mercury, Venus, Earth, and Mars – are quite close together. Jupiter and Saturn are much further out.
The planets' full orbits are shown in \in{figure}[fig:VisiblePlanets]. The scale in \in{figure}[fig:VisiblePlanets] shows $r$, the distance from the Sun.% and $\theta$ the angle measured around the Sun. (These are commonly called \keyterm{polar coordinates}.)

\startbuffer[TikZ:VisiblePlanets]
\environment env_physics
\environment env_TikZ
\setupbodyfont [libertinus,11pt]
\setoldstyle % Old style numerals in text
\startTEXpage\small
\starttikzpicture% tikz code
\startpolaraxis
 [	xticklabels=\empty,
 	ytick={0,5,...,15},
 	yticklabels={{},{},$1\units{Tm}$,{}},
 	minor y tick num={4},
	% yminorgrids=true,
	hide x axis,
	ymax = 15,
	scale only axis=true, width={11cm},
 	tick style={middlegray}, % Fixes ticks which are too light in ConTeXt
	major grid style = {middlegray},
	clip = false,
 	%ylabel={Distance from Sun $r$ ($\sci{9}\units{m}$)},
 ]
%    \addplot [ % Mars area at aphelion
%        draw=none, fill=black!20,
%        domain={156.08-4.36}:{156.08+4.36},
%        samples=20,
%    ]
%        {2.259/(1+0.0934*cos(x-336.08))}--(0,0)
%    ;
%    \addplot [ % Mars area at perihelion
%        draw=none, fill=black!20,
%        domain={336.08-6.35}:{336.08+6.35},
%        samples=20,
%    ]
%        {2.259/(1+0.0934*cos(x-336.08))}--(0,0)
%    ;
%    \addplot [ % Mercury area at perihelion
%        draw=none, fill=black!20,
%        domain={77.46-59.63}:{77.46+59.63},
%        samples=80,
%    ]
%        {0.5546/(1+0.20564*cos(x-77.46))}--(0,0)
%    ;
%    \addplot [ % Mercury area at aphelion
%        draw=none, fill=black!20,
%        domain={257.46-33.24}:{257.46+33.24},
%        samples=20,
%    ]
%        {0.5546/(1+0.20564*cos(x-77.46))}--(0,0)
%    ;
  	\node [name path=Sun] at (0,0) {\Sun};
    \addplot [ % Mercury
        thick,
        domain=0:360,
        samples=600,
    ]
        {0.5546/(1+0.20564*cos(x-77.46))}
  [yshift=-.5pt]
    node[pos=0.375] {\Mercury}
    ;
    \addplot [ % Venus
        thick,
        domain=0:360,
        samples=600,
    ]
        {1.082/(1+0.00676*cos(x-131.77))}
  [yshift=-1.7pt]
    node[pos=0.167] {\Venus}
    ;
    \addplot [ % Earth
    	name path=Earth,
        thick,
        domain=0:360,
        samples=600,
    ]
        {1.496/(1+0.0167*cos(x-102.93))}
    node[pos=0.25] {\Earth}
    ;
    \addplot [ % Mars
    	name path=Mars,
        thick,
        domain=0:360,
        samples=600,
    ]
        {2.259/(1+0.0934*cos(x-336.08))}
  [yshift=1pt, xshift=1.1pt]
    node[pos=0.625] {\Mars}
    ;
    \addplot [ % Jupiter
    	name path=Jupiter,
        thick,
        domain=0:360,
        samples=600,
    ]
        {7.76/(1+0.04854*cos(x-14.27))}
  %[yshift=1pt, xshift=1.1pt]
    node[below=0mm, pos=0.8] {\Jupiter}
    ;
    \addplot [ % Saturn
    	name path=Saturn,
        thick,
        domain=0:360,
        samples=600,
    ]
        {14.23/(1+0.05551*cos(x-92.86))}
  %[yshift=1pt, xshift=1.1pt]
    node[above, pos=0.2] {\Saturn}
    ;
%    \addplot [ % Uranus
%    	name path=Uranus,
%        thick,
%        domain=0:360,
%        samples=600,
%    ]
%        {28.642/(1+0.04686*cos(x-172.43))}
%  %[yshift=1pt, xshift=1.1pt]
%    node[below, pos=0.2] {\Uranus}
%    ;
%    \addplot [ % Neptune
%    	name path=Neptune,
%        thick,
%        domain=0:360,
%        samples=600,
%    ]
%        {44.981/(1+0.00895*cos(x-46.68))}
%  %[yshift=1pt, xshift=1.1pt]
%    node[below, pos=0.2] {\Neptune}
%    ;
%    \addplot [ % Halley
%    	name path=Halley,
%        thick,
%        domain=0:360,
%        samples=600,
%    ]
%        {1.7246/(1+0.96714*cos(x+52.91))}
%  %[yshift=1pt, xshift=1.1pt]
%    node[above right, pos=0.32] {Halley's Comet}
%    ;
\stoppolaraxis
\stoptikzpicture
\stopTEXpage
\stopbuffer

\placetextfloat[top][fig:VisiblePlanets] % location
{The orbits of the six visible planets. (A terameter is one trillion meters: $1\units{Tm} = 10^{12}\units{m}$).}	 % caption text
{\noindent\typesetbuffer[TikZ:VisiblePlanets]} % figure contents

%In the calculations for a circular, centered orbit, we took an illegal short-cut by assuming circular, centered orbits and uniform circular motion. 
The planets' orbits are slightly off center, or \keyterm{eccentric}, as can be seen for most of the planets in \in{figure}[fig:VisiblePlanets]. 
These eccentric orbits lead to changes in each planet's speed.
As a planet gets closer to the Sun on one side of its orbit, the planet's gravitational potential energy decreases and its kinetic energy increases, increasing its speed. As the planet gets farther from the Sun on the other side of its orbit, some of its kinetic energy gets converted back to gravitational potential energy, and the planet slows down. This speeding up and slowing down repeats with every orbit.
The planet's motion is not uniform circular.
The eccentric orbit is a compound motion, combining angular motion around the Sun with oscillating radial motion. 

The orbits' eccentricities and changing speeds were already apparent in Copernicus's model based on ancient observations, but neither the ancients nor Copernicus knew the cause. Kepler tried to accommodate the eccentric orbits and changing speeds in his model, but his own calculations showed that his model was not as accurate as he had originally hoped. However, he correctly recognized the Sun's important role as the center and the physical driver of the solar system.
Kepler sent copies of \booktitle{Mysterium Cosmographicum} to prominent astronomers, including Galileo. Galileo sent an encouraging reply, but it is not clear if he ever read the book before his own adventure with the spyglass fourteen years later.

The book did catch the attention of another astronomer, Tycho de Brahe. Tycho had been making precise observations of the positions of stars and planets for 35 years. After a spat with King Christian of Denmark in 1597, Tycho moved his formidable arsenal of observing equipment to a castle near Prague, where he became Imperial Mathematicus for Emperor Rudolf II. Even with the best assistants, Tycho was struggling to make a coherent mathematical model from his mountain of precise data. He did not believe Kepler's model was the solution, but Tycho recognized the young astronomer's tremendous mathematical skill and creativity. Kepler, in return, recognized Tycho's observations as the only firm foundation for his model.
Tycho invited Kepler to collaborate, but guarded his data jealously. Kepler expressed interest, and even made a visit, but was reluctant to leave his position in Graz.

In 1600, events intervened. Kepler was expelled from Graz in a religious purge. Tycho's top assistant left after failing to explain Mars's eccentric orbit. Kepler joined Tycho's team and Tycho granted access to his treasury of Mars observations. Kepler said he would explain Mars's orbit in eight days.
Kepler then worked diligently for eight months. Tycho died, and Kepler succeeded him is Imperial Mathematics in 1601, still without a solution for Mars's orbit.

\section{Kepler's first attempts}

\startbuffer[TikZ:KeplerMarsAncients]
\environment env_physics
\environment env_TikZ
\setupbodyfont [libertinus,11pt]
\setoldstyle % Old style numerals in text
\startTEXpage\small
\starttikzpicture% tikz code
\startpolaraxis
 [	xticklabels=\empty,
 	ytick={0,0.5,...,2.5},
 	yticklabels={{},{},$100\units{Gm}$,{},$200\units{Gm}$,{}},
 	minor y tick num={4},
	% yminorgrids=true,
	hide x axis,
	ymax = 2.5,
	scale only axis=true, width={11cm},
 	tick style={middlegray}, % Fixes ticks which are too light in ConTeXt
	major grid style = {middlegray},
 	% ylabel={Distance from Sun $r$ ($\sci{9}\units{m}$)},
 ]
%    \addplot [ % Mars area at aphelion
%        draw=none, fill=black!20,
%        domain={156.08-4.36}:{156.08+4.36},
%        samples=20,
%    ]
%        {2.259/(1+0.0934*cos(x-336.08))}--(0,0)
%    ;
%    \addplot [ % Mars area at perihelion
%        draw=none, fill=black!20,
%        domain={336.08-6.35}:{336.08+6.35},
%        samples=20,
%    ]
%        {2.259/(1+0.0934*cos(x-336.08))}--(0,0)
%    ;
    \addplot [ % Mars
    	name path=Mars,
        thick,
        domain=0:360,
        samples=600,
    ]
        {2.259/(1+0.0934*cos(x-336.08))}
  [yshift=1pt, xshift=1.1pt]
    node[pos=0.25] {\Mars}
    ;
	\node [name path=Sun] at (0,0) {\Sun}node[below=1mm]{Sun};
	\filldraw (156.08,0.213) circle[radius=.8pt]node[below left=0mm]{Orbit's center};
	\filldraw (156.08,0.426) circle[radius=.8pt]node[above right]{Equant};
	\draw  ({156.08-4.36},{2.259/(1+0.0934*cos(156.08-4.36-336.08))}) -- ({336.08-6.35},{2.259/(1+0.0934*cos(336.08-6.35-336.08))});
	\draw  ({156.08+4.36},{2.259/(1+0.0934*cos(156.08+4.36-336.08))}) -- ({336.08+6.35},{2.259/(1+0.0934*cos(336.08+6.35-336.08))});
\stoppolaraxis
\stoptikzpicture
\stopTEXpage
\stopbuffer

\placetextfloat[top][fig:KeplerMarsAncients] % location
{Kepler's model of Mars's orbit using a perfect circle, off center, and a constant angular velocity about an equant. The lines show the equal angle about the equant over two different $20\units{day}$ durations. The same angle produces a larger displacement on the side farther from the equant – and closer to the Sun. This model failed.}	 % caption text
{\noindent\typesetbuffer[TikZ:KeplerMarsAncients]} % figure contents

\noindent
Kepler recounts his struggle with Mars in his revolutionary \booktitle{New Astronomy.} His first model for Mars's orbit uses a perfect circle and constant angular velocity \quotation{in imitation of the ancients.} He combines these two ancient ideas in a novel way. He chooses a perfect circle for the orbit of Mars, but places that circle so that the Sun is not at the center. This matches the shape of Mars's orbit quite well. To get the correct speed, he uses constant angular velocity, but not around the circle's center, or around the Sun, but instead around a third point called the equant, shown in \in{figure}[fig:KeplerMarsAncients]. Constant angular velocity around the equant gives Mars a changing velocity along its orbit. When Mars is closer to the equant – and farther from the Sun – it will move along its orbit more slowly. When it is farther from the equant – and closer to the Sun – it will move more quickly.

Before he could check this model he needed to determine the orbit's center and radius and the equant's location. He used four of Tycho's observations in an extremely laborious calculation, explained in chapter 16 of \booktitle{New Astronomy.} At one point in the explanation he exclaims,

\startblockquote
If this wearisome method has filled you with loathing, it should more properly fill you with compassion for me, as I have gone through it at least seventy times at the expense of a great deal of time, and you will cease to wonder that the fifth year has now gone by since I took up Mars\dots.%, although the year of 1603 was nearly all given over to optical investigations.
\autocite{p.~190.}{Kepler1609}
\stopblockquote
We still have these calculations, in Kepler's small handwriting, filling nine hundred pages.

Then, to confirm the model, he used it to predict Mars's location on the dates of six other observations, and compared them to Tycho's records. This again required tremendous labor. All of the predictions agreed with the observations to remarkable precision. Tycho's observations were not perfect, of course, but they could be trusted to within two minutes of an arc. A minute of arc is one sixtieth of a degree. The full Moon's apparent diameter is about thirty minutes of arc, or half of a degree, so Tycho's observations were accurate to within about one fifteenth of the Moon's apparent diameter. Prior observations, going back to the Babylonians, were good only to ten minutes of an arc. Kepler's prediction for Mars's location matched Tycho's observations with an error of less than two minutes – exactly the range of error expected based on the observations' precision. These exciting results are presented in Kepler's chapter 18.

Chapter 19 begins, \quotation{Who would have thought it possible? This hypothesis, so closely in agreement with [chapter 18's] observations, is nonetheless false\dots.}\autocite{p.~208.}{Kepler1609}
Kepler then explains, in methodical detail, that a few additional predictions disagreed with Tycho's observations by up to eight minutes of an arc, or about one quarter of the Moon's apparent diameter. This is a tiny error! Prior to Tycho's observations, such an error could be attributed to the observations. But Tycho's observations were good to two minutes. A difference of eight minutes could only mean the failure of the model.

After years of work, Kepler responds to this devastating verdict with the most moving passage I have found in any science text.
\startblockquote
Since the divine benevolence has vouchsafed us Tycho Brahe, a most diligent observer, from whose observations the $8'$ error of this [model] is shown in Mars, it is fitting that we with thankful mind both acknowledge and honor this favor of God. For it is in this that we shall carry on, to find at length the true form of the celestial motions, supported as we are by the proofs showing our suppositions to be fallacious. In what follows, I shall myself, to the best of my ability, lead the way for others on this road. For if I had thought I could ignore eight minutes\dots I would have already made enough of a correction in Ch.~16. Now, because they could not be ignored, these eight minutes alone will have led the way to the reformation of all of astronomy\dots.\autocite{p.~211.}{Kepler1609}
\stopblockquote
If I had to point to a single moment when the old philosophical reasoning gave way to a new scientific practice, it would be this moment – when Kepler, after years of calculation, confronts the data proving his failure, gives thanks, and starts over.

\section{Kepler's area law}
Kepler could have continued \quotation{in imitation of the ancients} by adding a small extra wheel to the orbital motion or by adding a mysterious oscillation to the already mysterious equant. Instead, he recognized that a fundamental assumption of his model must be wrong.
\startblockquote
Therefore, something among those things we had assumed must be false. But what was assumed was: that the orbit upon which the planet moves is a perfect circle; and that there exists some unique point\dots about which point Mars describes equal angles in equal times. Therefore, of these, one or the other or perhaps both are false, for the observations used are not false.
\autocite{pp.~209-10.}{Kepler1609}
\stopblockquote
In chapter 20, Kepler shows that any attempt to use an equant will conflict with the observations.
\startblockquote
The blame for the discrepancy\dots(I am repeating this over and over so that it will be remembered) falls entirely upon the faulty assumptions deliberately entertained by me, in common with Tycho and all who have ever devised hypotheses. For the necessary consequence of this enquiry is that there is no single fixed point\dots about which the planet always sweeps out equal angles in equal times.\autocite{pp.~215-16.}{Kepler1609}
\stopblockquote
The model's failure for Mars called into question Kepler's model for Earth's motion, which also used a perfectly circular orbit and constant angular velocity. Kepler could not possibly make any progress in determining the position of Mars from Tycho's observations if he did not even know Earth's location, since Earth is where Tycho made the observations.
Kepler continued to believe that Earth's orbit is a perfect circle, slightly off center from the Sun (\in{fig.}[fig:KeplerTerestrial]), but he needed to find a new rule for Earth's speed. To do this, he again used Tycho's observations of Mars, but this time he reversed them to find the direction Earth would be seen from an observatory on Mars. This seems like a waste of time, since Mars position is also uncertain, but by using multiple observations of Mars when it was known to be in the same position, Kepler was nonetheless able to find Earth's motion. The details of the epic calculations fill chapters 22 through 31 of \booktitle{New Astronomy}. 

\startbuffer[TikZ:KeplerTerestrial]
\environment env_physics
\environment env_TikZ
\setupbodyfont [libertinus,11pt]
\setoldstyle % Old style numerals in text
\startTEXpage\small
\starttikzpicture% tikz code
\startpolaraxis
 [	xticklabels=\empty,
 	ytick={0,0.5,...,2.5},
 	yticklabels={{},{},$100\units{Gm}$,{},$200\units{Gm}$,{}},
 	minor y tick num={4},
	% yminorgrids=true,
	hide x axis,
	ymax = 2.5,
	scale only axis=true, width={11cm},
 	tick style={middlegray}, % Fixes ticks which are too light in ConTeXt
	major grid style = {middlegray},
 	% ylabel={Distance from Sun $r$ ($\sci{9}\units{m}$)},
 ]
    \addplot [ % Mars area at aphelion
        draw=none, fill=black!20,
        domain={156.08-4.36}:{156.08+4.36},
        samples=20,
    ]
        {2.259/(1+0.0934*cos(x-336.08))}--(0,0)
    ;
    \addplot [ % Mars area at perihelion
        draw=none, fill=black!20,
        domain={336.08-6.35}:{336.08+6.35},
        samples=20,
    ]
        {2.259/(1+0.0934*cos(x-336.08))}--(0,0)
    ;
    \addplot [ % Mercury area at perihelion
        draw=none, fill=black!20,
        domain={77.46-59.63}:{77.46+59.63},
        samples=80,
    ]
        {0.5546/(1+0.20564*cos(x-77.46))}--(0,0)
    ;
    \addplot [ % Mercury area at aphelion
        draw=none, fill=black!20,
        domain={257.46-33.24}:{257.46+33.24},
        samples=20,
    ]
        {0.5546/(1+0.20564*cos(x-77.46))}--(0,0)
    ;
  	\node [name path=Sun] at (0,0) {\Sun};
    \addplot [ % Mercury
        thick,
        domain=0:360,
        samples=600,
    ]
        {0.5546/(1+0.20564*cos(x-77.46))}
  [yshift=-.5pt]
    node[pos=0.25] {\Mercury}
    ;
    \addplot [ % Venus
        thick,
        domain=0:360,
        samples=600,
    ]
        {1.082/(1+0.00676*cos(x-131.77))}
  [yshift=-1.7pt]
    node[pos=0.25] {\Venus}
    ;
    \addplot [ % Earth
    	name path=Earth,
        thick,
        domain=0:360,
        samples=600,
    ]
        {1.496/(1+0.0167*cos(x-102.93))}
    node[pos=0.25] {\Earth}
    ;
    \addplot [ % Mars
    	name path=Mars,
        thick,
        domain=0:360,
        samples=600,
    ]
        {2.259/(1+0.0934*cos(x-336.08))}
  [yshift=1pt, xshift=1.1pt]
    node[pos=0.25] {\Mars}
    ;
\stoppolaraxis
\stoptikzpicture
\stopTEXpage
\stopbuffer

\placetextfloat[top][fig:KeplerTerestrial] % location
{The orbits of the inner planets. The orbits of Mercury and Mars are noticeably off-center. The gray pie-slices show the angle covered in $20\units{days}$ by Mars and Mercury when they are closest and farthest from the Sun. Notice that both the angle and the distance are greater when the planet is closer to the Sun. Both the angular velocity and the speed are greater when closer to the Sun.}	 % caption text
{\noindent\typesetbuffer[TikZ:KeplerTerestrial]} % figure contents

Knowing Earth's positions, and therefore its speed along its orbit, Kepler sought a mathematical model. Breaking with tradition, he did not seek inspiration in abstract geometry. Instead, he considers possible physical causes for the motion. \quotation{Physicists, prick your ears\dots for we are going to invade your territory.}\autocite{p.~329.}{Kepler1609} 
He no longer searches for imaginary wheels or imaginary points like the equant. Instead he looks for an explanation in the objects themselves, specifically in the Sun and planets.

\quotation{The power that moves the planets resides in the Sun,}\autocite{p.~278.}{Kepler1609} he suggests, and its strength diminishes as the planet gets farther away, decreasing the planet's speed. He attempts to develop this physical idea into a mathematical model and – after considerable stumbling – discovers that all along its orbit, Earth sweeps out \emph{equal areas in equal times.}

For Earth's nearly centered orbit there is only a tiny difference between a constant angular velocity (which sweeps out equal angles in equal times) and Kepler's new law of equal areas in equal times. Tycho's precise observations confirmed the tiny effect of Kepler's new area law for Earth's motion.

Kepler's area law, discovered first for Earth, is true for all planets. The effect is most pronounced for Mercury, which has the most eccentric orbit. Two shaded areas in \in{figure}[fig:KeplerTerestrial] show the areas swept out by Mercury during two equal times. The shaded, pie-shaped area above the Sun shows the area swept out by Mercury during the twenty days when it is closest to the Sun. The shaded area below the Sun shows the area swept out during the twenty days when Mercury is farthest from the Sun. The upper pie-slice has a smaller radius and a larger angle due to Mercury's higher speed near the Sun. The lower pie slice has a larger radius and a smaller angle due to Mercury's lower speed farther from the Sun. The two resulting areas are exactly the same.

Kepler's mysterious area law provides an interesting challenge for any theory of motion. Newton, to show the power of his laws of motion, takes on this challenge in Proposition 1 of the \booktitle{Principia}, where he proves the area law using geometric methods. Euler and Lagrange offered new perspectives on this law in their formulations of mechanics.
We will explain the area law using angular momentum, which is ideally suited to the orbital motions of the solar system.

\section{Conservation of angular momentum}
We will describe the planets' motions with the astronomers' coordinates $r$ and $\theta$ in \in{figure}[fig:polar], with the Sun at the center. These are called \keyterm{polar coordinates}. A planet's motion around the Sun is described by the position update formula for the angular coordinate $\theta$.
\startformula
  d\theta = \omega\,dt
\stopformula

\startbuffer[polar]
\environment env_physics
\environment env_TikZ
\setupbodyfont [libertinus,11pt]
\setoldstyle % Old style numerals in text
\startTEXpage\small
\starttikzpicture% tikz code
\startpolaraxis
 [	xticklabels=\empty,
 	ytick={0,0.5,...,1.5},
 	yticklabels={{},{},$100\units{Gm}$,{}},
 	minor y tick num={1},
	% yminorgrids=true,
	hide x axis,
	ymax = 1.6,
	scale only axis=true, width={5.5cm},
 	tick style={middlegray}, % Fixes ticks which are too light in ConTeXt
	major grid style = {middlegray},
 	% ylabel={Distance from Sun $r$ ($\sci{9}\units{m}$)},
	clip mode = individual,
 ]
%    \addplot [ % Mars area at perihelion
%        draw=none, fill=black!20,
%        domain={60}:{64.93},
%        samples=5,
%    ]
%        {1.496/(1+0.0167*cos(x-102.93))}--(0,0)
%    ;
%    \addplot [ % Venus
%        thick,
%        domain=0:360,
%        samples=600,
%    ]
%        {1.082/(1+0.00676*cos(x-131.77))}
%  [yshift=-1.3pt]
%    node[pos=0.25] {\Venus}
%    ;
    \addplot [ % Earth
    	name path=Earth,
        thick,
        domain=0:360,
        samples=600,
    ]
        {1.496/(1+0.0167*cos(x-102.93))}
    node[pos=0.1667] {\Earth}
    ;
    \addplot [ % theta
        ->,
        domain=0:63,
        samples=30,
    ]
        {0.5}
    node[above right=0pt, pos=0.5] {\theta}
    ;
	\node [name path=Sun] at (0,0) {\Sun};% node[below=1mm]{Sun};
%	\node [right] at (60,1.1) {$dA$};% node[below=1mm]{Sun};
	%\node [below right] at (60,0.75) {$r$};% node[below=1mm]{Sun};
	\node [white] at (225,2.12) {.}; % To make the bounding box slightly bigger
%	\draw[]  (60, 1.48) -- (60, 1.6);
%	\draw[]  (64.93, 1.48) -- (64.93, 1.6);
%	\draw[->]  (55, 1.55) -- (60, 1.55);
%	\draw[->]  (69.93, 1.55) -- (64.93, 1.55)node[above right]{$r\,d\theta$};
	%\draw[]  (0, 0) -- (150, 0.1);
	%\draw[->]  (150, 0.08) --node[left]{$r$} (67, 1.48);
	\draw[->]  (0, 0) --node[above left]{$r$} (60, 1.48);
\stoppolaraxis
\stoptikzpicture
\stopTEXpage
\stopbuffer

\marginTikZ{}{polar}{Earth's position in polar coordinates $r$ and $\theta$.} % vskip, name, caption

\noindent
If the angular velocity $\omega$ were constant, each small step $d\theta$ would be the same size. 
However, the angular velocity is not constant. The correct angular velocity is found from conservation of angular momentum.
\startformula
  L\si + \tau ∆t = L\sf
\stopformula
The central Sun's gravitational force pulls each planet directly toward the center, in the negative $r$ direction. The planets do not feel any torque $\tau$ in the clockwise or counter-clockwise $\theta$ direction.
With no torque, each planet's angular momentum $L$ is constant.

For a circular orbit and a constant distance $r$, the constant angular momentum would produce a constant angular velocity. But the planets' orbits are eccentric, $r$ is not constant, and the angular velocity changes as the planets get nearer to the Sun and then farther away. 

\startexample[ex:EarthDistances]
Earth's angular velocity around the Sun varies slightly, from $\unit{2.059e-7rad/s}$ in early January to $\unit{1.926e-7rad/s}$ in early July. Earth is closest to the Sun in early January, when the distance is $\unit{1.471e11m}$. What is Earth's greatest distance from the Sun?
\startsolution
Use conservation of angular momentum to relate the changing angular velocity to the changing distance. We will go from the closest point, where Earth's angular velocity is greatest, to the farthest point, where Earth's angular velocity is lowest.
\startformula\startmathalignment
\NC  L\si + \cancel{\tau∆t} \NC= L\sf \NR
\NC  I\si\omega\si \NC= I\sf\omega\sf \NR
\NC  mr\si^2\omega\si \NC= mr\sf^2\omega\sf \NR
\stopmathalignment\stopformula
\startformula
  r\sf = r\si\sqrt{\frac{\omega\si}{\omega\sf}}
       = \unit{1.471e11m}\sqrt{\frac{\unit{2.059e-7rad/s}}{\unit{1.926e-7rad/s}}}
       = \unit{1.521e11m}
\stopformula
At its farthest point, Earth is only slightly farther away from the Sun.
\stopsolution
\stopexample

Since a planet's constant angular momentum is $L = I\omega = mr^2\omega$, the planet's angular velocity is 
\startformula
  \omega = \frac{L}{mr^2}
\stopformula
This angular velocity $\omega$ allows us to find the correct position update formula for $\theta$ as the planet's distance $r$ changes.
\startformula
  d\theta = \frac{L}{mr^2}\,dt
\stopformula
Even as the angular momentum remains constant, the size of the angular steps $d\theta$ change as the planet moves closer and farther from the Sun. The angular steps are larger when the planet is closer to the Sun and $r$ smaller; the angular steps are smaller when the planet is farther from the Sun and $r$ is larger\dash exactly the speeds Kepler sought to model, first with the equant and then with his constant area law.

To see how our $\theta$ update formula compares with Kepler's area law, we can write the area law as an area update formula.
\startformula
	dA = (\text{const.})\,dt,
\stopformula
where $dA$ is a small change in area that occurs during the short duration $dt$, as shown in \in{figure}[fig:dA]. Kepler observed that equal areas are swept out in equal times, so the coefficient relating the area to the time must be a constant.

\startbuffer[dA]
\environment env_physics
\environment env_TikZ
\setupbodyfont [libertinus,11pt]
\setoldstyle % Old style numerals in text
\startTEXpage\small
\starttikzpicture% tikz code
\startpolaraxis
 [	xticklabels=\empty,
 	ytick={0,0.5,...,1.5},
 	yticklabels={{},{},$100\units{Gm}$,{}},
 	minor y tick num={1},
	% yminorgrids=true,
	hide x axis,
	ymax = 1.6,
	scale only axis=true, width={5.5cm},
 	tick style={middlegray}, % Fixes ticks which are too light in ConTeXt
	major grid style = {middlegray},
 	% ylabel={Distance from Sun $r$ ($\sci{9}\units{m}$)},
	clip mode = individual,
 ]
    \addplot [ % Mars area at perihelion
        draw=none, fill=black!20,
        domain={60}:{64.93},
        samples=5,
    ]
        {1.496/(1+0.0167*cos(x-102.93))}--(0,0)
    ;
%    \addplot [ % Venus
%        thick,
%        domain=0:360,
%        samples=600,
%    ]
%        {1.082/(1+0.00676*cos(x-131.77))}
%  [yshift=-1.3pt]
%    node[pos=0.25] {\Venus}
%    ;
    \addplot [ % Earth
    	name path=Earth,
        thick,
        domain=0:360,
        samples=600,
    ]
        {1.496/(1+0.0167*cos(x-102.93))}
    node[pos=0.1804] {\Earth}
    ;
%    \addplot [ % theta
%        ->,
%        domain=0:63,
%        samples=30,
%    ]
%        {0.5}
%    node[above right=0pt, pos=0.5] {\theta}
%    ;
	\node [name path=Sun] at (0,0) {\Sun};% node[below=1mm]{Sun};
	\node [right] at (60,1.1) {$dA$};% node[below=1mm]{Sun};
	%\node [below right] at (60,0.75) {$r$};% node[below=1mm]{Sun};
	\node [white] at (225,2.12) {.}; % To make the bounding box slightly bigger
	\draw[]  (60, 1.48) -- (60, 1.6);
	\draw[]  (64.93, 1.48) -- (64.93, 1.6);
	\draw[->]  (55, 1.55) -- (60, 1.55);
	\draw[->]  (69.93, 1.55) -- (64.93, 1.55)node[above right]{$r\,d\theta$};
	\draw[]  (0, 0) -- (150, 0.1);
	\draw[<->]  (150, 0.08) --node[left]{$r$} (67, 1.48);
	%\draw[->]  (0, 0) --node[above left]{$r$} (62.46, 1.48);
\stoppolaraxis
\stoptikzpicture
\stopTEXpage
\stopbuffer

\marginTikZ{}{dA}{The small area $dA$ swept out by Earth during a small duration $dt$.} % vskip, name, caption

In polar coordinates, the small area swept out by a planet is a long, thin triangle with length $r$ and width $r\,d\theta$. The thin triangle's area is
\startformula
	dA = \half (r)(r\,d\theta)
		= \half r^2\,d\theta.
\stopformula
We calculated the update $d\theta$ above, and we can use it here.
\startformula\pagereference[eq:AreaUpdate]
	dA = \half \cancel{r^2}\frac{L}{m\cancel{r^2}}\,dt.
		= \frac{L}{2m}\,dt.
\stopformula
Since the angular momentum $L$ is a constant of the motion, this is exactly Kepler's area law stating that equal areas $dA$ are swept out in equal times $dt$. Kepler's area law is the earliest description of conservation of angular momentum.

Now that we understand Earth's changing speed in terms of angular momentum, let's return to Kepler. 
With his new area law, Kepler had more precise positions for Earth, and he returned to the problem of Mars. His new effort is described in Book Five of \booktitle{New Astronomy}.

Mars's more eccentric orbit provides a more difficult challenge for Kepler's area law. \in{Figure}[fig:KeplerTerestrial] shows the areas swept out by Mars in the twenty days when it is farthest from the Sun (right) and the twenty days when it is closest (left). Once again, equal areas are swept out in equal times. Kepler checked areas along other parts of Mars's orbit, and all confirmed his area law.

The areas swept out by Mars are not the same as the areas swept out by Mercury. Both planets have constant angular momentum, but Mars's angular momentum is much greater than Mercury's. (This is true even if we correct for Mars's larger mass.) Mars sweeps out about twice the area that Mercury does in the same time. You can roughly check this by measuring the areas in \in{figure}[fig:KeplerTerestrial].

\section{Energy graphs for oribts}
Each planet's orbit is a compound motion. The planet moves around the Sun in the $\theta$-direction while also approaching and receding from the Sun in the $r$-direction. In the last section, we found the surprisingly simple update formulas for the angular motion. In this section we turn our attention to the puzzling motion in the radial direction. Energy graphs will be essential.

\startbuffer[Uterrestrial3D]
\startaxis[
 	  %axis line shift=1cm,
	   %axis lines*=left,
   hide x axis,
    hide y axis,
    hide z axis,
        axis lines=center,
        axis on top,
	 view={0}{45},
        width = 7cm,
    %z post scale = {1},
        clip mode = individual,
]
%    \addplot3 [
%        mesh, color = middlegray,
%        z buffer=sort,
%        samples=9,
%        domain=0.1:1,
%        y domain=0:2*pi,
%](
%{x * cos(deg(y))}, {x * sin(deg(y))}, {-10}
%    );
    \addplot3 [
        surf, faceted color = middlegray, color = gray,
        z buffer=sort,
        samples=5,
        domain=1:3,
        y domain=0:2*pi,
        samples y=25,
](
{x * cos(deg(y))}, {x * sin(deg(y))}, {-13.27/x}
    );
    \node at (-1.8,-1,-8) {$U$};
    \draw[->] (0,0,0) --node[above, pos = 0.98]{$r$} (3,0,0);
    \draw[shade, ball color = white] (0,0,0) circle[radius=1.2mm]node[above=0.8mm] {\Sun};
    \draw[shade, ball color = darkgray] (0.5546,0,0) circle[radius=0.2mm]node[above] {\Mercury};
    \draw[shade, ball color = darkgray] (1.082,0,0) circle[radius=0.6mm]node[above] {\Venus};
    %\draw[] (1.5,0,0) -- (1.5,0,-8.874);
    \draw[shade, ball color = darkgray] (1.496,0,0) circle[radius=0.6mm]node[above=0.8mm] {\Earth};
    \draw[shade, ball color = darkgray] (2.259,0,0) circle[radius=0.4mm]node[above=0.7mm] {\Mars};
    %\draw[shade, ball color = darkgray] (7.76,0,0) circle[radius=0.4mm]node[above=0.7mm] {\Jupiter};
    %\draw[shade, ball color = darkgray] (14.23,0,0) circle[radius=0.4mm]node[above=0.7mm] {\Saturn};
\stopaxis
\stopbuffer

\marginTikZ{}{Uterrestrial3D}{The gravitational potential energy of the inner planets, with their approximate distances from the Sun. The inner planets' full orbits are shown in \in{figure}[fig:KeplerTerestrial].} % vskip, name, caption

The energy graph in \in{figure}[fig:Uterrestrial3D] shows the gravitational potential energy for the inner planets. Above the graph are the central Sun and the inner planet's at their approximate distances. The gravitational potential energy alone does not offer much insight into the orbits. It looks as if the planets will spiral down the deep potential well. Fortunately, this does not happen due to the planets' angular motion. 
We can see the role of the angular motion by adding angular kinetic energy $K_\theta$ to each planet's energy graph.

\startbuffer[TikZ:EnergyGraphKU]
\environment env_physics
\environment env_TikZ
\setupbodyfont [libertinus,11pt]
\setoldstyle % Old style numerals in text
\startTEXpage\small
\starttikzpicture% tikz code
\startaxis
 [	width=11cm,y={.5cm},
   xlabel={Distance from Sun $r$ ($\sci{11}\units{m}$)},
   xmin=0, xmax=3,
   minor x tick num=4,
   ylabel={Energy per mass ($\sci{8}\units{J/kg}$)},
   ymin=-10, ymax=6,
   minor y tick num=3,
   clip mode = individual,
		every tick/.style={middlegray}, % Fixes ticks which are too light in ConTeXt
 ]
% \addplot[ % Us
%   thick,
%   domain=10:20,
%   samples=51
%  ]
%  {((x-20)^2/200)}
%  [xshift=9pt]
%    node[pos=0.03] {$U\sub{s}$}
%  ;
%\ifprintanswers
\addplot[ % axis
   thick,
   domain=0:3,
   samples=2
  ]
  {0}
%  [yshift=10pt]
%    node[pos=0.5] {$U$}
  ;
\addplot[ % Ug
   thick,
   domain=0.5:3,
   samples=150
  ]
  {-13.3/x}
  %[yshift=-10pt]
    node[below right,pos=0.9] {$U=-G\frac{mM}{r}$}
  ;
%\addplot[ % Circular Orbits
%   thick, dotted,
%   domain=0.5:5,
%   samples=150
%  ]
%  {-6.65/x}
%  ;
\addplot[ % K angular Earth
   thin,
   domain=1:3,
   samples=150
  ]
  {9.926/x^2}
  %[yshift=8pt]
    node[above right=0mm, pos=0.7] {$K_\theta=\frac{L^2}{2mr^2}$}
  ;
\addplot[ % K angular Earth
   thin,
   domain=0.5:3,
   samples=150
  ]
  {(-13.27/x)+(9.926/x^2)}
  %[yshift=8pt]
    node[above right,pos=0.6] {$U + K_\theta$}
  ;
\addplot[ % E Earth
   ultra thick,
   domain=1.471:1.521,
   samples=2
  ]
  {-4.437}
  %[yshift=10pt]
    node[below, pos=0.5] {$H = U + K_\theta + K_r$}
  ;
%\addplot[ % K angular Mars
%   thin,
%   domain=0.5:5,
%   samples=150
%  ]
%  {(-13.27/x)+(14.99/x^2)}
%  [yshift=8pt]
%    node[pos=0.5] {$H$}
%  ;
%\addplot[ % E Mars
%   very thick,
%   domain=2.067:2.492,
%   samples=2
%  ]
%  {-2.912}
%  [yshift=10pt]
%    node[pos=0.5] {$E$}
%  ;
    \draw[->] (0,7) --node[above, pos=.98]{$r$} (3,7);
    \draw[shade, ball color = white] (0,7) circle[radius=2.4mm]node[above =2mm] {\Sun}; % Sun
    \draw[shade, ball color = darkgray] (1.5,7) circle[radius=1.2mm]node[above=0.8mm] {\Earth}; % Earth
\stopaxis
\stoptikzpicture
\stopTEXpage
\stopbuffer

\placetextfloat[bottom][fig:EnergyGraphKU] % location
{Earth's rotational kinetic energy $K_\theta$ and potential energy $U$ plotted as functions of the distance from the Sun $r$. The sum $U +K_\theta$ cradles Earth at its minimum. $U$ prevents Earth from leaving the solar system, while $K_\theta$ keeps it from spiraling into the Sun. Earth is always also moving the $\theta$ direction (not shown) as it orbits the Sun.}	 % caption text
{\noindent\typesetbuffer[TikZ:EnergyGraphKU]} % figure contents

For planetary orbits, there are three contributions to the total energy: the gravitational potential energy $U$, the kinetic energy $K_\theta$ due to the planet's motion around the Sun in the $\theta$-direction, and the kinetic energy $K_r$ due to any motion toward or away from the Sun in the $r$-direction.
\startformula
	H = K_r + K_\theta + U
\stopformula
The gravitational potential energy is especially simple in polar coordinates, since it only depends on the planet's distance $r$ from the Sun, not on the angle $\theta$.
\startformula
	U = -G\frac{mM}{r}
\stopformula

The kinetic energy's radial part $K_r$ can be written in terms of the radial momentum $p_r$, as usual.
\startformula
	K_r = \frac{p_r^2}{2m}
\stopformula
Similarly, the kinetic energy's angular part can be written in terms of the \keyterm{angular momentum} $L$. However, when we work with an angular coordinate, the mass in the denominator is replaced by the moment of inertial $I$. 
\startformula
	K_\theta = \frac{L^2}{2I}
\stopformula
Angular momentum $L$ is a new type of momentum associated with angular motion like rotation or orbits. Euler and others developed special methods for working with angular quantities, but Hamilton's method treats angular coordinates and momenta just like any others. 


Earth's energy graph, in \in{figure}[fig:EnergyGraphKU], shows the angular kinetic energy $K_\theta$ along with the gravitational potential energy $U$. (This energy graph shows only the $r$-direction, not the $\theta$-direction.) The gravitational potential energy $U$ at the bottom should look familiar – rising toward zero far from the Sun and descending steeply close to the Sun.
Earth's angular kinetic energy trends in the opposite direction – descending toward zero far from the Sun and rising steeply close to the Sun, as shown in \in{figure}[fig:EnergyGraphKU].

Since gravitational potential energy and angular kinetic energy change in opposite directions, these energy changes balance when Earth's distance from the Sun changes slightly along its slightly eccentric orbit.
As Earth moves slightly toward the Sun, some gravitational potential energy is converted to angular kinetic energy. As Earth recedes slightly away from the Sun, some angular kinetic energy is converted back to gravitational potential energy. (The angular momentum $L$ stays constant during these energy changes.)

The angular kinetic energy graph is not \emph{exactly} the opposite the gravitational potential energy graph. The angular kinetic energy $K_\theta$ has $r^2$ in the denominator, while the gravitational potential energy $U$ has only $r$ in the denominator. This makes the angular kinetic energy's behavior a bit sharper than gravitational potential energy's. Angular kinetic energy rises more sharply close to the Sun, and falls off more sharply far away. As a result, there is only one distance where the changes in energy balance, so that the sum stays the same.

The last curve on Earth's energy graph in \in{figure}[fig:EnergyGraphKU] is this sum of the gravitational potential energy and the angular kinetic energy.
\startformula
	U + K_\theta = -G\frac{mM}{r} + \frac{L^2}{2mr^2}
\stopformula
This curve increases sharply close to the Sun, due to the sharp rise in the angular kinetic energy $K_\theta$. The curve also rises towards zero far from the Sun, where the sharper drop in angular kinetic energy allows the more persistent rise of gravitational potential energy $U$ to dominate.

In the middle is a minimum which cradles Earth, as shown by the Earth and Sun above the energy graph, and by the total energy $H$ marked at Earth's location on the graph.
Earth rocks gently in this minimum, where the two energy curves' slopes are in balance. At this minimum, the slope of the total $U + K_\theta$ is zero. Earth's total energy $H$ is very close to this minimum value of $U+K_\theta$. (The contribution of Earth's radial kinetic energy $K_r$ is too small to be seen on this graph. In the next section, Mars will help us understand the role of $K_r$.) Earth does not have enough energy to climb either side of the $U+K_\theta$ minimum, so Earth's orbit is stable.

\startbuffer[UK3D]
\startaxis[
   hide x axis,
    hide y axis,
    hide z axis,
        axis lines=center,
        axis on top,
	 view={0}{45},
        width = 7cm,
    z post scale = {1.67},
        clip mode = individual,
]
    \addplot3 [
        surf, faceted color = lightgray, color = white,
        z buffer=sort,
        samples=5,
        domain=1:3,
        y domain=0:2*pi,
        samples y=25,
](
{x * cos(deg(y))}, {x * sin(deg(y))}, {-13.27/x}
    );
    \path (-1.8,-1,-8)node {$U$};
    \addplot3 [
        surf, faceted color = middlegray, color = gray,
        z buffer=sort,
        samples=5,
        domain=1:3,
        y domain=0:2*pi,
        samples y=25,
](
{x * cos(deg(y))}, {x * sin(deg(y))}, {(-13.27/x)+(9.926/x^2)}
    );
    \node at (-2.4,1.8,-1.8) {$U + K_\theta$};
    \addplot3 [
        thick,
        domain=0:2*pi,
        samples=49,
        samples y=1,
   ](
        {1.5*sin(deg(x))},
        {1.5*cos(deg(x))},
        {-4.437}
    );
    \addplot3 [
        surf, faceted color = middlegray, color = gray,
        z buffer=sort,
        samples=2,
        domain=0.75:1,
        y domain=0:2*pi,
        samples y=25,
](
{x * cos(deg(y))}, {x * sin(deg(y))}, {(-13.27/x)+(9.926/x^2)}
    );
    \draw[->] (0,0,0) --node[above, pos=.98]{$r$} (3,0,0);
    \draw[shade, ball color = white] (0,0,0) circle[radius=1.2mm]node[above=0.8mm] {\Sun}; % Sun
    %\draw[] (1.5,0,0) -- (1.5,0,-4.437);
    \draw[shade, ball color = darkgray] (1.5,0,0) circle[radius=0.6mm];
    \draw[shade, ball color = darkgray] ({1.5*cos(15)},{1.5*sin(15)},0) circle[radius=0.6mm];
    \draw[shade, ball color = darkgray] ({1.5*cos(30)},{1.5*sin(30)},0) circle[radius=0.6mm]node[above right] {\Earth};
    \draw[shade, ball color = darkgray] ({1.5*cos(45)},{1.5*sin(45)},0) circle[radius=0.6mm];
    \draw[shade, ball color = darkgray] ({1.5*cos(60)},{1.5*sin(60)},0) circle[radius=0.6mm];
    \draw[shade, ball color = darkgray] ({1.5*cos(75)},{1.5*sin(75)},0) circle[radius=0.6mm];
    \draw[shade, ball color = darkgray] (0,1.5,0) circle[radius=0.6mm];
    \draw[shade, ball color = darkgray] ({1.5*cos(105)},{1.5*sin(105)},0) circle[radius=0.6mm];
    \draw[shade, ball color = darkgray] ({1.5*cos(120)},{1.5*sin(120)},0) circle[radius=0.6mm];
    \draw[shade, ball color = darkgray] ({1.5*cos(135)},{1.5*sin(135)},0) circle[radius=0.6mm];
    \draw[shade, ball color = darkgray] ({1.5*cos(150)},{1.5*sin(150)},0) circle[radius=0.6mm];
    \draw[shade, ball color = darkgray] ({1.5*cos(165)},{1.5*sin(165)},0) circle[radius=0.6mm];
    \draw[shade, ball color = darkgray] (-1.5,0,0) circle[radius=0.6mm];
    \draw[shade, ball color = darkgray] ({1.5*cos(15)},{-1.5*sin(15)},0) circle[radius=0.6mm];
    \draw[shade, ball color = darkgray] ({1.5*cos(30)},{-1.5*sin(30)},0) circle[radius=0.6mm];
    \draw[shade, ball color = darkgray] ({1.5*cos(45)},{-1.5*sin(45)},0) circle[radius=0.6mm];
    \draw[shade, ball color = darkgray] ({1.5*cos(60)},{-1.5*sin(60)},0) circle[radius=0.6mm];
    \draw[shade, ball color = darkgray] ({1.5*cos(75)},{-1.5*sin(75)},0) circle[radius=0.6mm];
    \draw[shade, ball color = darkgray] (0,-1.5,0) circle[radius=0.6mm];
    \draw[shade, ball color = darkgray] ({1.5*cos(105)},{-1.5*sin(105)},0) circle[radius=0.6mm];
    \draw[shade, ball color = darkgray] ({1.5*cos(120)},{-1.5*sin(120)},0) circle[radius=0.6mm];
    \draw[shade, ball color = darkgray] ({1.5*cos(135)},{-1.5*sin(135)},0) circle[radius=0.6mm];
    \draw[shade, ball color = darkgray] ({1.5*cos(150)},{-1.5*sin(150)},0) circle[radius=0.6mm];
    \draw[shade, ball color = darkgray] ({1.5*cos(165)},{-1.5*sin(165)},0) circle[radius=0.6mm];
\stopaxis
\stopbuffer

\marginTikZ{}{UK3D}{Earth's full orbit around the Sun and Earth's energy graph. Earth's orbit is at the minimum of $U + K_\theta$.} % vskip, name, caption

Earth's full orbit is shown on the energy graph in \in{figure}[fig:UK3D], where the $\theta$ direction has been restored. This graph reminds me again of Kepler's early drinking cup model, where Earth's orbit was confined to a spherical shell between two platonic solids (\at{p.}[fig:Kepler1596system]). In our energy graph, Earth's orbit is held between two competing energies. Rising gravitational potential energy surrounds Earth's orbit, preventing Earth from wandering out of the Solar System. Rising angular kinetic energy sits inside Earth's orbit, preventing Earth from spiraling into the Sun.

The gravitational potential energy graph is the same shape for every planet, but the relative size of the angular kinetic energy hill in the middle of the graph depends on the planet's angular momentum. Greater angular momentum increases the size of this hill, called the \keyterm{centrifugal barrier}. A larger centrifugal barrier pushes the planet's orbit farther out from the Sun, as we will see for Mars. Less angular momentum produces a smaller centrifugal barrier, allowing the orbit to be closer to the Sun, as for Mercury. The size of the centrifugal barrier is different for each planet, giving their orbits different sizes in the same gravitational potential energy well.

\startbuffer[TikZ:EnergyGraphKUH]
\environment env_physics
\environment env_TikZ
\setupbodyfont [libertinus,11pt]
\setoldstyle % Old style numerals in text
\startTEXpage\small
\starttikzpicture% tikz code
\startaxis
 [	width=11cm,y={1cm},
   xlabel={Distance from Sun $r$ ($\sci{11}\units{m}$)},
   xmin=0, xmax=3,
   minor x tick num=4,
   ylabel={Energy per mass ($\sci{8}\units{J/kg}$)},
   ymin=-8, ymax=0,
   minor y tick num=9,
   clip mode = individual,
		every tick/.style={middlegray}, % Fixes ticks which are too light in ConTeXt
 ]
% \addplot[ % Us
%   thick,
%   domain=10:20,
%   samples=51
%  ]
%  {((x-20)^2/200)}
%  [xshift=9pt]
%    node[pos=0.03] {$U\sub{s}$}
%  ;
%\ifprintanswers
\addplot[ % axis
   thick,
   domain=0:3,
   samples=2
  ]
  {0}
%  [yshift=10pt]
%    node[pos=0.5] {$U$}
  ;
\addplot[ % Ug
   thick,
   domain=1.5:3,
   samples=150
  ]
  {-13.3/x}
  %[yshift=-10pt]
    node[below right,pos=0.6] {$U$}
  ;
%\addplot[ % Circular Orbits
%   thick, dotted,
%   domain=0.5:5,
%   samples=150
%  ]
%  {-6.65/x}
%  ;
%\addplot[ % K angular Earth
%   thin,
%   domain=0.5:5,
%   samples=150
%  ]
%  {9.926/x^2}
%  %[yshift=8pt]
%    node[above, pos=0.95] {$K_\theta$}
%  ;
%\addplot[ % K angular Earth
%   thin,
%   domain=0.5:5,
%   samples=150
%  ]
%  {(-13.27/x)+(9.926/x^2)}
%  %[yshift=8pt]
%    node[above right,pos=0.75] {$K_\theta + U$}
%  ;
%\addplot[ % E Earth
%   very thick,
%   domain=1.471:1.521,
%   samples=2
%  ]
%  {-4.437}
%  %[yshift=10pt]
%    node[below, pos=0.5] {Earth}
%  ;
\addplot[ % K angular Mars
   thin,
   domain=1:3,
   samples=150
  ]
  {(-13.27/x)+(14.99/x^2)}
  %[yshift=8pt]
    node[below left,pos=0.45] {$U + K_\theta$}
  ;
\addplot[ % E Mars
   very thick,
   domain=2.067:2.492,
   samples=2
  ]
  {-2.912}
 % [yshift=10pt]
    node[above, pos=0.5] {$H = U + K_\theta + K_r$}
    %node[above, pos=0.5] {Mars}
  ;
    \draw[->] (0,0.5) --node[above, pos=.98]{$r$} (3,0.5);
    \draw[shade, ball color = white] (0,0.5) circle[radius=2.4mm]node[above =2mm] {\Sun};
    \filldraw[black!20] ({2.259/(1-0.0934*cos(120))},0.5) circle[radius=1.2mm];
    \filldraw[black!15] (2.259,0.5) circle[radius=1.2mm];
    \filldraw[black!20] ({2.259/(1-0.0934*cos(60))},0.5) circle[radius=1.2mm];
    \filldraw[black!30] ({2.259/(1-0.0934*cos(30))},0.5) circle[radius=1.2mm];
    \filldraw[black!30] ({2.259/(1-0.0934*cos(150))},0.5) circle[radius=1.2mm];
    \filldraw[black!40] (2.067,0.5) circle[radius=1.2mm];
    \draw[shade, ball color = darkgray] (2.492,0.5) circle[radius=1.2mm]node[above=0.8mm] {\Mars};
\stopaxis
\stoptikzpicture
\stopTEXpage
\stopbuffer

\placetextfloat[top][fig:EnergyGraphKUH] % location
{Mars's energy graph. The sum $U + K_\theta$ cradles Mars at its minimum. Mars oscillates between approximately $r\sub{min}=2\units{Gm}$ and $r\sub{max}=2.5\units{Gm}$, in agreement with the orbit shown in \in{figure}[fig:KeplerTerestrial].}	 % caption text
{\noindent\typesetbuffer[TikZ:EnergyGraphKUH]} % figure contents

Mars's more eccentric orbit leads to a more interesting energy graph, shown in \in{figure}[fig:EnergyGraphKUH]. Mars's energy graph shows the total energy $H$, including the kinetic energy $K_r$ due to Mars's motion in the $r$-direction. The total energy $H$ is slightly higher than the minimum of $K_\theta+U$, allowing Mars to oscillate over a significant range of distances from approximately $r\sub{min}=2\units{Gm}$ to $r\sub{max}=2.5\units{Gm}$, as shown by the range of Mars positions above the energy graph. These turning points in the energy graph are same maximum and minimum distances of Mars's orbit shown in \in{figure}[fig:KeplerTerestrial]. If you look closely at Earth's energy graph (\in{fig.}[fig:EnergyGraphKU]) you will see that it is also a small line segment showing the Earth's range of distances from the Sun.

Mars's full energy graph is shown in \in{figure}[fig:UKMars3D]. The orbit's eccentricity is clearly visible. The extra radial kinetic energy $K_r$ responsible for this eccentric motion has raised the total energy $H$ above the $U + K_\theta$ minimum, as if some water (shown in white) had been added to the moat around the central centrifugal barrier. As Mars circles the Sun, its orbit (shown in black) crosses this moat between the outer shore at $r_max$ on the right side of the centrifugal barrier, and the inner shore at $r_min$ on the left side of the centrifugal barrier. 

\startbuffer[UKMars3D]
\startaxis[
   hide x axis,
    hide y axis,
    hide z axis,
        axis lines=center,
        axis on top,
	 view={0}{45},
        width = 7cm,
    z post scale = {1.67},
        clip mode = individual,
]
    \addplot3 [
        surf, faceted color = lightgray, color = white,
        z buffer=sort,
        samples=5,
        domain=1:3,
        y domain=0:2*pi,
        samples y=25,
](
{x * cos(deg(y))}, {x * sin(deg(y))}, {-13.27/x}
    );
    \path (-1.8,-1,-8)node {$U$};
    \addplot3 [
        surf, faceted color = middlegray, color = gray,
        z buffer=sort,
        samples=2,
        domain=2.492:3,
        y domain=0:2*pi,
        samples y=25,
](
{x * cos(deg(y))}, {x * sin(deg(y))}, {(-13.27/x)+(15/x^2)}
    );
    \addplot3 [
        surf, faceted color = white, color = white,
        z buffer=sort,
        samples=2,
        domain=2.067:2.492,
        y domain=0:2*pi,
        samples y=25,
](
{x * cos(deg(y))}, {x * sin(deg(y))}, {-2.912}
    );
    \addplot3 [
        surf, faceted color = middlegray, color = gray,
        z buffer=sort,
        samples=2,
        domain=1.5:2.067, % 2.067
        y domain=0:2*pi,
        samples y=25,
](
{x * cos(deg(y))}, {x * sin(deg(y))}, {(-13.27/x)+(15/x^2)}
    );
    \addplot3 [
        surf, faceted color = middlegray, color = gray,
        z buffer=sort,
        samples=2,
        domain=2.492:3,
        y domain=0:2*pi,
        samples y=25,
](
{x * cos(deg(y))}, {x * sin(deg(y))}, {(-13.27/x)+(15/x^2)}
    );
    \node at (1.8,-2.4,-4.2) {$U + K_\theta$};
    \addplot3 [
        thick,
        domain=0:2*pi,
        samples=49,
        samples y=1,
   ](
        {(2.259/(1-0.0934*cos(deg(x))))*cos(deg(x))}, % -336.08
        {(2.259/(1-0.0934*cos(deg(x))))*sin(deg(x))}, % -336.08
        {-2.912}
    );
    \addplot3 [
        surf, faceted color = middlegray, color = gray,
        z buffer=sort,
        samples=2,
        domain=1.12:1.5, % 2.067
        y domain=0:2*pi,
        samples y=25,
](
{x * cos(deg(y))}, {x * sin(deg(y))}, {(-13.27/x)+(15/x^2)}
    );
    \draw[->] (0,0,0) --node[above, pos=.98]{$r$} (3,0,0);
    \draw[shade, ball color = white] (0,0,0) circle[radius=1.2mm]node[above=1mm] {\Sun}; % Sun
    %\draw[] (1.5,0,0) -- (1.5,0,-4.437);
    \draw[shade, ball color = darkgray] (2.492,0,0) circle[radius=0.6mm];
    \draw[shade, ball color = darkgray] ({2.259*cos(15)/(1-0.0934*cos(15))},{2.259*sin(15)/(1-0.0934*cos(15))},0) circle[radius=0.6mm];
    \draw[shade, ball color = darkgray] ({2.259*cos(30)/(1-0.0934*cos(30))},{2.259*sin(30)/(1-0.0934*cos(30))},0) circle[radius=0.6mm]node[above right] {\Mars};
    \draw[shade, ball color = darkgray] ({2.259*cos(45)/(1-0.0934*cos(45))},{2.259*sin(45)/(1-0.0934*cos(45))},0) circle[radius=0.6mm];
    \draw[shade, ball color = darkgray] ({2.259*cos(60)/(1-0.0934*cos(60))},{2.259*sin(60)/(1-0.0934*cos(60))},0) circle[radius=0.6mm];
    \draw[shade, ball color = darkgray] ({2.259*cos(75)/(1-0.0934*cos(75))},{2.259*sin(75)/(1-0.0934*cos(75))},0) circle[radius=0.6mm];
    \draw[shade, ball color = darkgray] (0,2.259,0) circle[radius=0.6mm];
    \draw[shade, ball color = darkgray] ({2.259*cos(105)/(1-0.0934*cos(105))},{2.259*sin(105)/(1-0.0934*cos(105))},0) circle[radius=0.6mm];
    \draw[shade, ball color = darkgray] ({2.259*cos(120)/(1-0.0934*cos(120))},{2.259*sin(120)/(1-0.0934*cos(120))},0) circle[radius=0.6mm];
    \draw[shade, ball color = darkgray] ({2.259*cos(135)/(1-0.0934*cos(135))},{2.259*sin(135)/(1-0.0934*cos(135))},0) circle[radius=0.6mm];
    \draw[shade, ball color = darkgray] ({2.259*cos(150)/(1-0.0934*cos(150))},{2.259*sin(150)/(1-0.0934*cos(150))},0) circle[radius=0.6mm];
    \draw[shade, ball color = darkgray] ({2.259*cos(165)/(1-0.0934*cos(165))},{2.259*sin(165)/(1-0.0934*cos(165))},0) circle[radius=0.6mm];
    \draw[shade, ball color = darkgray] (-2.067,0,0) circle[radius=0.6mm];
    \draw[shade, ball color = darkgray] ({2.259*cos(15)/(1-0.0934*cos(15))},{-2.259*sin(15)/(1-0.0934*cos(15))},0) circle[radius=0.6mm];
    \draw[shade, ball color = darkgray] ({2.259*cos(30)/(1-0.0934*cos(30))},{-2.259*sin(30)/(1-0.0934*cos(30))},0) circle[radius=0.6mm];
    \draw[shade, ball color = darkgray] ({2.259*cos(45)/(1-0.0934*cos(45))},{-2.259*sin(45)/(1-0.0934*cos(45))},0) circle[radius=0.6mm];
    \draw[shade, ball color = darkgray] ({2.259*cos(60)/(1-0.0934*cos(60))},{-2.259*sin(60)/(1-0.0934*cos(60))},0) circle[radius=0.6mm];
    \draw[shade, ball color = darkgray] ({2.259*cos(75)/(1-0.0934*cos(75))},{-2.259*sin(75)/(1-0.0934*cos(75))},0) circle[radius=0.6mm];
    \draw[shade, ball color = darkgray] (0,-2.259,0) circle[radius=0.6mm];
    \draw[shade, ball color = darkgray] ({2.259*cos(105)/(1-0.0934*cos(105))},{-2.259*sin(105)/(1-0.0934*cos(105))},0) circle[radius=0.6mm];
    \draw[shade, ball color = darkgray] ({2.259*cos(120)/(1-0.0934*cos(120))},{-2.259*sin(120)/(1-0.0934*cos(120))},0) circle[radius=0.6mm];
    \draw[shade, ball color = darkgray] ({2.259*cos(135)/(1-0.0934*cos(135))},{-2.259*sin(135)/(1-0.0934*cos(135))},0) circle[radius=0.6mm];
    \draw[shade, ball color = darkgray] ({2.259*cos(150)/(1-0.0934*cos(150))},{-2.259*sin(150)/(1-0.0934*cos(150))},0) circle[radius=0.6mm];
    \draw[shade, ball color = darkgray] ({2.259*cos(165)/(1-0.0934*cos(165))},{-2.259*sin(165)/(1-0.0934*cos(165))},0) circle[radius=0.6mm];
\stopaxis
\stopbuffer

\marginTikZ{}{UKMars3D}{A planet with angular momentum will not crash into the Sun, it will orbit.} % vskip, name, caption

If Mars were somehow given more radial kinetic energy $K_r$, the total energy $H$ would be higher, raising the level of the moat. This would make the moat wider. Mars's orbit, still traveling between the two shores, would be even more eccentric.

The energy graph shows the full compound motion discovered by Kepler. The Sun's mass produces the gravitational potential energy well that keeps the planets from leaving the solar system. Each planet's conserved angular momentum produces a centrifugal barrier that keeps the planet from spiraling into the Sun. The balance between the gravitational potential energy and the centrifugal barrier determines the approximate radius of the orbit. Additional radial kinetic energy determines the orbit's eccentricity.

\section{Orbits are elliptic}
\startblockquote
With the eccentricity and the [orbit's radius] established with the utmost certainty, it might appear strange to an astronomer that there remains yet another impediment in the way of astronomy's triumph\dots. Nevertheless\dots. \autocite{p.~336.}{Kepler1609}
\stopblockquote
During the years of work on the orbits of Mars and Earth, Kepler had developed unmatched skill in geometry and computation. He was testing many more predictions against Tycho's observations. The area law provided the correct speed for Mars along its orbit, but in some places the orbit was deviating from a circle.
Having already abandoned constant angular velocity, Kepler was forced to also abandon the ancient's circular orbits. \quotation{The orbit of the planet is not a circle but of an oval shape.} \autocite{p.~338.}{Kepler1609}

This presented a huge computational challenge. All of Kepler's geometrical methods depend on the geometry of a circle. He needed to find a new shape, but did not know what it should be. He tried some slightly pointy egg shapes, but needed to study many more observations to determine exactly how long and pointy the egg should be.

At one point in this dreadful analysis he writes to a friend, \quotation{if only the shape were a perfect ellipse all of the answers could be found in Archimedes' and Appollonius' works.}\autocite{v.~\convertnumber{KR}{14} p.~409.}{KeplerGW}
Ellipses are circles squashed symmetrically, no pointier at one end than the other. These had been studied extensively by the ancient Greeks. Almost anything that could be done with a circle had also been worked out for ellipses, which would make Kepler's work much easier. Alas, for his egg shape almost nothing was known. Kepler struggled to make progress. The struggle with the egg fills eleven chapters (45-55) of \booktitle{New Astronomy.} In the process he computes an impressive table of twenty-eight different observations of Mars, from 1582 to 1595, with Mars's distance from the Sun, position along its orbit, predicted position in the sky, its observed position, and the error. Many of the errors were several minutes – huge errors by Kepler's standards – causing the egg to \quotation{go up in smoke.}\autocite{p.~406.}{Kepler1609}

Kepler starts over again, using his table of twenty-eight distances. While considering these distances, he was playing around with another calculation and stumbled upon the number $1.00429$, which reminded him of one of the distances, which was off of the circle by $0.00432$. \quotation{It was as if I were awakened from a sleep to see a new light\dots.}\autocite{p.~407.}{Kepler1609}
He tried the same formula for other observations and it again gave the right distances! He did not realize that formula gives the distances on an ellipse, since he had simply stumbled on the formula while doing something else. As a result, he put these correct distances at the wrong positions and produced an orbit that was \quotation{puff-cheeked}\autocite{p.~428.}{Kepler1609} and did not match the observations. He therefore abandoned the formula to try something entirely new – an ellipse!
\startblockquote
Why should I mince words? The truth of Nature, which I had rejected and chased away, returned by stealth through the backdoor, disguising itself to be accepted. That is to say, I laid [the distance formula] aside, and fell back on ellipses, believing that this was quite a different hypothesis, whereas the two, as I will prove in the next chapter, are one and the same\dots.\autocite{p.~388.}{Kepler1609}
\stopblockquote
The ellipse provided exactly the formula he had stumbled upon and then abandoned. The ellipse also provided the correct angles and unlocked a wealth of geometric knowledge from ancient geometers. \quotation{Oh, ridiculous me!} \autocite{p.~430.}{Kepler1609}

Kepler finishes \booktitle{New Astronomy} with a detailed discussion of ellipses' properties and their application to Mars's orbit. His woodcut diagram is shown in \in{figure}[fig:Kepler1609Urania]. Kepler had finally triumphed over Mars.

\placefigure[margin][fig:Kepler1609Urania]{Woodcut from Kepler's \booktitle{New Astronomy} showing the elliptic orbit which explains the motion on Mars. The diagram includes Urania, the muse of astronomy, arriving on her 
triumphal chariot with a laurel wreath – a crown for Kepler to honor his victory over the war god, Mars.} {\externalfigure[Kepler1609Urania][width=\rightmarginwidth]}


All that remained was to triumph over the process of getting \booktitle{New Astronomy} actually published, which took another four years.

Kepler believed that his mathematical model of Mars's motion – the area law on an elliptic orbit – would apply to all of the planets. He was right. Telescopes, which were not available to Tycho or Kepler, allowed far more precise observations. Kepler's model continued to provided excellent predictions, not only for the planets' nearly circular orbits, but also for the highly elliptic orbits of comets like Halley's Comet, shown in \in{figure}[fig:HaleysComet]. 

\startbuffer[TikZ:HaleysComet]
\environment env_physics
\environment env_TikZ
\setupbodyfont [libertinus,11pt]
\setoldstyle % Old style numerals in text
\startTEXpage\small
\starttikzpicture% tikz code
\startpolaraxis
 [	xticklabels=\empty,
 	ytick={0,5,...,45},
 	yticklabels={{},{},$1\units{Tm}$,{},$2\units{Tm}$,{},$3\units{Tm}$,{},$4\units{Tm}$,{}},
 	minor y tick num={4},
	% yminorgrids=true,
	hide x axis,
	ymax = 45,
	scale only axis=true, width={11cm},
 	tick style={middlegray}, % Fixes ticks which are too light in ConTeXt
	major grid style = {middlegray},
	clip = false,
 	% ylabel={Distance from Sun $r$ ($\sci{9}\units{m}$)},
 ]
%    \addplot [ % Mars area at aphelion
%        draw=none, fill=black!20,
%        domain={156.08-4.36}:{156.08+4.36},
%        samples=20,
%    ]
%        {2.259/(1+0.0934*cos(x-336.08))}--(0,0)
%    ;
%    \addplot [ % Mars area at perihelion
%        draw=none, fill=black!20,
%        domain={336.08-6.35}:{336.08+6.35},
%        samples=20,
%    ]
%        {2.259/(1+0.0934*cos(x-336.08))}--(0,0)
%    ;
%    \addplot [ % Mercury area at perihelion
%        draw=none, fill=black!20,
%        domain={77.46-59.63}:{77.46+59.63},
%        samples=80,
%    ]
%        {0.5546/(1+0.20564*cos(x-77.46))}--(0,0)
%    ;
%    \addplot [ % Mercury area at aphelion
%        draw=none, fill=black!20,
%        domain={257.46-33.24}:{257.46+33.24},
%        samples=20,
%    ]
%        {0.5546/(1+0.20564*cos(x-77.46))}--(0,0)
%    ;
%  	\node [name path=Sun] at (0,0) {\Sun};
    \addplot [ % Mercury
        thick,
        domain=0:360,
        samples=600,
    ]
        {0.5546/(1+0.20564*cos(x-77.46))}
  %[yshift=-.5pt]
    %node[pos=0.25] {\Mercury}
    ;
    \addplot [ % Venus
        thick,
        domain=0:360,
        samples=600,
    ]
        {1.082/(1+0.00676*cos(x-131.77))}
  %[yshift=-1.7pt]
    %node[pos=0.25] {\Venus}
    ;
    \addplot [ % Earth
    	name path=Earth,
        thick,
        domain=0:360,
        samples=600,
    ]
        {1.496/(1+0.0167*cos(x-102.93))}
    %node[pos=0.25] {\Earth}
    ;
    \addplot [ % Mars
    	name path=Mars,
        thick,
        domain=0:360,
        samples=600,
    ]
        {2.259/(1+0.0934*cos(x-336.08))}
  %[yshift=1pt, xshift=1.1pt]
    node[below=0mm, pos=0.75] {\Mars}
    ;
    \addplot [ % Jupiter
    	name path=Jupiter,
        thick,
        domain=0:360,
        samples=600,
    ]
        {7.76/(1+0.04854*cos(x-14.27))}
  %[yshift=1pt, xshift=1.1pt]
    node[below=0mm, pos=0.8] {\Jupiter}
    ;
    \addplot [ % Saturn
    	name path=Saturn,
        thick,
        domain=0:360,
        samples=600,
    ]
        {14.23/(1+0.05551*cos(x-92.86))}
  %[yshift=1pt, xshift=1.1pt]
    node[above, pos=0.2] {\Saturn}
    ;
    \addplot [ % Uranus
    	name path=Uranus,
        thick,
        domain=0:360,
        samples=600,
    ]
        {28.642/(1+0.04686*cos(x-172.43))}
  %[yshift=1pt, xshift=1.1pt]
    node[below, pos=0.2] {\Uranus}
    ;
    \addplot [ % Neptune
    	name path=Neptune,
        thick,
        domain=0:360,
        samples=600,
    ]
        {44.981/(1+0.00895*cos(x-46.68))}
  %[yshift=1pt, xshift=1.1pt]
    node[below, pos=0.2] {\Neptune}
    ;
    \addplot [ % Halley
    	name path=Halley,
        thick,
        domain=0:360,
        samples=600,
    ]
        {1.7246/(1+0.96714*cos(x+52.91))}
  %[yshift=1pt, xshift=1.1pt]
    node[above right, pos=0.32] {Halley's Comet}
    ;
\stoppolaraxis
\stoptikzpicture
\stopTEXpage
\stopbuffer

\placetextfloat[bottom][fig:HaleysComet] % location
{The elliptic orbits of all eight planets and Halley's Comet. The planet's orbits are nearly circular, but off-center. Halley's Comet's orbit is highly elliptic, brining it near Earth approximately every $76$ years. It's next appearance will be in 2061. ($1\units{Tm} = 10^{12}\units{m}$)}	 % caption text
{\noindent\typesetbuffer[TikZ:HaleysComet]} % figure contents


 % The first few paragraphs here should probably move to the last section, before "Kepler finishes New Astronomy with...." Right now it is here because I added it late.
Kepler's distance formula, written in modern notation, is 
\startformula
	r = \frac{ r_0}{1+\epsilon \cos\theta}
\stopformula
\startuseMPgraphic{KeplerDistanceEllipse}
draw fullcircle xyscaled (4.5cm,3.6cm) shifted (1.35cm,0);
  path Radius, XAxis;
  Radius := origin -- ((1.44cm/(1-0.6*cosd(30))),0cm) rotated 30;
  XAxis := origin -- (3.6cm,0cm) ;
drawarrow origin -- (-0.9cm,0cm);
drawarrow origin -- (0cm,1.44cm);
drawarrow XAxis;
drawarrow origin -- (0cm,-1.44cm);
drawarrow origin -- Radius;
drawdot origin withpen pencircle scaled 3pt;
%dotlabel.bot("Sun",origin);
label.bot("$r\sub{min}$",(-0.5cm,0));
label.bot("$r\sub{max}$",(0.8cm,0));
label.rt("$r_0$",(0,1cm));
label.rt("$r_0$",(0,-1cm));
label.ulft("$r$",(1.7cm,1cm));
  begingroup;
    save anglelength ;
    anglelength := 1cm ;
    drawarrow anglebetween(XAxis, Radius, btex $\theta$ etex) ;  % draw 
  endgroup;
\stopuseMPgraphic

\startplacefigure[location=margin, reference=fig:KeplerDistanceEllipse, title={The ellipse described by Kepler's distance formula.}]
\small\vskip 2in
\reuseMPgraphic{KeplerDistanceEllipse}
\stopplacefigure

\startuseMPgraphic{MathEllipse}
draw fullcircle xyscaled (4.5cm,3.6cm) ;
drawarrow origin -- (2.5cm,0cm);
drawarrow origin -- (0cm,2.05cm);
draw origin -- (-2.5cm,0cm);
draw origin -- (0cm,-2.05cm);
label.bot("$a$",(-1.25cm,0));
label.bot("$a$",(1.25cm,0));
label.lft("$b$",(0,1cm));
label.lft("$b$",(0,-1cm));
label.top("$x$",(2.45cm,0));
label.rt("$y$",(0,2.0cm));
\stopuseMPgraphic

\startplacefigure[location=margin, reference=fig:MathEllipse, title={The ellipse as commonly described by mathematicians.}]
\small\vskip4.5in
\reuseMPgraphic{MathEllipse}
\stopplacefigure

\noindent
where $r_0$ gives the orbit's size and $\epsilon$ gives the orbit's eccentricity. As Kepler described, he did not recognize this formula as an ellipse, and you probably do not recognize it either. Even if you have studied ellipses is a mathematics course, you probably saw it described in terms of the semi-major axis $a$ and semi-minor axis $b$, as shown in \in{figure}[fig:MathEllipse]. These two shapes are the same, although it is not easy to prove. Kepler gives his proof in Chapter 59 of \booktitle{New Astronomy}.
We would like to relate the orbit's size and eccentricity to the planet's angular momentum $L$ and total energy $H$. Before tackling these relations, let us return briefly to circular orbits.

Elliptic orbits do not have a constant radius, so the formulas for angular momentum and total energy need to be modified to use the parameters of the ellipse – either $r$ and $\epsilon$ in Kepler's formula or $a$ and $b$ in the usual math description. The formulas are
\startformula
	L = m\sqrt{GMr_0}
	\qquad
	H = -G\frac{mM}{2a}.
\stopformula
The good news is that these formulas are almost identical to the formulas for circular orbits. The angular momentum is related to Kepler's $r_0$. Total energy is related to the semi-major axis $a$. The bad news is that one formula uses Kepler's description of the ellipse while the other uses the usual mathematics description. We do not need a complete conversion from one description to the other, but we do need to be able to find $r$ and $a$ in either case. The relations are:
\startformula
	r_0 = \frac{b^2}{a}
	\qquad
	a = \frac{r_0}{1-\epsilon^2}
\stopformula
You can use these to find formulas for $\epsilon$ and $b$ as well, but we will not need them.

\startbuffer[TikZ:EnergyGraphKUHMercury]
\environment env_physics
\environment env_TikZ
\setupbodyfont [libertinus,11pt]
\setoldstyle % Old style numerals in text
\startTEXpage\small
\starttikzpicture% tikz code
\startaxis
 [	width=11cm,y={0.5cm},
   xlabel={Distance from Sun $r$ ($\sci{11}\units{m}$)},
   xmin=0, xmax=1.5,
   minor x tick num=4,
   ylabel={Energy per mass ($\sci{8}\units{J/kg}$)},
   ymin=-15, ymax=0,
   minor y tick num=9,
   clip mode = individual,
		every tick/.style={middlegray}, % Fixes ticks which are too light in ConTeXt
 ]
% \addplot[ % Us
%   thick,
%   domain=10:20,
%   samples=51
%  ]
%  {((x-20)^2/200)}
%  [xshift=9pt]
%    node[pos=0.03] {$U\sub{s}$}
%  ;
%\ifprintanswers
\addplot[ % axis
   thick,
   domain=0:1.5,
   samples=2
  ]
  {0}
%  [yshift=10pt]
%    node[pos=0.5] {$U$}
  ;
\addplot[ % Ug
   thick,
   domain=0.8:1.5,
   samples=150
  ]
  {-13.3/x}
  %[yshift=-10pt]
    node[below right,pos=0.6] {$U$}
  ;
  \addplot[ % Circular Orbits
      thick, dotted,
      domain=0.4:1.5,
      samples=150]
    {-6.65/x};
  \addplot[ % K angular Mercury
      thin,
      domain=0.25:1.5,
      samples=150]
    {(-13.27/x)+(3.68/x^2)};
  \addplot[ % H Mercury
      very thick,
      domain=0.46:0.698,
      samples=2]
    {-11.457};
%    node[below, pos=0.5] {Mercury};
%\addplot[ % K angular Earth
%   thin,
%   domain=0.5:5,
%   samples=150
%  ]
%  {9.926/x^2}
%  %[yshift=8pt]
%    node[above, pos=0.95] {$K_\theta$}
%  ;
%\addplot[ % K angular Earth
%   thin,
%   domain=0.5:5,
%   samples=150
%  ]
%  {(-13.27/x)+(9.926/x^2)}
%  %[yshift=8pt]
%    node[above right,pos=0.75] {$K_\theta + U$}
%  ;
%\addplot[ % E Earth
%   very thick,
%   domain=1.471:1.521,
%   samples=2
%  ]
%  {-4.437}
%  %[yshift=10pt]
%    node[below, pos=0.5] {Earth}
%  ;
%\addplot[ % K angular Mars
%   thin,
%   domain=1:3,
%   samples=150
%  ]
%  {(-13.27/x)+(14.99/x^2)}
%  %[yshift=8pt]
%    node[below left,pos=0.45] {$U + K_\theta$}
%  ;
%\addplot[ % E Mars
%   very thick,
%   domain=2.067:2.492,
%   samples=2
%  ]
%  {-2.912}
% % [yshift=10pt]
%    node[above, pos=0.5] {$H = U + K_\theta + K_r$}
%    %node[above, pos=0.5] {Mars}
%  ;
    \draw[] (0.5546,{(-13.27/0.5546)+(3.68/0.5546^2)}) -- (0.5546,0)node[above]{$r_0\,$};
    \draw[] (0.46,-11.457) -- (0.46,0)node[above]{$r\sub{min}$};
    \draw[] (0.698,-11.457) -- (0.698,0)node[above]{$r\sub{max}$};
    \draw[] ({(0.698+0.46)/2},-11.457) -- ({(0.698+0.46)/2},0)node[above]{$\,_{\phantom{0}}a$};
    \draw[->] (0,1.5) --node[above, pos=.98]{$r$} (1.5,1.5);
    \draw[shade, ball color = white] (0,1.5) circle[radius=2.4mm]node[above =2mm] {\Sun};
    \filldraw[black!20] ({0.5546/(1+0.20564*cos(120))},1.5) circle[radius=1.2mm];
    \filldraw[black!15] ({0.5546},1.5) circle[radius=1.2mm];
    \filldraw[black!20] ({0.5546/(1+0.20564*cos(60))},1.5) circle[radius=1.2mm];
    \filldraw[black!30] ({0.5546/(1+0.20564*cos(30))},1.5) circle[radius=1.2mm];
    \filldraw[black!30] ({0.5546/(1+0.20564*cos(150))},1.5) circle[radius=1.2mm];
    \filldraw[black!40] ({0.5546/(1+0.20564)},1.5) circle[radius=1.2mm];
    \draw[shade, ball color = darkgray] ({0.5546/(1-0.20564)},1.5) circle[radius=1.2mm]node[above=0.8mm] {\Mercury};
\stopaxis
\stoptikzpicture
\stopTEXpage
\stopbuffer

\placetextfloat[top][fig:EnergyGraphKUHMercury] % location
{Mercury's energy graph. The sum $U + K_\theta$ cradles Mercury at its minimum. Mercury oscillates between approximately $r\sub{min}=0.46\units{Gm}$ and $r\sub{max}=0.70\units{Gm}$, in agreement with the orbit shown in \in{figure}[fig:KeplerTerestrial].}	 % caption text
{\noindent\typesetbuffer[TikZ:EnergyGraphKUHMercury]} % figure contents


\section{Harmonies of the World}

Having mastered the geometry of orbits, Kepler next attempted to unite the quadrivium by finding musical relationships to explain the sizes and eccentricities of the planets' orbits. The resulting book, \booktitle{Harmonies of the World,} is an amazing mixture of precise astronomical observations and calculations aligned with musical notation for chords and short musical phrases.

It must be admitted that in the early seventeenth century explaining astronomy with music was no more fanciful than introducing ellipses. However, ellipses have survived rigorous observations while the musical speculations have not. The book does contain one remarkable gem, which has come to be known as Kepler's third law. In studying the periods and sizes of the planets' orbits.
\startblockquote
I first believed I was dreaming... But it is absolutely certain and exact that the ratio which exists between the period times of any two planets is precisely the ratio of the \threehalves th power of the mean distance.\autocite{p.~180.}{Kepler1619}
\stopblockquote
This ratio can be used to relate the periods and radii of any two planets' orbits.
\startformula
\frac{T_1}{T_2} = \left(\frac{a_1}{a_2}\right)^{\threehalves}
\stopformula
where $T_1$ and $a_1$ are the period and radius of one planet's orbit, while $T_2$ and $a_2$ are those of the other. For elliptic orbits the radius is half of the ellipse's long axis. Formally, $a$ is called the semi-major axis.

Kepler believed that the \threehalves th power was a sign of a musical perfect fifth. It is not.

A planet's period can be calculated directly using Kepler's area law to fill the area of an elliptic orbit. The area $A$ of an ellipse is easiest to see in the standard mathematical description.
\startformula
	A = \pi ab
\stopformula
We saw Kepler's area law stated as an area update formula (p.\at[eq:AreaUpdate]).
\startformula
	dA = \frac{L}{2m}\,dt
\stopformula
Since the all of the dA are the same size, this total area $A$ covered during the orbit's period $T$ is
\startformula
	A = \frac{L}{2m}\,T
\stopformula
Using the area of an ellipse, the formula for the planet's angular momentum, and the relation $r_0=b^2/a$, we can calculate the period $T$
\startformula\startmathalignment
\NC	\pi ab	\NC = \frac{\cancel{m}\sqrt{GMr_0}}{2\cancel{m}}\,T \NR
\NC	\pi a\cancel{b}	\NC = \frac{\cancel{b}}{2}\sqrt{\frac{GM}{a}}\,T \NR
%\NC	2\pi a^{\threehalves}	\NC = \sqrt{GM}\,T \NR
\NC	T \NC = \frac{2\pi}{\sqrt{GM}}\,a^{\threehalves}	\NR
\stopmathalignment\stopformula

A planet's orbital period is proportional to the orbit's semi-major axis $a$ raised to the three-halves power.

When Newton presented his laws of motion and gravity in the \booktitle{Principia,} almost eighty years after \booktitle{New Astronomy}, his first application of those laws is a proof of Kepler's area law. Newton goes on to show that elliptic orbits arise only from a force that is inversely proportional to the squared distance. Finally, Newton showed that the \threehalves\ power ratio between period and radius is also a consequence of his theories of motion and gravity. 

Finally, Newton showed that even a tiny change in the gravitational force formula causes the elliptic orbit's long axis to rotate slowly, so that over time the planet traces a flower pattern rather than a simple ellipse.
Many planets' orbits do rotate slowly due to the slight pull of other planets. By 1859 it became clear that only Mercury's orbit rotates in a way that cannot be caused by other planets. This anomaly was not explained until 1915, when Albert Einstein was developing his own geometric model of gravity. For eight years, Einstein had been struggling to build his theory of curved space-time. In the final month, he discovered that his emerging theory predicts a slightly different path for planets close to the Sun. Einstein computed the effect on Mercury's orbit and found it exactly matched the well know anomaly in Mercury's orbit. He recalls, \quotation{for a few days, I was beside myself with joyous excitement.}
Two weeks later he completed the theory of General Relativity.

Kepler's model survived for over three-hundred years – from before Galileo's \booktitle{Starry Messenger} until Einstein's greatest achievement in the twentieth century.


\subject{Notes}
%\placefootnotes[criterium=chapter]
\placenotes[endnote][criterium=chapter]

\subject{Bibliography}
        \placelistofpublications  [criterium=chapter, method=local] % Citations for this chapter

\blank[.5in]
\startblockquote\it
When the storm rages and the state is threatened by shipwreck, we can do nothing more noble than to lower the anchor of our peaceful studies into the ground of eternity.\\% Koestler p. 427
	%\rightaligned{\it Discourse on Happiness}
	\rightaligned{\sc Johanes Kepler}
	%\rightaligned{1706–1749}}
\stopblockquote

\stopchapter
\stopcomponent




%
%\placetable
%    [margin]
%    [T:LinearAngular]
%    {Linear and angular quantities and some of their relations}
%    {\vskip18pt\small%\hbox{
	\starttabulate[|l|c|c|c|]
\FL[2]%\toprule
\NC				\NC Linear			\NC Angular 			\NC Conversion		\NR
\HL
\NC	 Coordinate	\NC $s$			\NC $\theta$			\NC $s=r\theta$	\NR
\NC Velocity		\NC \ $ds=v\,dt$	\NC $\omega$			\NC $v=r\omega$	\NR
\NC Inertia		\NC \ $m$			\NC $I$				\NC $I=mr^2$		\NR
\NC Momentum	\NC $p=mv$		\NC $L=I\omega$		\NC $L=rp$		\NR
\NC Force		\NC $dp=F\,dt$		\NC $dL = \tau\,dt$		\NC $\tau=rF$		\NR
\NC Kinetic Energy	\NC $K=\onehalf mv^2=\frac{p^2}{2m}$
									\NC $K=\onehalf I\omega^2=\frac{L^2}{2I}$
															\NC $K = K$				\NR
\NC Hamilton's Eq.	\NC $F=-\frac{\Delta U}{\Delta s}$
									\NC $\tau=-\frac{\Delta U}{\Delta\theta}$	\NC	\NR
\LL[2]%\bottomrule
    \stoptabulate%}

% Templates:

% Epigraph
\placefigure[margin,none]{}{\small
	\startalignment[flushleft]
	\stopalignment
	\startalignment[flushright]
	{\it }\\
	{\sc }\\
	–
	\stopalignment
}

% Margin image
\placefigure[margin][] % Location, Label
{} % Caption
{\externalfigure[chapter03/][width=144pt]} % File

% Margin Figure
\placefigure[margin][] % location
{}	% caption text
{\starttikzpicture	% tikz code
\stoptikzpicture}

% Aligned equation
\startformula\startmathalignment
\stopmathalignment\stopformula

% Aligned Equations
\startformula\startmathalignment[m=2,distance=2em]
\stopmathalignment\stopformula
