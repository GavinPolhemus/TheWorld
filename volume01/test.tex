\startcomponent test
\project project_world
\product prd_volume01

\startchapter[title=Test, reference=ch:Test]
\startplacefigure[location={margin,none}]
{\small
	\startalignment[flushleft]
	Pythagoras conceived that the first attention that should be given to everyone should be addressed to the senses, as when one perceives beautiful figures and forms, or hears beautiful rhythms and melodies.  Consequently he laid down that the first erudition was that which subsists through music's melodies and rhythms, and from these he obtained remedies of human manners and passions, and restored the pristine harmony of the faculties of the soul.
	\stopalignment
	\startalignment[flushright]
	\emph{Life of Pythagoras}\\
	\scaps{Iamblichus of Chalcis}\\
	c.245–c.325
	\stopalignment
}
\stopplacefigure

\Initial{O}{ur first exposure} to the beauty and precision of mathematics is through music. As infants, our sophisticated, language processing brains discover relationships of rhythm and pitch in both language and song. As our musical appreciation grows, we hear rhythmic patterns of repetition and variation, and harmonic patterns of consonance and dissonance that stir feelings of joy, foreboding, courage, and transcendence. When you sing and dance, you are probably not thinking about mathematical formulas and theorems (I admit to being odd in this way), but music's artistic expressiveness and mathematical precision are inseparable.

\stopchapter
\stopcomponent
