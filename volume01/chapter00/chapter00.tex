% !TEX useAlternatePath
% !TEX useConTeXtSyncParser
\startcomponent c_chapter00
\project project_world
\product prd_volume01

\doifmode{*product}{\setupexternalfigures[directory={chapter01/images}]}

\setupsynctex[state=start,method=max] % "method=max" or "min"

\starttext

%%%%%%%%%%%%%%%%%%%%%%%%%%%%%
\startchapter[title=Invitation, reference=ch:Invitation]
%%%%%%%%%%%%%%%%%%%%%%%%%%%%%
\placefigure[margin,none]{}{\small
	\startalignment[flushleft]
For a short time,\dots allow your thoughts to wander beyond this world to view another, wholly new one, which I shall cause to unfold before it in imaginary spaces.
	\stopalignment
	\startalignment[flushright]
	{\it The World, or a Treatise on Light}\\
	{\sc Rene Descartes}\\
	1596–1650
	\stopalignment
}

%\Initial{I}{ would like you to undertake an ambitious project.} At first, it may seem impossible, but I am confident you can succeed.

\Initial{I}{ invite you to build a world}\dash not a physical world, like the world we already inhabit and share, but an entirely new world, built in your imagination.
You already build worlds in your imagination whenever you read a story.
The story's author offers a sketch.
You add detail and life to this sketch using your own knowledge and creativity.

Let me guide your world-building for a moment.
Look at the items on your desk or table.
Imagine holding a glass of water above this surface, tipping the glass slowly, and pouring until the glass is empty.
Where will the water flow? Where will it splash?
Where might it run off the surface?
Which items will absorb the water?
Will any items float?

You answer these questions using mental models of the things in your imagined world.
You have a model of gravity, pulling everything in your imagined world downward.
You have a model of flowing water, guiding your imagined water across the surface.
You have models of paper, metal, wood, and plastic that include details about how these materials interact with water.
With these models, objects in your imagined world behave like objects in the real world, making your imagined world realistic.

You developed these models throughout your life.
As a baby, you grabbed everything. Whatever it was, you would squeeze it, bang it, drop it, taste it, throw it\dash learning everything you could about it.
This is how you created your mental models.
Gradually, your experiments became more sophisticated, testing your models in new situations.
You wouldn't just drop an object on the floor; you might take it to different rooms and drop it on different floors, and then drop it in a puddle!
Children's meticulous, experimental model building is called \keyterm{play}, and it is how we all build, test, and revise our models of the objects and people around us.

As you navigate the real world, your models constantly guide you, and you constantly revise your models.
Sometimes your models fail\dash something breaks when you thought it would hold; something is hot when you thought it would be cold; something slips or falls.
You pause, surprised, while your brain frantically revises your models, integrating this new evidence.
In the future, your models will be more faithful to the real world.

The chapters that follow will lead you through the explorations of philosophers and physicists, from the ancient Greeks to the nineteenth century, as they confront surprising evidence that challenged their models.
This evidence will challenge your models as well.
The philosophers and physicists will be your companions as you revise your models.

In Chapter 1, the ancient Greeks propose two distinct models: a model of musical harmony based on simple ratios, and a model of the heavens based on circular motion.
From the classical period through the Renaissance, experiments and observations challenged these models, prompting many revisions.
%For both music and astronomy, motion became the central issue.
%Could musical harmony and the rhythms of the heavens be united under one faithful model of motion?

In Chapter 2, Galileo challenges the widely accepted model of astronomy, passed down from Aristotle. His new description of motion provides the basic tools of the a new physics.

In Chapter 3, Descartes proposes that motion cannot be created or destroyed, but only moved from one object to another.
His model of conserved motion explains some motion experiments, but conflicts with many others.
Newton and Leibnitz propose two different revisions of the conserved motion model, prompting a long argument which continues well into the eighteenth century.
Newton's momentum is presented in Chapter 3. Leibniz's energy of motion (which he calls \visviva) is presented in Chapter 4.
Experimental evidence and insightful analysis lead to an unexpected resolution.

In Chapter 5, Daniel Bernoulli's experiments demonstrate that the energy of motion is not conserved, because it can be converted to a stationary form.
He revises the motion model to include this stationary energy and then makes many surprising predictions. Experiments provide tremendous support for his combined energy model.

By the end of Chapter 5, you will have two powerful models of motion\dash one based on conservation of momentum and another based on conservation of energy. 
With practice, momentum and energy will flow through your imagined world as naturally as water, driving the imagined objects' motion in ways that are faithful to real objects' motion.
These well-tested models offer new insights into motions far beyond our everyday experience, from the quick, microscopic jostling of molecules to the slow rotation of vast galaxies.
%Having two models of motion is convenient, because many questions about motion are easier to answer with one model or the other.
%However, having two models of motion can also be confusing.
%Both models match experiments, but are they both true?

Chapter 6 takes us to the end of the eighteenth century, when Lagrange finds a deep connection between the two models.
Hamilton, in 1837, uses this connection to forge a flexible and powerful new method for modeling both energy and momentum.

In the final two chapters, we will apply Hamilton's method to a detailed study of the two problems posed by the ancient Greeks in Chapter 1: musical harmony and celestial motion.
In Chapter 7, we follow Kepler's wild journey to a new model of astronomy.
Careful observations of the planets forced Kepler to abandon the basic tenets of ancient astronomy.
Hamilton's method provides a new, solid foundation for Kepler's model.
In Chapter 8, Hamilton's method leads to a mathematical model of waves. Using this model, Helmholtz solves the ancient puzzle of musical harmony.

Momentum, energy, and Hamilton's method will open vast new possibilities in your imagined world. 
To this day, these are the most useful ideas to emerge from our two thousand years of playing with motion and revising our models.
%Momentum, energy, and Hamilton's method also provide a strong foundation for twentieth century physics.
These chapters follow the most direct path to our modern understanding of motion.
This path is not solely for future physicists. It is for musicians who wish understand instruments, sound, and hearing; historians recognizing technology's role in shaping our cultural landscape; athletes looking for a competitive edge; and philosophers interested in our universe and our place in it. This path is for anyone curious about the world. 

I wrote this book out of respect for my students. Many did not take physics by choice, but every one gave me and my subject a chance. They each found something to enjoy\dash the puzzles, the stories, or the wonder. I humbly ask you do do the same. I cannot make physics easy, but I will make it worth your best effort. My students have shown me that this is enough. I have written this book for them, and for you.

Now, let's play!


\stopchapter

\stoptext

\stopcomponent

\subject{Notes}
\blank
\startcolumns
%\placefootnotes[criterium=chapter]
\placenotes[endnote][criterium=chapter, method=local]
\stopcolumns

\subject{Bibliography}
\placelistofpublications  [criterium=chapter, method=local]			% Citations for this chapter only
