% !TEX useAlternatePath
% !TEX useConTeXtSyncParser
\startcomponent c_chapter00
\project project_world
\product prd_volume01

\doifmode{*product}{\setupexternalfigures[directory={chapter01/images}]}

\setupsynctex[state=start,method=max] % "method=max" or "min"

\starttext

%%%%%%%%%%%%%%%%%%%%%%%%%%%%%
\startchapter[title=Invitation, reference=ch:Invitation]
%%%%%%%%%%%%%%%%%%%%%%%%%%%%%
\placefigure[margin,none]{}{\small
	\startalignment[flushleft]
For a short time,\dots allow your thoughts to wander beyond this world to view another, wholly new one, which I shall cause to unfold before it in imaginary spaces.
	\stopalignment
	\startalignment[flushright]
	{\it The World, or a Treatise on Light}\\
	{\sc Rene Descartes}\\
	1596–1650
	\stopalignment
}

%\Initial{I}{ would like you to undertake an ambitious project.} At first, it may seem impossible, but I am confident you can succeed.

\Initial{I}{ want you to build a world}\dash not a physical world, like the world we already inhabit and share, but an entirely new world, built in your own imagination.

You already build worlds in your imagination whenever you read a story or study history.
The story's author offers a sketch.
You add detail and life to this sketch using your own knowledge and creativity.
To make the imagined world believable, you build it using many sophisticated mental models. These models allow your imagined world to imitate the real world in important ways.

Let me guide your world-building for a moment, while you use your models\dash which are already in your mind\dash to fill in the details.
Look at the items on your desk or table.
Imagine holding a glass of water above this surface, tipping the glass slowly, and pouring until the glass is empty.
Where will the water flow? Where will it splash?
Where might it run off the surface?
Which items will absorb the water?
Will any items float?
%How will the people around you respond to this sudden outpouring?

You answer each of these questions using your mental models. You have a model of gravity, pulling everything in your imagined world downward.
You have a model of fluid flow, guiding your imagined water across the surface.
You have models of paper, metal, wood, and plastic that include details about how these materials interact with water.
%You even have, within your own mind, models of other people's minds, so that you might anticipate their responses.
Your models are not perfect, but they are amazing and effortless. % in their application. 
These models govern your imagined world in ways that are faithful to the behavior of the real world, making your imagined worlds realistic.

You were not born with these models. You developed them all yourself. As a baby, you grabbed everything. Whatever it was, you would squeeze it, bang it, drop it, taste it, throw it\dash learning everything you could about it.
As your models became more sophisticated, so did your experiments.
You wouldn't just drop an object on the floor; you would take it to different rooms and drop it on different floors, and then drop it in a puddle!
Children's meticulous, experimental model building is called \keyterm{play}, and it is how we all build, test, and refine our models of the objects and people around us.

As you navigate the real world, your models constantly guide you, and you constantly refine your models.
Sometimes your models fail\dash something breaks when you thought it would hold; something is hot when you thought it would be cold; something slips or falls.
You pause, surprised, while your brain frantically rewires your models, weighing this new evidence against a lifetime of evidence gathered in similar situations.
In the future, your models will be more faithful to the real world.

My goal is to challenge your existing models with new evidence.
I will present these challenges through the history of philosophers and physicists confronting surprising evidence and struggling to find their own new models consistent with it.
They will be your companions as you refine your own models.

In Chapter 1, the ancient Greeks propose two distinct mathematical models: a model of musical harmony based on simple ratios, and a model of the heavens based on circles.
From the classical period through the Renaissance, experiments and observations challenged these mathematical models, prompting many subtle refinements\dash and then unleashing a revolution in the seventeenth century.
For both music and astronomy, motion became the central issue.
Could musical harmony and the rhythms of the heavens be united under one faithful model of motion?

In Chapter 2, Galileo introduces new methods of quantifying and modeling motion.
Galileo uses the most childish of experiments\dash dropping a rock from a tower\dash to challenge well established models of the world.

In Chapter 3, Descartes proposes that motion cannot be created or destroyed, but only moved from one object to another.
He offers a universal formula for this conserved quantity of motion, but the formula quickly fails in experiments. Newton and Leibnitz each propose new formulas for the conserved quantity, prompting a long argument which continues well into the eighteenth century. Newton's quantity, momentum, is presented in Chapter 3. Leibniz's quantity, \visviva, is presented in Chapter 4.
Experimental evidence and insightful analysis lead to an unexpected resolution.

In Chapter 5, Daniel Bernoulli's experiments demonstrate that \visviva\ sometimes hides in motionless objects. He proposes that \visviva\ (now called energy) can transform between moving and stationary forms, and only the total is conserved.

By the end of Chapter 5, you will have two powerful models of motion\dash one model based on conservation of momentum and another based on conservation of energy. 
With practice, momentum and energy will flow through your imagined world as naturally as water, driving the imagined objects' motions in ways that are faithful to real objects' motions.
These well-tested models will give you new insights into motions far beyond your everyday experience, from the quick, microscopic jostling of molecules to the slow rotation of vast galaxies.
%Having two models of motion is convenient, because many questions about motion are easier to answer with one model or the other.
%However, having two models of motion can also be confusing.
%Both models match experiments, but are they both true?

Chapter 6 takes us to the end of the eighteenth century, when Lagrange finds a deep connection between the two models.
Hamilton, in 1837, uses this connection to forge a flexible and powerful new model incorporating both energy and momentum.
Applications of Hamilton's model are innumerable.
[Add some examples.]

In the final two chapters, we will apply Hamilton's method to a detailed study of the two problems posed by the ancient Greeks in Chapter 1: musical harmony and celestial motion.
In Chapter 7, we will decipher Kepler's wild and mysterious journey to a new astronomy based on ellipses and areas rather than circles.
In Chapter 8, we will study the wave motion at the center of Helmholtz's new understanding of harmony in music.
With Kepler and Helmholtz, we will have come full circle, building models of musical harmony and celestial motion on the solid foundation of Hamilton's method.

Momentum, energy, and Hamilton's method will open vast new possibilities in your imagined world. 
To this day, these are the most useful ideas to emerge from our two thousand years of playing with motion and refining our models.
Momentum, energy, and Hamilton's method also provide a strong foundation for twentieth century physics.

You have probably heard that quantum mechanics and Einstein's relativity replaced Newtonian physics. This is only partly true.
There was a crisis at the end of the nineteenth century.
New technologies and techniques in optics and chemistry, along with the discovery of radioactivity, provided new ways to study the microscopic world.
During this time, many experiments and observations showed that very small and very fast particles did not follow the rules of Newtonian physics.
To physicists at the time, the ground seemed to be crumbling beneath their feet.
They could not know what, if anything, would survive from the old models.
A century after that crisis, we do know: Conservation of energy and momentum survived. In 1923, Hamilton's method was de Broglie's key to unlock microscopic world.

This brings me to a final comment about the present book.
This is not a history of physics. This is a physics text with an historical approach and a modern perspective.
Important historical figures are left out if they could distract us from developing a modern understanding of the world.
The most striking example is Euler, in the middle of the eighteenth century, who placed acceleration (rather than momentum) at the center of his brilliant formulation of Newtonian physics. 
Euler's formulation was demolished by twentieth century experiments.
While Euler is a central historical figure, his physics is an unnecessary detour.

\blank


[Need to wrap it up....]

Every science conference, every academic journal, every physics textbook is an extension of our earliest explorations, giggling with delight at the startling comprehension of a wonderful world.

Let's play!


\stopchapter

\stoptext

\stopcomponent

\subject{Notes}
\blank
\startcolumns
%\placefootnotes[criterium=chapter]
\placenotes[endnote][criterium=chapter, method=local]
\stopcolumns

\subject{Bibliography}
        \placelistofpublications  [criterium=chapter, method=local]			% Citations for this chapter only


\page
%%%%%%%%%%%%%%%% EXERCISES %%%%%%%%%%%%%%%%
\startsubject[title=Exercises]
%\setuplayout[
%	leftmargin=36pt, % 1/2 in
%	leftmargindistance=9pt, % 3/8 in
%	width=477pt, % 4 1/4 in
%	rightmargindistance=9pt, % 3/8 in
%	rightmargin=36pt,  % 2 in
%]
%\setupheadertexts[text][section][\pagenumber][\pagenumber][chapter]
%\setupheadertexts[margin][][][][]
%
%\blank
%\startcolumns[n=2, tolerance=verytolerant]
\startitemize[n,packed]
\question
Find the frequency of the pitch one octave above middle A ($440\units{Hz}$).  What is the period of this higher pitch? %(Here \quote{A} is the name of the pitch.  It has nothing to do with the vibration’s amplitude, $A$.)
\blank

\question
Find the frequency of the pitch one perfect fifth below middle A.
\blank
%\question
%What the frequency of the highest A on a grand piano? What subscript should be put on this A?

\question Starting with the duration formula and working only with variables\nowhitespace
\startitemize[a,joinedup]%,packed]
\item solve for $t_{\rm f}$,
\item solve for $t_{\rm i}$.
\stopitemize
\blank

\question Starting with the frequency formula and working only with variables
\startitemize[a,packed,joinedup]
\item solve for $N$,
\item solve for $\Delta t$.
\stopitemize
\blank

\question
How many cycles will the middle A ($440\units{Hz}$) string oscillate in $2.5\units{s}$?
\blank

\question
How long will it take for the  D string with a frequency of $293\units{Hz}$ to oscillate one thousand times?
\blank
%\question
%What is the frequency of C$_0$? What is its period?
%\blank

\question Starting with the frequency relations and working only with variables
\startitemize[a,packed]
\item find $T$ in terms of $f$,
\item find $T$ in terms of $N$ and $\Delta t$.
\stopitemize
\blank[big]


%\question The Moon completes one revolution around Earth in 29.3 days. Find the speed of the Moon in its orbit around Earth.

\question The inner two moons of Jupiter, Io and Europa, are in a 1:2 resonance, which means that the ratio of Io’s period to Europa’s period is one-to-two. Europa’s orbital period is 3.55 days. What is Io’s orbital period?
%\begin{solution}[3in]
%\begin{align*}
%	\frac{T\sub{E}}{T\sub{I}} &= \frac{2}{1}	\\
%	T\sub{I} &= \half T\sub{E}	\\
%		&= \half 3.55\units{d}	\\
%		&= \answer{1.78\units{d}}
%\end{align*}
%\end{solution}

\question Jupiters second and third moons, Europa and Ganymede, are also in a 1:2 resonance. What is the Ganymede’s orbital period?
%\begin{solution}[3in]
%\begin{align*}
%	\frac{T\sub{E}}{T\sub{I}} &= \frac{2}{1}	\\
%	T\sub{I} &= \half T\sub{E}	\\
%		&= \half 3.55\units{d}	\\
%		&= \answer{1.78\units{d}}
%\end{align*}
%\end{solution}


\question Galileo published \textit{The Starry Messenger} in 1610. The Galileo spacecraft, which studied Jupiter and its moons up close, became the first craft to orbit Jupiter in 1995. How many orbits did Jupiter make between these two events? Jupiter orbits the Sun once every 11.86 years.

\question The Galileo spacecraft was launched on October 18, 1989. Its successful mission as the first man-made satellite of Jupiter ended on September 21, 2003 with an intentional crash into Jupiter’s atmosphere. Find the length of the Galileo mission in jovian years.

%\question Saturn orbits the Sun once every 29.5 years. How many orbits has it made since Galileo published \textit{The Starry Messenger} in 1610?
%\begin{solution}[3in]
%	\startformula
%T = \frac{1\units{cyc}\cdot\Delta t}{N}
%\stopformula
%First, find $\Delta t$.
%\startformula
%	\Delta t = t\sf - t\si = 2017 - 1610 = 407\units{yr}
%\stopformula
%Then we can find the number of cycles (or orbits).
%	\begin{align*}
%		T &= \frac{1\units{cyc}\cdot\Delta t}{N}	\\
%		N &= \frac{1\units{cyc}\cdot\Delta t}{T}	\\
%			&= \frac{1\units{cyc}\cdot(407\units{\ucan{yr}})}{29.5\units{\ucan{yr}}}	\\
%			& = \answer{13.8\units{cyc}}
%				\quad\text{\emph{or}}\quad
%				\answer{13.8\units{orbits}}
%	\end{align*}
%\end{solution}

\question The Cassini spacecraft journey to Saturn began with its launch on October 15, 1997. Cassini orbited Saturn, studying the planet, its many moons, and its amazing rings until September 15, 2017. How long was Cassini’s mission in Saturn’s years? Saturn’s orbit takes 29.5 Earth years.

\question Earth’s Moon orbits Earth with a frequency of $13.4\units{cyc/yr}$. What is the Moon’s orbital period in days?
%\begin{solution}[3in]
%\begin{align*}
%	T &= \frac{1\units{cyc}}{f}	\\
%		&= \frac{1\units{cyc}}{13.4\units{cyc/yr}}	\\
%		&= 0.0746\units{\ucan{yr}}\left(\frac{365\units{d}}	{1\units{yr}}\right)	\\
%		&= \answer{27.2\units{d}}
%\end{align*}
%Accept answer in years is question does not specify days (2017).
%\end{solution}


\question In 1967, Jocelyn Bell Burnell and Antony Hewish discovered a star that flashed brightly every $1.33\units{s}$. Eventually, pulsing stars like this became known as pulsars. Pulsars are rapidly spinning neutron stars which shine bright beams of light out of their magnetic poles. We see a flash every time one of the spinning pulsar’s beams points in our direction. What is this pulsar’s frequency?

\question In 1968 another pulsar was discovered in the nearby Crab Nebula with a period of only $33\units{ms}$. What is this pulsar’s frequency?

\question On August 17, 2017, an extremely sensitive device known as \scaps{ligo} (Laser Interferometer
Gravitational-Wave Observatory) detected a tiny vibration of space-time. These vibrations were gravitational waves from a pair of orbiting neutron stars which revealed the stars' orbital frequency to be $20\units{Hz}$. What was the orbital period of these two stars? (This ferocious dance did not last long. Over the next thirty seconds the neutron stars spiraled into each other, producing an enormous explosion. The merged stars then collapsed to form a black hole. Luckily, this violent event happened very, very far from us.)

\stopitemize
%\stopcolumns
\stopsubject