% !TEX useOldSyncParser
\startcomponent c_chapter01
\project project_world
\product prd_volume01

\setupsynctex[state=start,method=max] % "method=max" or "min"
\starttext


%%%%%%%%%%%%%%%%%%%%%%%%%%%%%
\startchapter[title=Four Paths,reference=ch:Music]
%%%%%%%%%%%%%%%%%%%%%%%%%%%%%
\placefigure[margin,none]{}{\small
	\startalignment[flushleft]
	Pythagoras conceived that the first attention that should be given to everyone should be addressed to the senses, as when one perceives beautiful figures and forms, or hears beautiful rhythms and melodies.  Consequently he laid down that the first erudition was that which subsists through music’s melodies and rhythms, and from these he obtained remedies of human manners and passions, and restored the pristine harmony of the faculties of the soul.
	\stopalignment
	\startalignment[flushright]
	{\it Life of Pythagoras}\\
	{\sc Iamblichus of Chalcis}\\
	c.245–c.325
	\stopalignment
}

\Initial{O}{ur first exposure} to the beauty and precision of mathematics is through music. As infants, our sophisticated, language processing brains discover relationships of rhythm and pitch in both language and song. As our musical appreciation grows, we hear rhythmic patterns of repetition and variation, and harmonic patterns of consonance and dissonance that stir feelings of joy, foreboding, courage, and transcendence. When you sing and dance, you are probably not thinking about mathematical formulas and theorems (I admit to being odd in this way), but music’s artistic expressiveness and mathematical precision are inseparable.

Several ancient cultures investigated music’s precise and expressive nature. Ancient Egyptians, Babylonians, and Chinese, among others, each developed unique musical styles, but they all found ways to blend certain pitches to produces the full, rich sounds that give music its expressive power. To achieve this \quotation{sounding together,} or \keyterm{consonance,} the voices or instruments must be perfectly in tune. If the pitches are off, then there is a \quotation{sounding together,} or \keyterm{dissonance.}

\placefigure[margin][fig:Pythagoras]{Pythagoras (right) and follower Philolaus play pipes demonstrating consonant sounds. The lengths of the pipes are labeled in this woodcut from Franchino Gafurio’s \booktitle{Theoria musice} (1492). Philolaus lived about a century after Pythagoras, so they did not actually perform together.} {\externalfigure[chapter01/gaffurio_pythagoras_pipes][width=\rightmarginwidth]}

Ancient musicians also discovered that {\em simple ratios produce consonant pitches.} For example, two pipes with lengths in the ratio of three-to-two or two-to-one (like in figure \in[fig:Pythagoras]) will produce consonant pitches, while pipes with less simple ratios, like nine-to-eight, will produce dissonant pitches. This mathematical model of music connects the purely mathematical idea of simple ratios to the physical phenomena of consonant sounds.

In the west these musical discoveries were attributed to Pythagoras,\index{Pythagoras} a Greek philosopher in the sixth century \scaps{bc}.  None of Pythagoras’s writings survive, so it is impossible to know if this attribution is correct.  His followers, the Pythagoreans, certainly possessed the simple mathematical model of consonance. As a result, they considered music to be a mathematical discipline like arithmetic.

The Pythagoreans also sought an understanding of the cycles of nature – day and night, the seasons of the year, the Moon’s phases, and even the motion of the visible planets. Simple ratios do not help. A year is about 365\onequarter\ days, a lunar month is about 29\onehalf\ days, so there are about 12\onethird\ lunar months in a year. None of these are simple ratios. The motions of the five visible planets – Mercury, Venus, Mars, Jupiter, and Saturn – are even more complicated.

\placefigure[margin][fig:LunarEclipse]{During a lunar eclipse the Moon passes through Earth’s shadow. The shadow’s curve reveals Earth’s round shape.} {\externalfigure[chapter01/1600px-Lunar_eclipse_of_2017_August_7][width=\rightmarginwidth]}

None the less, the Pythagoreans did recognize one interesting pattern: the objects that rule the rhythms of nature are all round. The Moon is obviously round. Seen through a thick veil of clouds, the Sun is clearly round. The Pythagoreans even knew Earth is round because Earth casts a round shadow on the Moon during a lunar eclipse, as seen in figure \in[fig:LunarEclipse]. Lunar eclipses occur fairly frequently, about once a year. Anyone on the night side of Earth can see a lunar eclipse. During the eclipse the curved shape of Earth’s shadow is quite obvious. Even in ancient times, Earth’s spherical shape was common knowledge among the educated.

Extrapolating boldly, the Pythagoreans concluded that everything in astronomy is circular, including the paths followed by celestial bodies. Therefore, the Sun and Moon,  planets, and stars seen passing overhead every day and night must be following circular paths around our spherical Earth. This was the Pythagorean’s second mathematical model: {\em celestial objects circle Earth.} The model connects a mathematical abstraction, the perfect circle, to the physical objects of astronomy. Because of this connection, Pythagorean’s considered astronomy to be a mathematical discipline like geometry.

Music, arithmetic, astronomy, and geometry were the \quotation{sounding together,} to knowledge, known together as the {\em quadrivium.} Arithmetic and geometry describe abstract numbers and shapes. Music and astronomy  describe concrete, observable phenomena. The physical and abstract we closely joined – musical consonance with arithmetic ratios, and astronomical motions with geometric circles. The quadrivium was a beautiful system which brought order to Europeans’ understanding of the world until the end of the sixteenth century.

During the late sixteenth century new ideas and observations caused a crisis for the quadrivium disciplines.  Bold new ideas were proposed, tested, and refined during the seventeenth, eighteenth, and nineteenth centuries, eventually revealing a new physics of motion that unified the quadrivium. This book is the story of that surprising and beautiful physical theory.

In this chapter we will lay the foundation for the new physics by studying the arithmetic model of music and the geometric model of astronomy. These two models followed very different paths. One developed into a sophisticated mathematical system, offering precise insights and transcendent beauty. The other languished, offering nothing but confusion and misdirection until it was abandoned and replaced. The discipline that blossomed into a true mathematical science was, of course, music.

%%%%%%%%%%%%%%%%%%%%%%%%%%%%%%%%%
\section{Simple ratios produce consonant sounds}
%%%%%%%%%%%%%%%%%%%%%%%%%%%%%%%%%

Before we apply the Pythagorean’s mathematical model of musical consonance, there is one thing that I want you to remember: this model is fantastic! Simple ratios really do produce consonant pitches. We will look at some different ideas about how to apply this model in musical practice, as well some conflicting ideas about why the model works, but the model itself is a solid foundation for all music theory.

Even so, the model faced several attacks over the years. The most serious challenges were brought by Vincenzo Galilei, a highly regarded Italian musician in the late sixteenth century. As we look at the Pythagorean model, Galilei’s criticism will help us understand and improve it – or perhaps abandon it.

\placefigure[margin][fig:PythagoreanPipesCons]{Pythagoras and Philolaus’s pipes, represented by vertical lines, can be paired in many ways to produce the five simple ratios of the Pythagorean musical system. These pairs will produce consonant pitches. Arrows that are the same length show the same ratios.} {\externalfigure[chapter01/PythagoreanPipesCons]}%[width=\rightmarginwidth]}

\placefigure[margin][fig:PythagoreanPipesDis]{Some pairs do not form simple ratios. These pairs will produce dissonant pitches.} {\externalfigure[chapter01/PythagoreanPipesDis]}%[width=\rightmarginwidth]}

We start with Pythagoras and and his follower Philolaus, who demonstrate the consonant pitches produced by simple ratios in figure \in[fig:Pythagoras]. The lengths of the pipes are shown so ratios can be calculated. The longest pipes produce the lowest pitches, and the shortest pipes produce the highest. Philolaus and Pythagoras are playing pipes whose lengths are in a two-to-one ratio, which produces a nice, consonant sound. They hold additional pipes that can produce every simple ratio in the Pythagorean musical system. The Pythagoreans had strong numerological beliefs about the number four being complete, so they built their musical system from the five simple ratios that can be made from the numbers one through four. Figure \in[fig:PythagoreanPipesCons] shows Pythagoras and Philolaus’s six pipes and the all of the simple ratios that can be made by pairing them. Each of these pairings will produces consonant pitches.

Not every pair in this pipe collection produces consonant pitches. The three pairs in figure \in[fig:PythagoreanPipesDis] are not simple ratios of small numbers. Each of these pairs will produce dissonant pitches.

%\placefigure[margin][fig:]{} {\externalfigure[chapter01/][width=\rightmarginwidth]}

\placetextfloat[top][fig:MonochordTyndall]{Monochord shown in John Tyndall’s 1875 textbook, \booktitle{Sound.}} {\externalfigure[chapter01/MonochordTyndallSound][width=\textwidth]}

\placefigure[margin,here][fig:monochord]{A monochord produces the lowest pitch when the full length of string is played.  Placing a finger on the string reduces the string’s length and raises its pitch. The three lengths shown can be paired to form the simple ratios 2:1, 3:2, and 4:3.}
{\vskip1in\noindent%
\starttikzpicture[thick,domain=-90:90]
	% First monochord
	\draw[fill=black!10] (0,4.75) rectangle (1.5,-4.75); % Box
	\draw (0.5,4.5)--(1,4.5); % Top bridge
	\draw (0.5,-4.5)--(1,-4.5); % Bottom bridge
	\draw[opacity=.5] (.75,4.5)--(.75,-4.5);
	\draw[opacity=.5] plot ({cos(\x)/10 + 0.75},{\x/20});
	\draw[opacity=.5] plot ({-cos(\x)/10 + 0.75},{\x/20});
	\draw[thin] (.25,4.5) -- (.45,4.5);
	\draw[thin] (.25,-4.5) -- (.45,-4.5);
	\draw [thin,<->](.32, 4.5) -- node[fill=black!10,inner sep=2pt]{\m{L}}  (.32, -4.5);
	% Second monochord
	\draw[fill=black!10] (1.75,4.75) rectangle (3.25,-4.75); % Box
	\draw (2.25,4.5)--(2.75,4.5); % Top bridge
	\draw (2.25,-4.5)--(2.75,-4.5); % Bottom bridge
	\draw (2.5,4.5)--(2.5,1.5);
	\draw[opacity=.5] (2.5,1.5)--(2.5,-4.5);
	\draw[thin] (2,1.5) -- (2.35,1.5);
	\draw[thin] (2,-4.5) -- (2.2,-4.5);
	\draw[opacity=.5] plot ({cos(\x)/10 + 2.5},{\x/30-1.5});
	\draw[opacity=.5] plot ({-cos(\x)/10 + 2.5},{\x/30-1.5});
	\draw [thin,<->](2.07, 1.5) -- node[fill=black!10,inner sep=2pt]{\m{\tfrac{2}{3}L}}  (2.07, -4.5);
	\draw[fill=white] (2.5,1.5) circle[radius=.1]; % Finger placement
	% Third monochord
	\draw[fill=black!10] (3.5,4.75) rectangle (5,-4.75); % Box
	\draw (4,4.5)--(4.5,4.5); % Top bridge
	\draw (4,-4.5)--(4.5,-4.5); % Bottom bridge
	\draw (4.25,4.5)--(4.25,0); % Straight string
	\draw[opacity=.5] (4.25,0)--(4.25,-4.5);
	\draw[opacity=.5] plot ({cos(\x)/10 + 4.25},{\x/40-2.25});
	\draw[opacity=.5] plot ({-cos(\x)/10 + 4.25},{\x/40-2.25});
	% Dimension of vibrating string
	\draw[thin] (3.75,0) -- (4.1,0);
	\draw[thin] (3.75,-4.5) -- (3.95,-4.5);
	\draw [thin,<->](3.82, 0) -- node[fill=black!10,inner sep=2pt]{\m{\tfrac{1}{2}L}}  (3.82, -4.5);
	\draw[fill=white] (4.25,0) circle[radius=.1]; % Finger placement
\stoptikzpicture
}
Pythagoras and Philolaus could have demonstrated their mathematical model of consonance with a string instrument. A monochord, like the one in figure \in[fig:MonochordTyndall], is often used for these demonstrations, but a cello or guitar works as well.  On any of these instruments a single string can produce many pitches. The string sounds its lowest pitch when its full length is free to vibrate. Pressing down on the string reduces the portion that can vibrate, producing a higher pitch. Making the string very short will raise the pitch dramatically.

To hear consonance and dissonance, you need to play two strings. To test simple ratios, the strings must be tuned to the same pitch. (Instruments with multiple strings always have each string tuned to a different pitch, so you need two instruments.) Strings tuned to the same open pitch produce consonant sounds when their lengths are reduced to form simple ratios like those on the monochords in figure \in[fig:monochord]. Ratios that aren’t simple will produce dissonant pitches, some much worse than others. Strings, like pipes, provide evidence supporting the Pythagorean model.

Galilei challenged the Pythagorean model by producing experimental evidence against it. In one set of experiments, he altered the tension in the strings by hanging different weights from them, as in figure \in[fig:GaffurioPythagorasStrings]. Increasing the weight increases the tension and raises the pitch. Using identical strings, Galilei found weights in the simple ratios of two-to-one and three-to-two produced dissonant pitches, but weights in the ratio of nine-to-four produced consonant pitches.
In the other experiment, Galilei altered the weights of the strings themselves, comparing thick, heavy strings with thin light strings, all stretched to the same length and tension. Lighter strings produce a higher pitch than heavy strings. Again he found exactly the same results as he had for the hanging weights. Simple ratios do not always produce consonant pitches!

\placefigure[margin][fig:GaffurioPythagorasStrings]{Pythagoras demonstrating consonant sounds in strings with different weights attached. This experiment does not produce the results that the Pythagoreans predicted. (Also from Franchino Gafurio’s \booktitle{Theoria musice.}} {\externalfigure[chapter01/gaffurio_pythagoras_strings][width=\rightmarginwidth]}

For advocates of the Pythagorean model, this challenge was embarrassing but helpful. It was embarrassing because many books on music theory, from ancient times right up until Galilei’s day, suggested that these experiments had been done, and the results supported the model. The woodcut in figure \in[fig:GaffurioPythagorasStrings], made decades before Galilei was born, depicts Pythagoras performing this experiment as a demonstration of the Pythagorean model! It is not entirely clear why this problem went unnoticed for over two thousand years. Some people probably noticed, and concluded that they should only use the model for lengths, not weights. It is certain that others simply assumed the experiment would produce the desired result, and they never bothered to check.

Galilei’s challenge was quickly met by the realization that the ratios must be applied to the instrument’s rate of vibration. This rate is called the \keyterm{frequency}. The mathematical model was refined to say \emph{simple frequency ratios produce consonant pitches}. Galilei’s experimental challenge brought about an extremely helpful clarification that turned attention from how the instrument is built to how it moves when producing sound.

%\stopsection
%%%%%%%%%%%%%%%%%%%%%%%%%%%%%%%%%
\section{Amplitude and frequency}
%%%%%%%%%%%%%%%%%%%%%%%%%%%%%%%%%

\placefigure[margin][fig:stringamp]{The amplitude of a string’s vibration is measured from the string’s rest position to its maximum displacement on either side.}
	{\hbox{\starttikzpicture[thick,domain=-90:90]
	\draw[fill=black!10] (0,4.75) rectangle (1.5,-4.75); % Box
	\draw (0.5,4.5)--(1,4.5); % Top bridge
	\draw (0.5,-4.5)--(1,-4.5); % Bottom bridge
	\draw[opacity=.5] (.75,4.5)--(.75,-4.5);
	\draw[opacity=.5] plot ({cos(\x)/4 + 0.75},{\x/20});
	\draw[opacity=.5] plot ({-cos(\x)/4 + 0.75},{\x/20});
	\draw [thin,->](.15, 0)--(0.5, 0) node[above left]{\m{A}};
	\draw [thin,->](1.1, 0)--(0.75, 0); %node[left]{\m{A}};
	\stoptikzpicture}%
}

All sounds, musical or not, are produced by something vibrating.  When the motion is disorganized, the resulting sound can be anything from a hiss to a clatter.  A musical pitch is produced by an organized vibration that repeats the same movement many times per second in exactly the same way.

A plucked string vibrates to produce a clear pitch.  The motion is too fast to follow with your eyes, but you can see the blurring of the string caused by its rapid vibrations.  The string’s vibration has two characteristics important to both physicists and musicians: amplitude and frequency.

The size of the vibrations is called the \keyterm{amplitude}, and is represented by the symbol \m{A}.  Amplitude is the distance that the string moves to either side of the center, as shown in Figure~\in[fig:stringamp].  The total width of the strings’ motion is twice the amplitude, or \m{2A}.  For stringed instruments, the amplitude of typical vibrations is a few millimeters or less.  The vibration’s amplitude determines the sound’s volume.

\keyterm{Frequency} is a measure of how quickly the string’s motion is repeated.  Each repetition is called a \keyterm{cycle}, so frequency is measured in cycles per second.  The long, heavy strings in a grand piano vibrate with a frequency of a few dozen cycles per second, producing low pitches.  The shortest strings’ frequencies are thousands of cycles per second, producing high pitches.
The unit of measure for frequency is \keyterm{hertz}, abbreviated as Hz.

\startformula
1\units{Hz} = 1\units{cyc/s}
\stopformula
Humans can hear a huge range of frequencies, from about from \m{5\units{Hz}} up to \m{20{,}000\units{Hz}}.  The highest frequencies are only audible to young people.

Many details of an instrument’s construction affect its sound by influencing the way it vibrates. Galilei’s experiments taught us that properties of the instruments, like length and tension, may affect the frequency of vibration in surprising ways.
The frequency produced by a string or pipe is inversely proportional to the length, so a simple three-to-two ratio in lengths will produce a consonant two-to-three ratio of frequencies, as the Pythagoreans knew. However, the frequency produced by a string is proportional to the square root of the string’s tension. A simple two-to-three ratio of tensions produces a dissonant \m{\sqrt 2}-to-\m{\sqrt 3} ratio of frequencies. A four-to-nine ratio of tensions produces a consonant two-to-three ratio of frequencies. One of our goals in studying the physics of motion will be understanding these surprising relationships.

\placefigure[margin][fig:GrandPiano]{A grand piano has strings spanning several octaves, with long, low strings on the left and shorter, higher strings on the right.} {\externalfigure[chapter01/piano-top][width=\rightmarginwidth]}

The Pythagorean model was demonstrated – and challenged – using simple pipes and monochords.
Performance instruments are more complicated, playing more pitches than the pipes and producing a fuller sound than the monochord, but they are based on the same principles. Wind instruments use pipes cut to different lengths like an organ, stretched like a trombone or a slide whistle, or made different lengths by covering and uncovering holes like a flute.
String instruments have many strings like a piano (figure \in[fig:GrandPiano]) or a few strings that can be shortened with fingers like a violin. All of these instruments different lengths are used to produce different frequencies, just like the different lengths of Pythagoras and Philolaus’ pipes.

Since it is frequencies, not lengths, that need to be in simple ratios to get consonant sounds, we should look again and Pythagoras and Philolaus’ pipe demonstration. The pipes lengths and the the simple length ratios were shown in figure \in[fig:PythagoreanPipesCons]. Figure \in[fig:PythagoreanPipesArrows] shows the same pipes with frequencies from the Pythagorean tuning (even though the Pythagoreans did not know the frequencies at the time).
Arrows show how the frequencies are related, with the ratio given as frequency multiplier. The frequency where an arrow starts, multiplied by the frequency multiplier, is frequency where the arrow ends. For example, the first pipe’s frequency multiplied by \fourthirds\ is the second pipe’s frequency.
\startformula
198\units{Hz} \times \frac{4}{3} = 264\units{Hz}
\stopformula
The change in frequency represented by an arrow is called an \keyterm{interval.} The most important interval is an \keyterm{octave}, which is always a factor of 2. Going up an octave doubles the frequency. Going down an octave cuts the frequency in half. Three intervals in figure \in[fig:PythagoreanPipesArrows] are octaves. Two frequencies separated by an octave are so consonant that they are given the same name. There are two Cs in this set, with frequencies in the ratio one-to-two. There are three Gs with frequencies in the ratio 1:2:4.

\placefigure[margin][fig:PythagoreanPipesArrows]{The frequencies of Pythagoras and Philolaus’s pipes, with arrows showing the simple frequency multipliers relating consonant pitches. Arrows that are the same length show the same frequency multiplier.} {\externalfigure[chapter01/PythagoreanPipesArrows]}%[width=\rightmarginwidth]}

The grand piano spans over seven octaves, from the low A with a frequency \m{27.5\units{Hz}} to the high C at \m{4186\units{Hz}}.  (This A has nothing to do with the amplitude \m{A}.  Notice that the name of the pitch is upright, while the symbol for amplitude is italic.)
If the strings were identical thickness and tension, the lengths would double seven times going from the highest C to the lowest C – making the longest string 128 times longer than the shortest! For the very lowest pitches, the strings are thicker and looser, which also makes them lower, so they don’t have to be quite so long. Even so, the exponential increase in string length going down the keyboard is quite obvious in the concert grand piano in figure \in[fig:GrandPiano], which is nine feet long. Large organs have pipes that are even longer to produce their lowest pitches.

\placefigure[margin][fig:PythagoreanPipesOct]{One octave of Pythagoras and Philolaus’s pipes, with arrows showing the intervals within the octave. The largest interval is the octave  from the low C to the high C. The small, dissonant interval from F to G is a whole step.} {\externalfigure[chapter01/PythagoreanPipesOct]}%[width=\rightmarginwidth]}

As we investigate simple frequency ratios in music, it is sufficient to look at a single octave. Frequencies and intervals for one octave of Pythagoras and Philolaus’ pipes are shown in figure \in[fig:PythagoreanPipesOct]. Frequencies in each higher octave can be found by doubling the frequencies in the lower octave. Likewise, frequencies in each lower octave can be found by halving the frequencies in the higher octave. In this way every pitch will be consonant with the pitch an octave above or below.

%To calculate the lengths, we need these relationships in mathematical language.  Let us use \m{L\sub{low}} to represent length of a string that plays a particular low pitch, and \m{L\sub{high}} to represent length of a string that plays a pitch one octave higher.  We can then say the ratio of \m{L\sub{low}} to \m{L\sub{high}} is the ratio of two-to-one.  Written in an equation, the octave ratio is
%
%\startformula
%\frac{L\sub{low}}{L\sub{high}} \NC= \frac{2}{1}
%\stopformula
%
%Suppose you are building a pipe organ.  You decide the longest, lowest pipe will be eight feet long.  How long should you make the pipe that will produce a pitch one octave above the lowest pitch? Clearly, it should be four feet long.  The algebraic steps required to find this length are shown in Example~\in[ex:organ].
%
%	%%%%%%%%%%%%%%%%%%%%%%%%%%%%%%%%%%%%%%%%%%%%%
%\startexample[ex:organ]
%You wish to build a pipe organ. The largest pipe that you can use is \m{8\units{ft}} long. Find the length of pipe that will play one octave higher than the largest pipe.
%\startsolution
%Start by writing the relationship for the octave, and then use algebra to solve for the length of the higher pitched pipe, \m{L\sub{high}}.  It is good practice to solve with variables first and then to plug in numbers.
%\startformula
%\frac{L\sub{low}}{L\sub{high}} = \frac{2}{1}
%\stopformula
%Multiply both sides of the equation by \m{L\sub{high}} to get it out of the denominator.
%\startformula
%\cancel{L\sub{high}}\frac{L\sub{low}}{\cancel{L\sub{high}}} = \frac{2}{1}L\sub{high}
%\stopformula
%To isolate \m{L\sub{high}}, divide both sides of the equation by \m{2}.
%\blank%[small]
%\startformula
%\frac{L\sub{low}}{2} = \frac{\cancel{2}}{\cancel{2}}L\sub{high}
%\stopformula
%
%Having solved with variables, we insert eight feet for the length \m{L\sub{low}}.
%
%\startformula
%L\sub{high} = \frac{L\sub{low}}{2} = \frac{8\units{ft}}{2} = 4\units{ft}
%\stopformula
%
%The higher pipe should be made four feet long, as expected.
%\stopsolution
%\stopexample
%	%%%%%%%%%%%%%%%%%%%%%%%%%%%%%%%%%%%%%%%%%%%%%

%	%%%%%%%%%%%%%%%%%%%%%%%%%%%%%%%%%%%%%%%%%%%%%
\section{Building scales with consonance}
%	%%%%%%%%%%%%%%%%%%%%%%%%%%%%%%%%%%%%%%%%%%%%%

Most instruments are built to play many specific pitches determined by the lengths of pipes or strings, or by the placement of holes or frets. Ideally, many of these pitches can be combined in consonance, which means many of the frequencies produced by the instruments should be related by simple ratios. Finding such a set of pitches is a challenging mathematical problem. Many cultures have adopted different solutions, each giving their music a distinctive sound.
The ancient Greeks produced several solutions. We will look at just one of them, the one that produces pitches most like the modern C major scale. I will use modern notation and terminology even though the exact pitches are slightly different.

\placefigure[margin][fig:TinyPiano]{The Pythagorean solution for the C major scale, along with several representations modern representations of the scale.} {\externalfigure[chapter01/TinyPiano]}

The C major scale starts on C and ends an a higher C. Greeks’ procedure for selecting pitches between these two Cs naturally produces a set of eight pitches. We say this scale covers one octave because it contains these eight pitches.
Several representations of the C major scale are shown in figure \in[fig:TinyPiano] – a keyboard, notes on a staff, and two common naming conventions. I will use letter names. The other representations are there to aid anyone familiar with them. You do not need to learn them to understand this chapter or work the problems. If you have access to a keyboard or know how to play the notes on another instrument, you can produce some of the consonant and dissonant combinations yourself. However, since modern tuning is slightly different, you will not hear all of the effects. In this chapter we will look at three versions of the C major scale, including the modern tuning. By the end of the chapter you should  understand what you can do on your modern instrument. If you can alter the tuning of your instrument, then you can experiment with the ancient scales as well.

Above the keyboard in figure \in[fig:TinyPiano] are lines showing the relative lengths that will produce the pitches. These lengths could be strings or pipes. For example, they could be one octave’s worth of strings in a grand piano, as in figure \in[fig:GrandPiano]. The first length, producing the low C, and the last length, producing the high C, are in the ratio of two-to-one. Galilei taught us that it is not the ratios of lengths that matter, but the ratios of frequencies. We already have frequencies for both Cs, F, and G, and the intervals connecting them in figure \in[fig:PythagoreanPipesOct]. The exact frequencies are not as important as the ratios, so only the ratios (as frequency multipliers) are shown if figure \in[fig:TinyPiano]. The \m{9/8} ratio from F to G is taken directly from Pythagoras and Philolaus’ pipes.
This interval is commonly called a \keyterm{whole step.}  The whole steps are obvious on the keyboard because there is always a black key in the middle of a whole step. (The black keys are sharps and flats. There were many arguments about the exact ratios for the sharps and flats, which we will avoid.)

The Pythagoreans used the whole step to find the remaining four notes of the C major scale. They used it twice on the left, moving up from C to D and then D to E. (Notice the black keys in these whole steps.) This leaves a small, dissonant interval between E and F called a \keyterm{half step}. They also used the whole step twice twice on the right, moving up from G to D and A to B, again leaving a dissonant half step between B and C. The half steps were small enough that the Pythagoreans did not subdivide them in this scale, so there is no black key in the half steps.

\placefigure[margin][fig:TinyPiano8th5th]{An octave divided into a perfect fifth and a perfect fourth.} {\externalfigure[chapter01/TinyPiano8th5th][width=\rightmarginwidth]}

None of the larger intervals (\m{4/3}, \m{3/2}, and \m{2}) are shown in figure \in[fig:TinyPiano] – there are just too many. However, every interval can be found using the whole steps and half steps once we have mastered some musical math with intervals. Our first calculation will use the intervals shown in figure \in[fig:TinyPiano8th5th], taken directly from Pythagoras and Philolaus’ pipes.

The first bit of interval math is counting.
Intervals are named according to the number of pitches they encompass. An octave encompasses all eight pitches shown in figure \in[fig:TinyPiano8th5th]. The fifth encompasses C–D–E–F–G, five pitches. The fourth encompasses four. (A whole step is sometimes called a second, since it encompasses two adjacent pitches.)

The ratio relating the low C to G is \m{3/2}, so they are consonant pitches. Since the pitches are consonant, this interval is called a \keyterm{perfect fifth}. The perfect fifth always corresponds to the ratio \m{3/2}. The ratio relating the G and the high C is \m{4/3}. This interval is also consonant and is called a \keyterm{perfect fourth.} The perfect fourth always corresponds to the fraction \m{4/3}.

The next bit of musical math is some multiplication. Notice what happens to the frequencies going from low C to G to high C. From G to C the frequency is multiplied by \m{3/2}. From G to C the frequency is multiplied again, this time by \m{4/3}. Going up from C to G to C multiplies frequency by \m{3/2} and then by \m{4/3} which is \m{\threehalves \times \fourthirds = 2}, an octave! Add intervals by multiplying their ratios, as shown in figure \in[fig:TinyPiano8th5th]. The full calculation then looks like this:
\startformula
	\text{perfect 5th} + \text{perfect 4th} = \frac{3}{2} \times \frac{4}{3} = 2 = \text{octave}
\stopformula

It is somewhat surprising that a fifth plus a fourth is an octave, since \m{5 + 4 \neq 8}.
This brings us to the last bit of musical math, which is some addition. Start the fifth on the low C, as shown in figure \in[fig:TinyPiano8th5th], and write five pitches: C–D–E–F–G. Then start the fourth on G and write four pitches: G–A–B–C. Together, these encompass eight pitches, from C to C. The fifth encompasses five pitches, but it only takes four steps to get from the first pitch to the fifth. The fourth only requires three steps. The octave is seven steps. The number of steps is always one less than the number of pitches. The calculation to add intervals can therefore be done like this:
\startformula
	\text{5th} + \text{4th} = 4\units{steps} + 3\units{steps} = 7\units{steps} = \text{octave}
\stopformula
Subtract one from every interval to get the number of steps. Calculate with steps. Add one to get the final interval.

	%%%%%%%%%%%%%%%%%%%%%%%%%%%%%%%%%%%%%%%%%%%%%
\startexample[ex:organ]
What interval is produced by two consecutive thirds?
\startsolution
We wish to add a third to a third.
\startformula
	\text{3rd} + \text{3rd} = 2\units{steps} + 2\units{steps} = 4\units{steps} = \text{5th}
\stopformula
Two thirds produce a fifth.
\stopsolution
\stopexample
	%%%%%%%%%%%%%%%%%%%%%%%%%%%%%%%%%%%%%%%%%%%%%
Steps are not all the same size, so this method is only useful for getting the step count.
When calculating frequencies and ratios, always add intervals by multiplying their ratios.
This musical multiplication can be used to find any interval from the whole steps and half steps in figure \in[fig:TinyPiano].

	%%%%%%%%%%%%%%%%%%%%%%%%%%%%%%%%%%%%%%%%%%%%%
\startexample[ex:TinyPianoPyEx]
What is the interval from D to A in the Pythagorean system?
\placefigure[margin][fig:TinyPianoPyEx]{The Pythagorean scale with the interval sought in example \in[ex:TinyPianoPyEx].} {\externalfigure[chapter01/TinyPianoPyEx][width=\rightmarginwidth]}
\startsolution
The interval is shown along with the Pythagorean steps in figure \in[fig:TinyPianoPyEx]. The interval encompasses D–E–F–G–A, so it is a fifth. To determine if it is a perfect fifth, we must add the four steps by multiplying their ratios. Using figure \in[fig:TinyPianoPyEx], the step from D to E is \m{9/8}, from E to F is \m{256/243}, etc.
\startformula
	\frac{9}{8} \times \frac{256}{243} \times \frac{9}{8} \times \frac{9}{8}
		= \frac{3^2}{2^3} \times \frac{2^8}{3^5} \times \frac{3^2}{2^3} \times \frac{3^2}{2^3}
		= \frac{2^8 \cdot 3^6}{2^9 \cdot 3^5}
		= \frac{3}{2}
\stopformula
The interval from D to A is a perfect fifth.
\stopsolution
\stopexample
	%%%%%%%%%%%%%%%%%%%%%%%%%%%%%%%%%%%%%%%%%%%%%
This horrible looking half step is just what is needed to make a beautiful perfect fifth! (Prime factorization makes these calculations a breeze, so you may want to refresh your powers of two and three. Every number in the Pythagorean system is a power of two or three.)

%These three lengths are enough to produce a three pitch chord, C–G–C, which contains all three of the simplest ratios. The chord is consonant with a wide-open, clear sound.
%%Aaron Copland’s \booktitle{Fanfare for the Common Man} opens with a soaring sequence.
%The fifth and fourth can trade places, still making an octave as in figure \in[fig:TinyPiano4thWhole]. This gives us the F length, which is ¾. The chord C–F–C also has an open sound, but it is darker than the C–G–C chord. Both chords contain the same three simple ratios, but the different arrangement alters the sound.

%\placefigure[margin][fig:TinyPiano4thWhole]{Flipping the order of the perfect fourth and perfect fifth gives a different division of the octave. A whole step is the difference between the perfect fourth and perfect fifth.} {\externalfigure[chapter01/TinyPiano4thWhole][width=\rightmarginwidth]}

The Pythagorean C major scale gives composers and musicians many perfect fifths and perfect fourths to enjoy. Perfect fourths can start on any pitch except F. The fourth encompassing F–G–A–B is three whole steps (three black keys), not two whole steps and one half step like the perfect fourths. The large F–B fourth is an augmented fourth and its sound is very far from perfect. Similarly, the only fifth that is not perfect is the B–F fifth. This fifth has two whole steps and two half steps, unlike the perfect fifths that have three whole steps and only one half step. The small B–F fifth is a diminished fifth. Like the augmented fourth, the diminished fifth is far from perfect.

While the Pythagorean scale has an abundance of consonant octaves, perfect fifths, and perfect fourths, every other interval is dissonant. Still, the Pythagoreans were happy with their scale. Every ratio that can be made from one, two, three, and four is in the Pythagorean system of octaves, perfect fifths and perfect fourths, so they believed their system was complete.

Not everyone was impressed with all of this calculating. Aristoxenus, in the fourth century \scaps{bc}, completely rejected the use of pure mathematical reasoning to determine consonant pitches, believing instead that the consonant pitches should be determined only by listening to whether they sound consonant. Music that sounds good, is good. His \booktitle{Harmonic Elements}, was first ridiculed and then largely ignored by music theorists, but his ideas continued to be influential, as we will see.

%\placefigure[margin][fig:TinyPiano3rds1Pt]{Ptolemy divided the perfect fifths into consonant major and minor thirds.} {\externalfigure[chapter01/TinyPiano3rds1Pt][width=\rightmarginwidth]}

Ptolemy, writing near the end of the classical period in the second century \scaps{bc}, took a more balanced view, seeking to satisfy both sense and reason. He was not committed to Pythagorean numerology, so he wished to use the simple ratios of four-to-five and five-to-six to provide more consonant intervals in the scale. His solution is shown in figure \in[fig:TinyPianoCompletePt].

\placefigure[margin][fig:TinyPianoCompletePt]{Potelomy’s C major scale with consonant thirds.} {\externalfigure[chapter01/TinyPianoCompletePt][width=\rightmarginwidth]}

Ptolemy made the whole steps from D to E and from G to A slightly smaller than the other Pythagorean whole steps. This makes the two half steps slightly larger. The result is consonant thirds! The third from C to E is
\startformula
	\frac{9}{8} \times \frac{10}{9} = \frac{5}{4}.
\stopformula
This consonant interval is called a \keyterm{major third.} The third from E to G is
\startformula
	\frac{16}{15} \times \frac{9}{8}
		= \frac{2^4}{3\cdot5} \times \frac{3^2}{2^3}
		= \frac{2^4 \cdot 3^2}{2^3 \cdot 3 \cdot 5}
		= \frac{2 \cdot 3}{5}
	= \frac{6}{5}.
\stopformula
This consonant interval is a \keyterm{minor third,} which is smaller than the major third.
These two consonant intervals were exactly the ones Ptolemy wished to add. Almost every third in the Ptolemaic scale is a consonant major or minor third. (You should find the exception.)

	%%%%%%%%%%%%%%%%%%%%%%%%%%%%%%%%%%%%%%%%%%%%%
\startexample[ex:TinyPianoPyEx]
What interval is the sum of a major third and a minor third?
\startsolution
Two thirds make a fifth. To determine if it is a perfect fifth, we must add the thirds by multiplying their ratios. \startformula
	\text{major 3rd} + \text{minor 3rd}
		= \frac{5}{4} \times \frac{6}{5}
		= \frac{3}{2}
		= \text{perfect fifth}
\stopformula
A major third and a minor third add to a make a perfect fifth.
\stopsolution
\stopexample
	%%%%%%%%%%%%%%%%%%%%%%%%%%%%%%%%%%%%%%%%%%%%%

%Triads\dots.

The introduction of consonant thirds does have some cost. While the Pythagorean scale had six perfect fifths and six perfect fourths, Ptolemy’s scale has only five of each. I will let you find the damaged fifth and fourth. These damaged intervals are called wolf intervals. If a composer puts one of these intervals in a piece, at the performance it will howl like a wolf! The wolf intervals can be avoided, and they are a small price to pay for the exciting selection of consonant thirds in the Ptolemaic scale.

Ptolemy presented this treasure chest of consonant intervals in \booktitle{Harmonics}. Unfortunately, \booktitle{Harmonics} was then lost as Europe descended into the the dark ages. Europeans spent the next twelve centuries subsisting on octaves, fifths, and fourths.

As Europe emerged into the Renaissance, musicians became more adventurous and began using thirds in their compositions. Music theorists reinvented the Ptolemaic scale, which became known as \keyterm{just intonation.} Then, part of the lost \booktitle{Harmonics} was found, adding Ptolemy’s authority to the system of just intonation. These developments were integrated into a sophisticated musical theory by the great Italian master Gioseffo Zarlino in \booktitle{Le Istitutioni harmoniche,} published in 1558.

\placefigure[margin][fig:TinyPianoCompleteEq]{The equal tempered tuning is built from twelfth roots two.} {\externalfigure[chapter01/TinyPianoCompleteEq][width=\rightmarginwidth]}

One of Zarlino’s most accomplished students was the musician Vincenzo Galilei, who we have already met. In addition to preforming the experiments with strings, Galilei proposed a radical alternative to just intonation.
Following a suggestion of Aristoxenus, Galilei divided the octave into twelve identical half steps. Galilei calculated that each half step would change the frequency by a factor of \m{\sqrt[12]{2}}. This method, called \keyterm{equal temperament,} is shown in figure \in[fig:TinyPianoCompleteEq]. All of the sharps and flats fit naturally into equal temperament.

	%%%%%%%%%%%%%%%%%%%%%%%%%%%%%%%%%%%%%%%%%%%%%
\startexample[ex:TinyPianoPyEx]
Find the ratio for the whole step in equal temperament.
\startsolution
Two half steps make a whole step. We add the half steps by multiplying their ratios.
\startformula
	\text{half step} + \text{half step}
		= \sqrt[12]{2} \times \sqrt[12]{2}
		= 2^{\onetwelfth} \times 2^{\onetwelfth}
		%= 2^{\twotwelfths}
		= 2^{\onesixth}
		= \sqrt[6]{2}
\stopformula
The equal tempered whole step is \m{\sqrt[6]{2}}.
\stopsolution
\stopexample
	%%%%%%%%%%%%%%%%%%%%%%%%%%%%%%%%%%%%%%%%%%%%%
	%%%%%%%%%%%%%%%%%%%%%%%%%%%%%%%%%%%%%%%%%%%%%
\startexample[ex:TinyPianoPyEx]
Find the ratio for the equal tempered perfect fifth.
\startsolution
The perfect fifth is three whole steps and one half step, which is a total of seven half steps.
\startformula
	7\;\text{half steps}
		= \left(\!\sqrt[12]{2}\right)^7
		= 2^{\seventwelfths}
		\approx 1.498
\stopformula
The equal tempered perfect fifth is \m{2^{\seventwelfths}}. This is not a simple fraction, but it is very close to \m{3/2}, the ratio of a just perfect fifth. The equal tempered perfect fifth sounds perfect to most listeners.
\stopsolution
\stopexample
	%%%%%%%%%%%%%%%%%%%%%%%%%%%%%%%%%%%%%%%%%%%%%

In equal temperament, the octave is the only interval produced by a simple ratio. In fact, no other interval is produced by any ratio of whole numbers – the ratios are all irrational roots of two. Yet equal tempered fifths, fourths, and thirds are nearly as consonant as the just intonation intervals. Better yet, all of the intervals can be started on any pitch. There are no wolves lurking in the equal tempered scale. This makes equal temperament extremely practical for composers and musicians. Galilei and his friends found and composed works demonstrating the practical benefits of equal temperament.

Many music purists were not impressed with equal temperament. The consonances in just intonation do sound better, but the difference is small. The obvious way to please everyone is to adjust the step size of equal temperament so the intervals perfectly match the simple ratios. Unfortunately, this is mathematically impossible. The mathematics of rational numbers proves that certain consonances will not quite line up. The mismatch can be concentrated in a few wolf intervals, which must be avoided in compositions, or the mismatch can be spread out among many intervals in the hope that they will be too small to notice. Just intonation takes the extreme of having the fewest, most howling wolves. Equal temperament takes the other extreme, spreading the mismatch equally throughout the scale so that it is everywhere a barely audible purr.

Theorists and musicians experimented with several other tunings which sought to preserve as many perfect consonances as possible while keeping the wolves from becoming too disruptive. In the end, equal temperament became the standard. Nearly all western instruments are tuned to the equal tempered scale.

%%%%%%%%%%%%%%%%%%%%%%%%%%%%%%%%%%%%%%%%%%%%%
\section{Equal temperament and frequency}
%%%%%%%%%%%%%%%%%%%%%%%%%%%%%%%%%%%%%%%%%%%%%
Consonance and dissonance, scales and tuning, intervals and musical math are all about ratios. The actual frequencies hardly matter. If Pythagoras and Philolaus’ pipes all had frequencies ten percent higher than the frequencies in figure \in[fig:PythagoreanPipesArrows] nothing about the ratios would change, and their demonstration of consonances and dissonances would work just as well. The same is true for any other instruments – as long as they are tuned in agreement when they play together.
Ensuring that instruments can play together requires choosing specific frequencies and building instruments to produce those specific frequencies.

In this section we will look more closely at how these frequencies are determined from the instruments pitch producing vibrations. Then we will calculate the specific frequencies of the modern equal tempered scale.

Audible frequencies are difficult to measure because the vibrations that produce audible sound are extremely fast. Even the lowest audible frequencies are many cycles per second. Marin Mersenne, in the generation after Galilei, was one of the first to carefully study the motions responsible for music. He determined the frequencies of audible pitches indirectly. First, he used a very long string vibrating at an inaudible \m{4\units{Hz}}, which was slow enough for him to measure. Then he increased the frequency by octaves until it was audible. Knowing that the frequency doubled with each octave allowed him to determine the audible frequency.

Over the years more direct mechanical and eventually electronic methods were developed. Today frequencies can be measured with tremendous accuracy due to the availability of high precision clocks and optical measurement techniques.

Whether the vibration is fast or slow, the most accurate way to find a vibration’s frequency is by timing a large number of cycles. The number of cycles is represented by \m{N}, and total time, or \keyterm{duration}, by \m{\Delta t}. The frequency is the number of cycles divided by the duration of those cycles.

\startformula
f = \frac{N}{\Delta t}
\stopformula

This will give the frequency in cycles per second, or hertz, the usual unit of frequency.

	%%%%%%%%%%%%%%%% EXAMPLE 1.3 %%%%%%%%%%%%%%%%
\startexample[ex:frequency]
	To determine the frequency of a violin string, you record the plucking of the string with a high-speed video camera.  Watching it in slow motion, you see the string plucked at \m{3.25\units{s}} into the video and count \m{55} cycles before ending at \m{3.50\units{s}}.  What is the frequency of the string’s vibration?

\startsolution
	The duration is \m{\Delta t = t\sf - t\si = 3.50\units{s}-3.25\units{s}= 0.25\units{s}}.  The frequency is:

	\startformula
		f = \frac{N}{\Delta t}
			= \frac{55\units{cyc}}{0.25\units{s}}
			= 220\units{cyc/s}
			= \answer{220\units{Hz}}.
	\stopformula
	Include cycles for the units of \m{N} to get the correct units in your answer!
\stopsolution
\stopexample
%%%%%%%%%%%%%%%%%%%%%%%%%%%%%%%%%%%%%%%%%%%%%

The greek letter \m{\Delta} (delta) represents difference or change.  In this case, \m{\Delta t} is the difference in time between when you start counting the cycles and when you stop.  For example, if you start timing when your timer says \m{15\units{s}} and stop when the timer says \m{45\units{s}}, then the duration is \m{\Delta t = 30\units{s}}.  The formula for duration is

\startformula
\Delta t = t\sf - t\si,
\stopformula

where \m{t\si} is the initial time and \m{t\sf} is the final time.  The subscripts \quote{i} and \quote{f} will be used frequently to represent initial and final values.  Note that final time comes first in the formula, and the initial time comes last.  This order, final minus initial, always gives a positive duration.

%\stopsection
%%%%%%%%%%%%%%%%%%%%%%%%%%%%%%%%%%
%\section{Frequency relations}
%%%%%%%%%%%%%%%%%%%%%%%%%%%%%%%%%%

The time required to complete one cycle is called the \keyterm{period}, represented by \m{T}.
The period can be related to the frequency using the frequency formula with the duration of one cycle being the period \m{T}.  This gives the extremely useful \keyterm{frequency relations}:

\startformula
	f = \frac{N}{\Delta t} = \frac{1\units{cyc}}{T}
\stopformula

Short periods correspond to high frequencies. Longer periods correspond to lower frequencies.

	%%%%%%%%%%%%%%%% EXAMPLE 1.6 %%%%%%%%%%%%%%%%
\startexample[ex:period]
	A piano string vibrates with a frequency of \m{220.0\units{Hz}}.  What is this vibration’s period?
\startsolution
	Use the frequency relations, crossing out the part we don’t need.

	\startformula
		f = \cancel{\frac{N}{\Delta t}} = \frac{1\units{cyc}}{T}
	\stopformula

	To solve for the period, multiply both sides by \m{T\!/\!f}.

	\startformula \startmathalignment
	\NC \frac{T}{\cancel{f}}\cancel{f} \NC= \frac{1\units{cyc}}{\cancel{T}}\frac{\cancel{T}}{f}	\NR
	\NC T \NC= \frac{1\units{cyc}}{f} \NR
	\stopmathalignment \stopformula

	Plug in values, and calculate.

	\startformula
		T = \frac{1\units{cyc}}{f}
			= \frac{1\units{cyc}}{220.0\units{Hz}}
			= \frac{1\units{\ucan{cyc}}}{220.0\units{\ucan{cyc}/s}}
			= \answer{4.545\units{ms}}
	\stopformula

	The string has a period of \m{4.45\sci{-3}\units{s}}.%, as shown in Figure~\in[fig:vibrationgraph].
\stopsolution
\stopexample
	%%%%%%%%%%%%%%%%%%%%%%%%%%%%%%%%%%%%%%%%%%%%%

The standard pitches of the modern equal tempered scale are based on a middle A with a frequency of \m{440\units{Hz}}. Higher As’ frequencies are found by doubling; lower As’ frequencies are found by halving. The different As are distinguished by subscripts. The middle A is A\m{_4}. Higher As have higher subscripts.

	%%%%%%%%%%%%%%%%%%%%%%%%%%%%%%%%%%%%%%%%%%%%%
\startexample[ex:FreqA5]
What is the frequency of A\m{_5}?
\startsolution
A\m{_5} is one octave above A\m{_4}. We will use \m{f_4} and \m{f_5} for the frequencies of A\m{_4} and A\m{_5}.
\startformula
	f_5 = 2 f_4
		= 2\cdot 440\units{Hz}
		= 880\units{Hz}
\stopformula
A\m{_5}\!’s frequency is \m{880\units{Hz}}.
\stopsolution
\stopexample
	%%%%%%%%%%%%%%%%%%%%%%%%%%%%%%%%%%%%%%%%%%%%%
	%%%%%%%%%%%%%%%%%%%%%%%%%%%%%%%%%%%%%%%%%%%%%
\startexample[ex:FreqA5]
What is the frequency of A\m{_0}, the lowest note on a grand piano?
\startsolution
A\m{_0} is four octaves below A\m{_4}.
\startformula
	f_0 = \frac{f_4}{2^4}
		= \frac{440\units{Hz}}{16}
		= 27.5\units{Hz}
\stopformula
A\m{_5}\!’s frequency is \m{880\units{Hz}}.
\stopsolution
\stopexample
	%%%%%%%%%%%%%%%%%%%%%%%%%%%%%%%%%%%%%%%%%%%%%

Frequencies of other notes can be found by using the ratios of equal temperament. The number subscripts are helpful for distinguishing the notes in different octaves. Each numbered octave starts with C, not A, so the sequence of notes with subscripts is

\startblockquote
\dots B\m{_3}\quad C\m{_4} D\m{_4} E\m{_4} F\m{_4} G\m{_4} A\m{_4} B\m{_4}\quad C\m{_5} D\m{_5} E\m{_5} F\m{_5} G\m{_5} A\m{_5} B\m{_5}\quad C\m{_6}\dots
\stopblockquote

The notes on a grand piano range from A\m{_0} to C\m{_8}.

	%%%%%%%%%%%%%%%%%%%%%%%%%%%%%%%%%%%%%%%%%%%%%
\startexample[ex:FreqC4]
What is the frequency of C\m{_4}, also known as middle C?
\startsolution
C\m{_4} is nine half steps below A\m{_4}. In equal temperament, moving down a half step divides the frequency by \m{\sqrt[12]{2}}.
\startformula
	f_C = \frac{f_A}{\left(\!\sqrt[12]{2}\,\right)^9}
		= \frac{440\units{Hz}}{1.6818}
		= 261.63\units{Hz}
\stopformula
Middle C’s frequency is \m{261.63\units{Hz}}.
\stopsolution
\stopexample
	%%%%%%%%%%%%%%%%%%%%%%%%%%%%%%%%%%%%%%%%%%%%%
	%%%%%%%%%%%%%%%%%%%%%%%%%%%%%%%%%%%%%%%%%%%%%
\startexample[ex:FreqC8]
What is the frequency of C\m{_8}, the highest note on a grand piano?
\startsolution
C\m{_8} is three octaves and three half steps above A\m{_4}.
\startformula
	f_C = 2^3\left(\!{\sqrt[12]{2}\,}\right)^3\!\! f_A
		= 8 \times 1.1892 \times 440\units{Hz}
		= 4186.0\units{Hz}
\stopformula
C\m{_8}\!’s frequency is \m{4186.0\units{Hz}}.
\stopsolution
\stopexample
	%%%%%%%%%%%%%%%%%%%%%%%%%%%%%%%%%%%%%%%%%%%%%

The success of equal temperament eliminated the Pythagorean simple ratios from music, yet it also represents a great triumph of for the Pythagorean’s mathematical model. The equal tempered tuning succeeds because it produces frequency ratios that are very close to Pythagoras and Ptolemy’s simple ratios. Table \in[T:Intervals] shows the ratios as decimal numbers for just intonation and equal temperament. The worst match is the minor third, which is off by less than one percent. The perfect fifth is off by approximately \m{0.1\%}. These tiny differences are hardly noticeable.

\placetable[margin][T:Intervals] % Label
    {Consonant intervals of Just intonation and equal temperament.} % Caption
    {\vskip9pt\small\starttabulate[|l|c|c|]
\FL[2]%\toprule
\NC Interval		\NC Just						\NC Equal							\NR
\HL
\NC Octave		\NC \m{2}						\NC \m{2}							\NR
\NC Perfect 5th		\NC \m{3/2 = 1.5\phantom{3\dots}}	\NC \m{2^{\seventwelfths} \approx 1.498}	\NR
\NC Perfect 4th		\NC \m{4/3 = 1.33\dots}			\NC \m{2^{\fivetwelfths} \approx 1.335}	\NR
\NC Major 3rd		\NC \m{5/4 = 1.25\phantom{\dots}}	\NC \m{2^{\onethird}\, \approx 1.260}		\NR
\NC Minor 3rd		\NC \m{6/5 = 1.2\phantom{3\dots}}	\NC \m{2^{\onequarter}\, \approx 1.189}	\NR
\LL[2]%\bottomrule
\stoptabulate}

Equal temperament divides the octave into twelve equal parts – rather than say eight or thirteen – because twelve comes so close to producing the simple ratios. Imagine a friend tells you they are building a piano with eight equally spaced pitches per octave. This seems like a natural thing to do to an octave, but you could calculate the intervals to see if any are close to the important simple ratios (\m{3/2}, \m{4/3}, \m{5/4}, and  \m{6/5}). You would discover that this piano will not to produce any intervals close to \m{3/2}, \m{4/3}, or \m{5/4}. The sound of this piano is going to be extremely jarring!

Twelve is not the only number that works. Five is not bad, and is called a pentatonic scale. Nineteen actually works quite well and instruments using an equal tempered nineteen step scale were built and used in performance. One appears in Zarlino’s book (figure \in[fig:Zarlino19detail]). Twelve produces the best perfect fifth of any of these, and has never faced a serious threat from other equal tempered divisions of the octave.

\placefigure[margin][fig:Zarlino19detail]{A harpsichord with nineteen keys per octave, shown in Gioseffo Zarlino’s \booktitle{Le Istitutioni Harmoniche} (1558).} {\externalfigure[chapter01/Zarlino19detail][width=\rightmarginwidth]}

	%%%%%%%%%%%%%%%%%%%%%%%%%%%%%%%%%%%%%%%%%%%%%
\section{Ancient Greeks and music in the modern age}
	%%%%%%%%%%%%%%%%%%%%%%%%%%%%%%%%%%%%%%%%%%%%%

The Pythagoreans produced one of the most successful mathematical models of all time. It was expanded by Ptolemy and clarified in response to Galilei, but the basic connection between the mathematical abstraction of simple ratios and the perceived, physical reality of consonant pitches remains unchanged. However, none of this tells us \emph{why} the model is successful. What makes simple frequency ratios sound so good? We do not have a \emph{theory} of consonance which would explain the importance of simple ratios. This is the deepest and most beautiful mystery we will solve in this book. Galilei points us in the right direction: the solution is in the physics of motion.

Galilei offered one more challenge to his teacher, Zarlino, and the musical tradition built on simple frequency ratios. The abundance of consonances in just intonation allowed composers to wright elaborate music with many voices singing many different pitches simultaneously. A skilled composer, writing for a professional choir or orchestra, could have these voices changing pitches at different times, some going up then others going down, creating tension through controlled dissonance and resolution through converging consonance. Zarlino’s music theories focused these grand polyphonic works.

While Galilei appreciated the intellectual beauty of polyphony, his study of ancient Greek music, which was predominately monophonic, left him wondering if something emotional was lost in the multitude of simultaneous sounds celebrated by his contemporaries. Galileo and his friends proposed a return to the tradition of Greek theatrical music, characterized by expressive melodies with intelligible words, embellished to engage and move the audience. They called their new form \quotation{sounding together,} It was a huge success.

%%%%%%%%%%%%%%%%%%%%%%%%%%%%%%%%%
\section{Celestial objects do not circle Earth}
%%%%%%%%%%%%%%%%%%%%%%%%%%%%%%%%%

Before we investigate the Pythagorean’s mathematical model of the cosmos, there is one thing that I want you to remember: this model is terrible! Celestial objects do not circle Earth. Only the moon goes around Earth, and none of the objects circle, they travel along ellipses. Some of these ellipses are close to circular, so a model made from circles could be a rough starting point for understanding the solar system. A model in which all of the objects go around Earth is irredeemable. This became clear almost immediately to anyone making regular observations of the planets.

% textwidth figure
\placetextfloat[top][fig:Ptolemaicsystem]{The geocentric model from Peter Apian’s \booktitle{Cosmographia,} 1524. Earth sits motionless in the center while the heavens spin around it. The outermost sphere, labeled \emph{Primu Mobile} or \quotation{primary mover,} spins most quickly, dragging the inner spheres along at slightly slower rates. This diagram, like others in the coming pages, uses common symbols for the visible planets. Table \in[T:Astrosym] shows these symbols, as well as the symbols of two more recently discovered planets.} {\externalfigure[chapter01/ptolemaicsystem][width=\textwidth]}

\define[2]\Astro{\NC #2 \NC #1 } % starfont

\placetable
    [margin,here]
    [T:Astrosym]
    {Astronomical Symbols for some important objects in our solar system. Most of these symbols can be found in Figure \in[fig:Ptolemaicsystem] above. The two outer planets, Uranus and Neptune, were discovered in 1781 and 1846, respectively.}
{\vskip330pt\noindent
\starttabulate[|l|c|c|l|c|]
\FL[2]
\Astro{\Sun}{Sun}\NC
\Astro{\Mars}{Mars}\NR
\Astro{\Mercury}{Mercury}\NC
\Astro{\Jupiter}{Jupiter}\NR
\Astro{\Venus}{Venus} \NC
\Astro{\Saturn}{Saturn}\NR
\Astro{\Earth}{Earth} \NC
\Astro{\Uranus}{Uranus}\NR
\Astro{\Moon}{Moon} \NC
\Astro{\Neptune}{Neptune}\NR
\LL[2]
\stoptabulate}
In the Pythagorean model (figure \in[fig:Ptolemaicsystem]) the sun, moon, planets, and stars all orbit the central, motionless Earth. This is a \keyterm{geocentric model}. (\quotation{Geo-} means Earth, so geocentric is \quotation{Earth centered.})
The sun completes one revolution around Earth every day. The moon, which is closer to Earth takes slightly more than a day to complete a revolution, about twenty-four hours and fifty minutes. The stars, which are farther away, complete a revolution in slightly less time than the sun, about twenty-three hours and fifty-six minutes. This pattern of longer revolutions for closer objects and shorter revolutions for father objects holds for all of the planets in the geocentric models. However, it only applies to their average revolution times. All of these objects have complicated rhythms of speeding up and slow down.

The geocentric model, the Sun, Moon, and stars all behave reasonably, traveling at nearly constant rates rates from east to west while moving somewhat north and south with the seasons. The planets, in contrast, have the most erratic movements, speeding up and slowing down, sometimes falling behind but at times moving faster than the stars. Venus and Mercury seem to be tethered to the Sun, sometimes falling behind but then aways overtaking the Sun, leading for a while, and falling behind again.
In addition, the planet get significantly brighter and then dimmer, as if they are approaching Earth and then receding away.

These motions inspired ancient Greeks to try several completely different models of the world (referring to Earth and the visible heavens). The most obvious modification is to put Mercury and Venus in orbit around the Sun, explaining why they never stray far from the Sun, and also why they move closer and then farther from Earth. Another model of the world puts Mars, Jupiter and Saturn in large orbits around the Sun, explaining their erratic behavior.

% textwidth figure
\placetextfloat[top][fig:CopernicanSystem]{The heliocentric model in \booktitle{De Revolutionibus Orbium Coelestium,} 1543. In this model the distant stars are immobile. Planets orbit the sun, with inner planets orbiting more quickly than outer planets. The moon is the one exception, orbiting Earth (\emph{Terra} in the diagram) rather than the sun (\emph{Sol.}).} {\externalfigure[chapter01/Copernican_heliocentrism_theory_diagram][width=\textwidth]}

The most daring modification was proposed by Aristarchus in the third century \scaps{bc}. He allowed Earth to join the other planets orbiting the Sun, as shown in figure \in[fig:CopernicanSystem]. This is the \keyterm{heliocentric model}. (\quotation{Helio-} means Sun.) In this model Earth moves in two ways. It rotates around its own axis, completing one rotation every day, and it orbits the sun, completing one revolution every year. All of the other known planets – Mercury, Venus, Mars, Jupiter, and Saturn – also orbit the sun. (Since all of these planets appeared as tiny dots in the sky, no one knew if they were rotating. We now know that they all rotate.)
In the heliocentric model, the moon does not orbit the sun like the other planets do; it orbits Earth, completing one revolution each lunar month (about \m{29} days).
The heliocentric model became widely known among astronomers, but it was not widely accepted. One reason was the moon’s exceptional behavior. Why would everything revolve around the central sun, except for the moon?

About this time, Eratosthenes measured Earth’s size –  an incredible achievement.
%Educated people of the time already knew that Earth was round
Eratosthenes had moved to Egypt to be librarian at the renowned Library of Alexandria, where the Nile flows into the Mediterranean Sea.
This put Eratosthenes in the perfect location to make this measurement.
Egypt is close enough to the equator for the sun to pass almost directly overhead in the summer.
Eratosthenes learned there was a well in southern Egypt, in the town of Syene, where the noon sun was \emph{exactly} overhead on the summer solstice, so that someone looking into the well would see the sun’s reflection in the water far below.
Farther north, in Alexandria, the sun was not quite directly overhead due to the Earth’s curvature. At noon on the summer solstice, Eratosthenes measured the shadow cast by a vertical post in Alexandria and found the sun was one fiftieth of a circle (about seven degrees) away from vertical. This meant  the distance from the Syene to Alexandria was one fiftieth of Earth’s circumference, as shown in Figure \in[fig:EarthRadius].

\placefigure[margin][fig:EarthRadius]{Eratosthenes determined Earth’s size. The well’s depth and the post’s height are greatly exaggerated.}{\hbox{\starttikzpicture[thick,domain=-90:90]
\clip (-1.25,-.5) rectangle (3.75,17);
\foreach \x in {-1,-.5,...,3.5}
	\draw[black!50,postaction={decorate}, decoration={markings,% switch on markings
mark=between positions 1cm and 3cm step 1.55cm with {\arrow{stealth}}}] (\x,17)--(\x,13.96);
\path (1.25,16.75) node[below,fill=white,inner sep=2pt]{Parallel rays from the sun}; % Rays label
\draw[thick, decoration={random steps, segment length=1mm, amplitude=.2mm},fill=black!10] (.1,13.96) -- (.1,14.96) % Well
	decorate{arc [start angle=89.62, end angle = -269.45, radius=14.96cm]} % Earth’s surface
	-- ( -.1,13.96) -- cycle;
\draw[->] (0,0)--(82.8:14.93) node[fill=black!10, inner sep=2pt, sloped, pos=0.98, left]{Post in Alexandria} ; % Angled radius
\draw[->] (0,0) -- node[fill=black!10, inner sep=2pt, sloped, pos=0.98, left]{Well in Syene} (0,13.94); % Vertical radius
\draw (0,4) arc [start angle=90, end angle = 82.8, radius=4cm]; % Earth’s angle
\draw (2,15.13) arc [start angle=270, end angle = 262.8, radius=.7cm]; % Post’s angle
\draw[<->,postaction={decorate, decoration={markings,
mark=at position .5 with {\node[above=2pt, fill=white, inner sep=2pt, transform shape]{5000 stadia};}}}] (0,15.11) arc [start angle=90, end angle = 82.8, radius=15.11cm]; % distance
\draw[very thick] (82.8:14.96) -- (82.8:15.96) ; % Post
\draw[<->,rounded corners] (2.03,15.13) -| (2.33,7) node[fill=black!10,inner sep=2pt]{\m{\tfrac{1}{50}} of a circle} |- (.52,3.972);
\draw[->,postaction={decorate, decoration={markings,
mark=at position .5 with {\node[above=2pt, fill=white, inner sep=2pt, transform shape]{North};}}}] (81:15.46) arc [start angle=81, end angle = 77.4, radius=15.46cm]; % North
\fill (0,0) circle[radius=1pt] node[right]{Earth’s Center}; % Vertical radius
	\stoptikzpicture}
}

Measuring the angle was the easy part. Measuring the distance from Alexandria to the well was more difficult.
Eratosthenes may have measure the distance by making the trip himself, or he may have used distance measurements made by others.
Either way, he found the distance to be about \m{5{,}000} stadia, or about \m{900\units{km}}. Multiplying by \m{50}, he estimated Earth’s circumference to be about about \m{250{,}000} stadia, or about \m{45{,}000\units{km}}.
Earth’s circumference is now known to be \m{40{,}000\units{km}} (or \m{4.00\sci{7}\units{m}}). Eratosthenes’s estimate is about \m{12\%} larger than the correct value, impressively accurate considering the difficulty of the measurement.
%If you measure the circumference around Earth’s equator, you get the 40,075 km figure I mentioned, But if you measure it from pole to pole, you get 40,007 km. [http://www.universetoday.com/26461/circumference-of-the-earth/]

Astronomical measurements improved significantly over the next few centuries, providing accurate distances for the Sun and Moon, and good estimates for the planets. Unfortunately, none of the models provided accurate predictions for the motions of the planets. Even in the heliocentric model planets strayed significantly from their proposed circular paths about the Sun. Accounting for these wanderings added tremendous complexity to the models.

The strongest argument for a motionless Earth, promoted by Aristotle, is that Earth does not \emph{seem} to be moving. Aristotle made several additional arguments based on everyday experience that supported the common belief that the Earth stands still. With all of the chaos in the skies, it seemed reasonable to trust the firm ground beneath our feet.

When Ptolemy attacked this problem, armed with the precise astronomical observations of the Babylonians, he produced a geocentric model with dozens of complicated, invisible wheels that moved the planets though their complex motions. Unlike his elegant system of intonation, his model of the world is insanely complex and arbitrary. He explained all of the details in the \booktitle{Almagest,} which was not lost. His geocentric model, the \keyterm{Ptolemaic system,} became the accepted model of the world through the middle ages.


During the Renaissance, when musicians were experimenting with their daring thirds, the Polish astronomer Nicholas Copernicus produced his own daring model of the world. Copernicus used the same Babylonian data as Ptolemy – it was still the best data in existance – to produce a heliocentric model. His 1543 book, \booktitle{On the Revolutions of Heavenly Bodies}, which is the source of figure \in[fig:CopernicanSystem], has all of the details – so many details! His model, like Ptolemy’s, was also insanely complex, requiring dozens of spheres to account for all of the planet’s motions. His heliocentric model is often called the \keyterm{Copernican system}.

Given the complexity of the Copernican system, Aristotle’s arguments for a motionless Earth remained persuasive. We will return to those arguments in Chapter~\in[ch:Motion]. Finding the correct model of the world would require two things: a willingness to abandon failed ideas, and new observations.

%%%%%%%%%%%%%%%%%%%%%%%%%%%%%%%%%%%%%%%%%%%%%
\startsection[title=Galileo’s spyglass]
%%%%%%%%%%%%%%%%%%%%%%%%%%%%%%%%%%%%%%%%%%%%%

While Vincenzo Galilei was making waves in the music world, his equally confrontational son, Galileo Galilei\index{Galileo}, became a professor of mathematics, first at the University of Pisa (1589--1592) and then at the University of Padua (1592--1610).
Galileo understood both the geocentric and heliocentric models and taught the geocentric model to students. His primary interest was not astronomy, but rather motion, fluids, and structures. He wrote several papers on these subjects for his students, but the papers were not published or distributed widely.

Then, in 1609, he heard of a device which seized his attention completely. He tells the story of this device and the discoveries that followed in his 1610 booklet, \booktitle{The Sidereal Messenger}. (Sidereal means \quotation{sounding together,})
\startblockquote
	About ten months ago a report reached my ears that a Dutchman had constructed a spyglass, by the aid of which visible objects, although at a great distance from the eye of the observer, were seen distinctly as if near....\autocite{p.~49}{Galileo1610}
\stopblockquote
Amazed by this report, he set out to understand and then construct his own spyglass, which we now call a telescope. He describes how he first built a telescope capable of magnifying objects by a factor of three, then eight, and finally, \quotation{sparing neither labor nor expense,} he constructed a telescope that magnified objects thirty times.
\startblockquote
	It would be altogether a waste of time to enumerate the number and importance of the benefits which this instrument may be expected to confer when used by land or sea. But without paying attention to its use for terrestrial objects, I betook myself to observations of the heavenly bodies.\autocite{p.~49--50}{Galileo1610}
\stopblockquote

Galileo shares his observations with delight and wonder. He presents his observations of the moon in careful drawings, Figure \in[fig:MoonSurface], that reveal its surface in greater detail than anyone had seen before.
% Margin image
\placefigure[margin][fig:MoonSurface]{The moon’s hills and valleys are clearly visible in Galileo’s original watercolors.} {\externalfigure[chapter01/MoonPhases6][width=\rightmarginwidth]}
\startblockquote
	...anyone may know with the certainty that is due to the use of our senses that the moon certainly does not possess a smooth and polished surface, but one rough and uneven, and, just like the face of the earth itself, it is everywhere full of vast protuberances, deep chasms, and sinuosities.\autocite{p.~48}{Galileo1610}
\stopblockquote
%Just as Vincenzio Galilei’s had overturned Aristotle’s understanding of a string’s vibration,
This observation challenged Aristotle’s view that objects in the heavens were perfectly smooth and round, made of an immutable substance unlike anything on Earth.

% Margin image
\placefigure[margin][fig:Orion]{Galileo writes, \quotation{I had determined to depict the entire constellation of Orion, but I was overwhelmed by the vast quantity of stars.... For this reason I have selected the three stars in Orion’s Belt and the six in his Sword, which have long been well known groups, and I have added eighty other stars recently discovered in their vicinity.... The well-known or old stars, for the sake of distinction, I have depicted of larger size....}\autocite{p.~65}{Galileo1610}} {\externalfigure[chapter01/orion][width=\rightmarginwidth]}

Galileo next shared his amazement when the telescope,
	\quotation{set distinctly before the eyes other stars in myriads, which have never been seen before, and which surpass the old, previously known, stars in number more than ten times.}\autocite{p.~48}{Galileo1610}\ He drew a small sample, Figure \in[fig:Orion], to show this how numerous the newly revealed stars are compared to those previously known.
This multitude of stars did not challenge any widely held view about the heavens, but it certainly demonstrated the telescope’s power.

Finally, Galileo relates his most amazing discovery.
\startblockquote
	There remains the matter that seems to me to deserve to be considered the most important in this work. That is, I should disclose and publish to the world the occasion of discovering and observing four planets never seen from the beginning of the world up to our own times, their positions, and the observations made during the last two months about their movements and their changes of magnitude. And I summon all astronomers to apply themselves to examine and determine their periodic times, which it has not been permitted me to achieve up to this day owing to the restriction of my time.\autocite{p.~67--68}{Galileo1610}
\stopblockquote

Galileo discovered these planets in the vicinity of Jupiter, observing three of them for the first time on January 7, 1610. Over the next few nights he observed them moving quickly around Jupiter, so that on some nights they appeared just to the east of Jupiter, and on other nights just to the west.
These observations are reproduced in Figure \in[fig:JupiterMoons].
%He reports that on January 11 the explanation for their motion became clear.
\startblockquote
	I therefore concluded, and decided unhesitatingly, that there were three stars in the heavens moving around Jupiter, like Venus and Mercury around the sun. This was finally established as clear as daylight by numerous other subsequent observations. These observations also established that there are not only three, but four, wandering sidereal bodies performing their revolutions around Jupiter.\autocite{p.~69}{Galileo1610}
\stopblockquote
The four celestial objects that Galileo discovered, which he referred to as \quotation{planets,} \quotation{stars} and \quotation{wandering sidereal bodies} in the passages above, are now called moons of Jupiter, or Jovian moons. Today the term \keyterm{star} is reserved for large, hot objects like the sun that produce their own light. The term \keyterm{planet} is reserved for objects orbiting the sun or another star. \emph{Moon}\index{moon} refers to an object orbiting a planet. Jupiter has many moons too small for Galileo  to have seen with his telescope.
Jupiter’s four large moons, often called the \keyterm{Galilean moons}, are huge, almost as large as the planet Mars.

% Margin image
\placefigure[margin][fig:JupiterMoons]{Galileo’s drawings of his first seven observations of Jupiter and its moons in January of 1610. He discovered four moons, but on many nights he saw fewer. Sometimes two of the moons were too close together to be seen separately, or moons were obscured by Jupiter. Clouds prevented observations on January 9 and 14.\autocite{p.~68--70}{Galileo1610}}
{\small
	\leftaligned{January 7, 1610}\\
	\externalfigure[chapter01/Jupiter1][width=\rightmarginwidth]\\
	\leftaligned{January 8}\\
	\externalfigure[chapter01/Jupiter2][width=\rightmarginwidth]\\
	\leftaligned{January 10}\\
	\externalfigure[chapter01/Jupiter3][width=\rightmarginwidth]\\
	\leftaligned{January 11}\\
	\externalfigure[chapter01/Jupiter4][width=\rightmarginwidth]\\
	\leftaligned{January 12}\\
	\externalfigure[chapter01/Jupiter5][width=\rightmarginwidth]\\
	\leftaligned{January 13}\\
	\externalfigure[chapter01/Jupiter6][width=\rightmarginwidth]\\
	\leftaligned{January 15}\\
	\externalfigure[chapter01/Jupiter7][width=\rightmarginwidth]
	\blank[small]
}

Galileo’s  discovery of Jupiter’s moons removed one of the primary objections to the Copernican system.
\startblockquote
	Additionally, we have a notable and splendid argument to remove the scruple of those who can tolerate the revolution of the planets around the sun in the Copernican system, but are so disturbed by the motion of one moon around the earth (while both accomplish an orbit of a year’s length around the sun) that they think this constitution of the universe must be rejected as impossible. For now we have not just one planet revolving around another while both traverse a vast orbit around the sun, but four planets which our sense of sight presents to us circling around Jupiter (like the moon around the earth) while the whole system travels over a mighty orbit around the sun in the period of twelve years.\autocite{p.~83--84}{Galileo1610}
\stopblockquote

%\section{Period of orbits and rotations}

When Galileo says that Jupiter completes an orbit \quotation{in the period of twelve years,} he is using \quotation{period} just as it was used for vibrations in Chapter \in[ch:Music]. Period\index{period} is still the time required for one \keyterm{cycle}.
In the case of the vibrating string, a cycle is one complete back-and-forth motion. In the case of an orbit, like Galileo describes, a cycle is one complete revolution of the orbiting object around the central object. In the case of rotation, a cycle is one complete rotation about the axis. The period of Earth’s rotation about its axis is \m{T=1\units{day}}. %In the case of an orbit, a cycle is one complete revolution about the orbit’s center.
The period of Earth’s revolution around the sun is \m{T=1\units{year}=365.24\units{days}}. %In all cases the period is the time required to complete a cycle.
Musical vibrations have periods of hundredths or thousandths of a second. Orbits can have periods of days, years, decades or longer. The same symbol \m{T} represents periods of any duration.

\section{Period relations}
The orbits of planets can be described by their frequencies rather than their periods. For example, Mercury orbits the sun with a period \m{T=88.0} days. The frequency can be found using the frequency relations from Chapter \in[ch:Music]. We could find the frequency in Hertz, but since the period is months long, let us find the frequency in cycles per year.
\startformula
	f = \frac{1\units{cyc}}{T}	%\\
		= \frac{1\units{cyc}}{88.0\ucan{d}}
			\left(\frac{365.24\ucan{d}}{1\units{yr}}\right)
		= 4.15\units{cyc/yr}
\stopformula
In practice, frequencies are rarely used to describe orbits and rotations because the periods are so long. However, there are some cases of extremely fast rotations and revolution where frequency is used. Some of the are explored in the problems at the end of the chapter.

%\question
%What is the frequency of Jupiter’s orbit around the sun?
%\begin{solution} Period is the reciprocal of frequency.
%	\begin{align*}
%		T &= \frac{1\units{cyc}}{f}	\\
%		f &= \frac{1\units{cyc}}{T}	%\\
%			= \frac{1\units{cyc}}{12\units{yr}}	%\\
%			%= \frac{1\units{\ucan{cyc}}}{440\units{\ucan{cyc}/s}}
%			= \answer{0.083\units{cyc/yr}}
%	\end{align*}
%Jupiter completes only 0.083 cycles of its orbit in one Earth year.
%We can convert this to Hertz.
%	\startformula
%		f = 0.083\frac{\units{cyc}}{\units{\ucan{yr}}}
%			\frac{1\units{\ucan{yr}}}{365\units{\ucan{days}}}
%			\frac{1\units{\ucan{day}}}{24\units{\ucan{hr}}}
%			\frac{1\units{\ucan{hr}}}{60\units{\ucan{min}}}
%			\frac{1\units{\ucan{min}}}{60\units{s}}
%		= \answer{2.6\sci{-9}\units{Hz}}
%	\stopformula
%\end{solution}

%Astronomers usually do not use frequency to describe orbits or rotations; they  use period. However, there are some exceptional events where frequency is useful. Two can be found in the problems at the end of this chapter.

%\section{Amplitude and radius}

Since period is the more common measure in astronomy, it is useful to solve the frequency relations for period, which gives the \keyterm{period relations}.
\startformula
	T = \frac{1\units{cyc}\cdot \Delta t}{N} = \frac{1\units{cyc}}{f}
\stopformula
As with frequency, the most accurate way to measure the period is to measure the duration on many cycles and then divide by the number of cycles, as shown in the first fraction in the period relations. The factor \m{1\units{cyc}} in the numerator is there to cancel the units of \m{N} (also cycles) so that the period is in seconds or years or any other time unit.

\booktitle{The Sidereal Messenger}, published in March of 1610, made Galileo a celebrity. In April, the grand duke of Tuscany, Cosimo II de' Medici, appointed Galileo to be his \quotation{Philosopher and Chief Mathematician.} Galileo quit his job at the University of Padua and moved to Florence where he refined his telescopes and studied the heavens in greater detail. He continued to write about his observations in letters and books.



%[Salviati helps Simplicio draw the solar system.]
\stopsection


%[Galileo uses the periods to make a stronger case for heliocentrism in the Dialogue. Larger orbits correspond to longer periods. [pp.206-7] Motionless stars are \quotation{so many suns.}[p.241] Note the 1:2:4 resonances of Jupiter’s inner three moons (octaves!). ]

\startsection[title={Solar system speeds}]

%[Galileo was not the only astronomer refining our model of the solar system. Tycho precision observations and Kepler sophisticated modeling.... A scale diagram of the solar system would be good here. Can I find a drawing of Kepler’s solar system? Alternatively, we could talk about the older precision data and with a teaser about better data which was not available to Galileo, but will be discussed in chapter 7.]

A circular orbit’s radius is measured from the orbit’s center of the orbit, just as amplitude is measured from the oscillating motion’s central point.
Planets and moons following circular orbits travel with a constant speed.
%\footnote{This footnote is only for savvy readers screaming \quotation{but the \emph{direction}!} Newton, in 1687, taught us that changes in the velocity’s direction are just as important as changes in the magnitude. In Newton’s view, the velocity \emph{is not} constant along a circular path because the direction changing. Lagrange responded, in 1790, that we can bend our coordinates to match our situation, and then consider velocities along the bent coordinates. After bending the coordinates to follow the circle, the velocity \emph{is} constant. Lagrange’s trick is both practical and powerful. We will take full advantage of it. The two methods are compared in Chapter \in[ch:Orbits].}
The speed can be calculated using \quotation{distance over duration.} During one cycle the distance is the circumference of the orbit, \m{2\pi R}, and the duration is the period \m{T}.
%\highlightbox{
\startformula[eq:vT]\pagereference[eq:vT]
	\text{speed} = \frac{2\pi R}{T}.
\stopformula
%}
%We have measured the displacement along the circular path.%, even though the planet returns to its original position after one orbit.

\startexample[ex:EarthSurfaceSpeed]
Earth completes one full rotation every day, so a person near the equator will travel Earth’s circumference each day. The tower in Florence is almost halfway between the equator and the North Pole, so each day the tower completes a somewhat smaller circle, with a radius of \m{4600\units{km}}. What is the speed of tower due to Earth’s rotation?
\startsolution
The formula for the speed of circular motion will provide the answer, with the help of several unit conversions.
\startformula\startmathalignment
	\NC \text{speed}	\NC = \frac{2\pi R}{T}		\NR
	\NC				\NC = \frac{2\pi\cdot4600\ucan{km}}{24\units{\ucan{hr}}}
				\left(\frac{1000\units{m}}{1\units{\ucan{km}}}\right)
				%\frac{1\units{\ucan{day}}}{24\units{\ucan{hr}}}
				\left(\frac{1\units{\ucan{hr}}}{60\units{\ucan{min}}}\right)
				\left(\frac{1\units{\ucan{min}}}{60\units{s}}\right)	\NR
	\NC				\NC = \answer{330\units{m/s}}
				%\quad \text{or}\quad
				%\answer{460\units{m/s}}
\stopmathalignment\stopformula
	The tower’s speed is \m{330\units{m/s}} in the direction of Earth’s rotation (west to east). That is fast!
\stopsolution
\stopexample

%\question
%What is Earth’s speed in its orbit about the sun?
%\begin{solution} Earth’s distance from the sun is \m{1.50\sci{11}\units{m}}.
%	\begin{align*}
%		\abs{v} &= \frac{2\pi R}{T}		\\
%			&= \frac{2\pi\cdot1.50\sci{11}\units{m}}{365\units{\ucan{days}}}
%				\left(\frac{1\units{\ucan{day}}}{24\units{\ucan{hr}}}\right)
%				\left(\frac{1\units{\ucan{hr}}}{60\units{\ucan{min}}}\right)
%				\left(\frac{1\units{\ucan{min}}}{60\units{s}}\right)	\\
%			&= \answer{3.0\sci{4}\units{m/s}}
%				\quad \text{or}\quad
%				\answer{30.\units{km/s}}
%	\end{align*}
%	\m{30.\units{km/s}} is very fast!
%\end{solution}




%The previous example required a rather tedious unit conversion from years to seconds. We can simplify this step by finding the number of seconds in a year.
%\startformula
%	1\units{yr} = 1\units{\ucan{yr}}
%				\frac{365\units{\ucan{days}}}{1\units{\ucan{yr}}}
%				\frac{24\units{\ucan{hr}}}{1\units{\ucan{day}}}
%				\frac{60\units{\ucan{min}}}{1\units{\ucan{hr}}}
%				\frac{60\units{s}}{1\units{\ucan{min}}}
%				= 3.15\sci{7}
%				\approx \pi\sci{7}\units{s}
%\stopformula
%The approximation at the end is a lucky coincidence. Since it is off by less than one-half of a percent, we can use it in calculations with three significant figures or less. Using this approximation, the solution to the last example problem is simple enough to do without a calculator.
%	\begin{align*}
%		v &= \frac{2\pi R}{T}		\\
%			&= \frac{2\cancel{\pi}\cdot1.50{\sci{\cancel{11}}}^{\,4}\units{m}}{1\units{\ucan{yr}}}
%				\frac{1\units{\ucan{yr}}}{\cancel{\pi}\times\cancel{10^{7}}\units{s}}	\\
%			&= \answer{3.00\sci{4}\units{m/s}}
%				\quad \text{or}\quad
%				\answer{30.0\units{km/s}}
%	\end{align*}
%Since the last significant digit is always uncertain, this answer matches the one we found before. Using the approximation doesn’t always work so nicely, but it always helps. Just remember that it cannot be used in problems requiring four or more significant figures.

In 1609, one year before Galileo’s \booktitle{Sidereal Messenger}, Johannes Kepler announced that the planets’ orbits are not described by circles – they are actually slightly elliptical. Ellipses are simple shapes, but they did not fit well into the mechanical models of spheres and wheels that drove both the Ptolemaic and Copernican world systems. Kepler found a mathematical model for the world’s motions, but the forces that could drive these off-center, elliptical motions were mysterious. The perspective was shifting, from a mechanical model to a new physics of motion. Galileo, who did not take much notice of Kepler’s announcement, did understand motion, which he had been studying before his sudden interest in astronomy. He used the discoveries from his lab to understand the world he discovered through his spyglass.

%[Scale map of the solar system for Kepler and speeds.]

%[Speeds make the transition to Chapter 3! Galileo distinguishes between astronomical and terrestrial arguments, and it might be good to quote him since we are going back to him.]
\stopsection

\subject{Notes}
\blank
\startcolumns
%\placefootnotes[criterium=chapter]
\placenotes[endnote][criterium=chapter]

%\subject{Bibliography}
%        \placelistofpublications

\stopcolumns


\page
%%%%%%%%%%%%%%%% EXERCISES %%%%%%%%%%%%%%%%
\startsubject[title=Exercises]
%\setuplayout[
%	leftmargin=36pt, % 1/2 in
%	leftmargindistance=9pt, % 3/8 in
%	width=477pt, % 4 1/4 in
%	rightmargindistance=9pt, % 3/8 in
%	rightmargin=36pt,  % 2 in
%]
%\setupheadertexts[text][section][\pagenumber][\pagenumber][chapter]
%\setupheadertexts[margin][][][][]
%
%\blank
%\startcolumns[n=2, tolerance=verytolerant]
\startitemize[n,packed]
\question
Find the frequency of the pitch one octave above middle A (\m{440\units{Hz}}).  What is the period of this higher pitch? %(Here \quote{A} is the name of the pitch.  It has nothing to do with the vibration’s amplitude, \m{A}.)
\blank

\question
Find the frequency of the pitch one perfect fifth below middle A.
\blank
%\question
%What the frequency of the highest A on a grand piano? What subscript should be put on this A?

\question Starting with the duration formula and working only with variables\nowhitespace
\startitemize[a,joinedup]%,packed]
\item solve for \m{t_{\rm f}},
\item solve for \m{t_{\rm i}}.
\stopitemize
\blank

\question Starting with the frequency formula and working only with variables
\startitemize[a,packed,joinedup]
\item solve for \m{N},
\item solve for \m{\Delta t}.
\stopitemize
\blank

\question
How many cycles will the middle A (\m{440\units{Hz}}) string oscillate in \m{2.5\units{s}}?
\blank

\question
How long will it take for the  D string with a frequency of \m{293\units{Hz}} to oscillate one thousand times?
\blank
%\question
%What is the frequency of C\m{_0}? What is its period?
%\blank

\question Starting with the frequency relations and working only with variables
\startitemize[a,packed]
\item find \m{T} in terms of \m{f},
\item find \m{T} in terms of \m{N} and \m{\Delta t}.
\stopitemize
\blank[big]


%\question The moon completes one revolution around Earth in 29.3 days. Find the speed of the moon in its orbit around Earth.

\question The inner two moons of Jupiter, Io and Europa, are in a 1:2 resonance, which means that the ratio of Io’s period to Europa’s period is one-to-two. Europa’s orbital period is 3.55 days. What is Io’s orbital period?
%\begin{solution}[3in]
%\begin{align*}
%	\frac{T\sub{E}}{T\sub{I}} &= \frac{2}{1}	\\
%	T\sub{I} &= \half T\sub{E}	\\
%		&= \half 3.55\units{d}	\\
%		&= \answer{1.78\units{d}}
%\end{align*}
%\end{solution}

\question Jupiters second and third moons, Europa and Ganymede, are also in a 1:2 resonance. What is the Ganymede’s orbital period?
%\begin{solution}[3in]
%\begin{align*}
%	\frac{T\sub{E}}{T\sub{I}} &= \frac{2}{1}	\\
%	T\sub{I} &= \half T\sub{E}	\\
%		&= \half 3.55\units{d}	\\
%		&= \answer{1.78\units{d}}
%\end{align*}
%\end{solution}


\question Galileo published \textit{The Starry Messenger} in 1610. The Galileo spacecraft, which studied Jupiter and its moons up close, became the first craft to orbit Jupiter in 1995. How many orbits did Jupiter make between these two events? Jupiter orbits the sun once every 11.86 years.

\question The Galileo spacecraft was launched on October 18, 1989. Its successful mission as the first man-made satellite of Jupiter ended on September 21, 2003 with an intentional crash into Jupiter’s atmosphere. Find the length of the Galileo mission in Jovian years.

%\question Saturn orbits the sun once every 29.5 years. How many orbits has it made since Galileo published \textit{The Starry Messenger} in 1610?
%\begin{solution}[3in]
%	\startformula
%T = \frac{1\units{cyc}\cdot\Delta t}{N}
%\stopformula
%First, find \m{\Delta t}.
%\startformula
%	\Delta t = t\sf - t\si = 2017 - 1610 = 407\units{yr}
%\stopformula
%Then we can find the number of cycles (or orbits).
%	\begin{align*}
%		T &= \frac{1\units{cyc}\cdot\Delta t}{N}	\\
%		N &= \frac{1\units{cyc}\cdot\Delta t}{T}	\\
%			&= \frac{1\units{cyc}\cdot(407\units{\ucan{yr}})}{29.5\units{\ucan{yr}}}	\\
%			& = \answer{13.8\units{cyc}}
%				\quad\text{\emph{or}}\quad
%				\answer{13.8\units{orbits}}
%	\end{align*}
%\end{solution}

\question The Cassini spacecraft journey to Saturn began with its launch on October 15, 1997. Cassini orbited Saturn, studying the planet, its many moons, and its amazing rings until September 15, 2017. How long was Cassini’s mission in Saturn’s years? Saturn’s orbit takes 29.5 Earth years.

\question Earth’s Moon orbits Earth with a frequency of \m{13.4\units{cyc/yr}}. What is the moon’s orbital period in days?
%\begin{solution}[3in]
%\begin{align*}
%	T &= \frac{1\units{cyc}}{f}	\\
%		&= \frac{1\units{cyc}}{13.4\units{cyc/yr}}	\\
%		&= 0.0746\units{\ucan{yr}}\left(\frac{365\units{d}}	{1\units{yr}}\right)	\\
%		&= \answer{27.2\units{d}}
%\end{align*}
%Accept answer in years is question does not specify days (2017).
%\end{solution}


\question In 1967, Jocelyn Bell Burnell and Antony Hewish discovered a star that flashed brightly every \m{1.33\units{s}}. Eventually, pulsing stars like this became known as pulsars. Pulsars are rapidly spinning neutron stars which shine bright beams of light out of their magnetic poles. We see a flash every time one of the spinning pulsar’s beams points in our direction. What is this pulsar’s frequency?

\question In 1968 another pulsar was discovered in the nearby Crab Nebula with a period of only \m{33\units{ms}}. What is this pulsar’s frequency?

\question On August 17, 2017, an extremely sensitive device known as \scaps{bc} (Laser Interferometer
Gravitational-Wave Observatory) detected a tiny vibration of space-time. These vibrations were gravitational waves from a pair of orbiting neutron stars which revealed the stars' orbital frequency to be \m{20\units{Hz}}. What was the orbital period of these two stars? (This ferocious dance did not last long. Over the next thirty seconds the neutron stars spiraled into each other, producing an enormous explosion. The merged stars then collapsed to form a black hole. Luckily, this violent event happened very, very far from us.)

\stopitemize
%\stopcolumns
\stopsubject

\stopchapter
%\page
%
%\setuplayout[
%	leftmargin=36pt, % 1/2 in
%	leftmargindistance=9pt, % 3/8 in
%	width=306pt, % 4 1/4 in
%	rightmargindistance=27pt, % 3/8 in
%	rightmargin=144pt,  % 2 in
%]
%\setupheadertexts[text][section][][][chapter]
%\setupheadertexts[margin][][\hfill\pagenumber][\pagenumber \hfill][]

\stoptext
\stopcomponent
