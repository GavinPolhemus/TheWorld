% !TEX TS-program = ConTeXt Suite
% !TEX useOldSyncParser
\startcomponent c_chapter07
\project project_world
\product prd_volume01

\setupsynctex[state=start,method=max] % "method=max" or "min"

%%%%%%%%%%%%%%%%%%%%%%%%%%%%%
\startchapter[title=Rotation, reference=ch:Rotation]
%%%%%%%%%%%%%%%%%%%%%%%%%%%%%

\placeinitial{T}\scaps{his chapter} only deals with orbital angular momentum in 2D and orbits.


\section{Reduced mass}
[It is possible to find the effective mass for motion where not every part is moving the same speed. For an angular coordinate this is the moment of inertia. For a radial coordinate between to movable masses this is the reduced mass. More generally every mass contributes
\startformula
	\mu_i = m_i\left\lvert\frac{d\vec{x}}{d\xi}\right\rvert^2
\stopformula
to the generalized reduced mass \m{\mu}.]



\section{The Vector Cross Product}
%\subsection{Right Hand Rule}
\section{Kinds of Angular Momentum}
\section{Angular Momentum is Conserverd}
\section{Torque}
\startformula
	\Gamma = \frac{dL}{dt}
\stopformula
\section{More Work}
\m{dW= \vec \Gamma \dotp d\theta}


\subject{Notes}
%\placefootnotes[criterium=chapter]
\placenotes[endnote][criterium=chapter]

%\subject{Bibliography}
%        \placelistofpublications

\stopchapter
\stopcomponent

\placetextfloat[top][fig:KeplerTerestrial]
{The orbits of the inner planets. These are the also called the terrestrial or rocky planets. The orbits of Mercury and Mars are off-center. Although their orbits appear nearly circular, they are actually ellipses, with the sun at one focus of the ellipse.}	% caption text
{\starttikzpicture	% tikz code
\startpolaraxis
 [	xticklabels=\empty,
 	ytick={0,0.5,...,2.5},
 	yticklabels={{},{},100\units{Gm},{},200\units{Gm},{}},
 	minor y tick num={4},
	% yminorgrids=true,
	hide x axis,
	ymax = 2.5,
	scale only axis=true, width={10cm},
  	% ylabel={Distance from Sun \m{r} (\m{\sci{9}\units{m}})},
 ]
  	\node [name path=Sun] at (0,0) {\Sun};
    \addplot [ % Mercury
        thick,
        domain=0:360,
        samples=600,
    ]
        {0.5546/(1+0.20564*cos(x-77.46))}
  [yshift=-.5pt]
    node[pos=0.25] {\Mercury}
    ;
    \addplot [ % Venus
        thick,
        domain=0:360,
        samples=600,
    ]
        {1.082/(1+0.00676*cos(x-131.77))}
  [yshift=-1.7pt]
    node[pos=0.25] {\Venus}
    ;
    \addplot [ % Earth
    	name path=Earth,
        thick,
        domain=0:360,
        samples=600,
    ]
        {1.496/(1+0.0167*cos(x-102.93))}
    node[pos=0.25] {\Terra}
    ;
    \addplot [ % Mars
    	name path=Mars,
        thick,
        domain=0:360,
        samples=600,
    ]
        {2.259/(1+0.0934*cos(x-336.08))}
  [yshift=1pt, xshift=1.1pt]
    node[pos=0.25] {\Mars}
    ;
%    \addplot fill between [
%        of=Mars and Earth,
%        %soft clip={domain=120:180},
%    ];
\stoppolaraxis
\stoptikzpicture}

%\stopchapter
%\stopcomponent

% Templates:

% Epigraph
\placefigure[margin,none]{}{\small
	\startalignment[flushleft]
	\stopalignment
	\startalignment[flushright]
	{\it }\\
	{\sc }\\
	–
	\stopalignment
}


% Margin image
\placefigure[margin][] % Location, Label
{} % Caption
{\externalfigure[chapter03/][width=144pt]} % File

% Margin Figure
\placefigure[margin][] % location
{}	% caption text
{\starttikzpicture	% tikz code
\stoptikzpicture}

% Aligned equation
\startformula\startmathalignment
\stopmathalignment\stopformula

% Aligned Equations
\startformula\startmathalignment[m=2,distance=2em]
\stopmathalignment\stopformula
