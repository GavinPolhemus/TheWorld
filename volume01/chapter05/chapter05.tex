% !TEX TS-program = ConTeXt Suite
% !TEX useOldSyncParser
\startcomponent c_chapter05
\project project_world
\product prd_volume01

\setupsynctex[state=start,method=max] % "method=max" or "min"
\starttext

%%%%%%%%%%%%%%%%%%%%%%%%%%%%%
\startchapter[title=Potential Energy, reference=ch:PotentialEnergy]
%%%%%%%%%%%%%%%%%%%%%%%%%%%%%

\placefigure[margin,none]{}{\small
	\startalignment[flushleft]
At this point it may be suitable to admonish – since to many anything new is suspect – that I conceived the whole Theory in my mind, wrote the treatise, communicated most of it privately among friends, even sketched some things in the presence of the Society, before I undertook any experiment,\dots at last the experiments were made before friends and they agreed with the Theory as much as I myself could barely hope.%\autocite{p.48}{Galileo1610}}
	\stopalignment
	\startalignment[flushright]
	{\it Hydrodynamics}\\
	{\sc Daniel Bernoulli}\\
	1700 – 1782
	\stopalignment
}

\lettrine{P}{endulums, especially pendulum collisions,} provided the first clues needed to understand both momentum and kinetic energy. Galileo discovered another intriguing property of pendulums, this time using the interrupted pendulum shown in figure~\ref{fig:InterruptedPendulum}, which he describes in \booktitle{Two New Sciences}.

%Newton taught us that momentum is conserved. Leibniz's \visviva\ was the first form of energy, another conserved quantity. Both of these conserved quantities appeared first in the laws of collision. Are there other conserved quantities, perhaps hiding somewhere else? Galileo noticed a candidate that seems to have nothing to do with collisions: height.

%When an object falls it gains kinetic energy. If the motion is redirected upwards this kinetic energy is lost as the object ascends. Galileo first noted that the heights of the object's descent and assent are equal even when the slopes are different. 

%Galileo discovered that the maximum height of a pendulum's swing is surprisingly consistent.

%  conservation of height with a with a simple interrupted pendulum,
\placefigure[margin][fig:InterruptedPendulum] % Location, Label
{Galileo's interrupted pendulum always swings to the same height, even when the string hits a nail placed below the pivot.} % Caption
{\externalfigure[chapter05/GalileoInturuptedPendulum][width=144pt]} % File

\startblockquote
	Imagine this page to represent a vertical wall with a nail driven into it; and from the nail let there be suspended a lead ball of one or two ounces by means of a fine vertical thread, AB, say three or four feet long;\dots\autocite{p.~343, “two or three cubits” changed to “three or four feet.”}{Galileo1638}
\stopblockquote
The thread must hang about two inches in front of the wall so that the lead ball can swing freely without touching the wall.
\startblockquote
	\dots on this wall draw a horizontal line DC, at right angles to the vertical thread AB\dots. Now bring the thread AB with the attached ball into the position AC and set it free; first it will be observed to descend along the arc CBD, to pass the point B, and to travel along the arc BD, till it almost reaches the horizontal CD, a slight shortage being caused by the resistance of the air and of the string; from this we may rightly infer that the ball in its descent through the arc CB acquired an impetus on reaching B that was just sufficient to carry it through a similar arc BC to the same height.\autocite{p.~343-4 for all of the remaining Galileo quotes in this section.}{Galileo1638}
\stopblockquote
As we have seen, attempts to quantify Galileo's impetus caused tremendous confusion because there are \emph{two} quantities, momentum and kinetic energy, that are both important for understanding an object's motion.
With luck, one of these will be the thing acquired in the ball's descent and sufficient for the ball's ascent to the same height.
\startblockquote
	Having repeated this experiment many times, let us now drive a nail into the wall\dots say at E\dots so that it projects out some five or six inches in order that the thread, again carrying the ball through the arc CB, may strike upon the nail E when the ball reaches B, and thus compel it to traverse the arc BG, described about E as center; 
\stopblockquote
The nail at E interrupts that pendulum's swing, and the ball travels up the steeper arc BG.
\startblockquote
	Now, gentleman, you will observe with pleasure that the ball swings to the point G in the horizontal\dots
\stopblockquote
%The interrupted pendulum, shown in figure~\ref{fig:InterruptedPendulum}, is a pendulum hanging from a nail on a wall (located at A in the diagram). The ball must swing freely from the release at C to the bottom of the swing at B up to the stopping point at D. The pendulum swings parallel to the wall but does not rub against the wall, so that D is almost the same height as C. The wall is important because another nail can be placed at E to get in the way of the string, interrupting the pendulum's swing so that it does not go from B to D but instead follows a steeper path from B to G.
Galileo noticed that G is also the same height as C and D. The interruption does not make G higher by reducing the length of the ascending path, nor does it make G lower by making the path steeper. The speed acquired by the pendulum during its descent from C to B is exactly enough to carry it up the ascent from B to the G.
\startblockquote
\dots and you would see the same thing happen if the obstacle were placed at some lower point, say at F, about which the ball would describe the arc BI, the rise of the ball always terminating exactly on the line CD.%\autocite{p.~344}{Galileo1638}
\stopblockquote
%Placing the nail even lower at F gives the same result. The path from B to I is even shorter and steeper, but I's height is the same as the height of C, D, and G.
In fact, the interrupting nail does not have to be directly below the pivot at A, it can be anywhere. I find this experiment surprisingly entertaining. Galileo clearly enjoyed it as well.
\startblockquote
	But when the nail is placed so low that the remainder of the thread below it will not reach to the height CD (which would happen if the nail were placed nearer to B than to the intersection of AB with the horizontal CD), then the thread leaps over the nail and twists itself about it.
\stopblockquote
If you know a five-year-old, then you have a perfect partner for this experiment. If you do not, you have my permission to try it anyway.

Huygens, who discovered the conservation of $mv$ and $mv^2$ in pendulum collisions, also studied the colliding pendulums' initial and final heights, and again he discovered something quite remarkable.
Pendulums of different masses may be released from various heights. Then, after the colliding with each other, those masses will ascend to other various heights. Huygens discovered that for elastic collisions the total of the masses times their heights (total $mh$, where $h$ is the mass's height) is the same before and after the collision.

Huygens then related the total $mh$ at release to the total $mv^2$ at the collision, showing that the total $mh$ lost in the masses' descent is proportional to the $mv^2$ right before the collision. Likewise the total $mv^2$ right after the collision is proportional to the total $mh$ gained in the masses' ascent.
%Huygens made all of the observations before Newton introduced momentum, before Leibniz introduced \visviva, and long before the concept of kinetic energy. None the less, we should take a moment to understand his observation in modern physics language.  
%Huygens then took on the problem of what happens before the system returns to rest. Galileo had shown that pendulum descends the loss of height is proportional to the increase in square of the pendulum's speed.
%Between the moments of initial and final rest the masses descend and pick up some kinetic energy, exchange that energy in the collisions, and then loose the kinetic energy as they ascend. 
Huygens extended his analysis to complicated systems with levers, pivots and jointed pendulums. Still, whenever the total $mh$ decreased, the total $mv^2$ would increase proportionally; when the total $mh$ increased, the total $mv^2$ would decrease proportionally. In Huygens' view, a system in motion always has the potential to ascend to a greater height and come to rest. Even though the system's actual height may be lower, this potential height is conserved. 

Johann Bernoulli, one of the early \visviva\ advocates, saw the situation from the opposite perspective. The initial heights represented a sort of stored \visviva. When the pendulums were released the \visviva\ could emerge during the descent only to return to the stored state during the ascent. In this view the conservation of \visviva\ must take into account the stored \visviva\ in the total $mh$ as well as the total $mv^2$ in the motion.

As a practical matter, Huygen's view and Bernoulli's view are interchangeable. The modern view does not prefer either form. Energy is changing from a static state to a moving state and then back. The moving state is kinetic energy. The static form of energy is called potential energy. 
But in the early days there was some confusion about the relationship between Huygens' heights and \visviva.

This was already getting very complicated when Johann Bernoulli's son, Daniel Bernoulli, decided to solve the problem of motions and pressures in fluids. The younger Bernoulli did not attempt to demonstrate that Huygen's relationship between total $mh$ and $mv^2$ applies to fluids (or, equivalently, that fluids obey conservation of \visviva). Instead he assumed, as a hypothesis, that these principles apply to fluids and then used the principles to make predictions. He describes this project in the introduction to his revolutionary work on fluids, \booktitle{Hydrodynamica}, published in 1738.

\startblockquote
	It is amazing how much utility this hypothesis may have in mechanical Philosophy\dots it is the same [hypothesis] that I employed for investigating in fluids the laws of motion arising from their own gravity\dots However, I preferred to adopt this hypothesis with Huygenian rather than Paternal words, and to mark it with the name of \textit{the equality between actual descent and potential ascent,} rather than by that other of \textit{conservation of \visviva,} which some even yet dislike, chiefly in England, I know not by what misfortune.
\stopblockquote
The complicated systems of pipes, tanks, and fountains studied by Daniel Bernoulli were infinitely more complex than the intricate systems considered by Huygens and the elder Bernoulli, but the energy conservation law Daniel Bernoulli used to analyze these systems is quite simple. In his study of fluids, Daniel Bernoulli was the first to carefully account for the system's kinetic and potential energy, taking into account any energy added or removed through the work of outside forces. (Heat transfer is not important in systems Daniel studied.) This careful application of a simple law produced a an abundant harvest of detailed and often surprising predictions. 
Daniel Bernoulli refers to these predictions as “Theorems” because they each follow logically from the law of conservation of energy. 
%\startblockquote
%	in this way much presents itself that was previously unknown, not only about the motion of water but also, as one can see, surprisingly, about its pressure which, with no Analysis yet performed, no one will have easily foreseen or expected.
%\stopblockquote
\startblockquote
	The Theorems which are presented are not only new, but the majority are unexpected, of the truth of all of which I was not able to convince myself clearly until I had conducted experiments which removed all my doubt.
\stopblockquote
Every experiment testing one of these predictions is also a test of the conservation law on which those predictions are based. The remarkable agreement between the many experiments and predictions gave Daniel Bernoulli great confidence in the law of conservation of energy.

Daniel Bernoulli's method has continued to provide insight into every know physical process, from particle interactions to the expansion of the universe. In the centuries since the publication of \booktitle{Hyrodynamica} much of the important progress in physics has come from properly identifying and quantifying the many forms of energy.

\section{Gravitational Potential Energy}

The journey starts with gravitational potential energy. The symbol for potential energy is $U$, and the formula for gravitational potential energy is
\startformula
	U = mgh,
\stopformula
where $g$ is the same $9.8\units{m/s^2}$ found in the formula for gravitational force, $F_y = -mg$. This is not a coincidence. A change in an object's gravitational potential energy is related to the work done by the gravitational force, so the appearance of $mg$ in the formula is not surprising.

The application of this formula is best demonstrated by a couple of simple examples.

\startexample[] How much work is required to lift a $3.0\units{kg}$ box from the floor a shelf that is $2.0\units{m}$ high?
\startsolution
	Following Daniel Bernoulli, we start with the law of conservation of energy. The total energy includes both kinetic and potential energy, that is $H=K+U$.
\startformula\startmathalignment
\NC	H\si + W + \cancel{Q}	\NC = H\sf			\NR
\NC	K\si + U\si + W			\NC = K\sf + U\sf	\NR
\stopmathalignment\stopformula
	The since the box starts motionless on the floor and ends motionless on the shelf, the initial and final kinetic energies are both zero. The initial height of the box is zero, so the initial gravitational potential energy is zero. Only the work and final gravitational potential energy remain.
	\startformula
		W = U\sf
			= mgh
			= (3.0\units{kg})(9.8\units{m/s^2})(2.0\units{m}) 
			= 59\units{J}
	\stopformula
	The person lifting will do $59\units{J}$ of work lifting the box.
\stopsolution
\stopexample

Whenever you are asked to calculate work you should consider using the work formula $W = F_x\Delta x + F_y\Delta y + F_z\Delta z$, but that would have been a poor choice here. You do not actually know the force that you exert on the box. It is approximately $mg$ in the vertical direction, but not exactly. At first you will need to exert more force than $mg$ to give the box some upwards momentum. You will exert slightly less force near the top as you allow gravity to slow the box. This complication makes the work formula into a calculus problem which you probably want to avoid. Conservation of energy only requires the direct calculation of initial and final energies, which is much easier.

\startexample[] The $3.0\units{kg}$ box was not placed on the $2.0\units{m}$ high shelf with care, and it falls off. What is the box's speed as it reaches the floor?
\startsolution
	The falling box is much like the descending pendulums studied by Huygens. The  box's gravitational potential energy will decrease during the descent, and the kinetic energy will increase by the same amount. All of this is accounted for using the law of conservation of energy, this time without any outside force doing work.
	\startformula\startmathalignment
	\NC	H\si + \cancel{W} + \cancel{Q}	\NC = H\sf		\NR
	\NC	K\si + U\si					\NC = K\sf + U\sf
	\stopmathalignment\stopformula
	The box starts motionless on the shelf, so the initial kinetic energy is zero. When the box reaches the floor, the gravitational potential energy is zero. Only the initial potential and final kinetic energies remain.
	\startformula\startmathalignment
	\NC	U\si	\NC = K\sf 	\NR
	\NC	mgh	\NC = \half mv^2	\NR
	\NC	v	\NC = \sqrt{2gh}
				= \sqrt{2(9.8\units{m/s^2})(2.0\units{m})}
				= 6.3\units{m/s}	
	\stopmathalignment\stopformula
	The box will probably have an inelastic, destructive collision with the floor which will bring it to a stop. However, as it reaches the floor, before the energy consuming destruction begins, the box's speed is $6.3\units{m/s}$.
\stopsolution
\stopexample

This problem could have been solved using the work done by the gravitational force on the box as it falls. The difference between my solution and the solution using work is the choice of the system being studied. The box's kinetic energy is a property of the box. To study the motion of the box we must include the box's kinetic energy in our system.

The gravitational potential energy is not actually in the box, nor is it in the Earth which is pulling down on the box. The gravitational potential energy is in the gravitational field surrounding the box. We can chose to include this potential energy in our system or not. If you chose to make the box the only thing in the system, then the gravitational force is an outside force that does work. Since the potential energy is outside the system it does not contribute to the total energy, therefore $H=K$. Using this method the solution looks like this:
	\startformula\startmathalignment
	\NC	H\si + W + \cancel{Q}		\NC = H\sf			\NR
	\NC	\cancel{K\si} + F_y\Delta y	\NC = K\sf			\NR
	\NC	(-mg)(-h)					\NC = \half mv^2	\NR
	\NC	v						\NC = \sqrt{2gh}
						= \sqrt{2(9.8\units{m/s^2})(2.0\units{m})}
							= 6.3\units{m/s}	
	\stopmathalignment\stopformula
	There are some tricky signs in the middle, but the final answer is exactly the same. Daniel Bernoulli taught us not to do the problem this way. It works here, but in most cases it will lead to a great deal of suffering.

If you include the gravitational potential energy in the system, as I did, then the gravitational force is not an external force, it is internal to the system. This force plays a roll in moving energy from one part of the system (the gravitational field) to another part (the box), but it does not add any energy to the system, nor does it remove any energy from the system. That is why there was no external work in my solution.

The gravitational field's ability to store potential energy was a great mystery until Einstein brilliantly explained that it has to do with the bending of space-time. This is extremely interesting and terribly complicated. Luckily, Daniel Bernoulli showed us how to deal with these sorts of mind-boggling complications.
He did not need to know the details of intermolecular forces in fluids, and you do not need to know the details of space-time bending. Just use the potential energy formula and everything is fine, provided you are careful to include the potential energy in the system and remember that the gravitational force is an internal force that does no work \emph{on the system.}

\startexample[ex:BalloonLaunch] As part of their quest for knowledge, physics students launch a water balloon from the top of the school onto the soccer field below . The water balloon has a mass of $0.50\units{kg}$ and is launched with with a speed of $15\units{m/s}$ and an angle $53\degree$ above the horizontal, as shown in figure~\ref{fig:BalloonLaunch}. What is the water ballon's speed when it hits the ground?

\placefigure[margin][fig:BalloonLaunch] % location
{The path of the projectile in example~\ref{ex:BalloonLaunch}}	% caption text
{\starttikzpicture
	\startaxis[%axis equal,
		footnotesize,
		width=2.25in,%\marginparwidth,
		y={0.1333cm},x={0.1333cm},
		xlabel={$x$ (m)},
		xmin=0, xmax=30,
		%xtick={0,1,...,4},
		%minor x tick num=9,
		ylabel={$y$ (m)},
		ymin=0, ymax=20,
		%ytick={0,1,...,6},
		%minor y tick num=4,
		clip=false,
		]
  \addplot[samples=100, variable=\t, domain=0:3.21]
    ({9*t}, {12+12*t-4.9*t^2});
  \addplot[samples=10, domain=0:3,
    % the default choice ’variable=\x’ leads to
    % unexpected results here!
  	mark = *, mark size={.4pt},
    variable=\t,
    quiver={
        u={9},
        v={12-9.8*t},
        scale arrows=0.333}, thick,
        ->]
    ({9*t}, {12+12*t-4.9*t^2});
  	\draw[fill=black!20] (0,0) rectangle (2,12);
	\stopaxis
\stoptikzpicture}

\startsolution
	The ballon starts with both potential and kinetic energy, but only has kinetic energy at the end.
	\startformula\startmathalignment
	\NC	H\si + \cancel{W} + \cancel{Q}	\NC = H\sf						\NR
	\NC	K\si + U\si					\NC = K\sf + \cancel{U\sf}			\NR
	\NC	\half mv\si^2 + mgh			\NC = \half mv\sf^2				\NR
	\NC	v\sf						\NC = \sqrt{v\si^2 + 2gh}			\NR
	\NC						\NC = \sqrt{(15\units{m/s})^2+2(9.8\units{m/s^2})(12.0\units{m})}\NR
	\NC					\NC = 21\units{m/s}	
	\stopmathalignment\stopformula
	The balloon is traveling quite a bit faster when it gets down to the soccer field.
\stopsolution
\stopexample
	Since energy is not directional there is no need to break anything into components. The balloon speed at impact is not affected by the launch angle. The launch angle will affect the distance and the time aloft, but the impact speed depends only on the initial speed and height. 

%Gravitational potential energy was the first form of potential energy to be identified and quantified, but there are other ways to store energy 

\section{Spring potential energy}

\placefigure[margin][fig:ballwallKU] % location
{As the ball bounces off of a wall, its kinetic energy $K$ is stored briefly as potential energy $U$.}	% caption text
{\starttikzpicture[thick]
\draw[->,ultra thick,opacity=.2] (0,1.5) --(1,1.5);
\draw[shade, ball color = white] (0,1.5) circle[radius=.4cm]node{$K$};
\shade[left color=gray] (4,.9) rectangle (4.2,2.1);
\draw (4,.9)--(4,2.1);
\draw[shade, ball color = white] (3.65,0) circle[radius=.4cm]node{$U$};
%\draw[->,ultra thick] (4,0) -- node[above, pos=.6]{$J$}(2,0);
\shade[left color=gray] (4,-.6) rectangle (4.2,.6);
\draw (4,-.6) rectangle (4,.6);
\draw[->,ultra thick,opacity=.2] (0,-1.5) -- (-1,-1.5);
\draw[shade, ball color = white] (0,-1.5) circle[radius=.4cm]node{$K$};
\shade[left color=gray] (4,-2.1) rectangle (4.2,-.9);
\draw (4,-2.1)--(4,-.9);
\stoptikzpicture}

At the end of Chapter 5 we studied a ball bouncing elastically off of a wall. Conservation of energy explained why the ball had the same speed after the bounce as it did before. The ball bouncing off of the wall still has one surprise for us. As the ball approaches the wall its momentum and kinetic energy are both positive. After it bounces its momentum is negative but its kinetic energy is still positive. However, at the wall the ball must come to rest momentarily as it changes direction. At that moment its momentum and kinetic energy are both zero. What has happened to the kinetic energy during this momentary stop?


As the ball collides with the wall, its shape is deformed and the kinetic energy is transformed into potential energy within the ball. %Potential energy, represented by $U$, is associated with an object's shape or position, not its motion. Potential energy got its name from the fact that it has the potential to turn into kinetic energy. Potential energy is real energy, it's just not \emph{kinetic} energy.
The ball only stores the potential energy briefly before springing back to its original shape. This pushes the ball away from the wall, and the potential energy is changed back into kinetic energy.

%Depending on the materials used to make the ball, it can be more or less effective at converting the kinetic energy to potential energy and then converting the potential energy back to kinetic energy. If the ball is made out of wet clay it will not be effective at all, and the ball will stick to the wall rather than rebounding. In this case the energy has been converted to a disorganized form that does not convert back into kinetic energy. This disorganized energy is detectable as a tiny increase in the temperature of the clay. We will study this disorganized energy later. For now it is sufficient to know that the energy can be turned into less useful forms, but it is never truly lost.
%\section{Springs store energy}

\placefigure[margin][fig:ballspringwall] % location
{A ball bounces off of a spring on a wall. The ball starts with kinetic energy $K$.
%(The arrow shows the direction of motion. The kinetic energy is not directional.)
During the bounce the energy is stored in the spring as potential energy $U$. Then the energy is returned to the ball as kinetic energy $K$ again.
}	% caption text%[-3in]
{\starttikzpicture[thick]
\draw[->,ultra thick,opacity=.2] (0,1.5) -- (1,1.5);
\draw[shade, ball color = white] (0,1.5) circle[radius=.4cm] node{$K$};
\draw[decorate,decoration={coil,segment length=2mm,amplitude=1mm}] (2.8,1.5) -- (4,1.5);
\shade[left color=gray] (4,.9) rectangle (4.2,2.1);
\draw (4,.9)--(4,2.1);
\draw[shade, ball color = white] (3,0) circle[radius=.4cm];
\draw[decorate,decoration={coil,segment length=.9mm,amplitude=1mm}] (3.4,0) -- node[above=1mm]{$U$}(4,0);
%\draw[->,ultra thick] (4,0) -- node[above, pos=.6]{$F$}(1.5,0);
\shade[left color=gray] (4,-.6) rectangle (4.2,.6);
\draw (4,-.6) rectangle (4,.6);
\draw[->,ultra thick,opacity=.2] (0,-1.5) -- (-1,-1.5);
\draw[shade, ball color = white] (0,-1.5) circle[radius=.4cm] node{$K$};
\draw[decorate,decoration={coil,segment length=2mm,amplitude=1mm}] (2.8,-1.5) -- (4,-1.5);
\shade[left color=gray] (4,-2.1) rectangle (4.2,-.9);
\draw (4,-2.1)--(4,-.9);
\stoptikzpicture}

The deformation of the ball is difficult to see in a real collision, so let us turn our attention to slightly different situation where the potential energy is easier to recognize. %The potential energy is easier to recognize in a spring, which deforms visibly.
Figure \ref{fig:ballspringwall} shows a ball bouncing off of a spring that is attached to the wall. During the collision the spring is squashed noticeably. A squashed spring contains potential energy that can be returned to the ball.

Springs are wonderfully simple energy storage devices. A relaxed spring does not have any energy stored in it. To add energy, just compress the spring. Compressing the spring requires applying a force in the direction that compresses the spring. A force acting in the same direction as the motion does positive work, increasing the energy of the spring.

Using conservation of energy on only the spring we see that the energy stored in the spring is equal to the work done in compressing it, as shown in Figure \ref{fig:SpringCompressWork}. The spring does not have kinetic energy, only potential energy, so $H=U$. We start with the spring uncompressed, with $U\si=0$, then do work $W$ to find the final potential energy in the spring, $U\sf$.
\startformula\startmathalignment
\NC	H\si + W + \cancel{Q}	\NC = H\sf		\NR
\NC	\cancel{U\si} + W		\NC = U\sf		\NR
\NC	W					\NC = U\sf
\stopmathalignment\stopformula
The work $W$ done on the spring is stored as potential energy $U\sf$.

\placefigure[margin][fig:SpringCompressWork] % location
{As the spring is compressed, the force acting on the left end of the spring is in the same direction as the displacement, so positive work is done on the spring.
}	% caption text
{\starttikzpicture[thick, scale=2]
%\draw[->,ultra thick] (0,1.5) -- (1,1.5);
%\draw[shade, ball color = white] (2.4,.9) circle[radius=.4cm] node{$K$};
\draw[decorate,decoration={coil,segment length=4mm, amplitude=1.9mm},very thick] (2.8,.9) -- node[above=2mm]{$U\si = 0$}(4,.9);
\shade[left color=gray,opacity=.2] (4,.6) rectangle (4.2,1.2);
\draw[opacity=.2] (4,.6)--(4,1.2);
\draw[shade, ball color = white,opacity=.2] (2.7,0) circle[radius=.4cm];
\draw[decorate,decoration={coil,segment length=2.9mm,amplitude=1.9mm},very thick] (3.1,0) --(4,0);
\draw[->,ultra thick,opacity=.4] (2.5,0) -- node[below]{$F$}(3.1,0);
\draw[-Implies,very thick,double] (2.9,.2) to [bend left=45]node[above]{$W$}(3.5,.15);
\shade[left color=gray,opacity=.2] (4,-.3) rectangle (4.2,.3);
\draw[opacity=.2] (4,-.3) rectangle (4,.3);
\draw[->,ultra thick,opacity=.4] (4.6,0) -- node[below]{$F$}(4,0);
%\draw[->,ultra thick] (0,-1.5) -- (-1,-1.5);
\draw[shade, ball color = white,opacity=.2] (3,-.9) circle[radius=.4cm];
\draw[decorate,decoration={coil,segment length=1.8mm,amplitude=1.9mm},very thick] (3.4,-.9) -- node[above=3mm, pos=.45]{$U\sf = W$}(4,-.9);
\shade[left color=gray,opacity=.2] (4,-1.2) rectangle (4.2,-.6);
\draw[opacity=.2] (4,-1.2)--(4,-.6);
\stoptikzpicture}

Notice that no work is done by the wall. The wall exerts a force on the right end of the spring, but there is no displacement on that end, so the force does not do any work.

The wall does play an important role: it prevents the spring from getting knocked out of the way by the ball. This is obvious, but it is worth taking a look at how this works using conservation of momentum. The ball and the wall exert opposite forces on the spring for the same amount of time, so they exert opposite impulses on the spring.
\startformula\startmathalignment
\NC	p\si + J							\NC = p\sf		\NR
\NC	\cancel{p\si} + J\sub{ball} + J\sub{wall}	\NC = p\sf		\NR
\NC	0								\NC = p\sf
\stopmathalignment\stopformula
In the last line the impulse from the wall, which is towards the left, cancels the impulse from ball, which is towards the right, so the spring does not gain any momentum and does not move out of the way.

The spring effectively separates the energy from the momentum during the compression. The energy gets stored in the spring, but the momentum passes through the spring to the wall. Energy an momentum commonly follow different paths through a system. Being able to use the two conservation laws allows us to understand the flow of each of them separately.

\placefigure[margin][fig:BallStopsWork] % location
{The force acting on the ball is opposite the direction of the displacement, so negative work is done on the ball. The ball loses all of its kinetic energy and comes to a stop.}	% caption text
{\starttikzpicture[thick, scale=2]
%\draw[->,ultra thick] (0,1.5) -- (1,1.5);
\draw[shade, ball color = white] (2,.9) circle[radius=.4cm] node{$K\si$};
\draw[decorate,decoration={coil,segment length=4mm, amplitude=1.9mm},very thick,opacity=.2] (2.8,.9) -- (4,.9);
\shade[left color=gray,opacity=.2] (4,.6) rectangle (4.2,1.2);
\draw[opacity=.2] (4,.6)--(4,1.2);
\draw[shade, ball color = white] (2.7,0) circle[radius=.4cm];
\draw[decorate,decoration={coil,segment length=2.9mm,amplitude=1.9mm},very thick,opacity=.2] (3.1,0) --(4,0);
\draw[->,ultra thick,opacity=.4] (3.7,0) -- node[below=2mm]{$F$}(3.1,0);
\draw[-Implies,very thick,double] (2.9,.2) to [bend left=45]node[above]{$W=-K\si$}(3.5,.15);
\shade[left color=gray,opacity=.2] (4,-.3) rectangle (4.2,.3);
\draw[opacity=.2] (4,-.3) rectangle (4,.3);
%\draw[->,ultra thick,opacity=.2] (4.6,0) -- node[below]{$F$}(4,0);
%\draw[->,ultra thick] (0,-1.5) -- (-1,-1.5);
\draw[shade, ball color = white] (3,-.9) circle[radius=.4cm] node{$K\sf=0$};
\draw[decorate,decoration={coil,segment length=1.8mm,amplitude=1.9mm},very thick,opacity=.2] (3.4,-.9) -- (4,-.9);
\shade[left color=gray,opacity=.2] (4,-1.2) rectangle (4.2,-.6);
\draw[opacity=.2] (4,-1.2)--(4,-.6);
\stoptikzpicture}

We have seen how the spring stores the work as potential energy. Let's take a look at the ball alone to see what happens to its initial kinetic energy $K\si$.
The ball has only kinetic energy, no potential energy, so when we conserve energy for the ball $H=K$. The work done on the ball is due to the force exerted on it by the spring. The direction of this force (to the left) is opposite the direction of the displacement (to the right) so the work done on the ball is negative.
\startformula\startmathalignment
\NC	H\si + W + \cancel{Q}	\NC = H\sf			\NR
\NC	K\si + W				\NC = \cancel{K\sf}	\NR
\NC	-K\si 					\NC = W
\stopmathalignment\stopformula
The negative work $W$ done on the ball takes away all of its kinetic energy, bringing it to a stop.

The force exerted by the spring on the ball also delivers an impulse (directed to the left) that cancels all of the ball's initial momentum (directed to the right). Momentum and energy don't always flow together, but in this case they do, both leaving the ball as it slows to a stop.

The work done on the spring by the ball is obviously related to the work done on the ball by the spring. By Newton's third law, the force exerted by the spring on the ball (to the left) is equal and opposite to the force exerted by the ball on the string (to the right). Since the displacements are the same, the work done by the spring on the ball (negative) is equal and opposite to the work done by the ball on the spring (positive). The negative work shown in Figure \ref{fig:BallStopsWork} is minus the positive work shown in Figure \ref{fig:SpringCompressWork}, which means that the final energy stored in the spring is equal to the initial kinetic energy of the ball. That is, $U\sf = K\si$. All of the ball's initial kinetic energy is stored by the spring as potential energy.

\placefigure[margin][fig:BallSpring] % location
{Energy is transferred from the ball to the spring, but the total energy in the system remains the same.}	% caption text
{\starttikzpicture[thick, scale=2]
%\draw[->,ultra thick] (0,1.5) -- (1,1.5);
\draw[shade, ball color = white] (2,.9) circle[radius=.4cm] node{$K\si$};
\draw[decorate,decoration={coil,segment length=4mm, amplitude=1.9mm},very thick] (2.8,.9) -- node[above=2mm, pos=.45]{$U\si = 0$}(4,.9);
\shade[left color=gray,opacity=.2] (4,.6) rectangle (4.2,1.2);
\draw[opacity=.2] (4,.6)--(4,1.2);
\draw[shade, ball color = white] (2.7,0) circle[radius=.4cm];
\draw[decorate,decoration={coil,segment length=2.9mm,amplitude=1.9mm},very thick] (3.1,0) --(4,0);
%\draw[->,ultra thick,opacity=.4] (3.7,0) -- node[below=2mm]{$F$}(3.1,0);
\draw[-Implies,very thick,double,opacity=.2] (2.9,.15) to [bend left=45]node[above]{$W$}(3.5,.15);
\shade[left color=gray,opacity=.2] (4,-.3) rectangle (4.2,.3);
\draw[opacity=.2] (4,-.3) rectangle (4,.3);
%\draw[->,ultra thick,opacity=.2] (4.6,0) -- node[below]{$F$}(4,0);
%\draw[->,ultra thick] (0,-1.5) -- (-1,-1.5);
\draw[shade, ball color = white] (3,-.9) circle[radius=.4cm] node{$K\sf=0$};
\draw[decorate,decoration={coil,segment length=1.8mm,amplitude=1.9mm},very thick] (3.4,-.9) -- node[above=3mm, pos=.45]{$U\sf = K\si$}(4,-.9);
\shade[left color=gray,opacity=.2] (4,-1.2) rectangle (4.2,-.6);
\draw[opacity=.2] (4,-1.2)--(4,-.6);
\stoptikzpicture}

One of the great benefits of conservation of energy is the ability to consider many types of energy stored in several different objects at the same time. 
Let's look at the flow energy from the ball to the spring one more time, considering the kinetic energy of the ball and the potential energy of the spring together, $H=K+U$. The work going from the ball to the spring doesn't alter the total energy in the system. Only work done by an external force contributes to the energy of the system.  In this case there is no external work done on the system, so $W=0$. 
\startformula\startmathalignment
\NC	H\si + \cancel{W} + \cancel{Q}	\NC = H\sf				\NR
\NC	K\si + \cancel{U\si}			\NC = \cancel{K\sf} + U\sf	\NR
\NC	K\si						\NC = U\sf
\stopmathalignment\stopformula
The energy in the system changes forms, but the total amount is unchanged. Choosing the system so there is no external work often saves a great deal of effort.



\placefigure[margin][fig:BallSpringFull] % location
{Energy is transferred from the ball to the spring and back.}	% caption text
{\starttikzpicture[thick, scale=2]
%\draw[->,ultra thick] (0,1.5) -- (1,1.5);
\draw[shade, ball color = white] (2,.9) circle[radius=.4cm] node{$K\si$};
\draw[decorate,decoration={coil,segment length=4mm, amplitude=1.9mm},very thick] (2.8,.9) -- node[above=2mm, pos=.45]{$U\si = 0$}(4,.9);
\shade[left color=gray,opacity=.2] (4,.6) rectangle (4.2,1.2);
\draw[opacity=.2] (4,.6)--(4,1.2);
\draw[shade, ball color = white] (2.7,0) circle[radius=.4cm];
\draw[decorate,decoration={coil,segment length=2.9mm,amplitude=1.9mm},very thick] (3.1,0) --(4,0);
%\draw[->,ultra thick,opacity=.4] (3.7,0) -- node[below=2mm]{$F$}(3.1,0);
\draw[-Implies,very thick,double,opacity=.2] (2.9,.2) to [bend left=45]node[above]{$W$}(3.5,.15);
\shade[left color=gray,opacity=.2] (4,-.3) rectangle (4.2,.3);
\draw[opacity=.2] (4,-.3) rectangle (4,.3);
%\draw[->,ultra thick,opacity=.2] (4.6,0) -- node[below]{$F$}(4,0);
%\draw[->,ultra thick] (0,-1.5) -- (-1,-1.5);
\draw[shade, ball color = white] (3,-.9) circle[radius=.4cm] node{$K=0$};
\draw[decorate,decoration={coil,segment length=1.8mm,amplitude=1.9mm},very thick] (3.4,-.9) -- node[above=3mm, pos=.45]{$U = K\si$}(4,-.9);
\shade[left color=gray,opacity=.2] (4,-1.2) rectangle (4.2,-.6);
\draw[opacity=.2] (4,-1.2)--(4,-.6);
\draw[shade, ball color = white] (2.7,-1.8) circle[radius=.4cm];
\draw[decorate,decoration={coil,segment length=2.9mm,amplitude=1.9mm},very thick] (3.1,-1.8) --(4,-1.8);
%\draw[->,ultra thick,opacity=.4] (3.7,0) -- node[below=2mm]{$F$}(3.1,0);
\draw[-Implies,very thick,double,opacity=.2] (3.5,-1.65) to [bend right=45]node[above]{$W$}(2.9,-1.7);
\shade[left color=gray,opacity=.2] (4,-1.5) rectangle (4.2,-2.1);
\draw[opacity=.2] (4,-1.5) rectangle (4,-2.1);
%\draw[->,ultra thick,opacity=.2] (4.6,0) -- node[below]{$F$}(4,0);
\draw[shade, ball color = white] (2,-2.7) circle[radius=.4cm] node{$K\sf=K\si$};
\draw[decorate,decoration={coil,segment length=4mm, amplitude=1.9mm},very thick] (2.8,-2.7) -- node[above=2mm, pos=.45]{$U\sf = 0$}(4,-2.7);
\shade[left color=gray,opacity=.2] (4,-2.4) rectangle (4.2,-3);
\draw[opacity=.2] (4,-2.4)--(4,-3);
\stoptikzpicture}

When the spring pushes the ball away from the wall the whole process is reversed. The system starts with potential energy $U\si$ in the spring and ends with kinetic energy $K\sf$ in the ball.
\startformula\startmathalignment
\NC	H\si + \cancel{W} + \cancel{Q}	\NC = H\sf				\NR
\NC	\cancel{K\si} + U\si			\NC = K\sf + \cancel{U\sf}	\NR
\NC	U\si						\NC = K\sf
\stopmathalignment\stopformula
All of the energy that was stored in the spring as potential energy $U\si$ is returned to the ball as kinetic energy $K\sf$.

The whole process of the bounce is shown in Figure \ref{fig:BallSpringFull}. Since the wall does no work, the ball's final kinetic energy is the same as the kinetic energy it had initially. Although the wall does no work, it does deliver the impulse required to change the direction of the balls's momentum, as shown if Figure \ref{fig:BallSpringFullp}. The spring has no momentum, so the momentum of the spring-ball system is entirely in the ball.
The ability to easily store and return energy makes spring incredibly useful.
In some cases the energy stored is quite small. Buttons on a keyboard have tiny springs in them that store some energy when the button is pressed. This energy is used to push the button back up so it is ready to be pressed again.
A bow for firing arrows is a different shape, but it is still a spring. When the string is pulled back the bow stores a larger amount of energy. When the string is released the energy stored in the bow is delivered to the arrow as kinetic energy.

The potential energy stored in a spring depends on the stiffness of the spring $k$ and the amount of stretch of the spring $s$. (The spring stiffness is often called the spring constant.) 
\highlightbox{
\startformula[eq:1DUspring]
	U\sub{spring} = \half ks^2
\stopformula
}
The stretch must always be measured from the relaxed length of the spring. If the spring is compressed rather than stretched, as in the ball bounce example above, then the “stretch” is negative. However, since the stretch is squared it does not matter whether the stretch is positive or negative; the energy stored is positive either way.

\placefigure[margin][fig:BallSpringFullp] % location
{The wall exerts a force to the left during the entire collision, providing the impulse necessary to reverse the direction of the ball's momentum.}	% caption text
{\starttikzpicture[thick, scale=2]
\draw[shade, ball color = white] (2,.9) circle[radius=.4cm];
\draw[->,ultra thick] (2,.9) --  node[above, pos=.4]{$p\si$}(2.35,.9);
\draw[decorate,decoration={coil,segment length=4mm, amplitude=1.9mm},very thick] (2.8,.9) -- (4,.9);
\shade[left color=gray,opacity=.2] (4,.6) rectangle (4.2,1.2);
\draw[opacity=.2] (4,.6)--(4,1.2);
\draw[shade, ball color = white] (2.7,0) circle[radius=.4cm];
\draw[decorate,decoration={coil,segment length=2.9mm,amplitude=1.9mm},very thick] (3.1,0) --(4,0);
\draw[->,ultra thick] (4,0) -- node[above=2mm]{$F$}(3.43,0);
\shade[left color=gray,opacity=.2] (4,-.3) rectangle (4.2,.3);
\draw[opacity=.2] (4,-.3) rectangle (4,.3);
%\draw[->,ultra thick,opacity=.2] (4.6,0) -- node[below]{$F$}(4,0);
%\draw[->,ultra thick] (0,-1.5) -- (-1,-1.5);
\draw[shade, ball color = white] (3,-.9) circle[radius=.4cm] node{$p=0$};
\draw[decorate,decoration={coil,segment length=1.8mm,amplitude=1.9mm},very thick] (3.4,-.9) -- (4,-.9);
\shade[left color=gray,opacity=.2] (4,-1.2) rectangle (4.2,-.6);
\draw[opacity=.2] (4,-1.2)--(4,-.6);
\draw[->,ultra thick] (4,-.9) -- node[above=2mm]{$F$}(3.25,-.9);
\draw[shade, ball color = white] (2.7,-1.8) circle[radius=.4cm];
\draw[decorate,decoration={coil,segment length=2.9mm,amplitude=1.9mm},very thick] (3.1,-1.8) --(4,-1.8);
%\draw[->,ultra thick,opacity=.4] (3.7,0) -- node[below=2mm]{$F$}(3.1,0);
\shade[left color=gray,opacity=.2] (4,-1.5) rectangle (4.2,-2.1);
\draw[opacity=.2] (4,-1.5) rectangle (4,-2.1);
\draw[->,ultra thick] (4,-1.8) -- node[above=2mm]{$F$}(3.43,-1.8);
\draw[shade, ball color = white] (2,-2.7) circle[radius=.4cm];
\draw[->,ultra thick] (2,-2.7) --  node[above=.5mm, pos=0]{$p\sf=-p\si$}(1.7,-2.7);
\draw[decorate,decoration={coil,segment length=4mm, amplitude=1.9mm},very thick] (2.8,-2.7) -- (4,-2.7);
\shade[left color=gray,opacity=.2] (4,-2.4) rectangle (4.2,-3);
\draw[opacity=.2] (4,-2.4)--(4,-3);
\stoptikzpicture}

%[Example finding the compression of the spring based on the Kinetic energy of the incoming ball.


%[Introduce energy buckets.]


%This leads to the equation for the force exerted by a spring, Hookes's law.
%\highlightbox{
%\startformula
%	F\sub{spring} = -kx
%\stopformula
%}
	
\section{Energy transformations in oscillations}

When ball bounced off of the spring the ball's energy was transferred to the spring and then back to the ball. Connecting the ball to the spring makes an oscillator, with the energy going back-and-forth between the ball an the spring over-and-over again.

The spring's maximum compression is the amplitude of the oscillation. That is the moment when all of the energy is in the spring as potential energy. When the ball passes through the midpoint of the oscillation all of the energy is in the kinetic energy of the ball. Using conservation of energy we can find the maximum speed of the ball (at the midpoint) from the amplitude.

Again we will consider the combined system of the ball and the spring: $H = K+U$. The initial position is at full stretch ($K=0$, $s=A$) and the final position is at the midpoint ($U=0$, $v=v\sub{max}$).
\startformula\startmathalignment
\NC	H\si + \cancel{W} + \cancel{Q}	\NC = H\sf					\NR
\NC	\cancel{K\si} + U\si			\NC = K\sf + \cancel{U\sf}		\NR
\NC	\half k A^2					\NC = \half mv\sub{max}^2	\NR
\NC	A\sqrt{\frac{k}{m}}			\NC = v\sub{max}
\stopmathalignment\stopformula
The maximum velocity, right at the midpoint, it proportional to the amplitude. 

Recall from our earlier discussion of oscillations that the maximum velocity can also be found from the amplitude and the period.
\highlightbox{
\startformula[eq:vmaxA]
	v\sub{max} = \frac{2\pi A}{T}
\stopformula
}
Combine this formula for $v\sub{max}$ with the formula for $v\sub{max}$ that we found above using conservation of energy.
\startformula
	\cancel{A}\sqrt{\frac{k}{m}} = \frac{2\pi \cancel{A}}{T}
\stopformula
Cleaning this up gives a useful formula an oscillator's period $T$.
\highlightbox{
\startformula
	T = 2\pi\sqrt{\frac{m}{k}}
\stopformula
}
The period $T$ depends on the mass of the ball and the stiffness of the spring, but it does not depend on the amplitude of the oscillations. If the oscillations are small, the speed is also small and each cycle takes time $T$. If the oscillations are big the speed is also big and each cycle is still completed in the same time $T$.

This can be used to find the frequency as well
\startformula
	f = \frac{1\units{cyc}}{T} = \frac{1\units{cyc}}{2\pi}\sqrt{\frac{k}{m}}
\stopformula
Again, the frequency does not depend on the amplitude, only on the mass and the spring constant. A stiffer spring (larger $k$) will increase the frequency, while a larger mass will lower the frequency. 


\section{Gravitational potential energy in the Solar System}

In the year that \booktitle{Hydrodynamica} was published, 1738, Daniel Bernoulli produced a paper on the motion of the Moon, which is pulled by both Earth and the Sun. Gravitational forces in the solar system had already been studied in some detail using Newton's methods of momentum and force, but Daniel Bernoulli attacked the problem using potential and kinetic energy.

The gravitational potential energy formula earlier in this chapter, $U=mgh$, is only useful for relatively small heights near the Earth's surface. Over the long distances between planets, moons, and the sun, the gravitational potential energy is a bit more complicated. The gravitational potential energy of any two objects in the solar system is
\startformula
	U = -G\frac{mM}{r},
\stopformula
where $m$ and $M$ are the objects' masses, $r$ is the distance between the objects' centers, and $G=6.67\sci{-11}\units{m^3/kg\.s^2}$ is the universal gravitational constant.

This formula is a bit intimidating, but extremely valuable if used correctly. First, notice that the distance between the objects is in the denominator. If the distance is extremely large, then their gravitational potential energy is extremely small. This is quite convenient because it means that very distant objects can be ignored. When studying the interaction of the Earth and Moon, it is not necessary to include the gravitational potential energy due to distant stars.

The second thing to notice is that the distance is between the objects' centers. When calculating the gravitational potential energy of an object on the Earth's surface the distance $r$ is Earth's radius ($r\sEarth=6.37\sci{6}\units{m}$) not the height above Earth's surface.

Third, the formula always gives a negative potential energy. This has alarmed many people, including great physicists, but it is not a problem. The gravitational potential energy is zero when the objects are far apart and it goes down as objects get closer together. Down from zero is negative. Kinetic energy is always positive, but there is no reason to be alarmed by negative potential energy.

Finally, the universal gravitational constant $G$ is extremely small, so the gravitational potential energy between every-day objects at every-day distances can be completely ignored. At least one of the objects must have a huge mass in order for the gravitational potential energy to be significant. Earth's mass of $5.97\sci{24}\units{kg}$ is enough to have a significant effect.

\startexample[] How much work is required to lift a $3.0\units{kg}$ box from Earth's surface to a very distant location?
\placefigure[margin][fig:BoxEarthGravU] % location
{The gravitational potential energy of a $3.0\units{kg}$ object in Earth's vicinity. The potential is not shown for locations inside Earth (the  gray region).}	% caption text
{\starttikzpicture
\startaxis
 [footnotesize, width=2.13in, height=2in,
   xlabel={$r$ ($\sci{6}\units{m}$)},
   xmin=0, xmax=50,
   ylabel={$U$ ($\sci{6}\unit{J}$)},
   ymin=-200, ymax=0,
   %ytick={-10,-8,...,0},
 ]
 \addplot[
   thick,
   domain=6.37:50,
   samples=201
  ]
  {-1190/x}
  ;
\startpgfonlayer{background}
\fill [black!20] (0, -200) rectangle (6.37, 0);
\stoppgfonlayer
%\draw [thick](0, -9.4) parabola (6.37, {(-1.99/6.37)});
\stopaxis
\stoptikzpicture}

\startsolution
	This problem is just like the earlier problem where the box was lifted onto a shelf, but now we must use the gravitational potential energy formula that works for large distances. The gravitational potential energy for the box and Earth are plotted in figure~\ref{fig:BoxEarthGravU}.
	\startformula\startmathalignment
	\NC	H\si + W + \cancel{Q}	\NC = H\sf			\NR
	\NC	K\si + U\si + W			\NC = K\sf + U\sf
	\stopmathalignment\stopformula
	Since the box starts motionless on the floor and ends motionless far away , the initial and final kinetic energies are both zero. The final potential energy is also zero because the box is very far from Earth. Only the work and initial potential energy remain. The initial distance between the box's center and Earth's center is Earth's radius.
	\startformula\startmathalignment
	\NC	U\si + W			\NC = 0										\NR
	\NC	-G\frac{mM}{r} + W	\NC = 0										\NR
	\NC	W				\NC = G\frac{mM}{r}								\NR
	\NC					\NC = (6.67\sci{-11}\units{m^3/kg\.s^2})
				\frac{(3.0\units{kg})(5.97\sci{24}\units{kg})}{6.37\sci{6}\units{m}}		\NR
	\NC					\NC = 1.88\sci{8}\units{J}
	\stopmathalignment\stopformula
	Moving the $3\units{kg}$ box from Earth's surface to a distant location requires a tremendous amount of work.
\stopsolution
\stopexample

\startexample[] The $3.0\units{kg}$ box was placed far from Earth, but even so it eventually is pulled back by Earth's gravitational attraction. How fast is the box going when it enters the atmosphere, approximately $100\units{km}$ above Earth's surface?
\startsolution
	Include the gravitational potential energy so that there is no work done on the system.
$100\units{km}$.
	\startformula\startmathalignment
	\NC	H\si + \cancel{W} + \cancel{Q}	\NC = H\sf		\NR
	\NC	K\si + U\si					\NC = K\sf + U\sf
	\stopmathalignment\stopformula
		The falling box starts with neither kinetic nor potential energy, so the initial total energy is zero. As the box falls its potential energy becomes more negative and its kinetic energy becomes more positive so that the total is unchanged at zero. 
	\startformula\startmathalignment
	\NC	0				\NC = \half mv^2	- G\frac{mM}{r}	\NR
	\NC	\half \cancel{m}v^2	\NC = G\frac{\cancel{m}M}{r}		\NR
	\NC	v				\NC = \sqrt{\frac{2GM}{r}}
	\stopmathalignment\stopformula
The final distance between Earth's center and the box is Earth's radius plus 
	\startformula\startmathalignment
	\NC	v	\NC = \sqrt{\frac{2(6.67\sci{-11}\units{m^3/kg\.s^2})(5.97\sci{24}\units{kg})}
				{6.37\sci{6}\units{m}+1.00\sci{5}\units{m}}}		\NR
	\NC		\NC = 1.11\sci{4}\units{m/s}
	\stopmathalignment\stopformula
	The box enters the atmosphere with a speed of about $11\units{km/s}$, causing it to burn up before hitting the ground.
\stopsolution
\stopexample

\subject{Notes}
%\placefootnotes[criterium=chapter]
\placenotes[endnote][criterium=chapter]
 
%\subject{Bibliography}
%        \placelistofpublications


\stopchapter
\stoptext
\stopcomponent

%[Example problem: Make a plot of the gravitational potential around a $10\units{kg}$ bowling ball. Make the plot from the surface of the bowling ball at $r=0.10\units{m}$ to a distance $r = 1\units{m}$ from the center of the of the ball.]
%\placefigure[margin][] % location
%{}	% caption text
%\hspace{-7pt}
%%{\starttikzpicture
%%\datavisualization [scientific axes, x axis={attribute=r,
%%label={[node style={inner sep=0pt,outer sep=0pt, below = 1mm}]$r$ (m)}, include value=0, length=3.9cm},
%%y axis={attribute=Phi, label={[node style={inner sep=0pt,outer sep=0pt}] $\Phi$ ($\sci{-9}\unit{m^2/s^2}$)}, include value=0,include value=-8, ticks={step=2}},  visualize as smooth line]
%%data [format=function] {
%%    var r : interval [.11:1];
%%    func Phi = -0.667/\value r;
%%}
%%info {
%%\fill [gray] (visualization cs: r=0, Phi=-8) rectangle (visualization cs: r=.11, Phi=0);
%%};
%%\stoptikzpicture}
%{\starttikzpicture
%\startaxis
% [footnotesize, width=2.13in, height=2in,
%   xlabel={$r$ (m)},
%   xmin=0, xmax=1,
%   ylabel={$\Phi$ ($\sci{-9}\unit{m^2/s^2}$)},
%   ymin=-8, ymax=0,
%   ytick={-8,-6,...,0},
% ]
% \addplot[
%   thick,
%   domain=.11:1,
%   samples=201
%  ]
%  {-0.667/x}
%  ;
%\startpgfonlayer{background}
%\fill [black!20] (0, -8) rectangle (.11, 0);
%\stoppgfonlayer
%\stopaxis
%\stoptikzpicture}
%\caption[Gravitational potential of a ball]{
%The gravitational potential in the vicinity of a $10\units{kg}$ bowling ball. The potential is extremely small, and can only be detected with the most sensitive instruments. The potential is not shown inside the bowling ball (the gray region).
%}
%\label{fig:BallGravPot}
%%



%The gravitational potential energy is always negative. There is no reason to be alarmed by this. It is negative because gravity is attractive, so the gravitational potential energy must decrease as objects approach each other. Since the potential energy is close to zero at large distances, the potential energy can only go down by going negative as the objects approach each other. Kinetic energy is always positive, as is the energy stored in a spring, but gravitational potential energy is negative.


%\subsection{Gravitational potential energy}
%Lifting a heavy thing requires energy. The energy can be extracted by lowing the thing. The stored energy is gravitational potential energy.
%\begin{equation}
%	U_G = mgy
%\end{equation}
%
%\section{Gravity}
%
%First we introduce the gravitational force. Notice that we are now moving in the $y$ direction.
%\begin{equation}
%	F_g = -mg
%\end{equation}
%This force can be used in problems using momentum or energy, time or distance.
%
%
%\placefigure[margin][] % location
%{}	% caption text
%	\begin{center}
%	\marginfig{GravPotentialEnergy}	
%%	\marginfig{matplotlibtest}	
%	\end{center}
%	\caption[Gravitational Potential Energy]{Gravitational potential energy is proportional to the mass and the height.} 
%
%
%
%\subsection{Freefall}
%\label{sec:freefall}
%
%Let's see what each of these can tell us about the motion of the cart, starting with conservation of energy. In this case we will consider the cart's kinetic energy, $K$, and gravitational potential energy, $U$.
%\startformula\startmathalignment[m=2,distance=2em]
%	H\sf \NC = H\si + \cancel{W} + \cancel{Q}	\NC \NC \text{no outside work or heat}\NR
%	K\sf + \cancel{U\sf} \NC = K\si + U\si		\NC \NC \text{set $U=0$ at the ground}\NR
%	\half mv\sf^2 \NC = \half mv\si^2 + mgy\si	\NC \NC \text{formulae for $K$ and $U$}\NR
%	\half v\sf^2 \NC = \half v\si^2 + gh		\NC \NC \text{divide both sides by $m$}\NR
%	v\sf^2 \NC = v\si^2 + 2gh					\NC \NC \text{multiply both sides by 2}\NR
%	v\sf \NC = \pm\sqrt{v\si^2 + 2gh}			\NC \NC \text{square root of both sides}
%\stopmathalignment\stopformula
%When taking the square root of in the last line, we must include a plus-or-minus. The mathematics cannot tell us which sign to pick, but often the correct sign is obvious from the problem. For example, if an object is falling to the ground, then the final velocity in negative, so replace the $\pm$ with a negative sign.
%
%
%The downward force is the object's weight. Objects with a small mass, like a gnat, have a small weight. Objects with a large mass also have a large weight.
%
%In fact, an object's weight, $F\sub{G}$ is proportional to the object's mass, $m$, and it is always downward. On Earth the force vector is 
%
%The words “mass” and “weight” are often treated as synonyms, but to a physicist they are different. %They 
%
%Take a moment to see that the equation fits with what you've experienced about gravitational force.  Weight is proportional to mass, meaning that the force pulling down on two identical boxes is double the force pulling down on only one.  
%
%
%\subsection{Incline}
%
%
%With Hamilton's equations, energy became the underlying driver of interactions between objects. 
%
%William Roan Hamilton writes
%
%\startblockquote
%	Lagrange has perhaps done more than any other analyst, to give extent and harmony to such deductive researches, by showing that the most varied consequences respecting the motions of systems of bodies may be derived from one radical formula; the beauty of the method so suiting the dignity of the results, as to make of his great work a kind of scientific poem. [Hamilton, \booktitle{On a General Method in Dynamics}, p.1]
%\stopblockquote
%
%
%\section{Motion with a constant force}
%
%For the programming I do not want to find average velocity. We start with a half time step to estimate the average momentum during the time step.
%
%We use that momentum to find the velocity and that velocity to update the position.
%Then we use a full time step to update the momentum and repeat the cycle.
%\placefigure[margin][] % location
%{}	% caption text
%	\marginfig{PavgConstF}
%	\caption{When the force is constant, the velocity at $t_\frac{1}{2}$ is the average velocity. This makes it easy to update the position.}
%
%
%\subsection{Momentum vs. time graphs}
%
%\begin{align}
%	x_1 \NC = x_0 + v\sub{avg} \Delta t \NR
%		\NC  =  x_0 + v\sub{avg} \Delta t
%\end{align}


% Templates: 

% Margin image
\placefigure[margin][] % Location, Label
{} % Caption
{\externalfigure[chapter03/][width=144pt]} % File

% Margin Figure
\placefigure[margin][] % location
{}	% caption text
{\starttikzpicture	% tikz code
\stoptikzpicture}

% Aligned equation
\startformula\startmathalignment
\stopmathalignment\stopformula

% Aligned Equations
\startformula\startmathalignment[m=2,distance=2em]
\stopmathalignment\stopformula


