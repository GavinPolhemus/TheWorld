% !TEX useAlternatePath
% !TEX useConTeXtSyncParser

\startcomponent *
\project project_world
\product prd_volume01

\doifmode{*product}{\setupexternalfigures[directory={chapter16/images}]}

\setupsynctex[state=start,method=max] % "method=max" or "min"

%%%%%%%%%%%%%%%%%%%%%%%%%%%%%
\startchapter[title={Cosmology}, reference=ch:Cosmology]
%%%%%%%%%%%%%%%%%%%%%%%%%%%%%

\placefigure[margin,none]{}{\small
	\startalignment[flushleft]
It is remarkable that we can say with some confidence what the universe was like far away and in the remote past.%\autocite{p.46}{Helmholtz1857}
	\stopalignment
	\startalignment[flushright]
	%{\it On the Physiological Causes\\
	%	of Harmony in Music}\\
	{\sc P.~J.~E.~Peebles}\\
	1935 --  %\scaps{BCE}
	\stopalignment
}

%%%%%%%%%%%%%%%%%%%%%%%%%%%%%

%\placetable[margin][T:Today] % Label
%    {{\bf Today} The seventeen particles of the Standard Model} % Caption
%    {\vskip9pt\tfb\starttabulate[|c|c|c|c|]
%\FL[2]
%\NS[3][c] $h$          			\NR
%\HL
%\NC $\nu_e$	   \NC $e$    \NC $d$ \NC $u$ \NR
%\NC $\nu_\mu$  \NC $\mu$  \NC $s$ \NC $c$ \NR
%\NC $\nu_\tau$ \NC $\tau$ \NC $b$ \NC $t$ \NR
%\HL
%\NC $W$	     \NC $Z$ \NC $\gamma$ \NC $g$ \NR
%\LL[2]
%\stoptabulate}

\noindent
At the beginning of the twentieth century, very little was known about our Universe beyond our own Milky Way Galaxy. In fact, there was a great debate about whether \emph{anything} was beyond our own Milky Way! During the twentieth century, we learned that the Universe is vast, containing innumerable galaxies. We also learned that the Universe is very old, and constantly expanding. The Universe's expansion is the most surprising and most important of these discoveries. 

%\section{Dark Matter}

%\section{Far away and long ago}
%The Universe is Homogeneous and Isotropic. [Ryden 2.2]


\section{The expanding Universe}
With modern telescopes we see a multitude of galaxies beyond our own. There are a few relatively close galaxies, and many more much farther away. The light from faraway galaxies is shifted slightly toward the red end of the spectrum. Farther galaxies show a greater redshift than closer galaxies.

We do not know the full size of the Universe, but the redshift tells us that the Universe is stretching out, causing far away galaxies to get ever farther. 
This stretching also stretches lightwaves, making them more red. The expansion is slow, so only light that has traveled for an extremely long time\dash across the vast distances between galaxies\dash will show this redshift. The longer the journey, the greater the redshift.

We describe this stretching with a \keyterm{scale factor}, represented by $a$, which relates the past and future distances between galaxies to the distances today. If the distance to a far off galaxy is $r_0$ today, then the distance $r$ to that galaxy at another time was (or will be)
\startformula
  r = ar_0.
\stopformula
In the past, distances were smaller than today, so $a$ was less than one. In the future, as distances become larger, $a$ will be greater than one. Cosmologists use the subscript zero for today's values, so $a_0=1$ by definition.

%2.5 Cosmic Microwave Background.
The scale factor also determines how much light is stretched during travel.
Light's wavelength $\lambda$ stretches in exactly the same way as the intergalactic distance $r$.
If the light is arriving with wavelength $\lambda_0$ today, then its original wavelength $\lambda$ was
\startformula
  \lambda = a\lambda_0,
\stopformula
where $a$ was the scale factor when the light was emitted.
Since the light was emitted in the past, when $a$ was less than one, the original wavelength $\lambda$ was shorter than today's wavelength $\lambda_0$. As the light traveled, it was redshifted from the original shorter wavelength $\lambda$ to the longer, redder wavelength $\lambda_0$ that we see today.

Not everything in the universe is stretching the way light and intergalactic distances stretch. For example, you are not stretching, because your atoms are held together by electromagnetic forces. Our solar system and our galaxy are not stretching because they are held together by gravitational forces. 
Only at very large, intergalactic distances are gravitational forces too weak to counteract the Universe's stretching.
Later in this chapter, when we know more of the Universe's story, we will return to the question of which large structures are held together by gravity and which stretch with Universe.

Light stretches with the Universe because there is no force holding the lightwaves together. Although the waves are very close to each other, each wave makes its long journey without any push or pull from the waves ahead or behind. Because the Universe is expanding, the journey from one galaxy to another is slightly shorter for earlier waves and slightly longer for later waves. With nothing holding the waves together, the earlier waves get slightly ahead, while the later waves fall slightly behind, stretching the wavelength. 


%%%%%%%%%%%%%%%%%%%%%%%%%%%%%%%%%%%%%%%%%%%%%%%%%%%
\startexample[ex:StretchingPhotons]
Light is made of photons. As photons travel through space, their wavelengths get longer, which will change their energy. Consider a photon with energy $E_0$ today. Assuming the photon is not absorbed, find its energy when the Universe has expanded by a scale factor $a$.
\startsolution
Using our knowledge of photons from earlier chapters, we find the photon's energy $E$ in terms of its momentum $p$ and then its wavelength $\lambda$.
\startformula
  E = p = \frac{hc\cdot1\units{cyc}}{\lambda}
\stopformula
For today's energy $E_0$ in terms of today's wavelength $\lambda_0$, that formula is
\startformula
  E_0 = \frac{hc\cdot1\units{cyc}}{\lambda_0}.
\stopformula
Take the ratio of the last two formulas and use the fact that the photon's wavelength $\lambda$ increases in proportion to the scale factor by $\lambda=a\lambda_0$.
\startformula
  \frac{E}{E_0} = \frac{hc\cdot1\units{cyc}/\lambda}{hc\cdot1\units{cyc}/\lambda_0}
  = \frac{\lambda_0}{\lambda}
  = \frac{\lambda_0}{a\lambda_0}
  = a^{-1}
\stopformula
Solve for $E$.
%\placeformula[eq:photonE]{(photons)}
\startformula
  \answer{ E = a^{-1}E_0 }
\stopformula
As the Universe expands by the scale factor $a$, the photon's energy decreases by the factor $a^{-1}$.
\stopsolution
\stopexample
%%%%%%%%%%%%%%%%%%%%%%%%%%%%%%%%%%%%%%%%%%%%%%%%%%%

The energy lost by an individual photon is very small. If we consider a large volume of space containing many, many photons going is all directions, the loss of energy is significant!
As the Universe expands by the scale factor $a$, every photon's energy decreases by the factor $a^{-1}$, so the total energy in the volume decreases by the factor $a^{-1}$. 

I often hear people state the law of conservation of energy as, \quotation{The total energy in the Universe is constant,} or \quotation{Energy cannot be created or destroyed.} In an expanding Universe full of photons\dash like our Universe\dash these statements are clearly false. As the Universe expands, the total photon energy decreases. That energy is not gained by anything else in the Universe. That energy is lost\dash or \quotation{destroyed} if you prefer more dramatic language.

The conservation of energy law is still true, it is simply necessary to state the law correctly, as we have done many times before:
\startformula
  H\si + W + Q = H\sf
\stopformula
For a universe full of \emph{only} photons (not like our Universe) the total energy $H$ is the energy of the photons $E$. There is no heat $Q$ either added or removed (as there is no place for it to come from or go to). However, there is work $W$ done by the Universe's expansion. This acts just like the expansion of a gas in a nineteenth-century steam engine. Recalling the work formula for engines, conservation of energy becomes 
\startformula
  E\si - P \Delta V = E\sf.
\stopformula
Where $P$ is the pressure of the gas\dash in this case a gas of photons\dash and $\Delta V$ is the change in volume due to the Universe's expansion. The photon gas has a positive pressure. The volume increases, so $\Delta V$ is also positive. The positive pressure and positive change in volume in the equation above result in a decrease in the energy, exactly as we expected based on the stretching wavelength of the photons. Carefully solving the conservation of energy law (which requires some calculus) shows that the change in energy is exactly what we calculated based on the photon stretching.

%%%%%%%%%%%%%%%%%%%%%%%%%%%%%%%%%%%%%%%%%%%%%%%%%%%%
%\startexample[ex:StretchingPhotons]
%This example is for students with calculus skills. The pressure $P$ of a photon gas is
%\startformula
%  P = \frac{1}{3}\frac{E}{V}
%\stopformula
%Where $E$ is the total energy of the photons and $V$ is the volume. Since the energy $E$ decreases as the volume $V$ increases, the pressure $P$ decreases as the Universe expands, and we constant use our constant pressure formulas. Instead, we need to consider small changes $dE$ in the energy due to small changes in volume $dV$. In this incremental approach, conservation of energy is 
%\startformula
%  dE = -P\,dV
%\stopformula
%Use the pressure for a photon gas to find $E$ as a function of $V$. You will need to separate variables and integrate. Consider the volume to be a large cube with edge length $r$, and find 
%\startformula
%  \answer{ E = a^{-1}E_0 }
%\stopformula
%\startsolution
%  Solution to be provided later....
%\stopsolution
%\stopexample
%%%%%%%%%%%%%%%%%%%%%%%%%%%%%%%%%%%%%%%%%%%%%%%%%%%%

Unlike the photons described above, matter in our universe does not produce significant pressure. The dust, rocks, stars, and interstellar gas produce almost no pressure at all. As the universe expands, this stuff generally spreads out, but no significant energy is lost. For typical matter, most of its energy is in its mass, while very little energy is kinetic or potential. The mass energy does not decrease as the universe expands, so no energy is lost.

The typical behavior of total photon energy and the total mass energy are shown in \in{figure}[fig:ScaleFactorPhotonsMatter]. As the universe expands, the total mass energy in an expanding volume remains constant. The total photon energy in that volume drops like $a^{-1}$.

\startuseMPgraphic{graph::ScaleFactorPhotonsMatter} % I'd like to add minor ticks, 0.667mm long.
vardef U =
	path p;
		for x = 0.01 step 0.01 until 1.1:
			y := 1/x; % lua.mp.morse(x);
			augment.p(x,y);
		endfor;
	p enddef;
draw begingraph(4.4cm,5cm);
	setrange(0,0, 1.1,5);
	for x=auto.x:
%		itick.bot(formatted("$@g$", x), x);
		itick.bot(formatted("@s", ""), x) withcolor "middlegray";
%		itick.top(formatted("@s", ""), x) withcolor "middlegray";
	endfor
	glabel.lft(textext("$E$"), OUT);
	glabel.bot(textext("$a$"), OUT);
	gdraw(U) withpen pencircle scaled 0.8pt;
	glabel.urt("Photons: $E_\gamma$",50); %  = a^{-1}E_{\gamma,0}
%	glabel(mydot,(80));
%	glabel(mydot,(210));
%	glabel(mydot,(340));
%	gfill(unitsquare xyscaled (6.37,-2)) withcolor "lightgray";
	gdraw((0,4) -- (1.1,4)) withpen pencircle scaled 0.8pt;
	glabel.bot("Matter: $E\sub{m}$",0.6); %  = E\sub{m,$0$}
%	for y=0 step -0.5 until -2:%auto.y:
%		itick.lft(formatted("$@g$", y), y);
%		itick.lft(formatted("@s", ""), y) withcolor "middlegray";
%		itick.rt(formatted("@s", ""), y) withcolor "middlegray";
%	endfor
endgraph 
\stopuseMPgraphic

\startplacefigure[location=margin, reference=fig:ScaleFactorPhotonsMatter, title={The typical behavior of the total photon energy ($E_\gamma$) and the total matter energy ($E\sub{m}$) in an expanding volume.}]
\small\reuseMPgraphic{graph::ScaleFactorPhotonsMatter}
\stopplacefigure

Our Universe has both photons and matter, and there is actually very little energy transfer between them on the vast scale of the universe. The matter particles are so spread out that they emit and absorb very few photons compared to the multitude of photons already zipping about. The matter particles' energy, which is mostly in their mass, is hardly affected by the gentle jostling of the photons.

\startuseMPgraphic{graph::ScaleFactorPhotonsMatterReal} % I'd like to add minor ticks, 0.667mm long.
vardef U =
	path p;
		for x = 0.001 step 0.001 until 1.1:
			y := 0.00535/x; % lua.mp.morse(x);
			augment.p(x,y);
		endfor;
	p enddef;
draw begingraph(4.4cm,17.5cm);
	setrange(0,0, 1.1,5);
	for x=auto.x:
		itick.bot(formatted("$@g$", x), x);
		itick.bot(formatted("@s", ""), x) withcolor "middlegray";
		itick.top(formatted("@s", ""), x) withcolor "middlegray";
	endfor
	glabel.lft(textext("$E$"), OUT);
	glabel.bot(textext("$a$"), OUT);
	gdraw(U) withpen pencircle scaled 0.8pt;
	glabel.urt("Photons: $E_\gamma$",50); %  = a^{-1}E_{\gamma,0}
%	glabel(mydot,(80));
%	glabel(mydot,(210));
%	glabel(mydot,(340));
%	gfill(unitsquare xyscaled (6.37,-2)) withcolor "lightgray";
	gdraw((0,4.8) -- (1.1,4.8)) withpen pencircle scaled 0.8pt;
	glabel.bot("Matter: $E\sub{m}$",0.5); %  = E\sub{m,$0$}
%	for y=0 step -0.5 until -2:%auto.y:
%		itick.lft(formatted("$@g$", y), y);
%		itick.lft(formatted("@s", ""), y) withcolor "middlegray";
%		itick.rt(formatted("@s", ""), y) withcolor "middlegray";
%	endfor
endgraph 
\stopuseMPgraphic

\startplacefigure[location=margin, reference=fig:ScaleFactorPhotonsMatterReal, title={In our Universe, the energy of photons ($E_\gamma$) has been much greater than the energy of matter ($E\sub{m}$) since very early times.}]
\small\reuseMPgraphic{graph::ScaleFactorPhotonsMatterReal}
\stopplacefigure

As a result of this independence, the energy of photons ($E_\gamma$) and the energy of matter ($E\sub{m}$) in our Universe each change exactly as described above. The energy of matter stays constant, while the energy of photons decreases like $a^{-1}$. These energies are shown in \in{figure}[fig:ScaleFactorPhotonsMatterReal] as a function of the scale factor $a$. Today, when the scale factor is $a=1$, the energy in the matter is much greater than the energy of the photons. In the distant past, when $a$ was tiny, the photons had more energy.

While the total energy of matter in an expanding volume stays constant, the energy density $\rho$ of that matter is decreasing as the volume grows. Consider a volume which is a large cube of edge length $r$. The volume is $V = r^3$. The energy density of matter in that volume is
\startformula
  \rho = \frac{E}{V}
    = \frac{E}{r^3}
    = \frac{E_0}{(ar_0)^3}
    = \frac{1}{a^3}\frac{E_0}{r_0^3}.
\stopformula
The energy density today is $\rho_0 = E_0/r_0^3$. This gives the density as a function of scale factor.
\startformula
  \rho = a^{-3}\rho_0
\stopformula
As the Universe expands, the energy density of matter decreases like $a^{-3}$.

As the Universe expands, the energy density of photons decreases for two reasons. First, the photons get more spread out, just like the matter. Second, the energy of the individual photons decreases, unlike the particles of matter. The energy density of photons in an expanding volume is
\startformula
  \rho = \frac{E}{V}
    = \frac{E}{r^3}
    = \frac{a^{-1}E_0}{(ar_0)^3}
    = \frac{1}{a^4}\frac{E_0}{r_0^3}.
\stopformula
The energy density of photons today is $\rho_0 = E_0/r_0^3$. This gives the density as a function of scale factor.
\startformula
  \rho = a^{-4}\rho_0
\stopformula
As the Universe expands, the energy density of photons decreases like $a^{-4}$.
These energy densities will play a key role in the Universe's story.

\section{Hubble's rate of expansion}
%Redshift is proportional to distance. [Ryden 2.3]
As the Universe expands, the distances between us and faraway galaxies grows. The farther the galaxy, the faster the growth. If fact, the rate at which the distance grows, $dr/dt$, is proportional to the distance $r$. This is Hubble's law.
\startformula
  \frac{dr}{dt} = Hr
\stopformula
The coefficient $H$ is called the \keyterm{Hubble rate}. A small value of $H$ represents a slower expansion, while a large value of $H$ represents a more rapid expansion. A contracting universe would have a negative $H$. %In a static universe, which is not expanding or contracting, $H$ would be zero.

The Hubble rate $H$ may change over time, as the expansion slows down or speeds up. Today, the Hubble rate is 
\startformula
  H_0 = 2.20\sci{-18}\units{s^{-1}}
    = 6.95\sci{-11}\units{yr^{-1}}
\stopformula
Today's Hubble rate, $H_0$, is known as the \keyterm{Hubble constant}.


%%%%%%%%%%%%%%%%%%%%%%%%%%%%%%%%%%%%%%%%%%%%%%%%%%%%
\startexample[ex:HubbleLaw]
A galaxy is currently at distance $r = 1.00\sci{6}\unit{yr} = 1.00\units{Myr}$ from us. At what rate is that distance increasing?
\startsolution
The rate of increase is $dr/dt$, which can be found using Hubble's law.
\startformula
  \frac{dr}{dt} = Hr
    = (6.95\sci{-11}\units{yr^{-1}})(1.00\sci{6}\unit{yr})
    = 6.95\sci{-5}
\stopformula
The distance to the galaxy is increasing at a rate $6.95\sci{-5}$. This is a unitless speed. To get a speed in meters per second, multiply by the speed of light: $6.95\sci{-5} c = 20.8\units{km/s}$.
\stopsolution
\stopexample
%%%%%%%%%%%%%%%%%%%%%%%%%%%%%%%%%%%%%%%%%%%%%%%%%%%%

The rate $dr/dt$ looks like a speed, and we will often refer to this as the faraway galaxy's speed. However, keep in mind that this speed is not due to the galaxy's motion. Rather, it is due to the stretching of space between us and the faraway galaxy. At great enough distances, this speed may be greater than the speed of light, but that is of no concern.
The galaxy is not moving faster than the speed of light. Rather, the space between us and the galaxy is so vast that the Universe's stretching contributes additional distance at a rate greater than $3.00\sci{8}\units{m/s}$.

\startuseMPgraphic{graph::ConstVel} % I'd like to add minor ticks, 0.667mm long.
path p;
p = (origin) -- (1.1,3) ;
draw begingraph(4.4cm,5cm);
	setrange(0,-2.5, 1.1,2.5);
	for x=auto.x:
%		itick.bot(formatted("$@g$", x), x);
		itick.bot(formatted("@s", ""), x) withcolor "middlegray";
%		itick.top(formatted("@s", ""), x) withcolor "middlegray";
	endfor
	glabel.lft(textext("$r$"), OUT);
	glabel.bot(textext("$t$"), OUT);
	for r=25 step -0.25 until -25:%auto.y:
	  gdraw(p) yscaled r withpen pencircle scaled 0.8pt;
	endfor
%	glabel.urt("Photons: $E_\gamma$",50); %  = a^{-1}E_{\gamma,0}
	gfill(unitsquare xyscaled (0.04,5) shifted (0,-2.5)) ;
	gdraw((0,0) -- (1.1,0)) withpen pencircle scaled 1.2pt;
%	glabel.bot("Matter: $E\sub{m}$",0.6); %  = E\sub{m,$0$}
%	for y=0 step -0.5 until -2:%auto.y:
%		itick.lft(formatted("$@g$", y), y);
%		itick.lft(formatted("@s", ""), y) withcolor "middlegray";
%		itick.rt(formatted("@s", ""), y) withcolor "middlegray";
%	endfor
endgraph 
\stopuseMPgraphic

\startplacefigure[location=margin, reference=fig:ConstVel, title={In a universe where each galaxy maintains a constant velocity, the Hubble rate decreases with time. $H=1/t$.}]
\small\reuseMPgraphic{graph::ConstVel}
\stopplacefigure

Consider, for a moment, a universe where every galaxy's speed $v$ remains constant as the universe expands, so that the galaxy's distance at time $t$ is $r=vt$. \in{Figure}[fig:ConstVel] shows many galaxies, each moving with a constant speed away from $r=0$. Some galaxies are fast, others slow, but for each the speed stays constant, $dr/dt = v$.

In this universe of constant-speed galaxies, Hubble's Law becomes
\startformula\startmathalignment
\NC \frac{dr}{dt} \NC= Hr \NR
\NC             v \NC= Hvt \NR
\stopmathalignment\stopformula
Solving for $H$ gives the Hubble rate as a function of time.
\startformula
 H = \frac{1}{t}
\stopformula
While each galaxy has a constant speed, we see that the Hubble rate is decreasing. At early times, $H$ is very large, because even very fast galaxies are still at small distances. At late times, $H$ is small, because even very slow galaxies have had time to move a large distance away.

This universe of constant speed galaxies is an example of a \keyterm{Big Bang} universe, in which everything starts with an explosive expansion at $t=0$.

%%%%%%%%%%%%%%%%%%%%%%%%%%%%%%%%%%%%%%%%%%%%%%%%%%%%
\startexample[ex:HubbleTime]
If we lived in a universe of constant speed galaxies, with the present-day Hubble constant given above, what would be the universe's present age?
\startsolution
In such a universe, the Hubble rate is related to the time by $H=1/t$. In particular, the Hubble rate today, $H_0$, is related to the present time, $t_0$, by
\startformula
 H_0 = \frac{1}{t_0}.
\stopformula
Solve for the present time, $t_0$.
\startformula
  t_0 = \frac{1}{H_0}
    = \frac{1}{6.95\sci{-11}\units{yr^{-1}}}
    = 1.44\sci{10}\units{yr}
    = 14.4\units{Gyr}
\stopformula
The present age would be $14.4\units{Gyr}$, or 14.4 billion years.
\stopsolution
\stopexample
%%%%%%%%%%%%%%%%%%%%%%%%%%%%%%%%%%%%%%%%%%%%%%%%%%%%

\startuseMPgraphic{graph::ConstH} % I'd like to add minor ticks, 0.667mm long.
vardef U =
	path p;
		for x = 0 step 0.01 until 1.1:
			y := mexp(256x); % lua.mp.morse(x);
			augment.p(x,y);
		endfor;
	p enddef;
draw begingraph(4.4cm,5cm);
	setrange(0,-2.5, 1.1,2.5);
	for x=auto.x:
%		itick.bot(formatted("$@g$", x), x);
		itick.bot(formatted("@s", ""), x) withcolor "middlegray";
%		itick.top(formatted("@s", ""), x) withcolor "middlegray";
	endfor
	glabel.lft(textext("$r$"), OUT);
	glabel.bot(textext("$t$"), OUT);
	for r=2.5 step -0.25 until -2.5:%auto.y:
	  gdraw(U) yscaled r withpen pencircle scaled 0.8pt;
	endfor
%	glabel.urt("Photons: $E_\gamma$",50); %  = a^{-1}E_{\gamma,0}
%	glabel(mydot,(80));
%	glabel(mydot,(210));
%	glabel(mydot,(340));
%	gfill(unitsquare xyscaled (6.37,-2)) withcolor "lightgray";
	gdraw((0,0) -- (1.1,0)) withpen pencircle scaled 1.2pt;
%	glabel.bot("Matter: $E\sub{m}$",0.6); %  = E\sub{m,$0$}
%	for y=0 step -0.5 until -2:%auto.y:
%		itick.lft(formatted("$@g$", y), y);
%		itick.lft(formatted("@s", ""), y) withcolor "middlegray";
%		itick.rt(formatted("@s", ""), y) withcolor "middlegray";
%	endfor
endgraph 
\stopuseMPgraphic

\startplacefigure[location=margin, reference=fig:ConstH, title={In a universe with a constant $H$, individual galaxies accelerate away in all directions.}]
\small\reuseMPgraphic{graph::ConstH}
\stopplacefigure

\noindent
Next, consider another universe where the Hubble rate is constant, $H = H_0$. \in{Figure}[fig:ConstH] shows many galaxies in this constant-$H$ universe. The galaxies speed up as they get farther away, so the Hubble rate stays the same. This is called an \keyterm{accelerating universe}, because the galaxies appear to accelerate as they get farther away. In fact, the added distance $r$ is simply causing a greater rate $dr/dt$, in accordance with Hubble's law.
The constant-$H$ universe expands forever in the future, and has been expanding forever in the past. It does not have a Big Bang.

For any of our universe models, the expansion can be concisely described by the scale factor $a$. Rather than showing the distances to many galaxies (as in \in{figs.}[fig:ConstVel] and \in[fig:ConstH]) we will show the scale factor $a$ as a function of time when discussing different cosmological models. \in{Figure}[fig:ScaleFactors] shows the scaled factors for the two universes we have been considering, one with constant speed galaxies and one with constant Hubble rate.
In the universe of constant speed galaxies, the scale factor grows linearly with time.
\startformula
  a = H_0 t
\stopformula
This is the straight line in \in{figure}[fig:ScaleFactors]. In the universe with constant Hubble rate, the scale factor grows exponentially with time.
\startformula
  a = e^{H_0(t-t_0)}
\stopformula
This is the curving line in \in{figure}[fig:ScaleFactors].

\startuseMPgraphic{graph::ScaleFactors} % I'd like to add minor ticks, 0.667mm long.
vardef U =
	path p;
		for x = 0 step 0.01 until 2:
			y := mexp(256(x-1)); % lua.mp.morse(x);
			augment.p(x,y);
		endfor;
	p enddef;
draw begingraph(4cm,3cm);
	setrange(0,0, 1.5,2);
	itick.bot(formatted("$t_0$", 1), 1);
	for x=auto.x:
		itick.bot(formatted("@s", ""), x) withcolor "middlegray";
		itick.top(formatted("@s", ""), x) withcolor "middlegray";
	endfor
	glabel.lft(textext("$a$"), OUT);
	glabel.bot(textext("$t$"), OUT) shifted(0,5pt);
	gdraw((1,0) -- (1,1) -- (0,1)) withpen pencircle scaled 0.6pt withcolor "middlegray";
%	glabel.rt("$t_0$",0.5); %  = E\sub{m,$0$}
	gdraw(U) withpen pencircle scaled 0.8pt;
	gdraw((0,0) -- (2,2)) withpen pencircle scaled 0.8pt;
%	glabel.bot("Matter: $E\sub{m}$",0.6); %  = E\sub{m,$0$}
	for y=0 step 1 until 2:%auto.y:
		itick.lft(formatted("$@g$", y), y);
		itick.lft(formatted("@s", ""), y) withcolor "middlegray";
		itick.rt(formatted("@s", ""), y) withcolor "middlegray";
	endfor
endgraph 
\stopuseMPgraphic

\startplacefigure[location=margin, reference=fig:ScaleFactors, title={The scale factor $a$ for two universes: a universe of constant-speed galaxies (straight line) and a universe with a constant Hubble rate (curved line).}]
\small\reuseMPgraphic{graph::ScaleFactors}
\stopplacefigure

In both models, the present-day, $t_0$, value of the scale factor is $a=1$.
The scale factor's rate of change, $da/dt$, is determined by the Hubble rate.
%\startformula\startmathalignment
%\NC \ddt r \NC= Hr \NR
%\NC \ddt a r_0 \NC= H a r_0 \NR
%\NC r_0 \ddt a \NC= H a r_0 \NR
%\NC \frac{da}{dt} \NC= Ha \NR
%\stopmathalignment\stopformula
\startformula
  \frac{da}{dt} = Ha
\stopformula
%The rate of growth today is $H_0 = \frac{da}{dt}$.
So far, we have looked at models that were simply guesses. To make realistic models, we need a tool for predicting the Hubble rate $H$. This tool is Einstein's theory of General Relativity.


\section{Einstein's Universe}
Einstein's theory of Relativity was widely accepted after Minkowski's lecture. Minkowski's description of a four-dimensional world governed by rules of four-dimensional geometry was both inspiring and practical. Einstein built on the four-dimensional ideas by developing a more general theory in which the four-dimensional world can be curved and dimpled. Large masses, like the Earth and Sun, create these dimples in space-time. The dimples then bend the paths of other objects, attracting them toward the large masses. This attraction is what we call gravity.

Einstein's theory of gravity, completed in 1915, became known as \keyterm{General Relativity}. (His original theory, described by Minkowski, became known as Special Relativity.) General Relativity correctly predicted the bending of starlight near the Sun and some strange behavior in Mercury's orbit. The mathematics of General Relativity is far more difficult than Special Relativity, so progress and acceptance we slower.

In addition to predicting the dimples responsible for gravity, Einstein's General Relativity predicts the large scale stretching of the Universe, just as we have been discussing. In fact, General Relativity predicts the exact value of the Hubble rate $H$. The formula for the Hubble rate was first worked out by Russian physicist Alexander Friedmann in 1922. The formula is now called the \keyterm{Friedmann equation}.
\startformula
  H^2 = \frac{8\pi G}{3}\rho
\stopformula
$G$ is Newton's gravitational constant, and $\rho$ is the average energy density.

The Friedmann equation is difficult to derive, but easy to use. It states that the Hubble rate $H$ is determined by the total energy density $\rho$. The total energy density includes all types of energy and mass\dash ordinary matter, photons, neutrinos, and dark matter. Since the total energy density is always greater than zero, the Hubble rate cannot be zero, and the Universe must be either expanding (positive $H$) or contracting (negative $H$). Since we know that our Universe is expanding, we will only consider positive $H$.

The coefficient in the Friedmann equation is
%\startformula
%  \frac{8\pi G}{3} = 5.59\sci{-10}\units{m^3\.kg^{-1}\.s^{-2}}
%    = 9.96\sci{-46}\units{m^3\.eV^{-1}\.s^{-2}}
%\stopformula
\startformula
  \frac{8\pi G}{3} = \unit{5.59e-10 m^3 / kg s^2}
    = \unit{9.91e-31 m^3 / eV yr^2}.
\stopformula

%%%%%%%%%%%%%%%%%%%%%%%%%%%%%%%%%%%%%%%%%%%%%%%%%%%%
\startexample[ex:PresentEnergyDensity]
Find the present energy density, $\rho_0$, in our Universe. Give values for the density in both $\units{kg/m^3}$ and $\units{eV/m^3}$.
\startsolution
Solve the Friedmann equation for today's energy density.
\startformula
 \rho = \frac{3}{8\pi G} H^2
\stopformula
Find today's energy density, $\rho_0$, from today's Hubble rate $H_0$.
\startformula\startmathalignment
\NC \rho_0 = \frac{3 H_0^2}{8\pi G}
    \NC= \frac{(2.20\sci{-18}\units{s^{-1}})^2}{\unit{5.59e-10 m^3 / kg s^2}}
    = 8.7\sci{-27}\units{kg/m^3} \NR
\NC \NC= \frac{(6.95\sci{-11}\units{yr^{-1}})^2}{\unit{9.91e-31 m^3 / eV yr^2}}
    = 4.9\units{GeV/m^3} \NR
\stopmathalignment\stopformula
\stopsolution
\stopexample
%%%%%%%%%%%%%%%%%%%%%%%%%%%%%%%%%%%%%%%%%%%%%%%%%%%%


Chapter 5: Model Universes (We do some real calculations here!)
5.3 Single-component universes. We want to look at how the energy and density change with scale factor (no calculus required). Quote formulas as function of time (and hopefully keep calc. kids busy.)

The big idea here is to understand how energy density, Hubble rate, and scale factor change with time.

Qualitative first: The easiest universe to understand is filled with matter (like dust and gas) which has no pressure. The density decreases (like $a^{-3}$) as the universe expands, so H also decreases (like $a^{-3/2}$), and the expansion slows down. This is not slowing due to the gravity of all the galaxies pulling on each other, it is slowing due to the Friedman equation with decreasing density.

What if we add pressure, like radiation? In this case the density falls more quickly (like $a^{-4}$) so the Hubble rate slows more quickly (like $a^{-2}$) and the expansion slows more quickly than in the matter case. This is very surprising! I would have thought that outward pressure would help keep the expansion going, so that is slows less than the zero-pressure matter case. But the opposite happens. Outward pressure actually puts on the brakes, slowing the expansion!

What if the pressure is negative, pulling inward like a stretched rubber band? In that case, expansion does positive work, and the density decreases more slowly, the Hubble rate decreases more slowly, and the expansion does not slow as quickly as in the matter case. Negative pressure (which pulls inward) actually helps keep the expansion going!

In the extreme case, where negative pressure keeps the density constant (a cosmological constant), the Hubble rate also stays constant and we have an accelerating universe (discussed in the previous section.

Quantitatively, we can summarize the results like this for three model universes with different content.

\startitemize[n, packed]
\item Matter: $H = H_0 a^{-3/2}$
\item Radiation: $H = H_0 a^{-2}$
\item Cosmological const: $H = H_0$ (from earlier section)
\stopitemize

It is a simple matter, with a little calculus, to put these into the Hubble law and get the scale factor as a function of time.

\startitemize[n, packed]
\item Matter: $da/dt = H_0 a^{-1/2}$ leads to $a = \left(\textfrac{3}{2}H_0 t\right)^{2/3}$
\item Radiation: $da/dt = H_0 a^{-1}$ leads to  $a = (2H_0 t)^{1/2}$
\item Cosmo cons: is exponential in earlier section.
\stopitemize

The matter and radiation cases can be checked with our simple derivative rules.

The benchmark model contains all three of these, and can only be described qualitatively. We had a Hot Big Bang, so the universe starts of radiation dominated, and expands like the radiation equations above. The density of radiation decreases until there is less radiation than matter, starting a matter dominated phase. Finally, the matter content falls below the cosmological constant and we enter a new cosmological constant dominated phase. We entered this last phase pretty recently (about a billion years ago) and now live in an accelerating universe.



\stopchapter
\stopcomponent
%%%%%%%%%%%%%%%%%%%%%%%%%%%%%%%%%%%%%%%%%%%%%%%%%%%
%%%%%%%%%%%%%%%%%%%%%%%%%%%%%%%%%%%%%%%%%%%%%%%%%%%



The $\Lambda$CDM model [Ryden 5.5]: All three densities change! Find scale factors where they cross.


\section{Measuring the Universe}
Chapter 6: Measuring Cosmological Parameters (No math, just the results)
6.4 Standard Candles and H. Paralax, Cephied variables gives $H$ (and $H$ gives the density).
6.5 Standard Candles and Acceleration. Type Ia supernovas give cosmological const $\Lambda$.


\section{Cosmic microwave background}
Chapter 8: Cosmic Microwave Background (very qualitative)
8.1 Observing the CMB.
8.3 Physics of recombination. We just want to know that equilibrium gives eq. 8.37, and the timeline in Table 8.1.
8.5 What Causes the Fluctuations. Understand why Figure 8.6 has bumps on the right, but not the left. Physics of the early universe was simple, and matches perfectly with this observation!


\section{The first three minutes}
Chapter 9: Nucleosynthesis and the Early Universe
9.2 Neutrons and Protons. Know that protons and neutrons were in equilibrium due to very high temps. As temp drops, lighter protons are favored. At freeze out five-to-one ratio. Calculate the ratio of helium to hydrogen that results.
9.3-4 Deuterium and Beyond: They can write out all the nuclear reactions, then follow the qualitative story of Figure 9.4. Another great fit with observations!


\section{Stars, galaxies, and clusters}
Chapter 11: Structure Formation: Gravitational Instability
We just want the qualitative understanding of the Matthew Effect.

Chapter 12: Structure Formation: Baryons and Photons
Story of how first stars, galaxies, and clusters formed (Figure 12.2). If we can talk about how later stars are different, that would be great.



%\subject{Notes}
%%\placefootnotes[criterium=chapter]
%\placenotes[endnote][criterium=chapter]

%\subject{Bibliography}
%        \placelistofpublications

\stopchapter
\stopcomponent
%%%%%%%%%%%%%%%%%%%%%%%%%%%%%%%%%%%%%%%%%%%%%%%%%%%
%%%%%%%%%%%%%%%%%%%%%%%%%%%%%%%%%%%%%%%%%%%%%%%%%%%

%$6.241509\sci{18}$ electrons is \emph{negative} one Coulomb. One Coulomb of charge from a one volt battery gives one joule of energy. How many electron volts are in one joule? Converting the other way, one electron volt is how many joules? How many electron volts of energy would be produced by one mole of electrons going through a potential difference on one volt. How many joules? Calories too?


% Templates:

% Margin image
\placefigure[margin][] % Location, Label
{} % Caption
{\externalfigure[chapter03/][width=144pt]} % File

% Margin Figure
\startbuffer[TikZ:NAME]
\environment env_physics
\environment env_TikZ
\setupbodyfont [libertinus,11pt]
\setoldstyle % Old style numerals in text
\startTEXpage\small
\starttikzpicture% tikz code
\stoptikzpicture
\stopTEXpage
\stopbuffer

\placefigure[margin][fig:NAME] % Location, Label
{}	 % caption text
{\noindent\typesetbuffer[TikZ:NAME]}

% Aligned equation
\startformula\startmathalignment
\stopmathalignment\stopformula

% Aligned Equations
\startformula\startmathalignment[m=2,distance=2em]
\stopmathalignment\stopformula
